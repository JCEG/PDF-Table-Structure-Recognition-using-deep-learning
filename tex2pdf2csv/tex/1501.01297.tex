\documentclass{article}



\usepackage{amsmath}
\usepackage{amsthm}
\usepackage{amssymb}
\usepackage{amscd}
\usepackage{xspace}
\usepackage{verbatim}

\pagestyle{plain}
\begin{comment}
%\textwidth 15cm \oddsidemargin 0.75cm \evensidemargin 0.75cm
%\addtolength{\textheight}{2.2cm}
%\addtolength{\textwidth}{2cm}
%\addtolength{\topmargin}{-1cm}

%%%%\setlength{\topmargin}{-1in}
\setlength{\topmargin}{-0.9in}


\setlength{\textwidth}{6.7in}

%%%\setlength{\textheight}{1.20\textheight}

\setlength{\textheight}{1.15\textheight}

%\setlength{\oddsidemargin}{2cm}
%\setlength{\evensidemargin}{2cm}

\setlength{\oddsidemargin}{-0.3in}

\setlength{\evensidemargin}{-0.3in}
\renewcommand{\baselinestretch}{1.2}
\end{comment}


\pagestyle{plain}

\setlength{\topmargin}{-0.6in} \setlength{\textwidth}{6.in}

\setlength{\textheight}{1.20\textheight}



\setlength{\oddsidemargin}{-0.in}

\setlength{\evensidemargin}{-0.25in}




\renewcommand{\baselinestretch}{1.5}

\newtheorem{theorem}{Theorem}[section]
\newtheorem{corollary}{Corollary}[section]
\newtheorem{lemma}{Lemma}[section]
\newtheorem{proposition}{Proposition}[section]
\theoremstyle{definition}
\newtheorem{definition}{Definition}[section]
\newtheorem*{acknowledgment}{Acknowledgment}
\theoremstyle{remark}
\newtheorem{remark}{Remark}[section]
%\numberwithin{equation}{section}
%%%%%%%%%%%%%%%%%%%%%%%%%
\newcommand{\rlemma}[1]{Lemma~\ref{#1}}
\newcommand{\rth}[1]{Theorem~\ref{#1}}
\newcommand{\rcor}[1]{Corollary~\ref{#1}}
\newcommand{\rdef}[1]{Definition~\ref{#1}}
\newcommand{\rprop}[1]{Proposition~\ref{#1}}
%%%%%%%%%%
\newcommand{\weakly}{\rightharpoonup}
\renewcommand{\ni}{\noindent}
\newcommand{\ov}{\overline}
\renewcommand{\a}{\alpha}
\renewcommand{\b}{\beta}
\renewcommand{\d}{\delta}
\newcommand{\e}{\varepsilon}
\newcommand{\en}{\e_n}
\newcommand{\D}{\Delta}
\newcommand{\g}{\gamma}
\newcommand{\G}{\Gamma}
\renewcommand{\L}{\mathbf L}
\newcommand{\Ls}{\mathbf L_s}
\renewcommand{\k}{\kappa}
\renewcommand{\O}{\Omega}
\newcommand{\OO}{\overline{\Omega}}
\renewcommand{\o}{\omega}
\newcommand{\s}{\sigma}
\renewcommand{\S}{\Sigma}
\renewcommand{\t}{\theta}
%%%%%%
\renewcommand{\liminf}{\varliminf}
\renewcommand{\limsup}{\varlimsup}
\newcommand{\cb}{\bar c}
\newcommand{\nue}{\nu_\e}
\newcommand{\circs}{{\scriptstyle{\circ}}}
\renewcommand{\vec}[1]{\mathbf{#1}}
%\newcommand{\field}[1]{\mathbb{#1}}
%\newcommand{\C}{\field{C}}
\newcommand{\R}{\mathbb{R}}
\newcommand{\Inter}{\mathop{\bigcap}\limits}
\newcommand{\Union}{\mathop{\bigcup}\limits}
\newcommand{\Prod}{\mathop{\prod}\limits}
\newcommand{\er}{\eqref}
\newcommand{\kpe}{\k_\e}
\newcommand{\kpen}{\k_{\e_n}}

\newcommand{\lame}{\lambda_\e}
\newcommand{\bs}{\bar s}
\newcommand{\bv}{\bar v}
\newcommand{\bu}{\bar u}
\newcommand{\uu}{\underbar u}
\newcommand{\uln}{u_{{\l}_n}}
\newcommand{\ue}{u_\e}
\newcommand{\uen}{u_{\e_n}}
\newcommand{\tue}{\tilde u_\e}
\newcommand{\ve}{v_\e}
\newcommand{\vn}{v_n}
\newcommand{\ven}{v_{\e_n}}
\newcommand{\we}{w_\e}
\newcommand{\ze}{z_\e}
\newcommand{\bze}{\bar z_\e}
\newcommand{\gr}{\nabla}
\newcommand{\Wz}{W^{1,p}_0({\O})}
\newcommand{\rec}[1]{\frac{1}{#1}}
\DeclareMathOperator{\Div}{div} \DeclareMathOperator{\dist}{dist}
\DeclareMathOperator{\supp}{supp}

\renewcommand{\ni}{\noindent}
\renewcommand{\a}{\alpha}
\renewcommand{\b}{\beta}
\renewcommand{\d}{\delta}
\renewcommand{\L}{\mathbf L}
\renewcommand{\k}{\kappa}
\renewcommand{\O}{\Omega}
\renewcommand{\o}{\omega}
\renewcommand{\S}{\Sigma}
\renewcommand{\t}{\theta}
\newcommand{\f}{\varphi}
%%%%%%
%\renewcommand{\vec}[1]{\boldsymbol{#1}}





%
%
%
%
%
\begin{comment}
% PDE Section IMU2002
%\documentclass{slides}
\documentclass{amsart}
\usepackage{amsmath}

\usepackage{amsthm}

\usepackage{amssymb}

\usepackage{amscd}

\usepackage{xspace}
%\usepackage[active]{srcltx}
\usepackage{verbatim}
%\pagestyle{empty}
%\newcommand{\section}[1]{\begin{center}\underline{\bf #1}\end{center}\par}
\newcommand{\N}{\mathbb{N}}
\newcommand{\Z}{\mathbb{Z}}
\newcommand{\Q}{\mathbb{Q}}
\renewcommand{\O}{\Omega}
\newcommand{\C}{{\mathbb{C}}}
\newcommand{\R}{{\mathbb{R}}}
\newcommand{\e}{\varepsilon}
\renewcommand{\d}{\delta}
\newcommand{\G}{\Gamma}
\newcommand{\ue}{u_\e}
\newcommand{\roe}{\rho_\e}
\newcommand{\ov}{\overline}
\newcommand{\gr}{\nabla}
\newcommand{\er}{\eqref}
\newcommand{\f}{\phi}
\DeclareMathOperator{\Div}{div}
\DeclareMathOperator{\Curl}{curl}
%\renewcommand{\vec}[1]{\mathbf{#1}}
\DeclareMathOperator{\sgn}{sgn}
\DeclareMathOperator{\degr}{deg}
\newtheorem{theorem}{Theorem}
\newtheorem{corollary}{Corollary}
\newtheorem{proposition}{Proposition}
\newtheorem{lemma}{Lemma}
\newtheorem{definition}{Definition}
\newtheorem{remark}{Remark}
\newcommand{\rth}[1]{Theorem~\ref{#1}}
\end{comment}
%
%
%
%
%
\date{}

\begin{document}
\title{
%On a n
Non-relativistic model of the laws of gravity and electromagnetism,
%in the non-relativistic space-time settings,
invariant under the change of inertial and non-inertial coordinate
systems.}
%$ $\\[3mm]
\maketitle
%{Arkady Poliakovsky \footnote{E-mail: arkady.pol@gmail.com}}
\begin{center}
\textsc{Arkady Poliakovsky \footnote{E-mails:
poliakov@math.bgu.ac.il, arkady.pol@gmail.com}}
\\[3mm]
Department of Mathematics, Ben Gurion University of the Negev,\\
P.O.B. 653, Be'er Sheva 84105, Israel
\\[2mm]
%\today
%\date{}
\end{center}
\begin{abstract}
Under the classical non-relativistic consideration of the space-time
we propose the model of the laws of gravity and Electrodynamics,
invariant under the galilean transformations and moreover, under
every change of non-inertial cartesian coordinate system. Being in
the frames of non-relativistic model of the space-time, we adopt
some general ideas of the General Theory of Relativity, like the
assumption of invariance of the most general physical laws in every
inertial and non-inertial coordinate system and equivalence of
factious forces in non-inertial coordinate systems and the force of
gravity. Moreover, in the frames of our model, we obtain that the
laws of Non-relativistic Quantum Mechanics are also invariant under
the change of inertial or non-inertial cartesian coordinate system.
%
%
%
\begin{comment}
\begin{equation*}
\begin{cases}
curl_{\vec x} \vec H\equiv \frac{4\pi}{c}\vec j+\frac{1}{c}\frac{\partial \vec D}{\partial t}\\
div_{\vec x} \vec D\equiv 4\pi\rho\\
curl_{\vec x} \vec E+\frac{1}{c}\frac{\partial \vec B}{\partial t}\equiv 0\\
div_{\vec x} \vec B\equiv 0,
\end{cases}
\end{equation*}
together with the following constitutive relations:
\begin{equation*}
\begin{cases}
\vec E=\vec D-\frac{1}{c}\,\vec v\times
\vec B\\
\vec H=\vec B+\frac{1}{c}\,\vec v\times \vec D,
\end{cases}
\end{equation*}
where $\vec v$ is the field of the aether velocity. The presented
model of Maxwell equations and the Lorentz force are invariant under
galilean transformations:
\begin{equation*}
\begin{cases}
\vec x'=\vec x+t\vec w\\
t'=t,
\end{cases}
\end{equation*}
together with the relations:
\begin{equation*}
\begin{cases}
\vec D'=\vec D\\
\vec B'=\vec B\\
\vec E'=\vec E-\frac{1}{c}\,\vec w\times \vec B\\
\vec H'=\vec H+\frac{1}{c}\,\vec w\times \vec D.
\end{cases}
\end{equation*}
Moreover, the form of these equations is preserved in every
non-inertial cartesian coordinate system. Further consequences of
the model are also derived.
\end{comment}
%
%
%
\end{abstract}











\section{Introduction}
\subsection{A new look to the Newtonian Gravity}
Consider the classical space-time where the change of some inertial
coordinate system $(*)$ to another inertial coordinate system $(**)$
is given by the Galilean Transformation:
\begin{equation}\label{noninchgravortbstrjgghguittu1int}
\begin{cases}
\vec x'=\vec x+\vec wt,\\
t'=t,
\end{cases}
\end{equation}
and the change of some non-inertial cartesian coordinate system
$(*)$ to another cartesian coordinate system $(**)$ is of the form:
\begin{equation}\label{noninchgravortbstrjgghguittu2int}
\begin{cases}
\vec x'=A(t)\cdot\vec x+\vec z(t),\\
t'=t,
\end{cases}
\end{equation}
where $A(t)\in SO(3)$ is a rotation, i.e. $A(t)\in \R^{3\times 3}$,
$det\, A(t)>0$ and $A(t)\cdot A^T(t)=I$, where $A^T$ is the
transpose of the matrix $A$ and $I$ is the identity matrix.

 Similarly to the General Theory of Relativity, we assume that
the most general laws of Classical Mechanics should be invariant in
every non-inertial cartesian coordinate system, i.e. they preserve
their form under transformations of the form
\er{noninchgravortbstrjgghguittu2int}. Moreover, again as in the
General Theory of Relativity, we assume that the fictitious forces
in non-inertial coordinate systems and the forces of Newtonian
gravitation have the same nature and represented by some field in
somewhat similar to the Electromagnetic field.

 We begin with some simple observation. Assume that we are away of
essential gravitational masses. Then consider two cartesian
coordinate systems $(*)$ and $(**)$, such that the system $(**)$ is
inertial and the change of coordinate system $(*)$ to coordinate
system $(**)$ is given by \er{noninchgravortbstrjgghguittu2int}.
Then the fictitious-gravitational force in the system $(**)$ is
trivial $\vec F'_0=0$. On the other hand, by
\er{noninchgravortbstrjgghguittu2int} the fictitious-gravitational
force in the system $(*)$ acting on the particle with inertial mass
$m$ is given by
\begin{equation}\label{noninchgravortbstrjgghguittu2gjgghhjhghjhjggint}
\vec F_0=m\left(-2A^T(t)\cdot\frac{dA}{dt}(t)\cdot\vec
u-A^T(t)\cdot\frac{d^2 A}{dt^2}(t)\cdot\vec
x-A^T(t)\cdot\frac{d^2\vec z}{dt^2}(t)\right).
\end{equation}
Thus if we define a vector field $\vec v:=\vec v(\vec x,t)$ by
\begin{equation}\label{noninchgravortbstrjgghguittu2gjgjhjhhklkint}
\vec v(\vec x,t):=-A^T(t)\cdot\frac{dA}{dt}(t)\cdot\vec
x-A^T(t)\cdot\frac{d\vec z}{dt}(t),
\end{equation}
then, by straightforward calculations we rewrite
\er{noninchgravortbstrjgghguittu2gjgghhjhghjhjggint} as
\begin{equation}\label{noninchgravortbstrjgghguittu2gjgghhjhghjhjgghgghghghtytytint}
\vec F_0=m\left(\frac{\partial\vec v}{\partial
t}+\frac{1}{2}\nabla_{\vec x}\left(|\vec v|^2\right)\right)+m\vec
u\times \left(-curl_{\vec x}\vec v\right)
\end{equation}
(see section \ref{gugyu} for details).




 Similarly, we assume that also in the general case of
gravitational masses there exists a vector field $\vec v:=\vec
v(\vec x,t)$ such that in some inertial or non-inertial cartesian
coordinate system the fictitious-gravitational force is given by
\er{noninchgravortbstrjgghguittu2gjgghhjhghjhjgghgghghghtytytint}.
Then we call the vector field $\vec v$ the vectorial gravitational
potential. We see here the following analogy with Electrodynamics:
denoting
\begin{equation*}
\tilde {\vec E}:=\partial_{t}\vec v+\nabla_{\vec
x}\left(\frac{1}{2}|\vec v|^2\right)\quad\text{and}\quad \tilde
{\vec B}:=-c\, curl_{\vec x}\vec v,
\end{equation*}
we rewrite
\er{noninchgravortbstrjgghguittu2gjgghhjhghjhjgghgghghghtytytint} as
\begin{equation*}
%\label{MaxVacFull1ninshtrgravortjhhjfhfhNewhjhint}
%m\frac{d\vec u}{dt}
\vec F_0=m\left(\tilde {\vec E}+\frac{1}{c}\vec u\times\tilde {\vec
B}\right),
%+\vec F,
%\frac{d\vec u}{dt}=-curl_{\vec x}\vec v\times(\vec u-\vec v)+\partial_t\vec v+(\nabla_{\vec x}\vec v)\cdot\vec v+\nabla_{\vec x}\psi_0+\frac{1}{m}\vec F.
\end{equation*}
where
\begin{equation*}
curl_{\vec x}\tilde {\vec E}+\frac{1}{c}\frac{\partial}{\partial
t}\tilde {\vec B}=0\quad\text{and}\quad div_{\vec x}\tilde {\vec
B}=0.
\end{equation*}






Next using
\er{noninchgravortbstrjgghguittu2gjgghhjhghjhjgghgghghghtytytint} we
rewrite the Second Law of Newton as
\begin{equation}\label{noninchgravortbstrjgghguittu2gjgghhjhghjhjgghgghghghtytythvfghfgghjggint}
m\frac{d^2\vec x}{dt^2}=m\frac{d\vec u}{dt}=\vec F_0+\vec
F=m\left(\frac{\partial\vec v}{\partial t}(\vec
x,t)+\frac{1}{2}\nabla_{\vec x}\left(\left|\vec
v\right|^2\right)(\vec x,t)\right)+m\vec u\times \left(-curl_{\vec
x}\vec v(\vec x,t)\right)+\vec F,
\end{equation}
where $\vec x:=\vec x(t)$, $\vec u:=\vec u(t)=\frac{d\vec x}{dt}(t)$
and $m$ are the place, the velocity and the inertial mass of some
given particle at the moment of time $t$, $\vec v:=\vec v(\vec x,t)$
is the vectorial gravitational potential and $\vec F$ is the total
non-gravitational force, acting on the given particle.
%APPNEW

















Once we considered the Second Law of Newton in the form
\er{noninchgravortbstrjgghguittu2gjgghhjhghjhjgghgghghghtytythvfghfgghjggint}
we show that this law is invariant under the change of inertial or
non-inertial cartesian coordinate system, provided that the law of
transformation of the vectorial gravitational potential, under the
change of coordinate system given by
\er{noninchgravortbstrjgghguittu2int}, is:
\begin{equation}\label{MaxVacFull1ninshtrgravortPPNintspd}\vec v'=A(t)\cdot \vec
v+\frac{dA}{dt}(t)\cdot\vec x+\frac{d\vec z}{dt}(t)
\end{equation}
i.e. it is the same as the transformation of a field of velocities.
More precisely we have the following theorem (see section
\ref{gugyu} for the proof):
\begin{theorem}\label{gjghghgghgint}
Consider that the change of some non-inertial cartesian coordinate
system $(*)$ to another cartesian coordinate system $(**)$ is given
by \er{noninchgravortbstrjgghguittu2int}. Next, assume that in the
coordinate system $(**)$ we observe a validity of the Second Law of
Newton in the form:
\begin{equation}\label{MaxVacFull1ninshtrgravortPPNint}
\frac{d\vec u'}{dt'}=-\vec u'\times curl_{\vec x'}\vec
v'+\partial_{t'}\vec v'+\nabla_{\vec x'}\left(\frac{1}{2}|\vec
v'|^2\right)+\frac{1}{m'}\vec F',
%\frac{d\vec u}{dt}=-curl_{\vec x}\vec v\times(\vec u-\vec v)+\partial_t\vec v+(\nabla_{\vec x}\vec v)\cdot\vec v+\nabla_{\vec x}\psi_0+\frac{1}{m}\vec F.
\end{equation}
where $\vec x':=\vec x'(t')$, $\vec u':=\vec u'(t')=\frac{d\vec
x'}{dt'}(t')$ and $m'$ are the place, the velocity and the inertial
mass of some given particle at the moment of time $t'$, $\vec
v':=\vec v'(\vec x',t')$ is the vectorial gravitational potential
and $\vec F'$ is a total non-gravitational force, acting on the
given particle in the coordinate system $(**)$. Then in the
coordinate system $(*)$ we have validity of the Second Law of Newton
in the same as \er{MaxVacFull1ninshtrgravortPPNint} form:
\begin{equation}\label{MaxVacFull1ninshtrgravortjhhjPPNjffjfint}
\frac{d\vec u}{dt}=-\vec u\times curl_{\vec x}\vec
v+\partial_{t}\vec v+\nabla_{\vec x}\left(\frac{1}{2}|\vec
v|^2\right)+\frac{1}{m}\vec F,
%\frac{d\vec u}{dt}=-curl_{\vec x}\vec v\times(\vec u-\vec v)+\partial_t\vec v+(\nabla_{\vec x}\vec v)\cdot\vec v+\nabla_{\vec x}\psi_0+\frac{1}{m}\vec F.
\end{equation}
provided that
\begin{align}
\label{NoIn5gravortPPN11int}\vec v'=A(t)\cdot \vec
v+\frac{dA}{dt}(t)\cdot\vec x+\frac{d\vec z}{dt}(t)\\
\label{NoIn1gravortPPN11int}\vec F'=A(t)\cdot\vec F,\\
\label{NoIn2gravortPPN11int}m'=m,\\
\label{NoIn3gravortPPN11int}\vec u'=A(t)\cdot \vec
u+\frac{dA}{dt}(t)\cdot\vec x+\frac{d\vec z}{dt}(t).
\end{align}
\end{theorem}
We call a vector field that transforms by
\er{MaxVacFull1ninshtrgravortPPNintspd} under the change of
cartesian coordinate system, by the name speed-like vector field.
Since the vectorial gravitational potential $\vec v$ is a speed-like
vector field, i.e. under the changes of inertial or non-inertial
coordinate system it behaves like a field of velocities of some
continuum, we could introduce a fictitious continuum medium covering
all the space, that we can call Aether, such that $\vec v(\vec x,t)$
is a fictitious velocity of this medium in the point $\vec x$ at the
time $t$. Furthermore, if some particle with the place $\vec r:=\vec
r(t)$, the velocity $\vec u:=\vec u(t)=\frac{d\vec r}{dt}(t)$ and
the inertial mass $m$ moves in the outer gravitational field with
the vectorial gravitational potential $\vec v:=\vec v(\vec x,t)$ in
the absence of non-gravitational forces, then we can associate a
Lagrangian with
\er{noninchgravortbstrjgghguittu2gjgghhjhghjhjgghgghghghtytythvfghfgghjggint}.
Indeed, for this case we define a Lagrangian:
\begin{equation}\label{vhfffngghkjgghfjjSYSPNint}
\mathcal{L}_0\left(\frac{d\vec r}{dt},\vec
r,t\right):=\frac{m}{2}\left|\frac{d\vec r}{dt}-\vec v(\vec
r,t)\right|^2.
\end{equation}
This Lagrangian is invariant under the change of non-inertial
cartesian coordinate systems, given by
\er{noninchgravortbstrjgghguittu2int}. Moreover, we can easily
deduce that a trajectory $\vec r(t):[0,T]\to\mathbb{R}^3$ is a
critical point of the functional
\begin{equation}\label{btfffygtgyggyijhhkkSYSPNint}
I_0=\int_0^T \mathcal{L}_0\left(\frac{d\vec r}{dt}(t),\vec
r(t),t\right)dt
\end{equation}
if and only if it satisfies
\begin{equation}\label{vhfffngghkjgghggtghjgfhjoyuiyuyhiyyukukyihyuSYSPNint}
-m\frac{d^2\vec r}{dt^2}+m\left(\frac{\partial}{\partial t}\vec
v(\vec r,t)+\nabla_{\vec x}\left(\frac{1}{2}\left|\vec v(\vec
r,t)\right|^2\right)-\frac{d\vec r}{dt}\times curl_{\vec x}\vec
v\left(\vec r,t\right)\right)=0,
\end{equation}
consistently with
\er{noninchgravortbstrjgghguittu2gjgghhjhghjhjgghgghghghtytythvfghfgghjggint}
for the case $\vec F=0$. Moreover, we would like to note that if in
some inertial or non-inertial cartesian coordinate system some
material body with the place $\vec r(t)$ and velocity $\vec
u(t)=\frac{d\vec r}{dt}(t)$ moves in the gravitational field, and,
except of the force of gravity all other forces, acting on the body,
are negligible then we can prove that the following equality for
some instant of time $t_0$:
\begin{equation*}
\vec u(t_0):=\frac{d\vec r}{dt}(t_0)=\vec v\left(\vec
r(t_0),t_0\right)
\end{equation*}
implies
\begin{equation*}
\vec u(t):=\frac{d\vec r}{dt}(t)=\vec v\left(\vec r(t),t\right),
\end{equation*}
for every instant of time. I.e. if the velocity of the particle for
some initial instant of time coincides with the local vectorial
gravitational potential, then it will coincide with it at any
instant of time and the trajectory of the motion will be tangent to
the direction of the local vectorial gravitational potential.

































Next, in order to fit the Second Law of Newton in the form
\er{noninchgravortbstrjgghguittu2gjgghhjhghjhjgghgghghghtytythvfghfgghjggint}
with the classical Second Law of Newton and the Newtonian Law of
Gravity we consider that in \underline{inertial} coordinate system
$(*)$, at least in the first approximation, we should have
\begin{equation}
\label{MaxVacFull1ninshtrgravortghhghgjkgghklhjgkghghjjkjhjkkggjkhjkhjjhhfhjhklkhkhjjklzzzyyyhjggjhgghhjhNWNWBWHWPPN222kgghjghjghjint}
\begin{cases}
curl_{\vec x}\vec v= 0,\\
\frac{\partial\vec v}{\partial t}+\frac{1}{2}\nabla_{\vec
x}\left(|\vec v|^2\right)= -\nabla_{\vec x}\Phi,
%\frac{d\vec u}{dt}=-curl_{\vec x}\vec v\times(\vec u-\vec v)+\partial_t\vec v+(\nabla_{\vec x}\vec v)\cdot\vec v+\nabla_{\vec x}\psi_0+\frac{1}{m}\vec F.
\end{cases}
\end{equation}
where $\Phi$ is a scalar Newtonian gravitational potential which
satisfies
\begin{equation}
\label{MaxVacFull1ninshtrgravortghhghgjkgghklhjgkghghjjkjhjkkggjkhjkhjjhhfhjhklkhkhjjklzzzyyyhjggjhgghhjhNWNWNWBWHWPPN222int}
\Delta_{\vec x}\Phi=4\pi GM,
\end{equation}
where $M$ is the gravitational mass density and $G$ is the
gravitational constant. Thus, since we require $curl_{\vec x}\vec v=
0$,
\er{MaxVacFull1ninshtrgravortghhghgjkgghklhjgkghghjjkjhjkkggjkhjkhjjhhfhjhklkhkhjjklzzzyyyhjggjhgghhjhNWNWBWHWPPN222kgghjghjghjint}
is equivalent to:
\begin{equation}
\label{MaxVacFull1ninshtrgravortghhghgjkgghklhjgkghghjjkjhjkkggjkhjkhjjhhfhjhklkhkhjjklzzzyyyhjggjhgghhjhNWNWBWHWPPN222int}
\begin{cases}
curl_{\vec x}\vec v= 0,\\
\frac{\partial\vec v}{\partial t}+d_\vec x\vec v\cdot\vec v=
-\nabla_{\vec x}\Phi,
%\frac{d\vec u}{dt}=-curl_{\vec x}\vec v\times(\vec u-\vec v)+\partial_t\vec v+(\nabla_{\vec x}\vec v)\cdot\vec v+\nabla_{\vec x}\psi_0+\frac{1}{m}\vec F.
\end{cases}
\end{equation}
where $d_{\vec x}\vec v$ is the Jacobian matrix of the vector field
$\vec v$. Clearly the law
\er{MaxVacFull1ninshtrgravortghhghgjkgghklhjgkghghjjkjhjkkggjkhjkhjjhhfhjhklkhkhjjklzzzyyyhjggjhgghhjhNWNWBWHWPPN222int}
is invariant under the change of inertial coordinate system, given
by \er{noninchgravortbstrjgghguittu1int}. Note also that, since in
the system $(*)$ we have $curl_{\vec x}\vec v=0$, we can write
\er{MaxVacFull1ninshtrgravortghhghgjkgghklhjgkghghjjkjhjkkggjkhjkhjjhhfhjhklkhkhjjklzzzyyyhjggjhgghhjhNWNWBWHWPPN222kgghjghjghjint}
as a Hamilton-Jacobi type equation:
\begin{equation}
\label{MaxVacFull1ninshtrgravortghhghgjkgghklhjgkghghjjkjhjkkggjkhjkhjjhhfhjhklkhkhjjklzzzyyyhjggjhgghhjhNWNWNWNWNWBWHWPPN222int}
\begin{cases}
\vec v=\nabla_{\vec x}Z,\\
\frac{\partial Z}{\partial t}+\frac{1}{2}\left|\nabla_{\vec
x}Z\right|^2=-\Phi,
%\frac{d\vec u}{dt}=-curl_{\vec x}\vec v\times(\vec u-\vec v)+\partial_t\vec v+(\nabla_{\vec x}\vec v)\cdot\vec v+\nabla_{\vec x}\psi_0+\frac{1}{m}\vec F.
\end{cases}
\end{equation}
where $Z$ is some scalar field. Next we introduce a law of gravity
which is invariant in every non-inertial cartesian coordinate system
and is equivalent to
\er{MaxVacFull1ninshtrgravortghhghgjkgghklhjgkghghjjkjhjkkggjkhjkhjjhhfhjhklkhkhjjklzzzyyyhjggjhgghhjhNWNWBWHWPPN222int}
in every inertial coordinate system. This law has the form:
\begin{equation}
\label{MaxVacFull1ninshtrgravortghhghgjkgghklhjgkghghjjkjhjkkggjkhjkhjjhhfhjhklkhkhjjklzzzyyyNWBWHWPPN222int}
\begin{cases}
curl_{\vec x}\left(curl_{\vec x}\vec v\right)= 0,\\
\frac{\partial}{\partial t}\left(div_{\vec x}\vec v\right)+div_{\vec
x}\left\{\left(div_{\vec x}\vec v\right)\vec
v\right\}+\frac{1}{4}\left|d_{\vec x}\vec v+\{d_{\vec x}\vec
v\}^T\right|^2-\left(div_{\vec x}\vec v\right)^2= -4\pi GM,
%\frac{d\vec u}{dt}=-curl_{\vec x}\vec v\times(\vec u-\vec v)+\partial_t\vec v+(\nabla_{\vec x}\vec v)\cdot\vec v+\nabla_{\vec x}\psi_0+\frac{1}{m}\vec F.
\end{cases}
\end{equation}
%
%
%
\begin{comment}
\begin{equation}
\label{MaxVacFull1ninshtrgravortghhghgjkgghklhjgkghghjjkjhjkkggjkhjkhjjhhfhjhklkhkhjjklzzzyyyNWBWHWPPN222jkgghggyuhgulkip}
\begin{cases}
curl_{\vec x}\left(curl_{\vec x}\vec v\right)= 0,\\
\frac{\partial}{\partial t}\left(div_{\vec x}\vec v\right)+\vec
v\cdot\nabla_{\vec x}\left(div_{\vec x}\vec
v\right)+\frac{1}{4}\left|d_{\vec x}\vec v+\{d_{\vec x}\vec
v\}^T\right|^2= -4\pi GM.
%\frac{d\vec u}{dt}=-curl_{\vec x}\vec v\times(\vec u-\vec v)+\partial_t\vec v+(\nabla_{\vec x}\vec v)\cdot\vec v+\nabla_{\vec x}\psi_0+\frac{1}{m}\vec F.
\end{cases}
\end{equation}
\end{comment}
%
%
%
(see section \ref{gugyu} for the details).




Next one can wonder: what should be possible values of the vectorial
gravitational potential $\vec v$ in the proximity of the Earth or
another massive body? We attempt to answer this question in remark
\ref{gygygygyggjh}. We obtain there that, if we consider a
\underline{non-rotating} cartesian coordinate system which center
coincides with the center of the Earth, then in this system we
should have either
\begin{equation}
\label{MaxVacFull1ninshtrgravortghhghgjkgghklhjgkghghjjkjhjkkggjkhjkhjjhhfhjhklkhkhjjklzzzyyyhjggjhgghhjhNWNWBWHWPPN222kkkgtytghjjhpoppkkkhhhkhjhjint}
\vec v(\vec x)= \frac{\sqrt{-2\Phi_1(|\vec x|)}}{|\vec x|}\vec x,
\end{equation}
or
\begin{equation}
\label{MaxVacFull1ninshtrgravortghhghgjkgghklhjgkghghjjkjhjkkggjkhjkhjjhhfhjhklkhkhjjklzzzyyyhjggjhgghhjhNWNWBWHWPPN222kkkgtytghjjhpoppkkkhhhkhjhjiuuint}
\vec v(\vec x)= -\frac{\sqrt{-2\Phi_1(|\vec x|)}}{|\vec x|}\vec x,
\end{equation}
where $\Phi_1$ is the usual Newtonian potential of the Earth, that
satisfies $\Phi_1(r)=\frac{Gm_0}{r}$ outside of the Earth. In
particular, on the Earth surface we have:
\begin{equation}
\label{MaxVacFull1ninshtrgravortghhghgjkgghklhjgkghghjjkjhjkkggjkhjkhjjhhfhjhklkhkhjjklzzzyyyhjggjhgghhjhNWNWBWHWPPN222kkkgtytghjjhpoppkkkhhhkhjhjiuuokokokint}
|\vec v|=\sqrt{\frac{2Gm_0}{r_0}},
\end{equation}
where $r_0$ is the Earth radius and $m_0$ is the Earth mass, i.e.
the absolute value of the vectorial gravitational potential on the
Earth surface approximately equals to the escape velocity and its
direction is normal to the Earth, either downward or upward.































\subsection{Non-relativistic model of Electrodynamics}
Similarly to the General Theory of Relativity we assume that the
electromagnetic field is influenced by the gravitational field. In
Section \ref{MaxRevsPPN} of this paper we propose the simple and
natural quantitative relations of Electrodynamics, substituting
(with minor changes) the classical Maxwell equations in the case of
an arbitrarily vectorial gravitational potential, and invariant
under Galilean Transformations. For this propose we appeal to the
Maxwell equations in a medium. It is well known that
%, up to rescaling,
the classical Maxwell equations in a medium have the following form
in the Gaussian unit system:
\begin{equation}
\label{MaxVackkkint}
\begin{cases}
curl_{\vec x} \vec H= \frac{4\pi}{c}\vec j+\frac{1}{c}\frac{\partial \vec D}{\partial t},\\
div_{\vec x} \vec D= 4\pi\rho,\\
curl_{\vec x} \vec E+\frac{1}{c}\frac{\partial \vec B}{\partial t}= 0,\\
div_{\vec x} \vec B= 0.
\end{cases}
\end{equation}
Here $\vec x\in\R^3$ and $t>0$ are the place and the time, $\vec E$
is the electric field, $\vec B$ is the magnetic field, $\vec D$ is
the electric displacement field, $\vec H$ is the $\vec H$-magnetic
field, $\rho$ is the charge density, $\vec j$ is the current density
and $c$ is the universal constant,
%usually
called
%"
speed of light.
%".
It is assumed in the Classical Electrodynamics that for the vacuum
we always have $\vec D= \vec E$ and $\vec H=\vec B$. We assume here
that the Maxwell equations in the vacuum have the usual form of
\er{MaxVackkkint} in every inertial coordinate system, as in any
other medium, however, we assume that, given some inertial
coordinate system, the relations $\vec D= \vec E$ and $\vec H=\vec
B$ in the vacuum are valid only for the parts of the space, where
the vectorial gravitational potential is negligible.

So we assume that, given some inertial coordinate system, if in some
point and at some instant the vectorial gravitational potential
vanishes, then in this point and at this time we have $\vec D= \vec
E$ and $\vec H=\vec B$. In order to obtain the relations $\vec D\sim
\vec E$ and $\vec H\sim \vec B$ in the general case we assume that
the equations \er{MaxVackkkint} and the Lorentz force
\begin{equation}
\label{LorenzChlljklljklkkzzzint} \vec F=\sigma \vec
E+\frac{\sigma}{c}\,\vec u\times \vec B
\end{equation}
%$\vec F:=\sigma \vec E+\frac{\sigma}{c}\,\vec u\times \vec B$
(where $\sigma$ is the charge of the test particle and $\vec u$ is
its velocity) are invariant under the Galilean transformations,
given by \er{noninchgravortbstrjgghguittu1int}. Then the analysis of
our assumptions, presented in section \ref{MaxRevsPPN}, implies that
the full system of Electrodynamics in the case of an arbitrarily
vectorial gravitational potential $\vec v:=\vec v(\vec x,t)$ has the
following form:
\begin{equation}\label{MaxVacFull1bjkgjhjhgjaaaint}
\begin{cases}
curl_{\vec x} \vec H= \frac{4\pi}{c}\vec j+\frac{1}{c}\frac{\partial
\vec D}{\partial t},\\
%\quad\text{for}\;\;(\vec x,t)\in\R^3\times[0,+\infty),\\
div_{\vec x} \vec D= 4\pi\rho,\\
%\quad\quad\text{for}\;\;(\vec x,t)\in\R^3\times[0,+\infty),\\
curl_{\vec x} \vec E+\frac{1}{c}\frac{\partial \vec B}{\partial t}=0,\\
%\quad\quad\text{for}\;\;(\vec x,t)\in\R^3\times[0,+\infty),\\
div_{\vec x} \vec B=0,\\
%\quad\quad\text{for}\;\;(\vec x,t)\in\R^3\times[0,+\infty),\\
\vec E=\vec D-\frac{1}{c}\,\vec v\times \vec B,\\
%\quad\quad\text{for}\;\;(\vec x,t)\in\R^3\times[0,+\infty)\\
\vec H=\vec B+\frac{1}{c}\,\vec v\times \vec D.
%\quad\quad\text{for}\;\;(\vec x,t)\in\R^3\times[0,+\infty).
\end{cases}
\end{equation}
It can be easily checked that system
\er{MaxVacFull1bjkgjhjhgjaaaint} and the expression of the Lorentz
force in \er{LorenzChlljklljklkkzzzint}
%$\vec F:=\sigma(\vec E+\frac{\vec u}{c}\times \vec B)$
are invariant under the Galilean transformations
\er{noninchgravortbstrjgghguittu1int}, provided that
\begin{equation}\label{EBDHTrans2kkkint}
\begin{cases}
\vec D'=\vec D,\\
\vec B'=\vec B,\\
\vec E'=\vec E-\frac{1}{c}\,\vec w\times \vec B,\\
\vec H'=\vec H+\frac{1}{c}\,\vec w\times \vec D\\
\vec v'=\vec v+\vec w.
\end{cases}
\end{equation}
In section \ref{INNONredPPN} we prove that the laws of
Electrodynamics in the form \er{MaxVacFull1bjkgjhjhgjaaaint} and the
law of the Lorentz force \er{LorenzChlljklljklkkzzzint}, preserve
their form also in non-inertial cartesian coordinate systems. More
precisely the following theorem is valid:
\begin{theorem}\label{gjghghgghgintint}
Consider that the change of some non-inertial cartesian coordinate
system $(*)$ to another cartesian coordinate system $(**)$ is given
by \er{noninchgravortbstrjgghguittu2int}. Next, assume that in the
coordinate system $(**)$ we observe a validity of Maxwell Equations
for the vacuum in the form:
\begin{equation}\label{MaxVacFull1ninshtrredPPNint}
\begin{cases}
curl_{\vec x'} \vec H'= \frac{4\pi}{c}\vec
j'+\frac{1}{c}\frac{\partial
\vec D'}{\partial t'},\\
%\quad\text{for}\;\;(\vec x,t)\in\R^3\times[0,+\infty),\\
div_{\vec x'} \vec D'= 4\pi\rho',\\
%\quad\quad\text{for}\;\;(\vec x,t)\in\R^3\times[0,+\infty),\\
curl_{\vec x'} \vec E'+\frac{1}{c}\frac{\partial \vec B'}{\partial t'}= 0,\\
%\quad\quad\text{for}\;\;(\vec x,t)\in\R^3\times[0,+\infty),\\
div_{\vec x'} \vec B'= 0,\\
%\quad\quad\text{for}\;\;(\vec x,t)\in\R^3\times[0,+\infty),\\
\vec E'=\vec D'-\frac{1}{c}\,\vec v'\times \vec B',\\
%\quad\quad\text{for}\;\;(\vec x,t)\in\R^3\times[0,+\infty)\\
\vec H'=\vec B'+\frac{1}{c}\,\vec v'\times \vec D'.
%\quad\quad\text{for}\;\;(\vec x,t)\in\R^3\times[0,+\infty).
\end{cases}
\end{equation}
Moreover, we assume that in coordinate system $(**)$ we observe a
validity of the expression for the Lorentz force in the form:
\begin{equation}\label{LorenzChredPPNint}
\vec F'=\sigma' \vec E'+\frac{\sigma'}{c}\,\vec u'\times \vec B'.
\end{equation}
Then in the coordinate system $(*)$ we have the validity of Maxwell
Equations for the vacuum in the same as
\er{MaxVacFull1ninshtrredPPNint} form:
\begin{equation}\label{MaxVacFull1ninshtrhjkkredPPNint}
\begin{cases}
curl_{\vec x} \vec H= \frac{4\pi}{c}\vec j+\frac{1}{c}\frac{\partial
\vec D}{\partial t},\\
%\quad\text{for}\;\;(\vec x,t)\in\R^3\times[0,+\infty),\\
div_{\vec x} \vec D= 4\pi\rho,\\
%\quad\quad\text{for}\;\;(\vec x,t)\in\R^3\times[0,+\infty),\\
curl_{\vec x} \vec E+\frac{1}{c}\frac{\partial \vec B}{\partial t}= 0,\\
%\quad\quad\text{for}\;\;(\vec x,t)\in\R^3\times[0,+\infty),\\
div_{\vec x} \vec B= 0,\\
%\quad\quad\text{for}\;\;(\vec x,t)\in\R^3\times[0,+\infty),\\
\vec E=\vec D-\frac{1}{c}\,\vec v\times \vec B,\\
%\quad\quad\text{for}\;\;(\vec x,t)\in\R^3\times[0,+\infty)\\
\vec H=\vec B+\frac{1}{c}\,\vec v\times \vec D,
%\quad\quad\text{for}\;\;(\vec x,t)\in\R^3\times[0,+\infty).
\end{cases}
\end{equation}
and we have the validity of the expression for the Lorentz force in
the same as \er{LorenzChredPPNint} form:
\begin{equation}\label{LorenzChlljklljkint}
\vec F=\sigma \vec E+\frac{\sigma}{c}\,\vec u\times \vec B,
\end{equation}
provided that
\begin{equation}\label{yuythfgfyftydtydtydtyddyyyhhddhhhredPPN111hgghjgintint}
\begin{cases}
\vec F'=A(t)\cdot\vec F,\\
\sigma'=\sigma,\\
\vec u'=A(t)\cdot \vec u+\frac{dA}{dt}(t)\cdot\vec x+\frac{d\vec z}{dt}(t),\\
\rho'=\rho,\\
\vec v'=A(t)\cdot \vec v+\frac{dA}{dt}(t)\cdot\vec x+\frac{d\vec z}{dt}(t),\\
\vec j'=A(t)\cdot \vec j+\rho\, \frac{dA}{dt}(t)\cdot\vec
x+\rho\,\frac{d\vec z}{dt}(t)
%\\ \vec v'=A(t)\cdot \vec v+\frac{dA}{dt}(t)\cdot\vec x+\frac{d\vec z}{dt}(t)\\ \vec F'=A(t)\cdot\vec F,\\ \sigma'=\sigma.
\end{cases}
\end{equation}
and
\begin{equation}\label{yuythfgfyftydtydtydtyddyyyhhddhhhredPPN111hgghjgint}
\begin{cases}
\vec D'=A(t)\cdot \vec D,\\
\vec B'=A(t)\cdot\vec B,\\
\vec E'=A(t)\cdot\vec E-\frac{1}{c}\,\left(\frac{dA}{dt}(t)\cdot\vec
x+\frac{d\vec z}{dt}(t)\right)\times \left(A(t)\cdot\vec B\right),\\
\vec H'=A(t)\cdot\vec H+\frac{1}{c}\,\left(\frac{dA}{dt}(t)\cdot\vec
x+\frac{d\vec z}{dt}(t)\right)\times \left(A(t)\cdot\vec D\right).
\end{cases}
\end{equation}
\end{theorem}











Next we would like to note that, since as already mentioned before,
the direction of the local vectorial gravitational potential is
normal to the Earth surface, in the frames of our model, we provide
a non-relativistic explanation of the classical Michelson-Morley
experiment. Indeed in this experiment the axes of the apparatus are
tangent to the Earth surface and thus the null result cannot be
affected by the vectorial gravitational potential. Since, the value
of the local vectorial gravitational potential equals to the escape
velocity, if we consider the vertical Michelson-Morley experiment,
where one of the axes of the apparatus is normal to the Earth
surface, then in the frames of our model the expected result should
be analogous to the positive result of Aether drift with the speed
equal to the escape velocity. However, regarding the vertical
Michelson-Morley experiment i.e. the modification of
Michelson-Morley experiment, where at least one of the axes of the
apparatus is not tangent to the Earth surface, we found only very
scarce and contradictory information.



































Next, as in the classical electrodynamics, by the third and the
fourth equations in \er{MaxVacFull1bjkgjhjhgjaaaint} we can find a
scalar field $\Psi:=\Psi(\vec x,t)$ and a vector field $\vec A:=\vec
A(\vec x,t)$ such that
\begin{equation}\label{MaxVacFull1bjkgjhjhgjgjgkjfhjfdghghligioiuittrPPNint}
\begin{cases}
\vec B\equiv curl_{\vec x} \vec A,\\
\vec E\equiv-\nabla_{\vec x}\Psi-\frac{1}{c}\frac{\partial\vec
A}{\partial t}.
%,\\ div_{\vec x}\vec A\equiv 0,
\end{cases}
\end{equation}
We call $\Psi$ and $\vec A$ the scalar and the vectorial
electromagnetic potentials. Then by
\er{MaxVacFull1bjkgjhjhgjgjgkjfhjfdghghligioiuittrPPNint} and
\er{MaxVacFull1bjkgjhjhgjaaaint} we also have
\begin{equation}\label{vhfffngghPPN333yuyuint}
\begin{cases}
\vec D=-\nabla_{\vec x}\Psi-\frac{1}{c}\frac{\partial\vec
A}{\partial t}+\frac{1}{c}\vec
v\times curl_{\vec x}\vec A\\
\vec H\equiv curl_{\vec x} \vec A+\frac{1}{c}\,\vec
v\times\left(-\nabla_{\vec x}\Psi-\frac{1}{c}\frac{\partial\vec
A}{\partial t}+\frac{1}{c}\vec v\times curl_{\vec x}\vec A\right).
%\\ div_{\vec x}\vec A\equiv 0.
\end{cases}
\end{equation}
We also define the proper scalar electromagnetic potential
$\Psi_0:=\Psi_0(\vec x,t)$ by
\begin{equation}\label{vhfffngghhjghhgPPNghghghutghffugghjhjkjjklint}
\Psi_0:=\Psi-\frac{1}{c}\vec A\cdot\vec v.
\end{equation}
The name "proper scalar potential" is clarified below. The
electromagnetic potentials are not uniquely defined and thus we need
to choose a calibration. It is clear that if
$(\tilde\Psi,\tilde\Psi_0,\tilde{\vec A})$ is another choice of
electromagnetic potentials with a different calibration then there
exists a scalar field $w:=w(\vec x,t)$ such that we have
\begin{equation}\label{MaxVacFull1bjkgjhjhgjgjgkjfhjfdghghligioiuittrPPNhjkjhkjgghhjjhjint}
\begin{cases}
\tilde\Psi=\Psi+\frac{1}{c}\frac{\partial w}{\partial t}\\
\tilde{\vec A}=\vec A-\nabla_{\vec x}w\\
\tilde\Psi_0=\Psi_0+\frac{1}{c}\left(\frac{\partial w}{\partial
t}+\vec v\cdot\nabla_{\vec x}w\right).
\end{cases}
\end{equation}
For definiteness we can take $\vec A$ to satisfy
\begin{equation}\label{MaxVacFull1bjkgjhjhgjgjgkjfhjfdghghligioiuittrPPN22int}
div_{\vec x}\vec A\equiv 0.
\end{equation}
In section \ref{ghfvdgfdjfg} we show that, consistently with
\er{yuythfgfyftydtydtydtyddyyyhhddhhhredPPN111hgghjgint}, under the
change of non-inertial cartesian coordinate system, given by
\er{noninchgravortbstrjgghguittu2int}, the electromagnetic
potentials transform as:
\begin{equation}\label{vhfffngghhjghhgPPNghghghutghfflklhjkjhjhjjgjkghhjint}
\begin{cases}
\Psi'=
%\Psi+\frac{1}{c}\vec A\cdot\left(A^T(t)\cdot\frac{dA}{dt}(t)\cdot\vec x+A^T(t)\cdot\frac{d\vec z}{dt}(t)\right)
\Psi+\frac{1}{c}\left(\frac{dA}{dt}(t)\cdot\vec x+\frac{d\vec
z}{dt}(t)\right)\cdot\left(A(t)\cdot\vec A\right)
\\
\vec A'=A(t)\cdot \vec A\\
\Psi'_0=\Psi_0.
\end{cases}
\end{equation}
The last equation in
\er{vhfffngghhjghhgPPNghghghutghfflklhjkjhjhjjgjkghhjint} clarifies
the name "proper scalar potential". The equalities
\er{vhfffngghhjghhgPPNghghghutghfflklhjkjhjhjjgjkghhjint} are
derived primarily under the choice of the calibration given by
\er{MaxVacFull1bjkgjhjhgjgjgkjfhjfdghghligioiuittrPPN22int}.
However, as can be easily seen by
\er{MaxVacFull1bjkgjhjhgjgjgkjfhjfdghghligioiuittrPPNhjkjhkjgghhjjhjint},
all the equalities in
\er{vhfffngghhjghhgPPNghghghutghfflklhjkjhjhjjgjkghhjint} still
remain to hold, under any other choice of calibration scalar
function $w$, provided that we have $w'=w$ under the transformation
\er{noninchgravortbstrjgghguittu2int}. In particular, under the
Galilean transformations \er{noninchgravortbstrjgghguittu1int} the
electromagnetic potentials transform as:
\begin{equation}\label{vhfffngghhjghhgPPNghghghutghfflklhjkjhjhjjgjkghhjhhhjhgjguint}
\begin{cases}
\Psi'=
%\Psi+\frac{1}{c}\vec A\cdot\left(A^T(t)\cdot\frac{dA}{dt}(t)\cdot\vec x+A^T(t)\cdot\frac{d\vec z}{dt}(t)\right)
\Psi+\frac{1}{c}\vec w\cdot\vec A
\\
\vec A'=\vec A\\
\Psi'_0=\Psi_0.
%\\ \left(\frac{1}{c}\vec A'\cdot\vec v'-\Psi'\right)=\left(\frac{1}{c}\vec A\cdot\vec v-\Psi\right).
\end{cases}
\end{equation}



















Next we can associate a Lagrangian density related to
electromagnetic field. Given known the charge distribution
$\rho:=\rho(\vec x,t)$, the current distribution $\vec j:=\vec
j(\vec x,t)$ and the vectorial gravitational potential $\vec v:=\vec
v(\vec x,t)$, consider a Lagrangian density $L_1$ defined by
\begin{multline}\label{vhfffngghkjgghPPNint}
L_1\left(\vec A,\Psi,\vec
x,t\right):=\frac{1}{8\pi}\left|-\nabla_{\vec
x}\Psi-\frac{1}{c}\frac{\partial\vec A}{\partial t}+\frac{1}{c}\vec
v\times curl_{\vec x}\vec A\right|^2-\frac{1}{8\pi}\left|curl_{\vec
x}\vec A\right|^2-\left(\rho\Psi-\frac{1}{c}\vec A\cdot\vec
j\right).
\end{multline}
Using \er{vhfffngghhjghhgPPNghghghutghfflklhjkjhjhjjgjkghhjint} we
can deduce that Lagrangian $L_1$ is invariant, under the change of
inertial or non-inertial cartesian coordinate system, given by
\er{noninchgravortbstrjgghguittu2int}. Moreover, if, consistently
with \er{MaxVacFull1bjkgjhjhgjgjgkjfhjfdghghligioiuittrPPNint},
\er{vhfffngghPPN333yuyuint} and
\er{vhfffngghhjghhgPPNghghghutghffugghjhjkjjklint}, we denote
\begin{equation}\label{guigjgjffghPPNint}
\begin{cases}
\vec D=-\nabla_{\vec x}\Psi-\frac{1}{c}\frac{\partial\vec
A}{\partial t}+\frac{1}{c}\vec
v\times curl_{\vec x}\vec A\\
\vec B=curl_{\vec x}\vec A
\\
\vec E=-\nabla_{\vec x}\Psi-\frac{1}{c}\frac{\partial\vec A}{\partial t}
%=\vec D-\frac{1}{c}\vec v\times\vec B
\\
\vec H=curl_{\vec x}\vec A+\frac{1}{c}\vec v\times\left(\nabla_{\vec
x}\Psi-\frac{1}{c}\frac{\partial\vec A}{\partial t}+\frac{1}{c}\vec
v\times curl_{\vec x}\vec A\right)
%=\vec B+\frac{1}{c}\vec v\times\vec D
\\
\Psi_0:=\Psi-\frac{1}{c}\vec A\cdot\vec v,
\end{cases}
\end{equation}
then:
\begin{multline*}
%\label{vhfffngghkjgghPPNggjgjjkgjint}
L_1\left(\vec A,\Psi,\vec x,t\right)=\frac{1}{8\pi}\left|\vec
D\right|^2-\frac{1}{8\pi}\left|\vec
B\right|^2-\left(\rho\Psi-\frac{1}{c}\vec A\cdot\vec
j\right)=\frac{1}{8\pi}\left|\vec
D\right|^2-\frac{1}{8\pi}\left|\vec
B\right|^2-\rho\Psi_0+\frac{1}{c}\vec A\cdot(\vec j-\rho\vec v).
\end{multline*}
Then in section \ref{bhjghjfghfg} we obtain that a configuration
$(\Psi,\vec A)$ is a critical point of the functional
\begin{equation}\label{btfffygtgyggyPPNint}
J_0=\int_0^T\int_{\mathbb{R}^3}L_1\left(\vec A(\vec x,t),\Psi(\vec
x,t),\vec x,t\right)d\vec x dt,
\end{equation}
if and only if we have
\begin{equation}\label{guigjgjffghguygjyfPPNint}
\begin{cases}
curl_{\vec x}\vec H=\frac{4\pi}{c}\vec j+\frac{\partial\vec
D}{\partial
t}\\
div_{\vec x}\vec D=4\pi\rho\\
curl_{\vec x}\vec E+\frac{1}{c}\frac{\partial\vec B}{\partial t}=0\\
div_{\vec x}\vec B=0\\
\vec E=\vec D-\frac{1}{c}\vec v\times\vec B\\
\vec H=\vec B+\frac{1}{c}\vec v\times\vec D,
\end{cases}
\end{equation}
where $(\vec D,\vec B,\vec E,\vec H)$ is given by
\er{guigjgjffghPPNint}. So we get a variational principle related to
Maxwell equations in the form \er{MaxVacFull1bjkgjhjhgjaaaint}.
















\subsection{Local gravitational time and Maxwell equations}
Consider an inertial or more generally a non-rotating cartesian
coordinate system $\bf{(*)}$. Then, as before, in this system we
have
\begin{equation}\label{jhhjgjhuiiuint}
\vec v(\vec x,t)=\nabla_{\vec x}Z(\vec x,t),
\end{equation}
where $\vec v$ is the vectorial gravitational potential and $Z$ is a
scalar field. Then define a scalar field $\tau:=\tau(\vec x,t)$ by
the following:
\begin{equation}\label{jhhjgjhuiiuiyint}
\tau(\vec x,t)=t+\frac{1}{c^2}Z(\vec x,t).
\end{equation}
We call the quantity $\tau(\vec x,t)$ by the name local
gravitational time. The name "local" and "gravitational" is quite
clear, since $\tau$ depend on the space and time variables and
derived by characteristic function $Z$ of the gravitational field.
The name "time" will be clarified bellow. Note also that, using
\er{noninchgravortbstrjgghguittu1intmmjhhj} in remark \ref{ugyugg},
one can easily deduce that under the change of inertial coordinate
system $\bf{(*)}$ to $\bf{(**)}$ given by the Galilean
Transformation \er{noninchgravortbstrjgghguittu1int} the local
gravitational time $\tau$ transforms as:
\begin{equation}\label{noninchgravortbstrjgghguittu1intmmjhhjhjhjint}
\tau'\,=\,\tau+\frac{1}{c^2}\vec w\cdot\vec x+\frac{|\vec
w|^2}{2c^2}t \,\approx\, \tau+\frac{1}{c^2}\vec w\cdot\vec x,
\end{equation}
where the last equality in
\er{noninchgravortbstrjgghguittu1intmmjhhjhjhjint} is valid if
$\frac{|\vec w|^2}{c^2}\ll 1$.

Next consider the Maxwell equations in the vacuum of the form
\er{MaxVacFull1bjkgjhjhgjaaaint} and consider a curvilinear change
of variables given by:
\begin{equation}\label{giuuihjghgghjgj78zzrrZZffhhhggygghghjhvbKKint}
\begin{cases}
t'=\tau(\vec x,t):=t+\frac{Z(\vec x,t)}{c^2}\\
\vec x'=\vec x.
\end{cases}
\end{equation}
Then, denoting
\begin{equation}\label{giuuihjghgghjgj78zzrrZZffhhhggygghghjhvbgghhjyuuyKKint}
\begin{cases}
\vec E^*:=\vec D-\frac{1}{c}\,\vec v\times \vec H=\vec
E-\frac{1}{c^2}\vec v\times\left(\vec v\times\vec D\right)
\\
\vec H^*:=\vec B+\frac{1}{c}\,\vec v\times \vec E=\vec
H-\frac{1}{c^2}\vec v\times\left(\vec v\times\vec B\right),
\end{cases}
\end{equation}
by \er{MaxVacFull1bjkgjhjhgjaaaint} we rewrite the Maxwell equations
in the new curvilinear coordinates in the case of time independent
$\vec v$ as:
\begin{equation}\label{MaxMedFullGGffgguiuiouiogghghhgghhjhjhjjgghhgKKint}
\begin{cases}
curl_{\vec x'}\vec H= \frac{4\pi}{c}\vec j+
\frac{1}{c}\frac{\partial \vec E^*}{\partial
t'},\\
div_{\vec x'}\vec E^*= 4\pi\left(\rho+\frac{1}{c^2}\,\vec v\cdot\vec j\right),\\
curl_{\vec x'}\vec E+\frac{1}{c}\frac{\partial\vec H^*}{\partial
t'}=0,\\
div_{\vec x'}\vec H^*=0,\\
\vec E^*=\vec E-\frac{1}{c^2}\vec v\times\left(\vec v\times\vec
D\right)
\\
\vec H^*=\vec H-\frac{1}{c^2}\vec v\times\left(\vec v\times\vec
B\right)\\
\vec E=\vec D-\frac{1}{c}\,\vec v\times \vec B,\\
%\quad\quad\text{for}\;\;(\vec x,t)\in\R^3\times[0,+\infty)\\
\vec H=\vec B+\frac{1}{c}\,\vec v\times \vec D,
%D_0:=E+\frac{1}{c}\,v\times
%B\quad\quad\text{for}\;\;(\vec x,t)\in\R^3\times[0,+\infty)\\
%H:=B+\frac{1}{c}\,v\times
%D_0\quad\quad\text{for}\;\;(\vec x,t)\in\R^3\times[0,+\infty).
\end{cases}
\end{equation}
(See section \ref{GIGIGU} for details). In particular, in the
approximation, up to the order $\left(\frac{|\vec v|}{c}\right)^2\ll
1$ we have $\vec E^*\approx\vec E$ and $\vec H^*\approx\vec H$ and
then the approximate Maxwell equations have the form:
\begin{equation}\label{MaxMedFullGGffgguiuiouiogghghhgghhjhjhjjgghhgjhgghhhjkiljklKKint}
\begin{cases}
curl_{\vec x'}\vec H= \frac{4\pi}{c}\vec j+
\frac{1}{c}\frac{\partial \vec E}{\partial
t'},\\
div_{\vec x'}\vec E= 4\pi\left(\rho+\frac{1}{c^2}\,\vec v\cdot\vec j\right),\\
curl_{\vec x'}\vec E+\frac{1}{c}\frac{\partial\vec H}{\partial
t'}=0,\\
div_{\vec x'}\vec H=0,\\
\vec E=\vec D-\frac{1}{c}\,\vec v\times \vec B,\\
%\quad\quad\text{for}\;\;(\vec x,t)\in\R^3\times[0,+\infty)\\
\vec H=\vec B+\frac{1}{c}\,\vec v\times \vec D.
%D_0:=E+\frac{1}{c}\,v\times
%B\quad\quad\text{for}\;\;(\vec x,t)\in\R^3\times[0,+\infty)\\
%H:=B+\frac{1}{c}\,v\times
%D_0\quad\quad\text{for}\;\;(\vec x,t)\in\R^3\times[0,+\infty).
\end{cases}
\end{equation}
The first four equations in
\er{MaxMedFullGGffgguiuiouiogghghhgghhjhjhjjgghhgjhgghhhjkiljklKKint}
form a following system of equation:
\begin{equation}\label{MaxMedFullGGffgguiuiouiogghghhgghhjhjhjjgghhgjhgghhhjkiljklKKHHint}
\begin{cases}
curl_{\vec x'}\vec H= \frac{4\pi}{c}\vec j^*+
\frac{1}{c}\frac{\partial \vec E}{\partial
t'},\\
div_{\vec x'}\vec E= 4\pi\rho^*,\\
curl_{\vec x'}\vec E+\frac{1}{c}\frac{\partial\vec H}{\partial
t'}=0,\\
div_{\vec x'}\vec H=0,
\end{cases}
\end{equation}
where
\begin{equation}\label{fgjyhyfgfjjjkint}
\vec j^*:=\vec
j\quad\text{and}\quad\rho^*:=\left(\rho+\frac{1}{c^2}\,\vec
v\cdot\vec j\right)
\end{equation}
The system
\er{MaxMedFullGGffgguiuiouiogghghhgghhjhjhjjgghhgjhgghhhjkiljklKKHHint}
coincides with the classical Maxwell equations of the usual
Electrodynamics. Therefore, given known $\vec v$, $\rho$ and $\vec
j$,
\er{MaxMedFullGGffgguiuiouiogghghhgghhjhjhjjgghhgjhgghhhjkiljklKKHHint}
could be solved as easy as the usual wave equation, for example by
the method of retarded potentials. Then backward to
\er{giuuihjghgghjgj78zzrrZZffhhhggygghghjhvbKKint} change of
variables could be made in order to deduce the electromagnetic
fields in coordinates $(\vec x,t)$. Next note that, since we defined
$t'=\tau$, all the above clarifies the name "time" of the quantity
$\tau$. Finally we would like to note that if we have a motion of
some material body with the place $\vec r(t)$ and the velocity $\vec
u(t):=\frac{d\vec r}{dt}(t)$ and we associate the local
gravitational time $\tau$ with this body then clearly
\begin{equation}\label{fgjyhyfgfjjjkjkkjint}
d\tau\,=\, \left(1+\frac{1}{c^2}\vec u(t)\cdot \vec v\left(\vec
r(t),t\right)\right)\,dt\,\approx\, dt,
\end{equation}
where the last equality in \er{fgjyhyfgfjjjkjkkjint} is valid if we
have
\begin{equation}\label{fgjyhyfgfjjjkintkk}
\left(\frac{|\vec v|}{c}\right)^2\ll
1\quad\text{and}\quad\left(\frac{|\vec u(t)|}{c}\right)^2\ll 1.
\end{equation}
So we can use the local gravitational time $\tau$ in the approximate
calculations instead of the true time $t$.



























































\subsection{Motion of the particles in the gravitational and electromagnetic fields and invariance of Shr\"{o}dinger and Pauli equations}
Given a classical particle with inertial mass $m$, charge $\sigma$,
place $\vec r(t)$ and velocity $\vec u(t)=\vec r'(t)$ in the outer
gravitational field with the vectorial gravitational potential $\vec
v(\vec x,t)$, the outer electromagnetic field with vectorial and
scalar potentials $\vec A(\vec x,t)$ and $\vec \Psi(\vec x,t)$, and
additional conservative field with scalar potential $V(\vec x,t)$ we
consider a Lagrangian:
\begin{equation}\label{vhfffngghkjgghfjjint}
L_0\left(\frac{d\vec r}{dt},\vec
r,t\right):=\frac{m}{2}\left|\frac{d\vec r}{dt}-\vec v(\vec
r,t)\right|^2-\sigma\left(\Psi(\vec r,t)-\frac{1}{c}\vec A(\vec
r,t)\cdot\frac{d\vec r}{dt}\right)+V(\vec r,t).
\end{equation}
Then this Lagrangian is invariant under the change of non-inertial
coordinate system, given by \er{noninchgravortbstrjgghguittu2int}.
Moreover, we can show that a trajectory $\vec
r(t):[0,T]\to\mathbb{R}^3$ is a critical point of the functional
\begin{equation}\label{btfffygtgyggyijhhkkint}
J_0=\int_0^T L_0\left(\frac{d\vec r}{dt}(t),\vec r(t),t\right)dt.
\end{equation}
if and only if, consistently with
\er{noninchgravortbstrjgghguittu2gjgghhjhghjhjgghgghghghtytythvfghfgghjggint}
and \er{LorenzChlljklljklkkzzzint}, we have
\begin{multline}\label{vhfffngghkjgghggtghjgfhjoyuiyuyhiyyukukyihyuint}
m\frac{d^2\vec r}{dt^2}=m\left(\frac{\partial}{\partial t}\vec
v(\vec r,t)+\nabla_{\vec x}\left(\frac{1}{2}\left|\vec v(\vec
r,t)\right|^2\right)-\frac{d\vec r}{dt}\times curl_{\vec x}\vec
v(\vec r,t)\right)+\nabla_{\vec x}V\left(\vec
r,t\right)\\+\sigma\vec E(\vec r,t)+\frac{\sigma}{c}\frac{d\vec
r}{dt}\times \vec B(\vec r,t),
\end{multline}
where $\vec E$ and $\vec B$ are given by
\er{MaxVacFull1bjkgjhjhgjgjgkjfhjfdghghligioiuittrPPNint}.
%consistently with \er{noninchgravortbstrjgghguittu2gjgghhjhghjhjgghgghghghtytythvfghfgghjggint} for the case of the presence of Lorentz force.
Next if we define the
generalized moment of the particle $m$ by
\begin{equation}\label{guytyurtydftyiujhint}
\vec P:=\nabla_{\vec r'}L_0\left(\vec r',\vec r,t\right)=m
\frac{d\vec r}{dt}-m\vec v(\vec r,t)+\frac{\sigma}{c}\vec A(\vec
r,t),
\end{equation}
and consider a Hamiltonian
\begin{equation}\label{vhfffngghkjgghfjjhyjjfgint}
H_0\left(\vec P,\vec r,t\right):=\vec P\cdot\frac{d\vec
r}{dt}-L_0\left(\frac{d\vec r}{dt},\vec r,t\right),
\end{equation}
then we obtain:
\begin{equation}\label{vhfffngghkjgghfjjghghghint} H_0\left(\vec
P,\vec r,t\right)= \vec P\cdot\vec v(\vec r,t)+
\frac{1}{2m}\left|\vec P-\frac{\sigma}{c}\vec A(\vec
r,t)\right|^2+\sigma\left(\Psi(\vec r,t)-\frac{1}{c}\vec A(\vec
r,t)\cdot\vec v(\vec r,t)\right)-V\left(\vec r,t\right).
\end{equation}
%\begin{equation}\label{vhfffngghkjgghfjjghghghint} H_0\left(\vec P,\vec r,t\right)=\frac{1}{2m}\left|\vec P+m\vec v(\vec r,t)-\frac{\sigma}{c}\vec A(\vec
%r,t)\right|^2-\frac{m}{2}\left|\vec v(\vec r,t)\right|^2+\sigma\Psi(\vec r,t)-V(\vec r,t).\end{equation}
See subsection \ref{hggyugyuy} for the generalizations of the
Lagrangian and Hamiltonian in the case of system of $n$ classical
particles.



Next if we consider the motion of a quantum micro-particle with
inertial mass $m$ and charge $\sigma$ in the outer gravitational
field with the vectorial gravitational potential $\vec v(\vec x,t)$,
the outer electromagnetic field with vectorial and scalar potentials
$\vec A(\vec x,t)$ and $\Psi(\vec x,t)$, and additional conservative
field with potential $V(\vec x,t)$, not taking into account the spin
interaction, then the Shr\"{o}dinger equation for this particle is
\begin{equation}\label{vhfffngghkjgghfjjghghghhjghjgghkghggkghghjghint}
i\hbar\frac{\partial\psi}{\partial t}=\hat H_0\cdot\psi,
\end{equation}
where $\psi:=\psi(\vec x,t)\in\mathbb{C}$ is a wave function and
$\hat H_0$ is the Hamiltonian operator. Thus, since by
\er{vhfffngghkjgghfjjghghghint} the Hamiltonian operator has the
form of:
\begin{multline}\label{vhfffngghkjgghfjjghghghhjghjgghkghgghjhggjjkgint}
\hat H_0\cdot\psi= -\frac{i\hbar}{2}div_{\vec x}\left\{\psi\vec
v(\vec x,t)\right\}-\frac{i\hbar}{2}\vec v(\vec
x,t)\cdot\nabla_{\vec x}\psi+
\left\{\frac{1}{2m}\left(-i\hbar\nabla_{\vec x}-\frac{\sigma}{c}\vec
A(\vec x,t)\right)\circ\left(-i\hbar\nabla_{\vec
x}-\frac{\sigma}{c}\vec A(\vec
x,t)\right)\right\}\cdot\psi\\+\sigma\left(\Psi(\vec
x,t)-\frac{1}{c}\vec A(\vec x,t)\cdot\vec v(\vec
x,t)\right)\cdot\psi-V\left(\vec x,t\right)\cdot\psi,
%\\ \\
%\\ \left\{\frac{1}{2m}\left(-i\hbar\nabla_{\vec x}+m\vec v(\vec x,t)-\frac{\sigma}{c}\vec A(\vec x,t)\right)\circ\left(-i\hbar\nabla_{\vec x}+m\vec v(\vec x,t)-\frac{\sigma}{c}\vec A(\vec
%x,t)\right)\right\}\cdot\psi\\+\left\{-\frac{m}{2}\left|\vec v(\vecx,t)\right|^2+\sigma\Psi(\vec x,t)-V\left(\vec x,t\right)\right\}\cdot\psi,
\end{multline}
we rewrite the corresponding Shr\"{o}dinger equation as
\begin{multline}\label{vhfffngghkjgghfjjghghghhjghjgghkghgghjhggjjkgfgdiyfgfjkjgjgint}
i\hbar\left(\frac{\partial\psi}{\partial t}+\vec v\cdot\nabla_{\vec
x}\psi\right)+\frac{i\hbar}{2}\left(div_{\vec x}\vec v\right)\psi=\\
-\frac{\hbar^2}{2m}\Delta_{\vec x}\psi+\frac{
i\hbar\sigma}{2mc}div_{\vec x}\left\{\psi\vec A\right\}+\frac{
i\hbar\sigma}{2mc}\vec A\cdot\nabla_{\vec
x}\psi+\left(\sigma\Psi-\frac{\sigma}{c}\vec A\cdot\vec
v+\frac{\sigma^2}{2mc^2}\left|\vec A\right|^2-V\right)\psi.
\end{multline}
Then we can deduce that, under the change of non-inertial cartesian
coordinate system , given by \er{noninchgravortbstrjgghguittu2int},
the Shr\"{o}dinger equation of the form
\er{vhfffngghkjgghfjjghghghhjghjgghkghgghjhggjjkgfgdiyfgfjkjgjgint}
stays invariant, provided that, under
\er{noninchgravortbstrjgghguittu2int} we have
\begin{equation}\label{vyfgjhgjhvhgghint}
\begin{cases}
\psi'=\psi
\\
V'=V
\\
\vec v'=A(t)\cdot \vec v+A'(t)\cdot\vec x+\frac{d\vec z}{dt}(t)\\
\vec A'=A(t)\cdot\vec A\\
\Psi'-\frac{1}{c}\vec A'\cdot\vec v'=\Psi-\frac{1}{c}\vec A\cdot\vec
v .
\end{cases}
\end{equation}
So the laws of Quantum Mechanics are also invariant in every
non-inertial cartesian coordinate system. Next, assume that we are
in some inertial coordinate system and observe the Newtonian Law of
Gravitation in the form of
\er{MaxVacFull1ninshtrgravortghhghgjkgghklhjgkghghjjkjhjkkggjkhjkhjjhhfhjhklkhkhjjklzzzyyyhjggjhgghhjhNWNWBWHWPPN222int}.
Then, as a consequence, we have
\er{MaxVacFull1ninshtrgravortghhghgjkgghklhjgkghghjjkjhjkkggjkhjkhjjhhfhjhklkhkhjjklzzzyyyhjggjhgghhjhNWNWNWNWNWBWHWPPN222int}
for some scalar field $Z$ and the scalar Newtonian gravitational
potential $\Phi$. Thus denoting
\begin{equation}\label{jhgghfiuihyhyiy}
\psi_1:=e^{\frac{im}{\hbar}Z}\psi,
\end{equation}
%and using the fact that $div_{\vec x}\vec A=0$
we rewrite
\er{vhfffngghkjgghfjjghghghhjghjgghkghgghjhggjjkgfgdiyfgfjkjgjgint}
in the given inertial coordinate system as:
\begin{multline*}
%\label{vhfffngghkjgghfjjghghghhjghjgghkghgghjhggjjkgfgdiyfgfjkjgjggjjgugyyjjkgghgjjhhjkkhjhjkint}
i\hbar\frac{\partial\psi_1}{\partial
t}=-\frac{\hbar^2}{2m}\Delta_{\vec x}\psi_1+\frac{
i\hbar\sigma}{2mc}div_{\vec x}\left\{\psi_1\vec A\right\}
%+\frac{i\hbar\sigma}{2mc}div_{\vec x}\left\{\psi\vec A\right\}
+\frac{ i\hbar\sigma}{2mc}\vec A\cdot\nabla_{\vec x}\psi_1
%-\frac{\sigma}{c}\left(\vec A\cdot\nabla_{\vec x}Z\right)e^{\frac{im}{\hbar}Z}\psi
+\left(\sigma\Psi+\frac{\sigma^2}{2mc^2}\left|\vec
A\right|^2-V+m\Phi\right)\psi_1,
\end{multline*}
which coincides with the classical Shr\"{o}dinger equation for this
case. Note also that by Remark \ref{ugyugg}, equality
\er{jhgghfiuihyhyiy} implies that under the change of coordinate
system given by the Galilean Transformation
\er{noninchgravortbstrjgghguittu1int} the quantity $\psi_1$
transforms as:
\begin{equation}\label{jhgghfhjhjhjhyuyuyint}
\psi'_1:=e^{\frac{im}{\hbar}(\vec w\cdot\vec x+\frac{1}{2}|\vec
w|^2t)}\psi_1,
\end{equation}
provided that $\psi'=\psi$. Moreover, \er{jhgghfhjhjhjhyuyuyint}
coincides with the classical law of transformation of the wave
function, under the Galilean Transformation (see section 17 in
\cite{LL}).

Next, again consider the motion of a quantum micro-particle having
the inertial mass $m$ and the charges $\sigma$ with the given
gravitational and electromagnetical fields with potentials $\vec
v(\vec x,t)$, $\vec A(\vec x,t)$ and $\Psi(\vec x,t)$ and additional
conservative field with potential $V(\vec x,t)$, not taking into the
account the the spin interaction. Then consider a Lagrangian density
$L_0$ defined by
\begin{multline}\label{vhfffngghkjgghDDmmkkkZZintkk}
L_0\left(\psi,\vec x,t\right):=
\frac{i\hbar}{2}\left(\left(\frac{\partial\psi}{\partial t}+\vec
v\cdot\nabla_{\vec
x}\psi\right)\cdot\bar\psi-\psi\cdot\left(\frac{\partial\bar\psi}{\partial
t}+\vec v\cdot\nabla_{\vec
x}\bar\psi\right)\right)-\frac{\hbar^2}{2m}\nabla_{\vec
x}\psi\cdot\nabla_{\vec x}\bar\psi\\
-\frac{\hbar\sigma i}{2mc}\left(\nabla_{\vec
x}\psi\cdot\bar\psi-\psi\cdot\nabla_{\vec x}\bar\psi\right)\cdot\vec
A-\frac{\sigma^2}{2mc^2}\left|\vec A\right|^2\psi\cdot\bar\psi
-\sigma\left(\Psi-\frac{1}{c}\vec v\cdot\vec
A\right)\psi\cdot\bar\psi+V\left(\vec x,t\right)\psi\cdot\bar\psi,
\end{multline}
where $\psi\in \mathbb{C}$ is a wave function. Then, as before, we
can prove that $L_0$ is invariant under the change of inertial or
non-inertial cartesian coordinate system, given by
\er{noninchgravortbstrjgghguittu2int}, provided that we take into
account \er{vyfgjhgjhvhgghint}. Moreover, if we consider a
functional
\begin{equation}\label{btfffygtgyggyDDmmkkkZZintkk}
J_0=\int_0^T\int_{\mathbb{R}^3}L_0\left(\psi,\vec x,t\right)d\vec x
dt,
\end{equation}
Then, by \er{vhfffngghkjgghDDmmkkkZZintkk} we get that the
Euler-Lagrange equation for \er{btfffygtgyggyDDmmkkkZZintkk}
coincides with the Shr\"{o}dinger equation in the form of
\er{vhfffngghkjgghfjjghghghhjghjgghkghgghjhggjjkgfgdiyfgfjkjgjgint}.
See subsection \ref{hggyugyuy1} for the generalizations of all
mentioned above about the Shr\"{o}dinger equation to the case of
system of $n$ quantum particles. Next we would like to note that the
Lagrangian density $L_0$, defined by
\er{vhfffngghkjgghDDmmkkkZZintkk} obeys $U(1)$ local symmetry, i.e.
for every scalar field $w:=w(\vec x,t)$ one can easily deduce that
$L_0$ in \er{vhfffngghkjgghDDmmkkkZZintkk} is invariant under the
transformation:
\begin{equation}\label{MaxVacFull1bjkgjhjhgjgjgkjfhjfdghghligioiuittrPPNhjkjhkjgghhjjhjintiiihh}
\begin{cases}
\psi\,\to\,e^{-\frac{i\sigma w}{c\hbar}}\,\psi
\\
\Psi\,\to\,\Psi+\frac{1}{c}\frac{\partial w}{\partial t}\\
\vec A\,\to\,\vec A-\nabla_{\vec x}w\\
\vec v\,\to\,\vec v.
\end{cases}
\end{equation}






















Next consider the motion of a spin-half quantum micro-particle with
inertial mass $m$ and the charge $\sigma$ in the outer gravitational
and electromagnetical field with potentials $\vec v(\vec x,t)$,
$\vec A(\vec x,t)$ and $\Psi(\vec x,t)$ and additional conservative
field with potential $V(\vec x,t)$. Since the Hamiltonian for a
macro-particle has the form \er{vhfffngghkjgghfjjghghghint}, we
built the Hamiltonian operator, taking into account the spin
interaction as
\begin{multline}\label{vhfffngghkjgghfjjghghghSYShmyuuiiuuhmhmiopoopnniukjhjkkkllkkkZZint}
\hat H_0\cdot\psi= -\frac{\hbar^2}{2m}\Delta_{\vec
x}\psi+\frac{i\hbar\sigma}{2mc}div_{\vec x}\left\{\psi_1\vec A(\vec
x,t)\right\}+\frac{i\hbar\sigma}{2mc}\nabla_{\vec x}\psi\cdot\vec
A(\vec x,t)+\frac{\sigma^2}{2mc^2}\left|\vec A(\vec
x,t)\right|^2\psi\\+\sigma\left(\Psi(\vec x,t)-\frac{1}{c}\vec
v(\vec x,t)\cdot\vec A(\vec x,t)\right)\psi-V\left(\vec
x,t\right)\psi-\frac{i\hbar}{2}div_{\vec x}\left\{\psi\vec v(\vec
x,t)\right\}-\frac{i\hbar}{2}\nabla_{\vec x}\psi\cdot\vec v(\vec
x,t)
\\-\frac{g\sigma\hbar}{2mc}\vec
S\cdot\left(curl_{\vec x}\vec A(\vec
x,t)\,\psi\right)+\frac{\hbar}{4}\vec S\cdot\left(curl_{\vec x}\vec
v(\vec x,t)\,\psi\right),
\end{multline}
where $\psi(\vec x,t)=\left(\psi_1(\vec x,t),\psi_2(\vec
x,t)\right)\in\mathbb{C}^2$ is a two-component wave function, $\hat
H_0$ is the Hamiltonian operator, $\vec S:=(S_1,S_2,S_3)$,
$$S_1=\left(\begin{matrix}0&1\\1&0\end{matrix}\right),\quad
S_2=\left(\begin{matrix}0&-i\\i&0\end{matrix}\right)\quad
S_3=\left(\begin{matrix}1&0\\0&-1\end{matrix}\right)$$ are Pauli
matrices and $g$ is a constant that depends on the type of the
particle (for electron we have $g=1$). Note that, in addition to the
classical term of the spin-magnetic interaction, we added another
term to the Hamiltonian, namely $\frac{\hbar}{4}\vec
S\cdot\left(curl_{\vec x}\vec v(\vec x,t)\,\psi\right)$. This term
vanishes in every non-rotating and, in particular, in every inertial
coordinate system, however it provides the invariance of the
Shr\"{o}dinger-Pauli equation, under the change of non-inertial
cartesian coordinate system, as can be seen in the following Theorem
\ref{gjghghgghgintintrrZZint}. The Shr\"{o}dinger-Pauli equation for
this particle is
\begin{equation}\label{vhfffngghkjgghfjjghghghhjghjgghkghggkghghjghSYSPNnnkkllkkkZZint}
i\hbar\frac{\partial\psi}{\partial t}=\hat H_0\cdot\psi.
\end{equation}
I.e,
\begin{multline}\label{vhfffngghkjgghfjjghghghSYShmyuuiiuuhmhmiopoopnniukjhjkk;l;lkhjjkihjjhkkkkkjjjZZint}
i\hbar\left(\frac{\partial\psi}{\partial t}+\frac{1}{2}div_{\vec
x}\left\{\psi\vec v(\vec x,t)\right\}+\frac{1}{2}\nabla_{\vec
x}\psi\cdot\vec v(\vec x,t)\right)\\=-\frac{\hbar^2}{2m}\Delta_{\vec
x}\psi+\frac{i\hbar\sigma}{2mc}div_{\vec x}\left\{\psi\vec A(\vec
x,t)\right\}+\frac{i\hbar\sigma}{2mc}\nabla_{\vec x}\psi\cdot\vec
A(\vec x,t)+\frac{\sigma^2}{2mc^2}\left|\vec A(\vec
x,t)\right|^2\psi\\
%-\frac{\sigma\hbar}{2mc}\vec S\cdot\left(curl_{\vec x}\vec A(\vec x,t)\,\psi\right)
%
%
%
%+\sum_{j=1}^{n}m_j\left(1+\frac{1}{c^2}\left|\frac{1}{m_j}\vec P_j-\frac{\sigma_j}{m_jc}\vec A(\vec r_j,t)\right|^2\right)^{-\frac{1}{2}}\left|\frac{1}{m_j}\vec P_j-\frac{\sigma_j}{m_jc}\vec A(\vec r_j,t)\right|^2
+\sigma\left(\Psi(\vec x,t)-\frac{1}{c}\vec v(\vec x,t)\cdot\vec
A(\vec x,t)\right)\psi-V\left(\vec
x,t\right)\psi+\frac{\hbar}{2}\vec
S\cdot\left(\left(\frac{1}{2}curl_{\vec x}\vec v(\vec
x,t)-\frac{g\sigma}{mc}curl_{\vec x}\vec A(\vec
x,t)\right)\,\psi\right).
\end{multline}
%where $\vec S:=(S_1,S_2,S_3)$ and
%$$S_1=\left(\begin{matrix}0&1\\1&0\end{matrix}\right),\quad
%S_2=\left(\begin{matrix}0&-i\\i&0\end{matrix}\right)\quad
%S_3=\left(\begin{matrix}1&0\\0&-1\end{matrix}\right)$$ are Pauli matrices.
In subsection \ref{hjjghjghggh} we prove the following:
\begin{theorem}\label{gjghghgghgintintrrZZint}
Consider that the change of some cartesian coordinate system $(*)$
to another cartesian coordinate system $(**)$ is given by
\er{noninchgravortbstrjgghguittu2int},
%\begin{equation}\label{noninchgravortbstrjgghguittu2intrrrZZint}\begin{cases}
%\vec x'=A(t)\cdot\vec x+\vec z(t),\\ t'=t,
%\end{cases}\end{equation}
where $A(t)\in SO(3)$ is a rotation. Next, assume that in the
coordinate system $(**)$ we observe a validity of the
Shr\"{o}dinger-Pauli equation of the form:
\begin{multline}\label{MaxVacFull1ninshtrredPPNintrrZZint}
i\hbar\left(\frac{\partial\psi'}{\partial t'}+\frac{1}{2}div_{\vec
x'}\left\{\psi'\vec v'\right\}+\frac{1}{2}\nabla_{\vec
x'}\psi'\cdot\vec v'\right)=-\frac{\hbar^2}{2m'}\Delta_{\vec
x'}\psi'+\frac{i\hbar\sigma'}{2m'c}div_{\vec x'}\left\{\psi'\vec
A'\right\}+\frac{i\hbar\sigma'}{2m'c}\nabla_{\vec x'}\psi'\cdot\vec
A'\\+\frac{(\sigma')^2}{2m'c^2}\left|\vec A'\right|^2\psi'
%-\frac{\sigma\hbar}{2mc}\vec S\cdot\left(curl_{\vec x}\vec A(\vec x,t)\,\psi\right)
%
%
%
%+\sum_{j=1}^{n}m_j\left(1+\frac{1}{c^2}\left|\frac{1}{m_j}\vec P_j-\frac{\sigma_j}{m_jc}\vec A(\vec r_j,t)\right|^2\right)^{-\frac{1}{2}}\left|\frac{1}{m_j}\vec P_j-\frac{\sigma_j}{m_jc}\vec A(\vec r_j,t)\right|^2
+\sigma'\left(\Psi'-\frac{1}{c}\vec v'\cdot\vec
A'\right)\psi'-V'\psi'+\frac{\hbar}{2}\vec
S\cdot\left(\left(\frac{1}{2}curl_{\vec x'}\vec
v'-\frac{g'\sigma'}{m'c}curl_{\vec x'}\vec A'\right)\,\psi'\right),
\end{multline}
where $\psi\in\mathbb{C}^2$. Then in the coordinate system $(*)$ we
have the validity of Shr\"{o}dinger-Pauli equation of the same as
\er{MaxVacFull1ninshtrredPPNintrrZZint} form:
\begin{multline}\label{MaxVacFull1ninshtrhjkkredPPNintrrZZint}
i\hbar\left(\frac{\partial\psi}{\partial t}+\frac{1}{2}div_{\vec
x}\left\{\psi\vec v\right\}+\frac{1}{2}\nabla_{\vec x}\psi\cdot\vec
v\right)=-\frac{\hbar^2}{2m}\Delta_{\vec
x}\psi+\frac{i\hbar\sigma}{2mc}div_{\vec x}\left\{\psi\vec
A\right\}+\frac{i\hbar\sigma}{2mc}\nabla_{\vec x}\psi\cdot\vec
A\\+\frac{\sigma^2}{2mc^2}\left|\vec A\right|^2\psi
%-\frac{\sigma\hbar}{2mc}\vec S\cdot\left(curl_{\vec x}\vec A(\vec x,t)\,\psi\right)
%
%
%
%+\sum_{j=1}^{n}m_j\left(1+\frac{1}{c^2}\left|\frac{1}{m_j}\vec P_j-\frac{\sigma_j}{m_jc}\vec A(\vec r_j,t)\right|^2\right)^{-\frac{1}{2}}\left|\frac{1}{m_j}\vec P_j-\frac{\sigma_j}{m_jc}\vec A(\vec r_j,t)\right|^2
+\sigma\left(\Psi-\frac{1}{c}\vec v\cdot\vec
A\right)\psi-V\psi+\frac{\hbar}{2}\vec
S\cdot\left(\left(\frac{1}{2}curl_{\vec x}\vec
v-\frac{g\sigma}{mc}curl_{\vec x}\vec A\right)\,\psi\right).
\end{multline}
provided that
\begin{equation}\label{yuythfgfyftydtydtydtyddyyyhhddhhhredPPN111hgghjgintintrrZZint}
\begin{cases}
g'=g\\
V'=V,\\
\sigma'=\sigma,\\
m'=m,\\
\vec v'=A(t)\cdot \vec v+\frac{dA}{dt}(t)\cdot\vec x+\frac{d\vec z}{dt}(t),\\
\vec A'=A(t)\cdot \vec A,\\
\Psi'-\vec v'\cdot\vec A'=\Psi-\vec v\cdot\vec A,\\
\psi'=U(t)\cdot\psi,
\end{cases}
\end{equation}
where $U(t)\in SU(2)$ is some special unitary $2\times 2$ matrix
i.e. $U(t)\in\mathbb{C}^{2\times 2}$, $det\,U(t)=1$, $U(t)\cdot
U^*(t)=I$ where $U^*(t)$ is the Hermitian adjoint to $U(t)$ matrix:
$U^*(t):=\bar U(t)^T$ and $I$ is the identity $2\times 2$ matrix.
Moreover, $U(t)$ is characterized by the equality:
\begin{equation}\label{gyfyfgfgfgghZZint}
U^*(t)\cdot\vec S\cdot U(t)=A(t)\cdot\vec S.
\end{equation}
%that means
%\begin{multline*}
%\left(U^*(t)\cdot S_1\cdot U(t),U^*(t)\cdot S_2\cdot U(t),U^*(t)\cdot S_3\cdot U(t)\right)=\\
%\left(a_{11}(t)S_1+a_{12}(t)S_2+a_{13}(t)S_3\,,\,a_{21}(t)S_1+a_{22}(t)S_2+a_{23}(t)S_3\,,\,a_{31}(t)S_1+a_{32}(t)S_2+a_{33}(t)S_3\right),
%\end{multline*}
%where $A(t)=\left\{a_{mk}(t)\right\}_{\{1\leq m,k\leq 3\}}$.
\end{theorem}
Next, again consider the motion of a quantum micro-particle with
spin-half, inertial mass $m$ and the charge $\sigma$ with the given
gravitational and electromagnetical fields with potentials $\vec
v(\vec x,t)$, $\vec A(\vec x,t)$ and $\Psi(\vec x,t)$ and additional
conservative field with potential $V(\vec x,t)$, taking into the
account spin interaction. Then consider a Lagrangian density $L$
defined by
\begin{multline}\label{vhfffngghkjgghDDmmkkkZZint}
L\left(\psi,\vec x,t\right):=
\frac{i\hbar}{2}\left(\left(\frac{\partial\psi}{\partial t}+\vec
v\cdot\nabla_{\vec
x}\psi\right)\cdot\bar\psi-\psi\cdot\left(\frac{\partial\bar\psi}{\partial
t}+\vec v\cdot\nabla_{\vec
x}\bar\psi\right)\right)-\frac{\hbar^2}{2m}\nabla_{\vec
x}\psi\cdot\nabla_{\vec x}\bar\psi\\
-\frac{\hbar\sigma i}{2mc}\left(\nabla_{\vec
x}\psi\cdot\bar\psi-\psi\cdot\nabla_{\vec x}\bar\psi\right)\cdot\vec
A-\frac{\sigma^2}{2mc^2}\left|\vec A\right|^2\psi\cdot\bar\psi
-\sigma\left(\Psi-\frac{1}{c}\vec v\cdot\vec
A\right)\psi\cdot\bar\psi\\ -\frac{\hbar}{2}\left(\left(\vec
S\cdot\left(\frac{1}{2}curl_{\vec x}\vec
v-\frac{g\sigma}{mc}curl_{\vec x}\vec
A\right)\right)\cdot\psi\right)\cdot\bar\psi+V\left(\vec
x,t\right)\psi\cdot\bar\psi,
\end{multline}
where $\psi\in \mathbb{C}^2$ is a two-component wave function. Then
similarly to the proof of Theorem \ref{gjghghgghgintintrrZZint} we
can prove that $L$ is invariant under the change of inertial or
non-inertial cartesian coordinate system, given by
\er{noninchgravortbstrjgghguittu2int}, provided that we take into
account
\er{yuythfgfyftydtydtydtyddyyyhhddhhhredPPN111hgghjgintintrrZZint}.
Moreover, if we consider a functional
\begin{equation}\label{btfffygtgyggyDDmmkkkZZint}
J=\int_0^T\int_{\mathbb{R}^3}L\left(\psi,\vec x,t\right)d\vec x dt,
\end{equation}
then, by \er{vhfffngghkjgghDDmmkkkZZint} we get that the
Euler-Lagrange equation for \er{btfffygtgyggyDDmmkkkZZint} coincides
with the Shr\"{o}dinger-Pauli equation in the form of
\er{vhfffngghkjgghfjjghghghSYShmyuuiiuuhmhmiopoopnniukjhjkk;l;lkhjjkihjjhkkkkkjjjZZint}.
Next we would like to note that, as before, the Lagrangian density
$L$, defined by \er{vhfffngghkjgghDDmmkkkZZint} obeys $U(1)$ local
symmetry, i.e. for every scalar field $w:=w(\vec x,t)$ one can
easily deduce that $L$ in \er{vhfffngghkjgghDDmmkkkZZint} is
invariant under the transformation:
\begin{equation}\label{MaxVacFull1bjkgjhjhgjgjgkjfhjfdghghligioiuittrPPNhjkjhkjgghhjjhjintiiihhintsp}
\begin{cases}
\psi\,\to\,e^{-\frac{i\sigma w}{c\hbar}}\,\psi
\\
\Psi\,\to\,\Psi+\frac{1}{c}\frac{\partial w}{\partial t}\\
\vec A\,\to\,\vec A-\nabla_{\vec x}w\\
\vec v\,\to\,\vec v.
\end{cases}
\end{equation}










































\subsection{Unified gravitational-electromagnetic field and conservation laws}
Similarly to our assumption that the electromagnetic field is
influenced by gravitational field, we also can assume that the
gravitational field is influenced by electromagnetic field. We
remind that we assume that the first approximation of the law of
gravitation is given by
\er{MaxVacFull1ninshtrgravortghhghgjkgghklhjgkghghjjkjhjkkggjkhjkhjjhhfhjhklkhkhjjklzzzyyyNWBWHWPPN222int}.
However, till now we said nothing about the relation between the
density of inertial and gravitational masses. If $\mu$ is the
density of inertial masses and $M$ is the density of gravitational
masses, then consistently with the classical Newtonian theory of
gravitation we assume that in the absence of essential
electromagnetic fields we should have
\begin{equation}\label{gghjgghfghdint}
M=\mu.
\end{equation}
In order to satisfy the conservation laws of linear and angular
momentums and energy, consider the following conserved scalar field
$Q$, that we call "electromagnetical-gravitational" mass density,
which is negligible in the absence of electromagnetic fields and
satisfies the identity
\begin{equation}
\label{MaxVacFull1ninshtrgravortghhghgjkgghklhjgkghghjjkjhjkkggjkhjkhjjhhfhjhklkhkhjjklzzzyyyNWNWBWHWPPNint}
\frac{\partial Q}{\partial t}+div_{\vec x}\left\{Q\vec v\right\}=-
div_\vec x\left\{\frac{1}{4\pi c}\vec D\times \vec B\right\}
%\frac{d\vec u}{dt}=-curl_{\vec x}\vec v\times(\vec u-\vec v)+\partial_t\vec v+(\nabla_{\vec x}\vec v)\cdot\vec v+\nabla_{\vec x}\psi_0+\frac{1}{m}\vec F.
\end{equation}
in the general case. Then, instead of \er{gghjgghfghdint}, for the
general case of gravitational-electromagnetic fields we consider the
following relation between the gravitational and inertial mass
densities
\begin{equation}\label{gghjgghfghdkjgjjint}
M=\mu+Q.
\end{equation}
Then by
\er{MaxVacFull1ninshtrgravortghhghgjkgghklhjgkghghjjkjhjkkggjkhjkhjjhhfhjhklkhkhjjklzzzyyyNWBWHWPPN222int}
and \er{gghjgghfghdkjgjjint} we have the following law of
gravitation:
\begin{equation}
\label{MaxVacFull1ninshtrgravortghhghgjkgghklhjgkghghjjkjhjkkggjkhjkhjjhhfhjhklkhkhjjklzzzyyyNWBWHWPPNint}
\begin{cases}
curl_{\vec x}\left(curl_{\vec x}\vec v\right)= 0,\\
\frac{\partial}{\partial t}\left(div_{\vec x}\vec v\right)+div_{\vec
x}\left\{\left(div_{\vec x}\vec v\right)\vec
v\right\}+\frac{1}{4}\left|d_{\vec x}\vec v+\{d_{\vec x}\vec
v\}^T\right|^2-\left(div_{\vec x}\vec v\right)^2=  -4\pi G(\mu+Q).
%\frac{d\vec u}{dt}=-curl_{\vec x}\vec v\times(\vec u-\vec v)+\partial_t\vec v+(\nabla_{\vec x}\vec v)\cdot\vec v+\nabla_{\vec x}\psi_0+\frac{1}{m}\vec F.
\end{cases}
\end{equation}
The laws
\er{MaxVacFull1ninshtrgravortghhghgjkgghklhjgkghghjjkjhjkkggjkhjkhjjhhfhjhklkhkhjjklzzzyyyNWNWBWHWPPNint}
and
\er{MaxVacFull1ninshtrgravortghhghgjkgghklhjgkghghjjkjhjkkggjkhjkhjjhhfhjhklkhkhjjklzzzyyyNWBWHWPPNint}
are invariant under the change of non-inertial cartesian coordinate
system, given by \er{noninchgravortbstrjgghguittu2int}, provided
that, under \er{noninchgravortbstrjgghguittu2int} we have $Q'=Q$ and
$\mu'=\mu$. In particular, in the inertial coordinate system $(*)$
we should have:
\begin{equation}
\label{MaxVacFull1ninshtrgravortghhghgjkgghklhjgkghghjjkjhjkkggjkhjkhjjhhfhjhklkhkhjjklzzzyyyhjggjhgghhjhNWNWBWHWPPNint}
\begin{cases}
curl_{\vec x}\vec v= 0,\\
\frac{\partial\vec v}{\partial t}+d_\vec x\vec v\cdot\vec v=
-\nabla_{\vec x}\Phi,
%\frac{d\vec u}{dt}=-curl_{\vec x}\vec v\times(\vec u-\vec v)+\partial_t\vec v+(\nabla_{\vec x}\vec v)\cdot\vec v+\nabla_{\vec x}\psi_0+\frac{1}{m}\vec F.
\end{cases}
\end{equation}
where $\Phi$ is the scalar gravitational potential which is a scalar
field satisfying in every coordinate system:
\begin{equation}
\label{MaxVacFull1ninshtrgravortghhghgjkgghklhjgkghghjjkjhjkkggjkhjkhjjhhfhjhklkhkhjjklzzzyyyhjggjhgghhjhNWNWNWBWHWPPNint}
\Delta_{\vec x}\Phi=4\pi G(\mu+Q).
\end{equation}
%Here $G$ is the gravitational constant.
\begin{remark}\label{ghghvghhgggh}
Lemma \ref{fgbfghfh} from Appendix gives some insight that the
"electromagnetical-gravitational" mass density $Q$ in
\er{MaxVacFull1ninshtrgravortghhghgjkgghklhjgkghghjjkjhjkkggjkhjkhjjhhfhjhklkhkhjjklzzzyyyNWNWBWHWPPNint}
should have the values of the same order as the quantity
$\frac{1}{c^2}\left(|\vec D|^2+|\vec B|^2\right)$ and therefore, in
the usual circumstances is negligible with respect to the inertial
mass density $\mu$. Thus we can write $Q\approx 0$ in
\er{MaxVacFull1ninshtrgravortghhghgjkgghklhjgkghghjjkjhjkkggjkhjkhjjhhfhjhklkhkhjjklzzzyyyNWBWHWPPNint},
i.e. the force of gravity in an inertial coordinate system
approximately equals to the classical Newtonian force of gravity.
\end{remark}
Next consider the Maxwell equation in the vacuum in the form
\er{MaxVacFull1bjkgjhjhgjaaaint} and consistently with
\er{noninchgravortbstrjgghguittu2gjgghhjhghjhjgghgghghghtytythvfghfgghjggint},
consider the second Law of Newton for the moving continuum with the
inertial mass density $\mu$ and the field of velocities $\vec u$:
\begin{equation}\label{MaxVacFull1ninshtrgravortghhghgjkgghklhjgkghghjjkjhjkkggjkhjkhjjhhfhjhkjkhbbgjhzzzyyykkknnnNWBWHWPPNint}
\mu\frac{\partial\vec u}{\partial t}+\mu d_{\vec x}\vec u\cdot\vec
u=-\mu\vec u\times curl_{\vec x}\vec v+\mu\partial_{t}\vec
v+\mu\nabla_{\vec x}\left(\frac{1}{2}|\vec v|^2\right)+\rho\vec
E+\frac{1}{c}\vec j\times\vec B+\vec G.
%\frac{d\vec u}{dt}=-curl_{\vec x}\vec v\times(\vec u-\vec v)+\partial_t\vec v+(\nabla_{\vec x}\vec v)\cdot\vec v+\nabla_{\vec x}\psi_0+\frac{1}{m}\vec F.
\end{equation}
where $\rho\vec E+\frac{1}{c}\vec j\times\vec B$ is the volume
density of the Lorentz force and $\vec G$ is the total volume
density of all non-gravitational and non-electromagnetic forces
acting on the continuum with mass density $\mu$. Then, in section
\ref{GravElectro} we prove that in inertial coordinate systems we
have conservation laws of the linear momentum, the angular momentum
and the energy. More precisely, we have the following theorem:
\begin{theorem}
Consider the Maxwell equation for the vacuum in the form
\er{MaxVacFull1bjkgjhjhgjaaaint} and the second Law of Newton for
the moving continuum in the form
\er{MaxVacFull1ninshtrgravortghhghgjkgghklhjgkghghjjkjhjkkggjkhjkhjjhhfhjhkjkhbbgjhzzzyyykkknnnNWBWHWPPNint}.
Next, assume that in some cartesian coordinate system $(*)$ we
observe the gravitational law in the form of
\er{MaxVacFull1ninshtrgravortghhghgjkgghklhjgkghghjjkjhjkkggjkhjkhjjhhfhjhklkhkhjjklzzzyyyhjggjhgghhjhNWNWBWHWPPNint},
\er{MaxVacFull1ninshtrgravortghhghgjkgghklhjgkghghjjkjhjkkggjkhjkhjjhhfhjhklkhkhjjklzzzyyyhjggjhgghhjhNWNWNWBWHWPPNint}
and
\er{MaxVacFull1ninshtrgravortghhghgjkgghklhjgkghghjjkjhjkkggjkhjkhjjhhfhjhklkhkhjjklzzzyyyNWNWBWHWPPNint}.
Then in the system $(*)$ we have the following laws of conservation
of the linear momentum, angular momentum and energy:
\begin{multline}\label{hvkgkjgkjbjkjjkgjglhhkhjyuyghjhhjhjfghfdhgdhdfdhzzzyyyjffjjkkhhkgjgkjhfhjhgfffjgjhgjffgjggjgjggjhgkkkggjgjgnnnmmmNWNWNWNWBWHWMPPNint}
\frac{\partial}{\partial t}\left(\mu\vec u+Q\vec v+\frac{1}{4\pi
c}\,\vec D\times \vec B\right)=\\-div_\vec x\left\{\mu\vec
u\otimes\vec u+Q\vec v\otimes\vec v+\left(\frac{1}{4\pi c}\vec
D\times \vec B\right)\otimes \vec v+\vec v\otimes\left(\frac{1}{4\pi
c}\vec D\times \vec
B\right)\right\}\\
+\frac{1}{4\pi}div_\vec x\left\{\vec D\otimes \vec D+\vec B\otimes
\vec B-\frac{1}{2}\left(|\vec D|^2+|\vec
B|^2\right)I-\frac{1}{G}\nabla_{\vec x}\Phi\otimes\nabla_{\vec
x}\Phi+\frac{1}{2G}\left|\nabla_{\vec x}\Phi\right|^2 I\right\}+\vec
G,
\end{multline}
\begin{multline}\label{hvkgkjgkjbjkjjkgjglhhkhjyuyghjhhjhjfghfdhgdhdfdhzzzyyyjffjjkkhhkgjgkjhfhjhgfffjgjhgjffgjggjgjggjhgkkkggjgjgmomnnnmmmNWNWBWHWMMPPNint}
\frac{\partial}{\partial t}\left(\vec x\times(\mu\vec u)+\vec
x\times(Q\vec v)+\vec x\times\left(\frac{1}{4\pi c}\,\vec D\times
\vec B\right)\right)=\\-div_\vec x\left\{\mu(\vec x\times\vec
u)\otimes\vec u+Q(\vec x\times\vec v)\otimes\vec v+\left(\vec
x\times\left(\frac{1}{4\pi c}\vec D\times \vec
B\right)\right)\otimes \vec v+(\vec x\times\vec
v)\otimes\left(\frac{1}{4\pi c}\vec D\times \vec
B\right)\right\}\\
+\frac{1}{4\pi}div_\vec x\left\{(\vec x\times\vec D)\otimes \vec
D+(\vec x\times\vec B)\otimes \vec B-\frac{1}{G}(\vec
x\times\nabla_{\vec x}\Phi)\otimes\nabla_{\vec
x}\Phi\right\}\\+\frac{1}{8\pi}curl_{\vec x}\left\{\left(|\vec
D|^2+|\vec B|^2-\frac{1}{G}\left|\nabla_{\vec
x}\Phi\right|^2\right)\vec x\right\}+ \vec x\times\vec G,
\end{multline}
and
\begin{multline}\label{MaxVacFull1ninshtrgravortghhghgjkgghklhjgkghghjjkjhjkkggjkhjkhjjhhfhjhkjkhbbgjhhkjhhklhzzzyyykkkgkhjjhgfhjjffghuikkgkjghhjkjknnnmmmNWNWBWHWhhjhhENPPNint}
\frac{\partial}{\partial t}\left(\frac{1}{2}\mu|\vec
u|^2+\frac{1}{2}Q\left|\vec v\right|^2+\frac{\vec D\cdot\vec E+\vec
B\cdot\vec H}{8\pi}-\frac{1}{8\pi G}\big|\nabla_{\vec
x}\Phi\big|^2\right)=\\
-div_\vec x\left\{\left(\frac{\mu|\vec u|^2}{2}\right)\vec
u+\left(\frac{Q|\vec v|^2}{2}\right)\vec v+\frac{1}{2}\left|\vec
v\right|^2\left(\frac{1}{4\pi c}\,\vec D\times \vec
B\right)+\left(\frac{\vec D\cdot\vec E+\vec B\cdot\vec
H}{8\pi}\right)\vec v\right\}
\\
+\frac{1}{4\pi}div_\vec x\left\{(\vec D\otimes \vec D+ \vec B\otimes
\vec B)\cdot \vec v-\frac{1}{2}\left(|\vec D|^2+|\vec
B|^2\right)\vec v-c \vec D\times \vec B\right\}\\-div_\vec
x\left\{\Phi\left(\mu\vec u+Q\vec v+\frac{1}{4\pi c}\,\vec D\times
\vec B\right)\right\}
%-div_{\vec x}\left\{\Phi\left(\frac{1}{4\pi c}\,\vec D\times \vec B\right)\right\}
-\frac{1}{4\pi G}div_{\vec x}\left\{\Phi\frac{\partial}{\partial
t}(\nabla_{\vec x}\Phi)\right\}+\vec G\cdot\vec u.
\end{multline}
\end{theorem}
%
%
%
\begin{comment}
\begin{multline}
\\
%\frac{\partial}{\partial t}\left(\frac{1}{2}\mu|\vec u|^2+\frac{1}{2}Q\left|\vec v\right|^2+\frac{|\vec D|^2+|\vec B|^2}{8\pi}+\frac{1}{4\pi
%c}\left(\vec D\times \vec B\right)\cdot\vec v-\frac{1}{8\pi G}\big|\nabla_{\vec x}\Phi\big|^2\right)=\\
-div_\vec x\left\{\left(\frac{\mu|\vec u|^2}{2}\right)\vec
u+\left(\frac{Q|\vec v|^2}{2}\right)\vec v+\left(\frac{|\vec
D|^2+|\vec B|^2}{8\pi}\right)\vec v+\left(\left(\frac{1}{4\pi c}\vec
D\times \vec B\right)\cdot\vec v\right)\vec v\right\}
\\
+\frac{1}{4\pi}div_\vec x\left\{(\vec D\otimes \vec D+ \vec B\otimes
\vec B)\cdot \vec v-\frac{1}{2}\left(|\vec D|^2+|\vec
B|^2\right)\vec v-c \vec D\times \vec B-4\pi\Phi(\mu\vec u+Q\vec
v)\right\}\\-div_{\vec x}\left\{\left(\Phi+\frac{1}{2}\left|\vec
v\right|^2\right)\left(\frac{1}{4\pi c}\,\vec D\times \vec
B\right)\right\}-\frac{1}{4\pi G}div_{\vec
x}\left\{\Phi\frac{\partial}{\partial t}(\nabla_{\vec
x}\Phi)\right\}+\vec G\cdot\vec u=
\end{multline}
\end{comment}
%
%
%











Next given known the distribution of inertial mass density of some
continuum medium $\mu:=\mu(\vec x,t)$, the field of velocities of
this medium $\vec u:=\vec u(\vec x,t)$, the charge density
$\rho:=\rho(\vec x,t)$ and the current density $\vec j:=\vec j(\vec
x,t)$ consider a Lagrangian density $L$ for the unified
gravitational-electromagnetic field, defined by
\begin{multline}\label{vhfffngghkjgghDDint}
L\left(\vec A,\Psi,\vec v,\Phi,\vec p,\vec
x,t\right):=\frac{1}{8\pi}\left|-\nabla_{\vec
x}\Psi-\frac{1}{c}\frac{\partial\vec A}{\partial t}+\frac{1}{c}\vec
v\times curl_{\vec x}\vec A\right|^2-\frac{1}{8\pi}\left|curl_{\vec
x}\vec A\right|^2-\left(\rho\Psi-\frac{1}{c}\vec A\cdot\vec
j\right)\\+\frac{\mu}{2}\left|\vec u-\vec v\right|^2+
%\frac{k}{2}
\frac{1}{2}\left(d_{\vec x}\vec v+\left\{d_{\vec x}\vec
v\right\}^T\right)\cdot\left(d_{\vec x}\vec p+\left\{d_{\vec x}\vec
p\right\}^T\right)-2\left(div_{\vec x}\vec v\right)\left(div_{\vec
x}\vec p\right)\\+\frac{1}{4\pi G}\left(div_{\vec x}\vec
v\right)\left(\frac{\partial \Phi}{\partial t}+\vec
v\cdot\nabla_{\vec x} \Phi\right)+\frac{1}{4\pi
G}\Phi\left(div_{\vec x}\vec v\right)^2-\frac{\Phi}{16\pi
G}\left|d_{\vec x}\vec v+\left\{d_{\vec x}\vec
v\right\}^T\right|^2+\frac{1}{8\pi G}\left|\nabla_{\vec
x}\Phi\right|^2,
\end{multline}
where $\vec p$ is some vector field. Then, as before, we can show
that $L$ is invariant under the change of non-inertial cartesian
coordinate system given by \er{noninchgravortbstrjgghguittu2int},
provided that, under \er{noninchgravortbstrjgghguittu2int} we have
\begin{equation}\label{vyfgjhgjhvhgghintintint}
\begin{cases}
\vec p'=A(t)\cdot\vec p\\
\Phi'=\Phi
\\
\vec v'=A(t)\cdot \vec v+A'(t)\cdot\vec x+\frac{d\vec z}{dt}(t)\\
\vec A'=A(t)\cdot\vec A\\
\Psi'-\frac{1}{c}\vec A'\cdot\vec v'=\Psi-\frac{1}{c}\vec A\cdot\vec
v .
\end{cases}
\end{equation}
Then in section \ref{ghjfhgfdhd} we obtain that a configuration
$(\vec A,\Psi,\vec v,\Phi,\vec p)$ is a critical point of the
functional
\begin{equation}\label{btfffygtgyggyDDint}
J=\int_0^T\int_{\mathbb{R}^3}L\left(\vec A,\Psi,\vec v,\Phi,\vec
p,\vec x,t\right)d\vec x dt.
\end{equation}
if and only if it satisfies
\begin{equation}\label{guigjgjffghguygjyfDDint}
\begin{cases}
curl_{\vec x}\vec H=\frac{4\pi}{c}\vec j+\frac{\partial\vec
D}{\partial
t}\\
div_{\vec x}\vec D=4\pi\rho\\
curl_{\vec x}\vec E+\frac{1}{c}\frac{\partial\vec B}{\partial t}=0\\
div_{\vec x}\vec B=0\\
\vec E=\vec D-\frac{1}{c}\vec v\times\vec B\\
\vec H=\vec B+\frac{1}{c}\vec v\times\vec D\\
curl_{\vec x}\left(curl_{\vec x}\vec v\right)=0
\\
\frac{\partial}{\partial t}\left\{div_{\vec x}\vec v\right\}+\vec
v\cdot\nabla_{\vec x}\left(div_{\vec x}\vec
v\right)+\frac{1}{4}\left|d_{\vec x}\vec v+\left\{d_{\vec x}\vec
v\right\}^T\right|^2=-\Delta_{\vec x}\Phi\\
\left(\mu\vec u-\mu\vec v+\frac{1}{4\pi c}\vec D\times\vec
B\right)=curl_{\vec x}\left(curl_{\vec x}\vec p\right)-\frac{1}{4\pi
G}\left(\frac{\partial}{\partial t}\left(\nabla_{\vec
x}\Phi\right)-curl_{\vec x}\left(\vec v\times\nabla_{\vec
x}\Phi\right)+\left(\Delta_{\vec x}\Phi\right)\vec v\right),
\end{cases}
\end{equation}
where, consistently with \er{guigjgjffghPPNint} we denote:
\begin{equation}\label{guigjgjffghDDint}
\begin{cases}
\vec D:=-\nabla_{\vec x}\Psi-\frac{1}{c}\frac{\partial\vec
A}{\partial t}+\frac{1}{c}\vec
v\times curl_{\vec x}\vec A\\
\vec B:=curl_{\vec x}\vec A
\\
\vec E:=-\nabla_{\vec x}\Psi-\frac{1}{c}\frac{\partial\vec A}{\partial t}\\
\vec H:=curl_{\vec x}\vec A+\frac{1}{c}\vec
v\times\left(-\nabla_{\vec x}\Psi-\frac{1}{c}\frac{\partial\vec
A}{\partial t}+\frac{1}{c}\vec v\times curl_{\vec x}\vec A\right).
\end{cases}
\end{equation}
In particular, using continuum equation $\partial_t\mu+div_{\vec
x}\left(\mu\vec u\right)=0$ from the last equality in
\er{guigjgjffghguygjyfDDint} we deduce
\begin{equation*}
%\label{vhfffngghkjgghggtghjgfhjhjkghghyuiuukjhjhhg}
\frac{\partial}{\partial t}\left(\frac{1}{4\pi G}\Delta_{\vec
x}\Phi-\mu\right)+div_{\vec x}\left\{\left(\frac{1}{4\pi
G}\Delta_{\vec x}\Phi-\mu\right)\vec v\right\}=-div_{\vec
x}\left\{\frac{1}{4\pi c}\vec D\times\vec B\right\}.
\end{equation*}
Thus denoting $Q=\Delta_{\vec x}\Phi/4\pi G-\mu$ we deduce the
following system of equation for the gravitational-electromagnetic
field, invariant under the change of non-inertial cartesian
coordinate system:
\begin{equation}\label{guigjgjffghguygjyfghggDDint}
\begin{cases}
curl_{\vec x}\vec H=\frac{4\pi}{c}\vec j+\frac{\partial\vec
D}{\partial
t}\\
div_{\vec x}\vec D=4\pi\rho\\
curl_{\vec x}\vec E+\frac{1}{c}\frac{\partial\vec B}{\partial t}=0\\
div_{\vec x}\vec B=0\\
\vec E=\vec D-\frac{1}{c}\vec v\times\vec B\\
\vec H=\vec B+\frac{1}{c}\vec v\times\vec D\\
curl_{\vec x}\left(curl_{\vec x}\vec v\right)=0
\\
%\frac{\partial}{\partial t}\left\{div_{\vec x}\vec v\right\}+\vec v\cdot\nabla_{\vec x}\left(div_{\vec x}\vec v\right)+\frac{1}{4}\left|d_{\vec x}\vec v+\left\{d_{\vec x}\vec v\right\}^T\right|^2=-4\pi G(\mu+Q)
\frac{\partial}{\partial t}\left(div_{\vec x}\vec v\right)+div_{\vec
x}\left\{\left(div_{\vec x}\vec v\right)\vec
v\right\}+\frac{1}{4}\left|d_{\vec x}\vec v+\{d_{\vec x}\vec
v\}^T\right|^2-\left(div_{\vec x}\vec v\right)^2=  -4\pi G(\mu+Q)
\\
\frac{\partial Q}{\partial t}+div_{\vec x}\left(Q\vec
v\right)=-div_{\vec x}\left\{\frac{1}{4\pi c}\vec D\times\vec
B\right\},
\end{cases}
\end{equation}
which is consistent with \er{MaxVacFull1bjkgjhjhgjaaaint},
\er{MaxVacFull1ninshtrgravortghhghgjkgghklhjgkghghjjkjhjkkggjkhjkhjjhhfhjhklkhkhjjklzzzyyyNWBWHWPPNint}
and
\er{MaxVacFull1ninshtrgravortghhghgjkgghklhjgkghghjjkjhjkkggjkhjkhjjhhfhjhklkhkhjjklzzzyyyNWNWBWHWPPNint}.















































































































\subsection{Transformations of general scalar and vector fields under the
change of cartesian coordinate system} In order to get the above
results we established some trivial calculus consequences about the
behavior of scalar, vector and matrix fields, under the change of
cartesian coordinate system of the form
\er{noninchgravortbstrjgghguittu2int}. We combine them in the form
of Proposition after the following definition:
\begin{definition}\label{bggghghgjint}
Consider the change of some non-inertial cartesian coordinate system
$(*)$ to another cartesian coordinate system $(**)$ of the form
\er{noninchgravortbstrjgghguittu2int} where $A(t)\in SO(3)$ is a
rotation.
%and $\vec z(t)$ is given vector depending only on the time $t$.
\begin{itemize}
\item
We say that a general scalar field $\psi:=\psi(\vec
x,t):\R^3\times[0,+\infty)\to\R$ is a proper scalar field if, under
every change of coordinate system given by
\er{noninchgravortbstrjgghguittu2int}, this field transforms by the
law:
\begin{equation}\label{uguyytfddddint}
\psi'(\vec x',t')=\psi(\vec x,t).
\end{equation}

\item
We say that a general vector field $\vec f:=\vec f(\vec
x,t):\R^3\times[0,+\infty)\to\R^3$ is a proper vector field if,
under every change of coordinate system given by
\er{noninchgravortbstrjgghguittu2int}, this field transforms by the
law:
\begin{equation}\label{uguyytfddddgghjjgint}
\vec f'(\vec x',t')=A(t)\cdot\vec f(\vec x,t),
\end{equation}


\item
We say that a general vector field $\vec v:=\vec v(\vec
x,t):\R^3\times[0,+\infty)\to\R^3$ is a speed-like vector field if,
under every change of coordinate system given by
\er{noninchgravortbstrjgghguittu2int}, this field transforms by the
law:
\begin{equation}
\label{NoIn5redbstrint}\vec v'(\vec x',t')=A(t)\cdot \vec v(\vec
x,t)+
%A'(t)
\frac{d A}{dt}(t)\cdot\vec x+
%\frac{d\vec z}{dt}(t)
\vec w(t),
\end{equation}
where we set
\begin{equation}\label{buitguihjkint}
\vec w(t):=\frac{d\vec z}{dt}(t)\quad\quad\forall\,t.
\end{equation}

\item
We say that a general matrix valued field $T:=T(\vec
x,t):\R^3\times[0,+\infty)\to\R^{3\times 3}$ is a proper matrix
field if, under every change of coordinate system given by
\er{noninchgravortbstrjgghguittu2int}, this field transforms by the
law:
\begin{equation}\label{uguyytfddddgghjjghjjjint}
T'(\vec x',t')=A(t)\cdot T(\vec x,t)\cdot A^T(t)=A(t)\cdot T(\vec
x,t)\cdot \left\{A(t)\right\}^{-1}.
\end{equation}
\end{itemize}
\end{definition}
\begin{proposition}\label{yghgjtgyrtrtint}
If $\psi:\R^3\times[0,+\infty)\to\R$ is a proper scalar field, $\vec
f:\R^3\times[0,+\infty)\to\R^3$ and $\vec
g:\R^3\times[0,+\infty)\to\R^3$ are proper vector fields, $\vec
v:\R^3\times[0,+\infty)\to\R^3$ and $\vec
u:\R^3\times[0,+\infty)\to\R^3$ are speed-like vector fields and
$T:\R^3\times[0,+\infty)\to\R^{3\times 3}$ is a proper matrix field,
then:
\begin{itemize}
\item[{\bf(i)}] scalar fields defined in every coordinate system as $\,\vec f\cdot\vec g$, $div_{\vec x} \vec f$ and $div_{\vec x} \vec
v$ are proper scalar fields;

\item[{\bf(ii)}] vector fields defined in every coordinate system as $\nabla_{\vec x}\psi$, $div_{\vec x} T$, $curl_{\vec x}\vec
f$, $\vec f\times\vec g$,
%$\Delta_{\vec x}\vec v$
$div_{\vec x}\left(d_{\vec x} \vec v+\left\{d_{\vec x} \vec
v\right\}^T\right)$, $\nabla_{\vec x}\left(div_{\vec x}\vec
v\right)$, $\Delta_{\vec x}\vec v$, $curl_{\vec x}\left(curl_{\vec
x}\vec v\right)$ and $(\vec u-\vec v)$ are proper vector fields;

\item[{\bf(iii)}] matrix fields defined in every coordinate system as $\,d_{\vec x} \vec f$ and $\left(d_{\vec x} \vec v+\left\{d_{\vec x} \vec v\right\}^T\right)$ are proper
matrix fields;

\item[{\bf(iv)}] scalar fields $\xi:\R^3\times[0,+\infty)\to\R$ and $\zeta:\R^3\times[0,+\infty)\to\R$, defined
in every coordinate system by
\begin{equation}\label{vfyutuyfffhhgfhgfhgtgint}
\xi:=\frac{\partial\psi}{\partial t}+\vec v\cdot\nabla_{\vec x}\vec
\psi\quad \text{and}\quad \zeta:=\frac{\partial\psi}{\partial
t}+div_{\vec x}\left\{ \psi\vec v\right\}
\end{equation}
are proper scalar fields;

\item[{\bf(v)}] vector fields $\vec\Theta:\R^3\times[0,+\infty)\to\R^3$ and $\vec\Xi:\R^3\times[0,+\infty)\to\R^3$, defined
in every coordinate system by
\begin{equation}\label{vfyutuyfffhhgfhgfhint}
\vec\Theta:=\frac{\partial \vec f}{\partial t}- curl_{\vec
x}\left(\vec v\times \vec f\right)+\left({div}_{\vec x}\vec
f\right)\vec v\quad\text{and}\quad \vec\Xi:=\frac{\partial \vec
f}{\partial t}- \vec v\times curl_{\vec x}\vec f+\nabla_{\vec
x}\left(\vec v\cdot\vec f\right),
\end{equation}
are proper vector fields and
\begin{equation}
%\begin{multline}
\label{vhfffngghhjghhgjlkhjhkPPPint} \vec\Xi
%\frac{\partial\vec f}{\partial t}-\vec v\times curl_{\vec x}\vec f+\nabla_{\vec x}\left(\vec f\cdot\vec v\right)
=
%\left(\frac{\partial\vec f}{\partial t}-curl_{\vec x}\left(\vec v\times\vec f\right)+\left(div_{\vec x}\vec f\right)\vec v\right)
\vec\Theta-\left(div_{\vec x}\vec v\right)\vec f+ \left(d_{\vec
x}\vec v+\left\{d_{\vec x}\vec v\right\}^T\right)\cdot\vec f.
%\end{multline}
\end{equation}
\end{itemize}
\end{proposition}



































\subsection{Covariant formulation of the physical laws in the
four-dimensional non-relativistic space-time} In Section \ref{CVFRM}
we present the covariant (tensor) formulations of the Maxwell
Equations and the Lagrangian density of the electromagnetic field
and the covariant form of the Lagrangian of motion of charged
particles in the outer gravitational and electromagnetic fields.


\subsubsection{Four-vectors, four-covectors and tensors in the
four-dimensional non-relativistic space-time} First of all we would
like to remind the definitions of the vectors, covectors and
covariant and contravariant tensors of second order in
$\mathbb{R}^4$.
\begin{definition}
Given $\mathcal{S}$, that is a certain subgroup of the group of all
smooth non-degenerate invertible transformations from $\mathbb{R}^4$
onto $\mathbb{R}^4$ having the form
\begin{equation}\label{fgjfjhgghyuyyuint}
\begin{cases}
x'^0=f^{(0)}(x^0,x^1,x^2,x^3),\\
x'^1=f^{(1)}(x^0,x^1,x^2,x^3),\\
x'^2=f^{(2)}(x^0,x^1,x^2,x^3),\\
x'^3=f^{(3)}(x^0,x^1,x^2,x^3),
\end{cases}
\end{equation}
we say that a one-component field $a:=a(x^0,x^1,x^2,x^3)$ is a
scalar field on the group $\mathcal{S}$, if under the coordinate
transformation in the group $\mathcal{S}$ of the form
\er{fgjfjhgghyuyyuint} this field transforms as:
\begin{equation}\label{fgjfjhgghhgjgihhiint}
a'=a.
\end{equation}
Next we say that a four-component field $(a^0,a^1,a^2,a^3)$ is a
four-vector field on the group $\mathcal{S}$, if under the
coordinate transformation in the group $\mathcal{S}$ of the form
\er{fgjfjhgghyuyyuint} every of four components of this field
transforms as:
\begin{equation}\label{fgjfjhgghhgjgint}
a'^j=\sum_{k=0}^{3}\frac{\partial f^{(j)}}{\partial
x^k}a^k\quad\quad\forall j=0,1,2,3.
\end{equation}
Next we say that a four-component field $(a_0,a_1,a_2,a_3)$ is a
four-covector field on the group $\mathcal{S}$, if under the
coordinate transformation in the group $\mathcal{S}$ of the form
\er{fgjfjhgghyuyyuint} every of four components of this field
transforms as:
\begin{equation}\label{fgjfjhgghhgjghjhjint}
a_j=\sum_{k=0}^{3}\frac{\partial f^{(k)}}{\partial
x^j}a'_k\quad\quad\forall j=0,1,2,3.
\end{equation}
Furthermore, we say that a $16$-component field
$\{a_{mn}\}_{m,n=0,1,2,3}$ is a two times covariant tensor field on
the group $\mathcal{S}$, if under the coordinate transformation in
the group $\mathcal{S}$ of the form \er{fgjfjhgghyuyyuint} every of
$16$ components of this field transforms as:
\begin{equation}\label{fgjfjhgghhgjghjhjkkkkjjkint}
a_{mn}=\sum_{j=0}^{3}\sum_{k=0}^{3}\frac{\partial f^{(k)}}{\partial
x^m}\frac{\partial f^{(j)}}{\partial x^n}a'_{kj}\quad\quad\forall\,
m,n=0,1,2,3.
\end{equation}
Next we say that a $16$-component field $\{a^{mn}\}_{m,n=0,1,2,3}$
is a two times contravariant tensor field on the group
$\mathcal{S}$, if under the coordinate transformation in the group
$\mathcal{S}$ of the form \er{fgjfjhgghyuyyuint} every of $16$
components of this field transforms as:
\begin{equation}\label{fgjfjhgghhgjghjhjkkkkggghint}
a'^{mn}=\sum_{j=0}^{3}\sum_{k=0}^{3}\frac{\partial f^{(m)}}{\partial
x^k}\frac{\partial f^{(n)}}{\partial x^j}a^{kj}\quad\quad\forall\,
m,n=0,1,2,3.
\end{equation}
%
%
%
\begin{comment}
Then it is well known that for every two four-vectors
$(a^0,a^1,a^2,a^3)$ and $(b^0,b^1,b^2,b^3)$ on $\mathcal{S}$, the
$16$-component field $\{c^{mn}\}_{m,n=0,1,2,3}$, defined in every
coordinate system by
\begin{equation}\label{fgjfjhgghhgjghjhjkkkkgjghghjljlint}
c^{mn}:=a^mb^n\quad\quad\forall\, m,n=0,1,2,3,
\end{equation}
is a two times contravariant tensor on $\mathcal{S}$. Moreover, for
every two four-covectors $(a_0,a_1,a_2,a_3)$ and $(b_0,b_1,b_2,b_3)$
on $\mathcal{S}$, the $16$-component field
$\{c_{mn}\}_{m,n=0,1,2,3}$, defined in every coordinate system by
\begin{equation}\label{fgjfjhgghhgjghjhjkkkkgjghghkkkjint}
c_{mn}:=a_mb_n\quad\quad\forall\, m,n=0,1,2,3,
\end{equation}
is a two times covariant tensor on $\mathcal{S}$. It is also well
known that if $\{a^{mn}\}_{m,n=0,1,2,3}$ is a two times
contravariant tensor field on the group $\mathcal{S}$ and if a
$16$-component field $\{b_{mn}\}_{m,n=0,1,2,3}$ satisfies
\begin{equation}\label{fgjfjhgghhgjghjhjkkkkgjghghuiiiuint}
\sum_{k=0}^{3}a^{mk}b_{kn}=\begin{cases}
1\quad\text{if}\quad m=n\\
0\quad\text{if}\quad m\neq n
\end{cases}\quad\quad\forall\, m,n=0,1,2,3,
\end{equation}
then $\{b_{mn}\}_{m,n=0,1,2,3}$ is a two times covariant tensor on
$\mathcal{S}$. Next it is well known that, given a four-covector
$(a_0,a_1,a_2,a_3)$ a four-vector $(b^0,b^1,b^2,b^3)$, a two times
covariant tensor $\{c_{mn}\}_{m,n=0,1,2,3}$ and a two times
contravariant tensor $\{d^{mn}\}_{m,n=0,1,2,3}$ on the group
$\mathcal{S}$, the quantities
\begin{equation}\label{fgjfjhgghhgjghjhjkkkkgjghghuiiiukljkint}
\sum_{k=0}^{3}a_kb^k\quad\text{and}\quad
\sum_{m=0}^{3}\sum_{n=0}^{3}c_{mn}d^{mn}
\end{equation}
are scalars on $\mathcal{S}$, the four-component fields defined by
\begin{equation}\label{fgjfjhgghhgjghjhjkkkkgjghghuiiiulkkjint}
\Big\{\sum_{k=0}^{3}d^{mk}a_{k}\Big\}_{m=0,1,2,3}\quad\text{and}\quad
\Big\{\sum_{k=0}^{3}c_{mk}b^{k}\Big\}_{m=0,1,2,3}
\end{equation}
are four-vector and four-covector on $\mathcal{S}$ and moreover,
$16$-component fields $\{\hat c^{mn}\}_{m,n=0,1,2,3}$ and $\{\hat
d_{mn}\}_{m,n=0,1,2,3}$ defined by
\begin{equation}\label{fgjfjhgghhgjghjhjkkkkgjghghuiiiulkkjlkklint}
\hat
c^{mn}:=\sum_{k=0}^{3}\sum_{j=0}^{3}d^{mj}d^{nk}c_{jk}\quad\text{and}\quad
%\Big\{\sum_{k=0}^{3}c_{mk}b^{k}\Big\}_{m=0,1,2,3}
\hat
d_{mn}:=\sum_{j=0}^{3}\sum_{k=0}^{3}c_{mj}c_{nk}d^{jk}\quad\quad\forall\,
m,n=0,1,2,3,
\end{equation}
are two times contravariant and two times covariant tensors on
$\mathcal{S}$. Next, it is also well known that given a two times
covariant tensor $\{c_{mn}\}_{m,n=0,1,2,3}$ and a two times
contravariant tensor $\{d^{mn}\}_{m,n=0,1,2,3}$ on the group
$\mathcal{S}$ the $16$-component fields $\{c_{nm}\}_{m,n=0,1,2,3}$
and $\{d^{nm}\}_{m,n=0,1,2,3}$ are also two times covariant and two
times contravariant tensors on $\mathcal{S}$. Finally, it is well
known that, if $a:=a(x^0,x^1,x^2,x^3)$ is a scalar field on the
group $\mathcal{S}$, then the four-component field
$(w_0,w_1,w_2,w_3)$ defined by:
\begin{equation}\label{fgjfjhgghhgjghjhjkkkkgjghghuiiiulkkjlkklplikklint}
w_j:=\frac{\partial a}{\partial x^j}\quad\quad\forall\,j=0,1,2,3,
\end{equation}
is a \underline{four-covector} field on the group $\mathcal{S}$.
\end{comment}
%
%
%
\end{definition}
Next consider the four-dimensional space-time $\mathbb{R}^4$, such
that for every point in space $\vec x=(x_1,x_2,x_3)\in\mathbb{R}^3$
and every instant of time $t$ we correspond the point
$(x^0,x^1,x^2,x^3)\in\mathbb{R}^4$ that has the form:
\begin{equation}\label{fgjfjhgghint}
(x^0,x^1,x^2,x^3):=(ct,x_1,x_2,x_3)=\left(ct,\vec x\right),
\end{equation}
where $c$ is the universal constant in Maxwell equations for vacuum.
In this space we denote by $\mathcal{S}_0$, the subgroup of the
group of smooth non-degenerate invertible mappings, containing
transformations of the form
\begin{equation}\label{noninchgravortbstrjgghguittu2intrrrZZygjyghhjint}
\begin{cases}
x'^0=x^0
\\
x'^j=\sum\limits_{k=1}^{3}A_{jk}\left(\frac{x^0}{c}\right)\,x_k+z_j\left(\frac{x^0}{c}\right)\quad\forall
j=1,2,3,
\end{cases}
\end{equation}
where $$\left\{A_{jk}(t)\right\}_{j,k=1,2,3}=A(t):\mathbb{R}\to
SO(3)$$ is a rotation, smoothly dependent on $t$ and
$$\left(z_1(t),z_2(t),z_3(t)\right)=\vec
z(t):\mathbb{R}\to\mathbb{R}^3$$ also smoothly dependent on $t$.
Then in the terms of time $t$ and three-dimensional space we rewrite
\er{noninchgravortbstrjgghguittu2intrrrZZygjyghhjint} as
\er{noninchgravortbstrjgghguittu2int}, i.e.:
\begin{equation}\label{noninchgravortbstrjgghguittu2intrrrZZygjygint}
\begin{cases}
\vec x'=A(t)\cdot\vec x+\vec z(t),\\
t'=t,
\end{cases}
\end{equation}
where $A(t)\in SO(3)$ is a rotation. I.e. the group $\mathcal{S}_0$
represents all transformations of cartesian non-inertial coordinate
systems in the non-relativistic space-time. It can be easily checked
by trivial calculations that $\mathcal{S}_0$ is indeed a group, i.e.
for every two transformations $f,g\in\mathcal{S}_0$ the composition
$g\circ f$ and the inverse transformation $f^{(-1)}$ are also
contained in $\mathcal{S}_0$, thats mean that they also have a form
of \er{noninchgravortbstrjgghguittu2intrrrZZygjyghhjint}. Next
assume that a four-covector $(a_0,a_1,a_2,a_3)$ and a four-vector
$(b^0,b^1,b^2,b^3)$ on the group $\mathcal{S}_0$ are given. Then, by
inserting \er{noninchgravortbstrjgghguittu2intrrrZZygjyghhjint} into
\er{fgjfjhgghhgjgint} and \er{fgjfjhgghhgjghjhjint} in Section
\ref{CVFRM}
%
%
%
\begin{comment}
we obtain the following laws of transformations under the acting in
the group $\mathcal{S}_0$:
\begin{equation}\label{fgjfjhgghhgjghjhjijhojint}
\begin{cases}
a_0=a'_0+\sum_{k=1}^{3}\frac{1}{c}\left(\sum\limits_{j=1}^{3}\frac{dA_{kj}}{dt}\left(\frac{x^0}{c}\right)\,x_j+\frac{d
z_k}{dt}\left(\frac{x^0}{c}\right)\right)\,a'_k
\\
a_j=\sum_{k=1}^{3}A_{kj}\left(\frac{x^0}{c}\right)\,a'_k\quad\quad\forall
j=1,2,3,
\end{cases}
\end{equation}
and
\begin{equation}\label{fgjfjhgghhgjgiuouoiuuint}
\begin{cases}
b'^0=b^0
\\
b'^j=\frac{1}{c}\left(\sum\limits_{k=1}^{3}\frac{dA_{jk}}{dt}\left(\frac{x^0}{c}\right)\,x_k+\frac{d
z_j}{dt}\left(\frac{x^0}{c}\right)\right)\,b^0+\sum_{k=1}^{3}A_{jk}\left(\frac{x^0}{c}\right)\,b^k\quad\quad\forall
j=1,2,3.
\end{cases}
\end{equation}
In particular, since $A(t)\in SO(3)$ and thus
\begin{equation}\label{fgjfjhgghhgjgiuouoiuu1int}
\sum_{j=1}^{3}A_{mj}(t)A_{nj}(t)=\begin{cases}1\quad\text{if}\quad m=n\\
0\quad\text{if}\quad m\neq n
\end{cases}
\quad\quad\quad\forall m,n=1,2,3,
\end{equation}
by \er{fgjfjhgghhgjghjhjijhojint} we deduce:
\begin{equation}\label{fgjfjhgghhgjghjhjijhojpiiiint}
\begin{cases}
a'_0=a_0-\sum_{k=1}^{3}\frac{1}{c}\left(\sum\limits_{j=1}^{3}\frac{dA_{kj}}{dt}\left(\frac{x^0}{c}\right)\,x_j+\frac{d
z_k}{dt}\left(\frac{x^0}{c}\right)\right)\left(\sum_{j=1}^{3}A_{kj}\left(\frac{x^0}{c}\right)\,a_j\right)
\\
a'_k=\sum_{j=1}^{3}A_{kj}\left(\frac{x^0}{c}\right)\,a_j\quad\quad\forall
k=1,2,3.
\end{cases}
\end{equation}
So, by \er{fgjfjhgghhgjghjhjijhojpiiiint} and
\er{fgjfjhgghhgjgiuouoiuuint}
\end{comment}
%
%
%
we obtained the following laws of transformation of four-covectors
and four-vectors on the group $\mathcal{S}_0$, i.e. under the change
of non-inertial cartesian coordinate systems:
\begin{equation}\label{fgjfjhgghhgjghjhjijhojpiiihjhjint}
\begin{cases}
a'_0=a_0-\sum_{k=1}^{3}\frac{1}{c}\left(\sum\limits_{j=1}^{3}\frac{dA_{kj}}{dt}\left(\frac{x^0}{c}\right)\,x_j+\frac{d
z_k}{dt}\left(\frac{x^0}{c}\right)\right)\left(\sum_{j=1}^{3}A_{kj}\left(\frac{x^0}{c}\right)\,a_j\right)
\\
a'_k=\sum_{j=1}^{3}A_{kj}\left(\frac{x^0}{c}\right)\,a_j\quad\quad\forall
k=1,2,3,
\end{cases}
\end{equation}
and
\begin{equation}\label{fgjfjhgghhgjgiuouoiuujkjkjkint}
\begin{cases}
b'^0=b^0
\\
b'^j=\frac{1}{c}\left(\sum\limits_{k=1}^{3}\frac{dA_{jk}}{dt}\left(\frac{x^0}{c}\right)\,x_k+\frac{d
z_j}{dt}\left(\frac{x^0}{c}\right)\right)\,b^0+\sum_{k=1}^{3}A_{jk}\left(\frac{x^0}{c}\right)\,b^k\quad\quad\forall
j=1,2,3.
\end{cases}
\end{equation}
%
%
%
\begin{comment}
\begin{equation}\label{fgjfjhgghhgjghjhjijhojihjhjint}
\begin{cases}
a'_0=\frac{\partial f^{(0)}}{\partial
x^0}a_0+\sum_{k=1}^{3}\frac{\partial f^{(k)}}{\partial x^0}a_k
\\
a'_j=\frac{\partial f^{(0)}}{\partial
x^j}a_0+\sum_{k=1}^{3}\frac{\partial f^{(k)}}{\partial
x^j}a_k\quad\quad\forall j=1,2,3,
\end{cases}
\end{equation}
and
\begin{equation}\label{fgjfjhgghhgjgiuouoiuuhjhjint}
\begin{cases}
b'^0=\frac{\partial f^{(0)}}{\partial
x^0}b^0+\sum_{k=1}^{3}\frac{\partial f^{(0)}}{\partial x^k}b^k
\\
b'^j=\frac{\partial f^{(j)}}{\partial
x^0}b^0+\sum_{k=1}^{3}\frac{\partial f^{(j)}}{\partial
x^k}b^k\quad\quad\forall j=1,2,3.
\end{cases}
\end{equation}
\end{comment}
%
%
%
Therefore, if we denote the four-vector $(b^0,b^1,b^2,b^3)$ and the
four-covector $(a_0,a_1,a_2,a_3)$ on the group $\mathcal{S}_0$ as:
\begin{equation}\label{fgjfjhgghhgjghjhjijhojihjhjjijhjjjint}
\begin{cases}
(b^0,b^1,b^2,b^3)=\left(\sigma,\frac{1}{c}\vec
b\right)\quad\text{where}\quad
\sigma:=b^0\;\;\text{and}\;\;\vec b:=c(b^1,b^2,b^3)\in\mathbb{R}^3,\\
(a_0,a_1,a_2,a_3)=(\psi,-\vec a)\quad\text{where}\quad
\psi:=a_0\;\;\text{and}\;\;\vec a:=-(a_1,a_2,a_3)\in\mathbb{R}^3,
\end{cases}
\end{equation}
then by \er{fgjfjhgghhgjghjhjijhojpiiihjhjint} and
\er{fgjfjhgghhgjgiuouoiuujkjkjkint} in the terms of time $t$ and
three-dimensional space $\vec x$, we obtain the following laws of
transformations of $\sigma$, $\vec b$, $\psi$ and $\vec a$ under the
change of non-inertial cartesian coordinate system:
\begin{equation}\label{fgjfjhgghhgjgiuouoiuujkjkjkojkoiuint}
\begin{cases}
\sigma'=\sigma
\\
\vec b'=A\left(t\right)\cdot\vec
b+\left(\frac{dA}{dt}\left(t\right)\cdot\vec x+\frac{d \vec
z}{dt}\left(t\right)\right)\sigma,
\end{cases}
\end{equation}
and
\begin{equation}\label{fgjfjhgghhgjghjhjijhojpiiihjhjuioujint}
\begin{cases}
\psi'=\psi+\frac{1}{c}\left(\frac{dA}{dt}\left(t\right)\cdot\vec
x+\frac{d \vec z}{dt}\left(t\right)\right)\cdot\left(A(t)\cdot\vec
a\right)
\\
\vec a'=A(t)\cdot\vec a.
\end{cases}
\end{equation}
In particular, if $\sigma:=b^0$ is the first coordinate of an
arbitrary four-vector $(b^0,b^1,b^2,b^3)$ on the group
$\mathcal{S}_0$, then $\sigma$ is a proper scalar field in the
frames of Definition \ref{bggghghgjint}. Moreover, if $\vec
a:=-(a_1,a_2,a_3)$, where $a_1,a_2,a_3$ are the last three
coordinates of an arbitrary four-covector $(a_0,a_1,a_2,a_3)$ on the
group $\mathcal{S}_0$, then $\vec a$ is a proper vector field in the
frames of Definition \ref{bggghghgjint}.

Next, since by Definition \ref{bggghghgjint} every three-dimensional
speed-like vector field, $\vec u$ transforms under the change of
non-inertial cartesian coordinate system as:
\begin{equation}
\label{NoIn3redPPN'int}\vec u'=A(t)\cdot \vec
u+\frac{dA}{dt}\left(t\right)\cdot\vec x+\frac{d \vec
z}{dt}\left(t\right),
\end{equation}
by comparing \er{NoIn3redPPN'int} with
\er{fgjfjhgghhgjgiuouoiuujkjkjkojkoiuint} we deduce that for every
speed-like vector field $\vec u$ the four-component field
$(u^0,u^1,u^2,u^3)$ defined by
\begin{equation}\label{fgjfjhgghhgjghjhjijhojihjhjjijhjjjjjuiiint}
(u^0,u^1,u^2,u^3):=\left(1,\frac{1}{c}\vec
u\right)\quad\text{where}\quad
u^0=1\;\;\text{and}\;\;(u^1,u^2,u^3)=\frac{1}{c}\vec
u\in\mathbb{R}^3,
\end{equation}
is a four-vector field on the group $\mathcal{S}_0$. We call such
four-vectors by the name vectors of type $1$. In particular, if
$\vec u$ is the velocity field, then the quantity defined by
\er{fgjfjhgghhgjghjhjijhojihjhjjijhjjjjjuiiint} is a a four-vector
field on the group $\mathcal{S}_0$ that we call the four-dimensional
speed.
%
%
%
\begin{comment}
Regarding the field of velocity $\vec u$ we also can give a
different argumentation that the four-component field
$(u^0,u^1,u^2,u^3)$ defined by
\er{fgjfjhgghhgjghjhjijhojihjhjjijhjjjjjuiiint} is a four-vector
field on the group $\mathcal{S}_0$: indeed it is well known from
Tensor Analysys that if $\left(x^0(s),x^1(s),x^2(s),x^3(s)\right)$
is a curve in $\mathbb{R}^4$, parameterized by some scalar parameter
$s$, then the four-component field
$\left(\frac{dx^0}{ds}(s),\frac{dx^1}{ds}(s),\frac{dx^2}{ds}(s),\frac{dx^3}{ds}(s)\right)$
is a four-vector field on an arbitrary group $\mathcal{S}$ and, in
particular, on the group $\mathcal{S}_0$.
\end{comment}
%
%
%
Thus, in particular, if $\vec
r(t)=\left(r_1(t),r_2(t),r_3(t)\right)$ is a three-dimensional
trajectory of the motion of some particle, parameterized by the
global time $t$, then if we consider a curve
$\frac{1}{c}\left(ct,r_1(t),r_2(t),r_3(t)\right)$ in $\mathbb{R}^4$,
parameterized by the global time $t$, then the four-component field:
\begin{equation}\label{fgjfjhgghhgjghjhjijhojihjhjjijhjjjjjuiijhjhhint}
\left(1,\frac{1}{c}\frac{d\vec r}{dt}(t)\right):=
\left(1,\frac{1}{c}\frac{dr_1}{dt}(t),\frac{1}{c}\frac{dr_2}{dt}(t),\frac{1}{c}\frac{dr_3}{dt}(t)\right)
\end{equation}
is a four-vector field on the group $\mathcal{S}_0$.


 Similarly, if $\vec v$ is the vectorial gravitational
potential, then since $\vec v$ is a speed-like vector field, the
four-component field $(v^0,v^1,v^2,v^3)$ defined by
\begin{equation}\label{fgjfjhgghhgjghjhjijhojihjhjjijhjjjjjuiijjjkint}
(v^0,v^1,v^2,v^3):=\left(1,\frac{1}{c}\vec
v\right)\quad\text{where}\quad
v^0=1\;\;\text{and}\;\;(v^1,v^2,v^3)=\frac{1}{c}\vec v,
\end{equation}
is also a four-vector field on the group $\mathcal{S}_0$ that we
call the four-dimensional gravitational potential.

Moreover, by \er{fgjfjhgghhgjgiuouoiuujkjkjkojkoiuint},
%
%
%
\begin{comment}
for every speed-like vector field $\vec u$ and every proper scalar
field $\sigma$ the four-component field $(b^0,b^1,b^2,b^3)$ defined
by
\begin{equation}\label{fgjfjhgghhgjghjhjijhojihjhjjijhjjjjjuiijjjklint}
(b^0,b^1,b^2,b^3):=\left(\sigma,\frac{\sigma}{c}\vec
u\right)\quad\text{where}\quad
b^0=\sigma\;\;\text{and}\;\;(b^1,b^2,b^3)=\frac{\sigma}{c}\vec u,
\end{equation}
is also a four-vector field on the group $\mathcal{S}_0$. In
particular,
\end{comment}
%
%
%
if we consider the field of four-dimensional moment of a particle
$(p^0,p^1,p^2,p^3)$ defined by
\begin{equation}\label{fgjfjhgghhgjghjhjijhojihjhjjijhjjjjjuiijjjklihhint}
(p^0,p^1,p^2,p^3):=\left(m,\frac{1}{c}(m\vec
u)\right)\quad\text{where}\quad
p^0=m\;\;\text{and}\;\;(p^1,p^2,p^3)=\frac{1}{c}(m\vec u),
\end{equation}
where $m$ is the mass of the particle and $\vec u$ is the velocity
of the particle, then $(p^0,p^1,p^2,p^3)$ is also a four-vector on
the group $\mathcal{S}_0$. Moreover, by comparing
\er{yuythfgfyftydtydtydtyddyyyhhddhhhredPPN111hgghjgintint}
%\er{NoIn4redPPNint} and \er{NoIn6redPPNint}
with
\er{fgjfjhgghhgjgiuouoiuujkjkjkojkoiuint} we deduce that if we
consider the field of four-dimensional electric current
$(j^0,j^1,j^2,j^3)$ defined by
\begin{equation}\label{fgjfjhgghhgjghjhjijhojihjhjjijhjjjjjuiijjjklihhojjjoint}
(j^0,j^1,j^2,j^3):=\left(\rho,\frac{1}{c}\,\vec
j\right)\quad\text{where}\quad
j^0=\rho\;\;\text{and}\;\;(j^1,j^2,j^3)=\frac{1}{c}\,\vec j,
\end{equation}
where $\rho$ is the electric charge density and $\vec j$ is the
electric current density, then $(j^0,j^1,j^2,j^3)$ is also a
four-vector on the group $\mathcal{S}_0$.

On the other hand, for every proper three-dimensional vector field
$\vec G$ that satisfies due to Definition \ref{bggghghgjint}:
\begin{equation}
\label{NoIn1redPPN'int}\vec G'=A(t)\cdot\vec G,
\end{equation}
by comparing \er{NoIn1redPPN'int} with
\er{fgjfjhgghhgjgiuouoiuujkjkjkojkoiuint} we deduce that the
four-component field $(G^0,G^1,G^2,G^3)$ defined by
\begin{equation}\label{fgjfjhgghhgjghjhjijhojihjhjjijhjjjjjuiiklkllo;int}
(G^0,G^1,G^2,G^3):=\left(0,\vec G\right)\quad\text{where}\quad
G^0=0\;\;\text{and}\;\;(G^1,G^2,G^3)=\vec G,
\end{equation}
is also a four-vector field on the group $\mathcal{S}_0$.  We call
such four-vectors by the name vectors of type $0$.

Next, since by
\er{vhfffngghhjghhgPPNghghghutghfflklhjkjhjhjjgjkghhjint} the scalar
electromagnetic potential $\Psi$ and the vector electromagnetic
potential $\vec A$, under the change of non-inertial cartesian
coordinate system transform as:
\begin{equation}\label{vhfffngghhjghhgPPNghghghutghfflklhjkjhjhjjgjkghhjhhjhjkhjghlkkint}
\begin{cases}
\Psi'=
%\Psi+\frac{1}{c}\vec A\cdot\left(A^T(t)\cdot\frac{dA}{dt}(t)\cdot\vec x+A^T(t)\cdot\frac{d\vec z}{dt}(t)\right)
\Psi+\frac{1}{c}\left(\frac{dA}{dt}(t)\cdot\vec x+\frac{d\vec
z}{dt}(t)\right)\cdot\left(A(t)\cdot\vec A\right)
\\
\vec A'=A(t)\cdot \vec A,
\end{cases}
\end{equation}
by comparing
\er{vhfffngghhjghhgPPNghghghutghfflklhjkjhjhjjgjkghhjhhjhjkhjghlkkint}
with \er{fgjfjhgghhgjghjhjijhojpiiihjhjuioujint} we deduce that the
four-component field $(A_0,A_1,A_2,A_3)$ defined as
\begin{equation}\label{fgjfjhgghhgjghjhjijhojihjhjjijhjjjljljpkint}
(A_0,A_1,A_2,A_3)=(\Psi,-\vec A)\quad\text{where}\quad
A_0=\Psi\;\;\text{and}\;\;(A_1,A_2,A_3)=-\vec A,
\end{equation}
is a four-\underline{covector} field on the group $\mathcal{S}_0$.
We call this  four-covector field by the name four dimensional
electromagnetic potential. Next, since $(A_0,A_1,A_2,A_3)$ is a
four-covector field on the group $\mathcal{S}_0$, then it is well
known from the tensor analysis that the $16$-component field
$\{F_{ij}\}_{0\leq i,j\leq 3}$ defined in every non-inertial
cartesian coordinate system by
\begin{equation}\label{huohuioy89gjjhjffffff3478zzrrZZZhjhhjhhjjhhffGGhjjhint}
F_{ij}:=\frac{\partial A_j}{\partial x^i}-\frac{\partial
A_i}{\partial x^j}\quad\quad\forall\, i,j=0,1,2,3\,,
\end{equation}
is an antisymmetric two times covariant tensor field on the group
$\mathcal{S}_0$, which we call the covariant tensor of the
electromagnetic field. In particular, by inserting
\er{fgjfjhgghhgjghjhjijhojihjhjjijhjjjljljpkint} and
\er{fgjfjhgghint} into
\er{huohuioy89gjjhjffffff3478zzrrZZZhjhhjhhjjhhffGGhjjhint}
%
%
%
\begin{comment} we deduce:
\begin{equation}\label{huohuioy89gjjhjffffff3478zzrrZZZhjhhjhhjjhhffGGhjjhiuuiint}
\begin{cases}
F_{00}=0\\ F_{0j}=-F_{j0}=-\frac{1}{c}\frac{\partial(-A_j)}{\partial
t}-\frac{\partial \Psi}{\partial
x^j}\quad\quad\forall\, j=1,2,3\\
F_{jj}=0\quad\quad\forall\, j=1,2,3
\\
F_{ij}=-F_{ji}=\frac{\partial (-A_i)}{\partial x^j}-\frac{\partial
(-A_j)}{\partial x^i}\quad\quad\forall\, i\neq j=1,2,3\,,
\end{cases}
\end{equation}
Thus if as in
\er{MaxVacFull1bjkgjhjhgjgjgkjfhjfdghghligioiuittrPPNint} we denote:
\end{comment}
%
%
%
and denoting:
\begin{equation}\label{MaxVacFull1bjkgjhjhgjgjgkjfhjfdghghligioiuittrPPNkkkint}
\begin{cases}
(B_1,B_2,B_3)=\vec B:= curl_{\vec x} \vec A,\\
(E_1,E_2,E_3)=\vec E:=-\nabla_{\vec
x}\Psi-\frac{1}{c}\frac{\partial\vec A}{\partial t},
%,\\ div_{\vec x}\vec A\equiv 0,
\end{cases}
\end{equation}
%then denoting
%$\vec E:=(E_1,E_2,E_3)$ and $\vec B:=(B_1,B_2,B_3)$,
%by \er{MaxVacFull1bjkgjhjhgjgjgkjfhjfdghghligioiuittrPPNkkkint}
%% and \er{fgjfjhgghhgjghjhjijhojihjhjjijhjjjljljpk}
%we rewrite
%\er{huohuioy89gjjhjffffff3478zzrrZZZhjhhjhhjjhhffGGhjjhiuuiint} as:
we deduce:
\begin{equation}\label{huohuioy89gjjhjffffff3478zzrrZZZhjhhjhhjjhhffGGhjjhiuuijkjjkint}
\begin{cases}
F_{00}=0\\ F_{0j}=-F_{j0}=E_j\quad\quad\forall\, j=1,2,3\\
F_{jj}=0\quad\quad\forall\, j=1,2,3
\\
F_{12}=-F_{21}=-B_3
%\frac{\partial (-A_1)}{\partial x^2}-\frac{\partial(-A_2)}{\partial x^1}
\\
F_{13}=-F_{31}=B_2
%\frac{\partial (-A_1)}{\partial x^3}-\frac{\partial(-A_3)}{\partial x^1}
\\
F_{23}=-F_{32}=-B_1\,.
%\frac{\partial (-A_2)}{\partial x^3}-\frac{\partial (-A_3)}{\partial x^2}
\end{cases}
\end{equation}


Next assume that $T:=\left\{T_{ij}\right\}_{i,j=1,2,3}\in
\R^{3\times 3}$ is a $9$-component proper matrix valued field,
which, being a proper matrix field, by Definition \ref{bggghghgjint}
satisfies:
\begin{equation}\label{uguyytfddddgghjjghjjjihohjjkint}
T'=A(t)\cdot T\cdot A^T(t)=A(t)\cdot T\cdot
\left\{A(t)\right\}^{-1}.
\end{equation}
Next consider a $16$-component field $\{\mathcal{T}^{ij}\}_{0\leq
i,j\leq 3}$ defined in every non-inertial cartesian coordinate
system by
\begin{equation}\label{huohuioy89gjjhjffffff3478zzrrZZZhjhhjhhjjhhffGGhjjhuiiuihhjint}
\begin{cases}
\mathcal{T}^{00}=0
\\
\mathcal{T}^{0j}=\mathcal{T}^{j0}=0\quad\quad\forall\, j=1,2,3
\\
\mathcal{T}^{ij}:=T_{ij}\quad\quad\forall\, i,j=1,2,3\,,
\end{cases}
\end{equation}
Then, by inserting
\er{noninchgravortbstrjgghguittu2intrrrZZygjyghhjint} and
\er{uguyytfddddgghjjghjjjihohjjkint} into
\er{fgjfjhgghhgjghjhjkkkkggghint} in Section \ref{CVFRM} we  prove
that the field $\{\mathcal{T}^{ij}\}_{0\leq i,j\leq 3}$ defined by
\er{huohuioy89gjjhjffffff3478zzrrZZZhjhhjhhjjhhffGGhjjhuiiuihhjint}
is a two times \underline{contravariant} tensor field on the group
$\mathcal{S}_0$.
%
%
%
\begin{comment}
Indeed, by \er{fgjfjhgghhgjghjhjkkkkggghint} for every two times
contravariant tensor field $\{a^{ij}\}_{0\leq i,j\leq 3}$ we have
\begin{multline}\label{fgjfjhgghhgjghjhjkkkkggghkjjkint}
a'^{mn}=\frac{\partial f^{(m)}}{\partial x^0}\frac{\partial
f^{(n)}}{\partial x^0}a^{00}+\sum_{k=1}^{3}\frac{\partial
f^{(m)}}{\partial x^k}\frac{\partial f^{(n)}}{\partial
x^0}a^{k0}+\sum_{j=1}^{3}\frac{\partial f^{(m)}}{\partial
x^0}\frac{\partial f^{(n)}}{\partial
x^j}a^{0j}\\+\sum_{j=1}^{3}\sum_{k=1}^{3}\frac{\partial
f^{(m)}}{\partial x^k}\frac{\partial f^{(n)}}{\partial
x^j}a^{kj}\quad\quad\forall\, m,n=0,1,2,3.
\end{multline}
Then, since by \er{noninchgravortbstrjgghguittu2intrrrZZygjyghhjint}
we have $\frac{\partial f^{(0)}}{\partial x^0}=1$, $\frac{\partial
f^{(0)}}{\partial x^k}=0$ $\forall k=1,2,3$ and $\frac{\partial
f^{(m)}}{\partial x^k}=A_{mk}\left(\frac{x^0}{c}\right)$ $\forall
k,m=1,2,3$, in the case where $a^{00}=0$ and $a^{0j}=a^{j0}=0$
$\forall j=1,2,3$  we rewrite \er{fgjfjhgghhgjghjhjkkkkggghkjjkint}
as:
\begin{equation}\label{fgjfjhgghhgjghjhjkkkkggghjjkjhgjint}
\begin{cases}
a'^{00}=0\\
a'^{j0}=a'^{0j}=0\quad\quad\forall\, j=1,2,3,
\\
a'^{mn}=\sum_{j=1}^{3}\sum_{k=1}^{3}A_{mk}\left(\frac{x^0}{c}\right)A_{nj}\left(\frac{x^0}{c}\right)a^{kj}\quad\quad\forall\,
m,n=1,2,3.
\end{cases}
\end{equation}
that is compatible with
\er{huohuioy89gjjhjffffff3478zzrrZZZhjhhjhhjjhhffGGhjjhuiiuihhjint}
and \er{uguyytfddddgghjjghjjjihohjjkint}.
\end{comment}
%
%
%



In particular, if we consider the $9$-component matrix field $I$
that defined in every cartesian coordinate system as
$I:=\left\{\delta_{ij}\right\}_{1,j=1,2,3}\in \R^{3\times 3}$, where
\begin{equation}\label{huohuioy89gjjhjffffff3478zzrrZZZhjhhjhhjjhhffGGhjjhuiiuihhjkklpklint}
\delta_{ij}:=
\begin{cases}
1\quad\text{if}\quad i=j
\\
0\quad\text{if}\quad i\neq j,
\end{cases}
\end{equation}
which is a proper matrix field, since
\begin{equation}\label{uguyytfddddgghjjghjjjihohjjkkkookint}
I=A(t)\cdot I\cdot \left\{A(t)\right\}^{-1},
\end{equation}
then the $16$-component field $\{{\Theta}^{ij}\}_{0\leq i,j\leq 3}$
defined in every non-inertial cartesian coordinate system by
\begin{equation}\label{huohuioy89gjjhjffffff3478zzrrZZZhjhhjhhjjhhffGGhjjhuiiuihhjkkoioiint}
\begin{cases}
{\Theta}^{00}=0
\\
{\Theta}^{0j}={\Theta}^{j0}=0\quad\quad\forall\, j=1,2,3
\\
{\Theta}^{ij}:=\delta_{ij}\quad\quad\forall\, i,j=1,2,3
\end{cases}
\end{equation}
is a two times contravariant tensor field on the group
$\mathcal{S}_0$ and moreover, this tensor is symmetric. We call
$\{{\Theta}^{ij}\}_{0\leq i,j\leq 3}$ the contravariant tensor of
the three-dimensional geometry.

Finally the scalar field $\tau:=\tau(x^0,x^1,x^2,x^3)$, defined in
every cartesian coordinate system as
\begin{equation}\label{uguyytfddddgghkjhhjuuiuiint}
\tau:=\frac{x^0}{c}=t,
\end{equation}
is a scalar on the group $\mathcal{S}_0$. Here $t$ is the global
non-relativistic time. Moreover,
%by \er{fgjfjhgghhgjghjhjkkkkgjghghuiiiulkkjlkklplikklint} and
by \er{uguyytfddddgghkjhhjuuiuiint}, the four-component field
$(v_0,v_1,v_2,v_3)$ defined as a gradient of the global time by:
\begin{equation}\label{fgjfjhgghhgjghjhjkkkkgjghghuiiiulkkjlkklplikklkjljghhgghkint}
v_0:=c\,\frac{\partial \tau}{\partial
x^0}(x^0,x^1,x^2,x^3)=1\quad\text{and}\quad v_j:=c\,\frac{\partial
\tau}{\partial x^j}(x^0,x^1,x^2,x^3)=0\quad\quad\forall\,j=1,2,3,
\end{equation}
is a \underline{four-covector} field on the group $\mathcal{S}_0$.
\subsubsection{Pseudo-metric tensors of the four-dimensional
space-time} Consider $\{g^{ij}\}_{0\leq i,j\leq 3}$ to be a two
times contravariant tensor field on the group $\mathcal{S}_0$,
defined by
\begin{equation}\label{hoyuiouigyfghgjh3478zzrrZZffGGjkkjojjint}
g^{ij}:=v^iv^j-{\Theta}^{ij}\quad\quad\forall\,i,j=0,1,2,3,
\end{equation}
where $\{{\Theta}^{ij}\}_{0\leq i,j\leq 3}$ is the contravariant
tensor of the three-dimensional geometry, defined by
\er{huohuioy89gjjhjffffff3478zzrrZZZhjhhjhhjjhhffGGhjjhuiiuihhjkkoioiint}
and being a two times contravariant tensor, and $(v^0,v^1,v^2,v^3)$
is the four-dimensional gravitational potential, defined by
\er{fgjfjhgghhgjghjhjijhojihjhjjijhjjjjjuiijjjkint} and being a
four-vector. Then,
%by \er{fgjfjhgghhgjghjhjkkkkgjghghjljlint}
in Section \ref{CVFRM} we obtain that $\{g^{ij}\}_{0\leq i,j\leq 3}$
is indeed a two times contravariant tensor field on the group
$\mathcal{S}_0$ and moreover, this tensor is symmetric. Moreover, by
\er{huohuioy89gjjhjffffff3478zzrrZZZhjhhjhhjjhhffGGhjjhuiiuihhjkkoioiint}
and \er{fgjfjhgghhgjghjhjijhojihjhjjijhjjjjjuiijjjkint} we have:
\begin{equation}\label{hoyuiouigyfghgjh3478zzrrZZffGGjkkjint}
\begin{cases}
g^{00}=1\\
g^{ij}=-\delta_{ij}+\frac{v_iv_j}{c^2}\quad\forall 1\leq i,j\leq 3\\
g^{0j}=g^{j0}=\frac{v_j}{c}\quad\forall 1\leq j\leq 3.
\end{cases}
\end{equation}
We call the tensor $\{g^{ij}\}_{0\leq i,j\leq 3}$ the contravariant
pseudo-metric tensor of the four-dimensional space-time. Next
consider a $16$-component field $\{g_{ij}\}_{0\leq i,j\leq 3}$
defined by
\begin{equation}\label{hoyuiouigyfg3478zzrrZZffGGhhjhjint}
\begin{cases}
g_{00}=1-\frac{|\vec v|^2}{c^2}\\
g_{ij}=-\delta_{ij}\quad\forall 1\leq i,j\leq 3\\
g_{0j}=g_{j0}=\frac{v_j}{c}\quad\forall 1\leq j\leq 3.
\end{cases}
\end{equation}
where $\vec v=(v_1,v_2,v_3)$ is the three-dimensional vectorial
gravitational potential. Then in Section \ref{CVFRM}
%
%
%
\begin{comment}
\begin{equation*}
\sum_{k=0}^{3}g_{0k}g^{k0}=g_{00}g^{00}+\sum_{k=1}^{3}g_{0k}g^{k0}=
1-\frac{|\vec v|^2}{c^2}+\frac{|\vec v|^2}{c^2}=1,
\end{equation*}
\begin{equation*}
\sum_{k=0}^{3}g_{ik}g^{kj}=g_{i0}g^{0j}+\sum_{k=1}^{3}g_{ik}g^{kj}=
\frac{v_iv_j}{c^2}+\delta_{ij}-\frac{v_iv_j}{c^2}=\delta_{ij}\quad\forall
1\leq i,j\leq 3,
\end{equation*}
and
\begin{equation*}
\sum_{k=0}^{3}g_{ik}g^{k0}=g_{i0}g^{00}+\sum_{k=1}^{3}g_{ik}g^{k0}=\frac{v_i}{c}-\frac{v_i}{c}=0
\quad\forall 1\leq i\leq 3,
\end{equation*}
\begin{multline*}
\sum_{k=0}^{3}g_{0k}g^{kj}=g_{00}g^{0j}+\sum_{k=1}^{3}g_{0k}g^{kj}=
\left(1-\frac{|\vec v|^2}{c^2}\right)\frac{v_j}{c}
-\sum_{k=1}^{3}\frac{v_k}{c}\left(\delta_{kj}-\frac{v_kv_j}{c^2}\right)=0
\quad\forall 1\leq j\leq 3,
\end{multline*}
where $\{g^{ij}\}_{0\leq i,j\leq 3}$ is the contravariant
pseudo-metric tensor of the four-dimensional space-time, defined by
\er{hoyuiouigyfghgjh3478zzrrZZffGGjkkjint}. So,
\end{comment}
%
%
%
we deduce:
\begin{equation}\label{fgjfjhgghhgjghjhjkkkkgjghghuiiiuujhjhint}
\sum_{k=0}^{3}g^{ik}g_{kj}=\begin{cases}
1\quad\text{if}\quad i=j\\
0\quad\text{if}\quad i\neq j
\end{cases}\quad\quad\forall\, i,j=0,1,2,3.
\end{equation}
Therefore,
%by comparing \er{fgjfjhgghhgjghjhjkkkkgjghghuiiiuujhjhint} and \er{fgjfjhgghhgjghjhjkkkkgjghghuiiiuint}
we obtain that $\{g_{ij}\}_{i,j=0,1,2,3}$ is a two times covariant
tensor on the group $\mathcal{S}_0$, and moreover, this tensor is
symmetric. We call the tensor $\{g_{ij}\}_{0\leq i,j\leq 3}$
covariant pseudo-metric tensor of the four-dimensional space-time.
Using \er{fgjfjhgghhgjghjhjkkkkgjghghuiiiuujhjhint} we also obtain
that the pseudo-metric tensors $\{g_{ij}\}_{i,j=0,1,2,3}$ and
$\{g^{ij}\}_{0\leq i,j\leq 3}$ are non-degenerate.
% and they are inverse of each other.
Moreover, it can be easily calculated that if
we consider the $4\times 4$-matrix:
\begin{equation}\label{fgjfjhgghhgjghjhjkkkkgjghghuiiiuujhjhjkljint}
G=\{g_{ij}\}_{0\leq i,j\leq 3},
\end{equation}
then
\begin{equation}\label{fgjfjhgghhgjghjhjkkkkgjghghuiiiuujhjh1int}
\text{det}\,G=-1.
\end{equation}

Thus, with the covariant and contravariant pseudo-metric tensors we
can lower and lift indexes of arbitrary tensors. In particular given
a four-covector $(a_0,a_1,a_2,a_3)$ and a four-vector
$(b^0,b^1,b^2,b^3)$ on the group $\mathcal{S}_0$ we can define the
corresponding lifted four-vector $(a^0,a^1,a^2,a^3)$ and the
corresponded lowered four-covector $(b_0,b_1,b_2,b_3)$ by
\begin{equation}\label{fgjfjhgghhgjghjhjkkkkgjghghuiiiulkkjKKint}
(a^0,a^1,a^2,a^3):=\Big\{\sum_{k=0}^{3}g^{mk}a_{k}\Big\}_{m=0,1,2,3}\quad\text{and}\quad
(b_0,b_1,b_2,b_3):=\Big\{\sum_{k=0}^{3}g_{mk}b^{k}\Big\}_{m=0,1,2,3}
\end{equation}
Then by \er{hoyuiouigyfghgjh3478zzrrZZffGGjkkjint},
\er{hoyuiouigyfg3478zzrrZZffGGhhjhjint} and
\er{fgjfjhgghhgjghjhjkkkkgjghghuiiiulkkjKKint} we have:
%
%
%
\begin{comment}
\begin{equation}\label{fgjfjhgghhgjghjhjkkkkgjghghuiiiulkkjKKyuyyuint}
a^0=a_{0}+\sum_{k=1}^{3}\frac{1}{c}v_k a_{k}\quad\text{and}\quad
a^m=-a_{m}+\frac{1}{c}\bigg(a_{0}+\sum_{k=1}^{3}\frac{1}{c}v_k
a_{k}\bigg)v_m\quad\quad\forall m=1,2,3,
\end{equation}
and
\begin{multline}\label{fgjfjhgghhgjghjhjkkkkgjghghuiiiulkkjKKyuyyuuuuiint}
b_0=\left(1-\frac{|\vec
v|^2}{c^2}\right)b^{0}+\sum_{k=1}^{3}\frac{1}{c}v_k
b^{k}=b^0-\sum_{k=1}^{3}\frac{1}{c}v_k\left(-
b^{k}+\frac{1}{c}b^0v_k\right)\\ \quad\text{and}\quad
b_m=-b^{m}+\frac{1}{c} b^{0}v_m\quad\quad\forall m=1,2,3.
\end{multline}
We also can rewrite
\er{fgjfjhgghhgjghjhjkkkkgjghghuiiiulkkjKKyuyyuint} and
\er{fgjfjhgghhgjghjhjkkkkgjghghuiiiulkkjKKyuyyuuuuiint} as:
\end{comment}
%
%
%
\begin{equation}\label{fgjfjhgghhgjghjhjkkkkgjghghuiiiulkkjKKyuyyuiouiint}
a^0=a_{0}+\sum_{k=1}^{3}\frac{1}{c}v_k a_{k}\quad\text{and}\quad
a^m=-a_{m}+\frac{1}{c}a^0v_m\quad\quad\forall m=1,2,3,
\end{equation}
and
\begin{equation}\label{fgjfjhgghhgjghjhjkkkkgjghghuiiiulkkjKKyuyyuuuuiiuiiint}
b_0=b^0-\sum_{k=1}^{3}\frac{1}{c}v_kb_k \quad\text{and}\quad
b_m=-b^{m}+\frac{1}{c} b^{0}v_m\quad\quad\forall m=1,2,3.
\end{equation}
In particular, we have:
%
%
%
\begin{comment}
if we consider the scalar field $\Lambda$ on the group
$\mathcal{S}_0$ defined by:
\begin{equation}\label{fgjfjhgghhgjghjhjkkkkgjghghuiiiulkkjKKyuyyu0ioint}
\Lambda:=b^0a_0+\sum_{k=1}^{3}b^ka_k
\end{equation}
then by inserting
\er{fgjfjhgghhgjghjhjkkkkgjghghuiiiulkkjKKyuyyuiouiint} and
\er{fgjfjhgghhgjghjhjkkkkgjghghuiiiulkkjKKyuyyuuuuiiuiiint} into
\er{fgjfjhgghhgjghjhjkkkkgjghghuiiiulkkjKKyuyyu0ioint} we deduce:
\begin{equation}\label{fgjfjhgghhgjghjhjkkkkgjghghuiiiulkkjKKyuyyu0ioioioint}
\Lambda=b^0\left(a^0-\sum_{k=1}^{3}\frac{1}{c}v_ka_k\right)+\sum_{k=1}^{3}\left(-b_{k}+\frac{1}{c}b^0v_k\right)a_k=b^0a^0-\sum_{k=1}^{3}b_{k}a_k.
\end{equation}
So,
\end{comment}
%
%
%
\begin{equation}\label{fgjfjhgghhgjghjhjkkkkgjghghuiiiulkkjKKyuyyu0ioioiogghghghint}
%\Lambda=
b^0a_0+\sum_{k=1}^{3}b^ka_k=b^0a^0-\sum_{k=1}^{3}b_{k}a_k.
\end{equation}

Next, if for every speed-like vector field $\vec u$ we consider the
four-vector field $(u^0,u^1,u^2,u^3)$ defined by
\er{fgjfjhgghhgjghjhjijhojihjhjjijhjjjjjuiiint} as:
\begin{equation}\label{fgjfjhgghhgjghjhjijhojihjhjjijhjjjjjuiikkjint}
(u^0,u^1,u^2,u^3):=\left(1,\frac{1}{c}\vec
u\right)\quad\text{where}\quad
u^0=1\;\;\text{and}\;\;(u^1,u^2,u^3)=\frac{1}{c}\vec
u\in\mathbb{R}^3,
\end{equation}
then, by \er{fgjfjhgghhgjghjhjkkkkgjghghuiiiulkkjKKyuyyuuuuiiuiiint}
the corresponding lowered four-covector field $(u_0,u_1,u_2,u_3)$
satisfies:
\begin{multline}\label{fgjfjhgghhgjghjhjijhojihjhjjijhjjjjjuiikkjjnjint}
(u_0,u_1,u_2,u_3):=\left(1+\frac{1}{c^2}\left(\vec u-\vec
v\right)\cdot\vec v,-\frac{1}{c}\left(\vec u-\vec
v\right)\right)\quad\text{where}\quad\\
u_0=1+\frac{1}{c^2}\left(\vec u-\vec v\right)\cdot\vec
v\;\;\text{and}\;\;(u_1,u_2,u_3)=-\frac{1}{c}\left(\vec u-\vec
v\right)\in\mathbb{R}^3.
\end{multline}
Moreover, in the case where $(u^0,u^1,u^2,u^3)$  is a
four-dimensional speed, we call the corresponding lowered
four-covector field $(u_0,u_1,u_2,u_3)$ by the name four-dimensional
cospeed. In particular, if we consider the four-dimensional
gravitational potential $(v^0,v^1,v^2,v^3)$ defined by
\er{fgjfjhgghhgjghjhjijhojihjhjjijhjjjjjuiijjjkint}:
\begin{equation}\label{fgjfjhgghhgjghjhjijhojihjhjjijhjjjjjuiijjjkhjhjhjint}
(v^0,v^1,v^2,v^3):=\left(1,\frac{1}{c}\vec
v\right)\quad\text{where}\quad
v^0=1\;\;\text{and}\;\;(v^1,v^2,v^3)=\frac{1}{c}\vec v,
\end{equation}
then by \er{fgjfjhgghhgjghjhjijhojihjhjjijhjjjjjuiikkjjnjint} we
obtain that the corresponding lowered four-covector field
$(v_0,v_1,v_2,v_3)$, that we call the four-covector of gravitational
potential, satisfies:
\begin{equation}\label{fgjfjhgghhgjghjhjijhojihjhjjijhjjjjjuiikkjjnj1int}
(v_0,v_1,v_2,v_3):=\left(1,0\right)\quad\text{where}\quad
v_0=1\;\;\text{and}\;\;(v_1,v_2,v_3)=0:=(0,0,0).
\end{equation}
Note that the four-covector of gravitational potential, defined by
\er{fgjfjhgghhgjghjhjijhojihjhjjijhjjjjjuiikkjjnj1int} coincides
with the four-covector defined by
\er{fgjfjhgghhgjghjhjkkkkgjghghuiiiulkkjlkklplikklkjljghhgghkint} as
the gradient of the scalar of global time. Moreover, by
\er{fgjfjhgghhgjghjhjijhojihjhjjijhjjjjjuiijjjkhjhjhjint} and
\er{fgjfjhgghhgjghjhjijhojihjhjjijhjjjjjuiikkjjnj1int} we clearly
have:
\begin{equation}\label{fgjfjhgghhgjghjhjijhojihjhjjijhjjjjjuiijjjkhjhjhjuiiuuuyuint}
c^2\left(\sum_{j=0}^{n}\sum_{k=0}^{n}\,g^{jk}\frac{\partial\tau}{\partial
x^j}\frac{\partial\tau}{\partial
x^k}\right)=\sum_{j=0}^{n}\sum_{k=0}^{n}g^{jk}v_jv_k=\sum_{j=0}^{n}\sum_{k=0}^{n}g_{jk}v^jv^k=\sum_{j=0}^{n}v^jv_j=1,
\end{equation}
where $\tau$ is the scalar of the global time on the group
$\mathcal{S}_0$, defined by \er{uguyytfddddgghkjhhjuuiuiint}.


%
%
%
\begin{comment}
More generally, if for every speed-like vector field $\vec u$ and
every proper scalar field $\sigma$  we consider the four-vector
field $(b^0,b^1,b^2,b^3)$ on the group $\mathcal{S}_0$ defined by
\er{fgjfjhgghhgjghjhjijhojihjhjjijhjjjjjuiijjjklint} as:
\begin{equation}\label{fgjfjhgghhgjghjhjijhojihjhjjijhjjjjjuiijjjklhhhint}
(b^0,b^1,b^2,b^3):=\left(\sigma,\frac{\sigma}{c}\vec
u\right)\quad\text{where}\quad
b^0=\sigma\;\;\text{and}\;\;(b^1,b^2,b^3)=\frac{\sigma}{c}\vec u,
\end{equation}
then by \er{fgjfjhgghhgjghjhjkkkkgjghghuiiiulkkjKKyuyyuuuuiiuiiint}
the corresponding lowered four-covector field $(b_0,b_1,b_2,b_3)$
satisfies:
\begin{multline}\label{fgjfjhgghhgjghjhjijhojihjhjjijhjjjjjuiikkjjnjjbjint}
(b_0,b_1,b_2,b_3):=\left(\sigma\left(1+\frac{1}{c^2}\left(\vec
u-\vec v\right)\cdot\vec v\right),-\frac{\sigma}{c}\left(\vec u-\vec
v\right)\right)\quad\text{where}\quad\\
b_0=\sigma\left(1+\frac{1}{c^2}\left(\vec u-\vec v\right)\cdot\vec
v\right)\;\;\text{and}\;\;(b_1,b_2,b_3)=-\frac{\sigma}{c}\left(\vec
u-\vec v\right).
\end{multline}
In particular,
\end{comment}
%
%
%
Moreover, if we consider the field of four-vector of the moment of a
particle $(p^0,p^1,p^2,p^3)$ defined by
\er{fgjfjhgghhgjghjhjijhojihjhjjijhjjjjjuiijjjklihhint} as
\begin{equation}\label{fgjfjhgghhgjghjhjijhojihjhjjijhjjjjjuiijjjklihhjjjint}
(p^0,p^1,p^2,p^3):=\left(m,\frac{1}{c}(m\vec
u)\right)\quad\text{where}\quad
p^0=m\;\;\text{and}\;\;(p^1,p^2,p^3)=\frac{1}{c}(m\vec u),
\end{equation}
where $m$ is the mass of the particle and $\vec u$ is the velocity
of the particle, then the corresponding lowered four-covector field
$(p_0,p_1,p_2,p_3)$, which we call the four-covector of momentum,
satisfies:
\begin{multline}\label{fgjfjhgghhgjghjhjijhojihjhjjijhjjjjjuiikkjjnjjbjhjhint}
(p_0,p_1,p_2,p_3):=\left(m\left(1+\frac{1}{c^2}\left(\vec u-\vec
v\right)\cdot\vec v\right),-\frac{m}{c}\left(\vec u-\vec
v\right)\right)\quad\text{where}\quad\\
p_0=m\left(1+\frac{1}{c^2}\left(\vec u-\vec v\right)\cdot\vec
v\right)\;\;\text{and}\;\;(p_1,p_2,p_3)=-\frac{m}{c}\left(\vec
u-\vec v\right).
\end{multline}
In particular, by
\er{fgjfjhgghhgjghjhjkkkkgjghghuiiiulkkjKKyuyyu0ioioiogghghghint} we
have:
%the scalar field $J_0$ defined by
%\begin{equation}\label{fgjfjhgghhgjghjhjkkkkgjghghuiiiulkkjKKyuyyu0ioioiogghghghgghghhgint}
%J_0:=-\frac{c^2}{2m}\left(p^0p_0+\sum_{k=1}^{3}p^kp_k\right),
%\end{equation}
%by \er{fgjfjhgghhgjghjhjkkkkgjghghuiiiulkkjKKyuyyu0ioioiogghghghint},
%\er{fgjfjhgghhgjghjhjijhojihjhjjijhjjjjjuiijjjklihhjjjint} and
%\er{fgjfjhgghhgjghjhjijhojihjhjjijhjjjjjuiikkjjnjjbjhjhint} satisfies:
\begin{equation}\label{fgjfjhgghhgjghjhjkkkkgjghghuiiiulkkjKKyuyyu0ioioiogghghghgghghint}
-\frac{c^2}{2m}\left(p^0p_0+\sum_{k=1}^{3}p^kp_k\right)=\frac{mc^2}{2}\left(\frac{1}{c^2}\left|\vec
u-\vec v\right|^2-1\right)=\frac{m}{2}\left|\vec u-\vec
v\right|^2-\frac{mc^2}{2}.
\end{equation}
Moreover, if we consider the  four-dimensional electric current
$(j^0,j^1,j^2,j^3)$ defined by
\er{fgjfjhgghhgjghjhjijhojihjhjjijhjjjjjuiijjjklihhojjjoint} as
\begin{equation}\label{fgjfjhgghhgjghjhjijhojihjhjjijhjjjjjuiijjjklihhojjjokjkjint}
(j^0,j^1,j^2,j^3):=\left(\rho,\frac{1}{c}\,\vec
j\right)\quad\text{where}\quad
j^0=\rho\;\;\text{and}\;\;(j^1,j^2,j^3)=\frac{1}{c}\,\vec j,
\end{equation}
where $\rho$ is the electric charge density and $\vec j$ is the
electric current density, then the corresponding lowered
four-covector field $(j_0,j_1,j_2,j_3)$, which we call the
four-covector of current, satisfies:
\begin{multline}\label{fgjfjhgghhgjghjhjijhojihjhjjijhjjjjjuiikkjjnjjbjhjhiyuyugint}
(j_0,j_1,j_2,j_3):=\left(\rho+\frac{1}{c^2}\left(\vec j-\rho\vec
v\right)\cdot\vec v,-\frac{1}{c}\left(\vec j-\rho\vec
v\right)\right)\quad\text{where}\quad\\
j_0=\rho+\frac{1}{c^2}\left(\vec j-\rho\vec v\right)\cdot\vec
v\;\;\text{and}\;\;(j_1,j_2,j_3)=-\frac{1}{c}\left(\vec j-\rho\vec
v\right).
\end{multline}
Finally, if $\Psi$ is the scalar electromagnetic potential and $\vec
A$ is the vector electromagnetic potential and we consider the
four-\underline{covector} field of four dimensional electromagnetic
potential $(A_0,A_1,A_2,A_3)$, defined by
\er{fgjfjhgghhgjghjhjijhojihjhjjijhjjjljljpkint} as:
\begin{equation}\label{fgjfjhgghhgjghjhjijhojihjhjjijhjjjljljpkikhjint}
(A_0,A_1,A_2,A_3)=(\Psi,-\vec A)\quad\text{where}\quad
A_0=\Psi\;\;\text{and}\;\;(A_1,A_2,A_3)=-\vec A,
\end{equation}
then by inserting
\er{fgjfjhgghhgjghjhjijhojihjhjjijhjjjljljpkikhjint} into
\er{fgjfjhgghhgjghjhjkkkkgjghghuiiiulkkjKKyuyyuiouiint} we deduce
that the corresponding lifted four-vector field $(A^0,A^1,A^2,A^3)$,
which we call the four-vector of electromagnetic potential,
satisfies:
\begin{equation}\label{fgjfjhgghhgjghjhjkkkkgjghghuiiiulkkjKKyuyyuiouigjguuuiuiint}
A^0=\Psi-\frac{1}{c}\vec v\cdot\vec
A\quad\text{and}\quad(A^1,A^2,A^3)=\vec
A+\frac{1}{c}\left(\Psi-\frac{1}{c}\vec v\cdot\vec A\right)\vec v.
\end{equation}
On the other hand, the proper scalar electromagnetic potential
$\Psi_0$ was defined by
\er{vhfffngghhjghhgPPNghghghutghffugghjhjkjjklint} as:
\begin{equation}\label{vhfffngghhjghhgPPNghghghutghffugghjhjkjjklhjghint}
\Psi_0:=\Psi-\frac{1}{c}\vec A\cdot\vec v.
\end{equation}
Thus we rewrite
\er{fgjfjhgghhgjghjhjkkkkgjghghuiiiulkkjKKyuyyuiouigjguuuiuiint} as:
\begin{equation}\label{fgjfjhgghhgjghjhjkkkkgjghghuiiiulkkjKKyuyyuiouigjguuuiui1int}
A^0=\Psi_0\quad\text{and}\quad(A^1,A^2,A^3)=\vec
A+\frac{1}{c}\Psi_0\vec v.
\end{equation}




Next given a two times covariant tensor $\{c_{mn}\}_{m,n=0,1,2,3}$
%and a two times contravariant tensor $\{d^{mn}\}_{m,n=0,1,2,3}$
on the group $\mathcal{S}_0$
%by \er{fgjfjhgghhgjghjhjkkkkgjghghuiiiulkkjlkklint}
we consider two times contravariant lifted tensor
%and two times covariant tensors
on $\mathcal{S}_0$: $\{c^{mn}\}_{m,n=0,1,2,3}$
%and $\{d_{mn}\}_{m,n=0,1,2,3}$
defined by:
\begin{equation}\label{fgjfjhgghhgjghjhjkkkkgjghghuiiiulkkjlkklKKint}
c^{mn}:=\sum_{k=0}^{3}\sum_{j=0}^{3}g^{mj}g^{nk}c_{jk}
%\quad\text{and}\quadd_{mn}:=\sum_{j=0}^{3}\sum_{k=0}^{3}g_{mj}g_{nk}d^{jk}
\quad\quad\forall\, m,n=0,1,2,3.
\end{equation}
%
%
%
\begin{comment}
We rewrite \er{fgjfjhgghhgjghjhjkkkkgjghghuiiiulkkjlkklKKint} as:
\begin{multline}\label{khjhhkfgjfjhgghhgjghjhjkkkkgjghghuiiiulkkjlkklKKgfgint}
c^{mn}=g^{m0}g^{n0}c_{00}+\sum_{k=1}^{3}g^{m0}g^{nk}c_{0k}+\sum_{j=1}^{3}g^{mj}g^{n0}c_{j0}+\sum_{k=1}^{3}\sum_{j=1}^{3}g^{mj}g^{nk}c_{jk}
%\quad\text{and}\quad\quad\\ d_{mn}=g_{m0}g_{n0}d^{00}+\sum_{k=1}^{3}g_{m0}g_{nk}d^{0k}+\sum_{j=1}^{3}g_{mj}g_{n0}d^{j0}+\sum_{k=1}^{3}\sum_{j=1}^{3}g_{mj}g_{nk}d^{jk}
\quad\forall\, m,n=0,1,2,3.
\end{multline}
In particular, by inserting
\er{hoyuiouigyfghgjh3478zzrrZZffGGjkkjint} and
\er{hoyuiouigyfg3478zzrrZZffGGhhjhjint} into
\er{khjhhkfgjfjhgghhgjghjhjkkkkgjghghuiiiulkkjlkklKKgfgint} we
deduce:
\begin{equation}\label{khjhhkfgjfjhgghhgjghjhjkkkkgjghghuiiiulkkjlkklKKgfgjhjjghgjhint}
\begin{cases}
c^{00}=c_{00}+\sum_{k=1}^{3}\frac{v_k}{c}c_{0k}+\sum_{j=1}^{3}\frac{v_j}{c}c_{j0}+\sum_{k=1}^{3}\sum_{j=1}^{3}\frac{v_j}{c}\frac{v_k}{c}c_{jk}
\\
c^{m0}=
%\frac{v_m}{c}c_{00}+\sum_{k=1}^{3}\frac{v_m}{c}\frac{v_k}{c}c_{0k}+\sum_{j=1}^{3}\frac{v_m}{c}\frac{v_j}{c}c_{j0}+\sum_{k=1}^{3}\sum_{j=1}^{3}\frac{v_k}{c}\frac{v_m}{c}\frac{v_j}{c}c_{jk}
\frac{v_m}{c}c^{00}-c_{m0}-\sum_{k=1}^{3}\frac{v_k}{c}c_{mk}
\quad\forall\, m=1,2,3,
\\
c^{0n}=\frac{v_n}{c}c^{00}
%\frac{v_n}{c}c_{00}+\sum_{k=1}^{3}\frac{v_n}{c}\frac{v_k}{c}c_{0k}+\sum_{j=1}^{3}\frac{v_j}{c}\frac{v_n}{c}c_{j0}+\sum_{k=1}^{3}\sum_{j=1}^{3}\frac{v_j}{c}\frac{v_n}{c}\frac{v_k}{c}c_{jk}
-c_{0n} -\sum_{j=1}^{3}\frac{v_j}{c}c_{jn} \quad\forall\, n=1,2,3,
\\
c^{mn}=\frac{v_m}{c}\frac{v_n}{c}c^{00}
%\frac{v_m}{c}\frac{v_n}{c}c_{00}+\sum_{k=1}^{3}\frac{v_m}{c}\frac{v_n}{c}\frac{v_k}{c}c_{0k}+\sum_{j=1}^{3}\frac{v_n}{c}\frac{v_m}{c}\frac{v_j}{c}c_{j0}+\sum_{k=1}^{3}\sum_{j=1}^{3}\frac{v_m}{c}\frac{v_n}{c}\frac{v_j}{c}\frac{v_k}{c}c_{jk}
%-\sum_{k=1}^{3}\sum_{j=1}^{3}\frac{v_m}{c}\frac{v_n}{c}\frac{v_j}{c}\frac{v_k}{c}c_{jk}
-\sum_{k=1}^{3}\frac{v_n}{c}\frac{v_k}{c}c_{mk}-\sum_{j=1}^{3}\frac{v_m}{c}\frac{v_j}{c}c_{jn}
-\frac{v_m}{c}c_{0n}-\frac{v_n}{c}c_{m0}+c_{mn}\quad\forall\,
m,n=1,2,3.
\end{cases}
\end{equation}
We rewrite
\er{khjhhkfgjfjhgghhgjghjhjkkkkgjghghuiiiulkkjlkklKKgfgjhjjghgjhint}
as:
\begin{equation}\label{khjhhkfgjfjhgghhgjghjhjkkkkgjghghuiiiulkkjlkklKKgfgjhjjghgjh1int}
\begin{cases}
c^{00}=c_{00}+\sum_{k=1}^{3}\frac{v_k}{c}c_{0k}+\sum_{j=1}^{3}\frac{v_j}{c}c_{j0}+\sum_{k=1}^{3}\sum_{j=1}^{3}\frac{v_j}{c}\frac{v_k}{c}c_{jk}
\\
c^{m0}=
%\frac{v_m}{c}c_{00}+\sum_{k=1}^{3}\frac{v_m}{c}\frac{v_k}{c}c_{0k}+\sum_{j=1}^{3}\frac{v_m}{c}\frac{v_j}{c}c_{j0}+\sum_{k=1}^{3}\sum_{j=1}^{3}\frac{v_k}{c}\frac{v_m}{c}\frac{v_j}{c}c_{jk}
\frac{v_m}{c}c^{00}-c_{m0}-\sum_{k=1}^{3}\frac{v_k}{c}c_{mk}
\quad\forall\, m=1,2,3,
\\
c^{0n}=\frac{v_n}{c}c^{00}
%\frac{v_n}{c}c_{00}+\sum_{k=1}^{3}\frac{v_n}{c}\frac{v_k}{c}c_{0k}+\sum_{j=1}^{3}\frac{v_j}{c}\frac{v_n}{c}c_{j0}+\sum_{k=1}^{3}\sum_{j=1}^{3}\frac{v_j}{c}\frac{v_n}{c}\frac{v_k}{c}c_{jk}
-c_{0n} -\sum_{j=1}^{3}\frac{v_j}{c}c_{jn} \quad\forall\, n=1,2,3,
\\
c^{mn}=\frac{v_m}{c}c^{0n}+\frac{v_n}{c}c^{m0}-\frac{v_m}{c}\frac{v_n}{c}c^{00}
%\frac{v_m}{c}\frac{v_n}{c}c_{00}+\sum_{k=1}^{3}\frac{v_m}{c}\frac{v_n}{c}\frac{v_k}{c}c_{0k}+\sum_{j=1}^{3}\frac{v_n}{c}\frac{v_m}{c}\frac{v_j}{c}c_{j0}+\sum_{k=1}^{3}\sum_{j=1}^{3}\frac{v_m}{c}\frac{v_n}{c}\frac{v_j}{c}\frac{v_k}{c}c_{jk}
%-\sum_{k=1}^{3}\sum_{j=1}^{3}\frac{v_m}{c}\frac{v_n}{c}\frac{v_j}{c}\frac{v_k}{c}c_{jk}
%%+2\frac{v_m}{c}\frac{v_n}{c}c^{00}-\sum_{k=1}^{3}\frac{v_n}{c}\frac{v_k}{c}c_{mk}-\sum_{j=1}^{3}\frac{v_m}{c}\frac{v_j}{c}c_{jn}-\frac{v_m}{c}c_{0n}-\frac{v_n}{c}c_{m0}
+c_{mn}\quad\forall\, m,n=1,2,3.
\end{cases}
\end{equation}
In particular if the tensor $\{c_{mn}\}_{m,n=0,1,2,3}$ is
antisymmetric, i.e. $c_{mn}=-c_{nm}\;\forall \,m,n=0,1,2,3$, then we
simplify
\er{khjhhkfgjfjhgghhgjghjhjkkkkgjghghuiiiulkkjlkklKKgfgjhjjghgjh1int}
as
\begin{equation}\label{khjhhkfgjfjhgghhgjghjhjkkkkgjghghuiiiulkkjlkklKKgfgjhjjghgjhhjhjint}
\begin{cases}
c^{00}=0
\\
c^{mm}=0 \quad\forall\, m=1,2,3,
\\
c^{0m}=-c^{m0}=
%\frac{v_m}{c}c_{00}+\sum_{k=1}^{3}\frac{v_m}{c}\frac{v_k}{c}c_{0k}+\sum_{j=1}^{3}\frac{v_m}{c}\frac{v_j}{c}c_{j0}+\sum_{k=1}^{3}\sum_{j=1}^{3}\frac{v_k}{c}\frac{v_m}{c}\frac{v_j}{c}c_{jk}
-c_{0m}+\sum_{k=1}^{3}\frac{v_k}{c}c_{mk} \quad\forall\, m=1,2,3,
\\
%\frac{v_n}{c}c_{00}+\sum_{k=1}^{3}\frac{v_n}{c}\frac{v_k}{c}c_{0k}+\sum_{j=1}^{3}\frac{v_j}{c}\frac{v_n}{c}c_{j0}+\sum_{k=1}^{3}\sum_{j=1}^{3}\frac{v_j}{c}\frac{v_n}{c}\frac{v_k}{c}c_{jk}
%%c^{0n}=-c_{0n} -\sum_{j=1}^{3}\frac{v_j}{c}c_{jn} \quad\forall\, n=1,2,3,\\
c^{mn}=\frac{v_m}{c}c^{0n}-\frac{v_n}{c}c^{0m}
%\frac{v_m}{c}\frac{v_n}{c}c_{00}+\sum_{k=1}^{3}\frac{v_m}{c}\frac{v_n}{c}\frac{v_k}{c}c_{0k}+\sum_{j=1}^{3}\frac{v_n}{c}\frac{v_m}{c}\frac{v_j}{c}c_{j0}+\sum_{k=1}^{3}\sum_{j=1}^{3}\frac{v_m}{c}\frac{v_n}{c}\frac{v_j}{c}\frac{v_k}{c}c_{jk}
%-\sum_{k=1}^{3}\sum_{j=1}^{3}\frac{v_m}{c}\frac{v_n}{c}\frac{v_j}{c}\frac{v_k}{c}c_{jk}
%%+2\frac{v_m}{c}\frac{v_n}{c}c^{00}-\sum_{k=1}^{3}\frac{v_n}{c}\frac{v_k}{c}c_{mk}-\sum_{j=1}^{3}\frac{v_m}{c}\frac{v_j}{c}c_{jn}-\frac{v_m}{c}c_{0n}-\frac{v_n}{c}c_{m0}
+c_{mn}\quad\forall\, m,n=1,2,3.
\end{cases}
\end{equation}
\end{comment}
%
%
%
In particular, if $\{F_{ij}\}_{0\leq i,j\leq 3}$ is the
antisymmetric two times covariant tensor field of the
electromagnetic field on the group $\mathcal{S}_0$, which by
\er{huohuioy89gjjhjffffff3478zzrrZZZhjhhjhhjjhhffGGhjjhiuuijkjjkint}
satisfies:
\begin{equation}\label{huohuioy89gjjhjffffff3478zzrrZZZhjhhjhhjjhhffGGhjjhiuuijkjjkihjhjhint}
\begin{cases}
F_{00}=0\\ F_{0j}=-F_{j0}=E_j\quad\quad\forall\, j=1,2,3\\
F_{jj}=0\quad\quad\forall\, j=1,2,3
\\
F_{12}=-F_{21}=-B_3
%\frac{\partial (-A_1)}{\partial x^2}-\frac{\partial(-A_2)}{\partial x^1}
\\
F_{13}=-F_{31}=B_2
%\frac{\partial (-A_1)}{\partial x^3}-\frac{\partial(-A_3)}{\partial x^1}
\\
F_{23}=-F_{32}=-B_1\,,
%\frac{\partial (-A_2)}{\partial x^3}-\frac{\partial (-A_3)}{\partial x^2}
\end{cases}
\end{equation}
%where  $\vec E:=(E_1,E_2,E_3)$ and $\vec B:=(B_1,B_2,B_3)$,
then by inserting
\er{huohuioy89gjjhjffffff3478zzrrZZZhjhhjhhjjhhffGGhjjhiuuijkjjkihjhjhint}
into \er{fgjfjhgghhgjghjhjkkkkgjghghuiiiulkkjlkklKKint}, using
\er{hoyuiouigyfghgjh3478zzrrZZffGGjkkjint} and denoting:
\begin{equation}\label{MaxVacFullPPNhjjghint}
\begin{cases}
(D_1,D_2,D_3)=\vec D:=\vec E+\frac{1}{c}\,\vec v\times
\vec B\\
(H_1,H_2,H_3)=\vec H:=\vec B+\frac{1}{c}\,\vec v\times \vec D,
\end{cases}
\end{equation}
we deduce:
%
%
%
\begin{comment}
\begin{equation}\label{khjhhkfgjfjhgghhgjghjhjkkkkgjghghuiiiulkkjlkklKKgfgjhjjghgjhhjhjhhhhhint}
\begin{cases}
F^{00}=0
\\
F^{jj}=0 \quad\forall\, j=1,2,3,
\\
%\frac{v_m}{c}c_{00}+\sum_{k=1}^{3}\frac{v_m}{c}\frac{v_k}{c}c_{0k}+\sum_{j=1}^{3}\frac{v_m}{c}\frac{v_j}{c}c_{j0}+\sum_{k=1}^{3}\sum_{j=1}^{3}\frac{v_k}{c}\frac{v_m}{c}\frac{v_j}{c}c_{jk}
F^{01}=-F^{10}=-F_{01}+\frac{v_2}{c}F_{12}+\frac{v_3}{c}F_{13}=-\left(E_1+\frac{1}{c}\left(v_2B_3-v_3B_2\right)\right)\\
F^{02}=-F^{20}=-F_{02}+\frac{v_1}{c}F_{21}+\frac{v_3}{c}F_{23}=-\left(E_2+\frac{1}{c}\left(v_3B_1-v_1B_3\right)\right)\\
F^{03}=-F^{30}=-F_{03}+\frac{v_1}{c}F_{31}+\frac{v_2}{c}F_{32}=-\left(E_3+\frac{1}{c}\left(v_1B_2-v_2B_1\right)\right)
\\
%\frac{v_n}{c}c_{00}+\sum_{k=1}^{3}\frac{v_n}{c}\frac{v_k}{c}c_{0k}+\sum_{j=1}^{3}\frac{v_j}{c}\frac{v_n}{c}c_{j0}+\sum_{k=1}^{3}\sum_{j=1}^{3}\frac{v_j}{c}\frac{v_n}{c}\frac{v_k}{c}c_{jk}
%%c^{0n}=-c_{0n} -\sum_{j=1}^{3}\frac{v_j}{c}c_{jn} \quad\forall\, n=1,2,3,\\

%\frac{v_m}{c}\frac{v_n}{c}c_{00}+\sum_{k=1}^{3}\frac{v_m}{c}\frac{v_n}{c}\frac{v_k}{c}c_{0k}+\sum_{j=1}^{3}\frac{v_n}{c}\frac{v_m}{c}\frac{v_j}{c}c_{j0}+\sum_{k=1}^{3}\sum_{j=1}^{3}\frac{v_m}{c}\frac{v_n}{c}\frac{v_j}{c}\frac{v_k}{c}c_{jk}
%-\sum_{k=1}^{3}\sum_{j=1}^{3}\frac{v_m}{c}\frac{v_n}{c}\frac{v_j}{c}\frac{v_k}{c}c_{jk}
%%+2\frac{v_m}{c}\frac{v_n}{c}c^{00}-\sum_{k=1}^{3}\frac{v_n}{c}\frac{v_k}{c}c_{mk}-\sum_{j=1}^{3}\frac{v_m}{c}\frac{v_j}{c}c_{jn}-\frac{v_m}{c}c_{0n}-\frac{v_n}{c}c_{m0}
F^{12}=-F^{21}=\frac{v_1}{c}F^{02}-\frac{v_2}{c}F^{01}+F_{12}=-\left(B_3+\frac{1}{c}\left(v_1F^{20}-v_2F^{10}\right)\right)\\
F^{13}=-F^{31}=\frac{v_1}{c}F^{03}-\frac{v_3}{c}F^{01}+F_{13}=B_2+\frac{1}{c}\left(v_3F^{10}-v_1F^{30}\right)\\
F^{23}=-F^{32}=\frac{v_2}{c}F^{03}-\frac{v_3}{c}F^{02}+F_{23}=-\left(B_1+\frac{1}{c}\left(v_2F^{30}-v_3F^{20}\right)\right).
\end{cases}
\end{equation}
Thus, as before in \er{MaxVacFull1bjkgjhjhgjaaaint}, denoting:
\begin{equation}\label{MaxVacFullPPNhjjghint}
\begin{cases}
\vec D=\vec E+\frac{1}{c}\,\vec v\times
\vec B\\
\vec H=\vec B+\frac{1}{c}\,\vec v\times \vec D,
\end{cases}
\end{equation}
and denoting $\vec D:=(D_1,D_2,D_3)$ and $\vec H:=(H_1,H_2,H_3)$ we
rewrite
\er{khjhhkfgjfjhgghhgjghjhjkkkkgjghghuiiiulkkjlkklKKgfgjhjjghgjhhjhjhhhhhint}
as:
\end{comment}
%
%
%
\begin{equation}\label{khjhhkfgjfjhgghhgjghjhjkkkkgjghghuiiiulkkjlkklKKgfgjhjjghgjhhjhjhhhhhghhgint}
\begin{cases}
F^{00}=0
\\
%\frac{v_m}{c}c_{00}+\sum_{k=1}^{3}\frac{v_m}{c}\frac{v_k}{c}c_{0k}+\sum_{j=1}^{3}\frac{v_m}{c}\frac{v_j}{c}c_{j0}+\sum_{k=1}^{3}\sum_{j=1}^{3}\frac{v_k}{c}\frac{v_m}{c}\frac{v_j}{c}c_{jk}
F^{0j}=-F^{j0}=-D_j\quad\forall\, j=1,2,3,\\
F^{jj}=0 \quad\forall\, j=1,2,3,\\
%F^{01}=-F^{10}=-D_1\\
%-\left(E_1+\frac{1}{c}\left(v_2B_3-v_3B_2\right)\right)\\
%F^{02}=-F^{20}=-D_2\\
%-\left(E_2+\frac{1}{c}\left(v_3B_1-v_1B_3\right)\right)\\
%F^{03}=-F^{30}=-D_3
%-\left(E_3+\frac{1}{c}\left(v_1B_2-v_2B_1\right)\right)
%\\
%\frac{v_n}{c}c_{00}+\sum_{k=1}^{3}\frac{v_n}{c}\frac{v_k}{c}c_{0k}+\sum_{j=1}^{3}\frac{v_j}{c}\frac{v_n}{c}c_{j0}+\sum_{k=1}^{3}\sum_{j=1}^{3}\frac{v_j}{c}\frac{v_n}{c}\frac{v_k}{c}c_{jk}
%%c^{0n}=-c_{0n} -\sum_{j=1}^{3}\frac{v_j}{c}c_{jn} \quad\forall\, n=1,2,3,\\
%\frac{v_m}{c}\frac{v_n}{c}c_{00}+\sum_{k=1}^{3}\frac{v_m}{c}\frac{v_n}{c}\frac{v_k}{c}c_{0k}+\sum_{j=1}^{3}\frac{v_n}{c}\frac{v_m}{c}\frac{v_j}{c}c_{j0}+\sum_{k=1}^{3}\sum_{j=1}^{3}\frac{v_m}{c}\frac{v_n}{c}\frac{v_j}{c}\frac{v_k}{c}c_{jk}
%-\sum_{k=1}^{3}\sum_{j=1}^{3}\frac{v_m}{c}\frac{v_n}{c}\frac{v_j}{c}\frac{v_k}{c}c_{jk}
%%+2\frac{v_m}{c}\frac{v_n}{c}c^{00}-\sum_{k=1}^{3}\frac{v_n}{c}\frac{v_k}{c}c_{mk}-\sum_{j=1}^{3}\frac{v_m}{c}\frac{v_j}{c}c_{jn}-\frac{v_m}{c}c_{0n}-\frac{v_n}{c}c_{m0}
F^{12}=-F^{21}=-H_3\\
%-\left(B_3+\frac{1}{c}\left(v_1F^{20}-v_2F^{10}\right)\right)\\
F^{13}=-F^{31}=H_2\\
%B_2+\frac{1}{c}\left(v_3F^{10}-v_1F^{30}\right)\\
F^{23}=-F^{32}=-H_1.
%-\left(B_1+\frac{1}{c}\left(v_2F^{30}-v_3F^{20}\right)\right)
\end{cases}
\end{equation}
In particular, by
\er{huohuioy89gjjhjffffff3478zzrrZZZhjhhjhhjjhhffGGhjjhiuuijkjjkihjhjhint}
and
\er{khjhhkfgjfjhgghhgjghjhjkkkkgjghghuiiiulkkjlkklKKgfgjhjjghgjhhjhjhhhhhghhgint},
using \er{MaxVacFullPPNhjjghint} we deduce that:
%
%
%
\begin{comment}
the scalar field on the group $\mathcal{S}_0$: $L_e$, defined as:
\begin{equation}\label{MaxVacFullPPNhjjghjjkjhhint}
L_e:=\sum_{j=0}^{3}\sum_{k=0}^{3}F^{jk}F_{jk},
\end{equation}
satisfies
\begin{multline}\label{MaxVacFullPPNhjjghjjkjhh1int}
L_e=F^{00}F_{00}+\sum_{k=1}^{3}F^{0k}F_{0k}+\sum_{j=1}^{3}F^{j0}F_{j0}+\sum_{j=1}^{3}\sum_{k=1}^{3}F^{jk}F_{jk}=-2\vec
E\cdot\vec D+2\vec B\cdot\vec H=\\-2\left(\left(\vec
D-\frac{1}{c}\,\vec v\times \vec B\right)\cdot\vec D-\vec
B\cdot\left(\vec B+\frac{1}{c}\,\vec v\times \vec
D\right)\right)=-2\left(|\vec D|^2-|\vec B|^2\right).
\end{multline}
\end{comment}
%
%
%
\begin{equation}\label{MaxVacFullPPNhjjghjjkjhh1int}
-\sum_{j=0}^{3}\sum_{k=0}^{3}\frac{1}{4}F^{jk}F_{jk}=\frac{1}{2}|\vec
D|^2-\frac{1}{2}|\vec B|^2.
\end{equation}

\subsubsection{Maxwell equations in covariant formulation}
%
%
%
\begin{comment}
It is well known from Tensor Analysys that if $\{S^{ij}\}_{0\leq
i,j\leq 3}$ is the antisymmetric two times contravariant tensor and
if $\{\xi_{ij}\}_{0\leq i,j\leq 3}$ is a symmetric two times
covariant and non-degenerate tensor, both on the certain group
$\mathcal{S}$, then the four-component field $\{\theta_{k}\}_{0\leq
k\leq 3}$ defined by
\begin{equation}\label{MaxVacFullPPNhjjghjjkjhhoujiiint}
\theta_k:=\sum_{j=0}^{3}\frac{\partial S^{kj}}{\partial
x^j}+\sum_{j=0}^{3}\frac{S^{kj}}{\sqrt{|\text{det}\,\xi|}}\frac{\partial}{\partial
x^j}\left(\sqrt{|\text{det}\,\xi|}\right)\quad\quad\forall\,
k=0,1,2,3,
\end{equation}
is a four-vector on $\mathcal{S}$. Here $\xi$ is a $4\times
4$-matrix defined by:
\begin{equation}\label{fgjfjhgghhgjghjhjkkkkgjghghuiiiuujhjhjkljjhjhjjiint}
\xi=\{\xi_{ij}\}_{0\leq i,j\leq 3}.
\end{equation}
In particular,
\end{comment}
%
%
%
In Section \ref{CVFRM} we prove that, since the lifted contravariant
tensor of the electromagnetic field $\{F^{ij}\}_{0\leq i,j\leq 3}$
on the group $\mathcal{S}_0$, considered in
\er{khjhhkfgjfjhgghhgjghjhjkkkkgjghghuiiiulkkjlkklKKgfgjhjjghgjhhjhjhhhhhghhgint}
is antisymmetric, then
%as in \er{MaxVacFullPPNhjjghjjkjhhoujiiint}
the following four-component field:
\begin{equation}\label{MaxVacFullPPNhjjghjjkjhhoujiiikjjihjhiuiuinthhint}
\left\{\sum_{j=0}^{3}\frac{\partial F^{kj}}{\partial
x^j}+\sum_{j=0}^{3}\frac{F^{kj}}{\sqrt{|\text{det}\,G|}}\frac{\partial}{\partial
x^j}\left(\sqrt{|\text{det}\,G|}\right)\right\}_{0\leq k\leq 3}
\end{equation}
is a four-vector on the group $\mathcal{S}_0$, where the $4\times
4$-matrix $G$ is defined as $G:=\{g_{ij}\}_{0\leq i,j\leq 3}$.
%by \er{fgjfjhgghhgjghjhjkkkkgjghghuiiiuujhjhjkljint} as:
%\begin{equation}\label{fgjfjhgghhgjghjhjkkkkgjghghuiiiuujhjhjklj1int}
%G=\{g_{ij}\}_{0\leq i,j\leq 3}.
%\end{equation}
Then, since the matrix $G$ satisfies $\text{det}\,G=-1$ in every
cartesian coordinate system,
%\er{fgjfjhgghhgjghjhjkkkkgjghghuiiiuujhjh1int} in every cartesian coordinate system, i.e.
%\begin{equation}\label{fgjfjhgghhgjghjhjkkkkgjghghuiiiuujhjhhjjgint}
%\text{det}\,G=-1.
%\end{equation}
then
\begin{equation}\label{MaxVacFullPPNhjjghjjkjhhoujiiikjjihjhiuiuint}
\left\{\sum_{j=0}^{3}\frac{\partial F^{kj}}{\partial
x^j}+\sum_{j=0}^{3}\frac{F^{kj}}{\sqrt{|\text{det}\,G|}}\frac{\partial}{\partial
x^j}\left(\sqrt{|\text{det}\,G|}\right)\right\}_{0\leq k\leq
3}=\left\{\sum_{j=0}^{3}\frac{\partial F^{kj}}{\partial
x^j}\right\}_{0\leq k\leq 3}.
\end{equation}
Then, by
\er{khjhhkfgjfjhgghhgjghjhjkkkkgjghghuiiiulkkjlkklKKgfgjhjjghgjhhjhjhhhhhghhgint},
denoting $(x^0,x^1,x^2,x^3):=(ct,x_1,x_2,x_3)=(ct,\vec x)$, we
deduce:
%
%
%
\begin{comment}
\begin{equation}\label{khjhhkfgjfjhgghhgjghjhjkkkkgjghghuiiiulkkjlkklKKgfgjhjjghgjhhjhjhhhhhghhgtyytojjjint}
\begin{cases}
\sum_{j=0}^{3}\frac{\partial F^{0j}}{\partial x^j}=-div_{\vec x}\vec D\\
\sum_{j=0}^{3}\frac{\partial F^{1j}}{\partial x^j}=\frac{1}{c}\frac{\partial D_1}{\partial t}-\left(\frac{\partial H_3}{\partial x_2}-\frac{\partial H_2}{\partial x_3}\right)\\
\sum_{j=0}^{3}\frac{\partial F^{2j}}{\partial x^j}=\frac{1}{c}\frac{\partial D_2}{\partial t}-\left(\frac{\partial H_1}{\partial x_3}-\frac{\partial H_3}{\partial x_1}\right)\\
\sum_{j=0}^{3}\frac{\partial F^{3j}}{\partial
x^j}=\frac{1}{c}\frac{\partial D_3}{\partial t}-\left(\frac{\partial
H_2}{\partial x_1}-\frac{\partial H_1}{\partial x_2}\right).
\end{cases}
\end{equation}
I.e.:
\end{comment}
%
%
%
\begin{equation}\label{khjhhkfgjfjhgghhgjghjhjkkkkgjghghuiiiulkkjlkklKKgfgjhjjghgjhhjhjhhhhhghhgtyytojjjjjint}
\left(\sum_{j=0}^{3}\frac{\partial F^{0j}}{\partial
x^j},\sum_{j=0}^{3}\frac{\partial F^{1j}}{\partial
x^j},\sum_{j=0}^{3}\frac{\partial F^{2j}}{\partial
x^j},\sum_{j=0}^{3}\frac{\partial F^{3j}}{\partial
x^j}\right)=\left(-div_{\vec x}\vec D,
\left(\frac{1}{c}\frac{\partial \vec D}{\partial t}-curl_{\vec
x}\vec H\right)\right).
\end{equation}
Therefore, by
\er{khjhhkfgjfjhgghhgjghjhjkkkkgjghghuiiiulkkjlkklKKgfgjhjjghgjhhjhjhhhhhghhgtyytojjjjjint},
the first pair of Maxwell Equations in
\er{MaxVacFull1bjkgjhjhgjaaaint}:
\begin{equation}\label{MaxVacFullPPNhjjghyghghiyyhhhjint}
\begin{cases}
curl_{\vec x} \vec H=\frac{4\pi}{c}\vec j+\frac{1}{c}\frac{\partial \vec D}{\partial t}\\
div_{\vec x} \vec D= 4\pi\rho,
\end{cases}
\end{equation}
is equivalent to the following equations:
\begin{equation}\label{khjhhkfgjfjhgghhgjghjhjkkkkgjghghuiiiulkkjlkklKKgfgjhjjghgjhhjhjhhhhhghhgtyytojjjjjuyiyint}
\left(\sum_{j=0}^{3}\frac{\partial F^{0j}}{\partial
x^j},\sum_{j=0}^{3}\frac{\partial F^{1j}}{\partial
x^j},\sum_{j=0}^{3}\frac{\partial F^{2j}}{\partial
x^j},\sum_{j=0}^{3}\frac{\partial F^{3j}}{\partial x^j}\right)
%=\left(-div_{\vec x}\vec D,\left(\frac{1}{c}\frac{\partial \vec D}{\partial t}-curl_{\vec x}\vec H\right)\right)
=-4\pi(j^0,j^1,j^2,j^3),
\end{equation}
where $(j^0,j^1,j^2,j^3)$ is the four-vector of electric current on
the group $\mathcal{S}_0$ defined by
\er{fgjfjhgghhgjghjhjijhojihjhjjijhjjjjjuiijjjklihhojjjoint} as:
\begin{equation}
\label{fgjfjhgghhgjghjhjijhojihjhjjijhjjjjjuiijjjklihhojjjoouuoiuiuint}
(j^0,j^1,j^2,j^3):=\left(\rho,\frac{1}{c}\,\vec j\right)
%\quad\text{where}\quad j^0=\rho\;\;\text{and}\;\;(j^1,j^2,j^3)=\frac{1}{c}\,\vec j,
\end{equation}
%where $\rho$ is the electric charge density and $\vec j$ is the electric current density, then $(j^0,j^1,j^2,j^3)$ is also the four-vector on the group $\mathcal{S}_0$.
Note that in both sides of equation
\er{khjhhkfgjfjhgghhgjghjhjkkkkgjghghuiiiulkkjlkklKKgfgjhjjghgjhhjhjhhhhhghhgtyytojjjjjuyiyint}
we have four-vectors and thus
\er{khjhhkfgjfjhgghhgjghjhjkkkkgjghghuiiiulkkjlkklKKgfgjhjjghgjhhjhjhhhhhghhgtyytojjjjjuyiyint}
is a covariant form of \er{MaxVacFullPPNhjjghyghghiyyhhhjint}. On
the other hand, the second pair of Maxwell Equations in
\er{MaxVacFull1bjkgjhjhgjaaaint}:
\begin{equation}\label{MaxVacFullPPNhjjghyghghiyyint}
\begin{cases}
curl_{\vec x} \vec E+\frac{1}{c}\frac{\partial \vec B}{\partial t}=0\\
div_{\vec x} \vec B=0,
\end{cases}
\end{equation}
is equivalent to
\er{MaxVacFull1bjkgjhjhgjgjgkjfhjfdghghligioiuittrPPNkkkint}, i.e.
to the following:
\begin{equation}\label{MaxVacFull1bjkgjhjhgjgjgkjfhjfdghghligioiuittrPPNkkkouiuiuuiuiint}
\begin{cases}
\vec B= curl_{\vec x} \vec A,\\
\vec E=-\nabla_{\vec x}\Psi-\frac{1}{c}\frac{\partial\vec
A}{\partial t},
%,\\ div_{\vec x}\vec A\equiv 0,
\end{cases}
\end{equation}
On the other hand, as before, by
\er{huohuioy89gjjhjffffff3478zzrrZZZhjhhjhhjjhhffGGhjjhiuuijkjjkint}
we can rewrite
\er{MaxVacFull1bjkgjhjhgjgjgkjfhjfdghghligioiuittrPPNkkkouiuiuuiuiint}
in the form of
\er{huohuioy89gjjhjffffff3478zzrrZZZhjhhjhhjjhhffGGhjjhint}:
\begin{equation}\label{huohuioy89gjjhjffffff3478zzrrZZZhjhhjhhjjhhffGGhjjhuyuyuint}
F_{ij}=\frac{\partial A_j}{\partial x^i}-\frac{\partial
A_i}{\partial x^j}\quad\quad\forall\, i,j=0,1,2,3\,,
\end{equation}
where $(A_0,A_1,A_2,A_3)$ is the four-covector of the
electromagnetic potential on the group $\mathcal{S}_0$ defined by
\er{fgjfjhgghhgjghjhjijhojihjhjjijhjjjljljpkint} as:
\begin{equation}\label{fgjfjhgghhgjghjhjijhojihjhjjijhjjjljljpkyuuyyuint}
(A_0,A_1,A_2,A_3)=(\Psi,-\vec A).
%\quad\text{where}\quad A_0=\Psi\;\;\text{and}\;\;(A_1,A_2,A_3)=-\vec A,
\end{equation}
Note that in both sides of equation
\er{huohuioy89gjjhjffffff3478zzrrZZZhjhhjhhjjhhffGGhjjhuyuyuint} we
have two time covariant tensors, and thus
\er{huohuioy89gjjhjffffff3478zzrrZZZhjhhjhhjjhhffGGhjjhuyuyuint} is
a covariant form of \er{MaxVacFullPPNhjjghyghghiyyint}. Finally,
%by \er{khjhhkfgjfjhgghhgjghjhjkkkkgjghghuiiiulkkjlkklKKgfgjhjjghgjhhjhjhhhhhint},
the relations between $(\vec E,\vec B)$ and $(\vec D,\vec H)$ in
\er{MaxVacFull1bjkgjhjhgjaaaint}:
\begin{equation}\label{MaxVacFullPPNhjjghojjkjkiouint}
\begin{cases}
\vec D=\vec E+\frac{1}{c}\,\vec v\times
\vec B\\
\vec H=\vec B+\frac{1}{c}\,\vec v\times \vec D,
\end{cases}
\end{equation}
are equivalent to the following covariant equations:
\begin{equation}\label{fgjfjhgghhgjghjhjkkkkgjghghuiiiulkkjlkklKKkjkjint}
F^{mn}:=\sum_{k=0}^{3}\sum_{j=0}^{3}g^{mj}g^{nk}F_{jk}
%\quad\text{and}\quadd_{mn}:=\sum_{j=0}^{3}\sum_{k=0}^{3}g_{mj}g_{nk}d^{jk}
\quad\quad\forall\, m,n=0,1,2,3.
\end{equation}
Thus by
\er{huohuioy89gjjhjffffff3478zzrrZZZhjhhjhhjjhhffGGhjjhuyuyuint},
\er{fgjfjhgghhgjghjhjkkkkgjghghuiiiulkkjlkklKKkjkjint} and
\er{khjhhkfgjfjhgghhgjghjhjkkkkgjghghuiiiulkkjlkklKKgfgjhjjghgjhhjhjhhhhhghhgtyytojjjjjuyiyint}
together, we deduce that the full system of Maxwell Equations in
\er{MaxVacFull1bjkgjhjhgjaaaint}:
\begin{equation}\label{MaxVacFullPPNhjjghyghghiyyhhint}
\begin{cases}
curl_{\vec x} \vec H=\frac{4\pi}{c}\vec j+\frac{1}{c}\frac{\partial \vec D}{\partial t}\\
div_{\vec x} \vec D=4\pi\rho\\
curl_{\vec x} \vec E+\frac{1}{c}\frac{\partial \vec B}{\partial t}=0\\
div_{\vec x} \vec B=
0\\
\vec E=\vec D-\frac{1}{c}\,\vec v\times
\vec B\\
\vec H=\vec B+\frac{1}{c}\,\vec v\times \vec D,
\end{cases}
\end{equation}
is equivalent to the following covariant equations:
\begin{equation}\label{khjhhkfgjfjhgghhgjghjhjkkkkgjghghuiiiulkkjlkklKKgfgjhjjghgjhhjhjhhhhhghhgtyytojjjjjuyiyuuint}
\sum_{j=0}^{3}\frac{\partial}{\partial
x^j}\left(\sum_{m=0}^{3}\sum_{n=0}^{3}g^{km}g^{jn}\left(\frac{\partial
A_n}{\partial x^m}-\frac{\partial A_m}{\partial
x^n}\right)\right)=-4\pi j^k\quad\quad\forall\, k=0,1,2,3.
\end{equation}
Note that equations
\er{khjhhkfgjfjhgghhgjghjhjkkkkgjghghuiiiulkkjlkklKKgfgjhjjghgjhhjhjhhhhhghhgtyytojjjjjuyiyuuint}
are fully analogous to the covariant formulation of Maxwell
equations in Special Relativity and the only difference is the
choice of the pseudo-metric tensor $\{g^{ij}\}_{0\leq i,j\leq 3}$
(Note that for the Special Relativity case we also have
$\text{det}\,G=-1$). As for the cases of the General relativity, the
covariant formulation of Maxwell equations is still similar to
\er{khjhhkfgjfjhgghhgjghjhjkkkkgjghghuiiiulkkjlkklKKgfgjhjjghgjhhjhjhhhhhghhgtyytojjjjjuyiyuuint},
however, in addition to the different choice of the pseudo-metric
tensor $\{g^{ij}\}_{0\leq i,j\leq 3}$ we also have
$\text{det}\,G\neq Const.$ and thus for the full analogy equations
\er{khjhhkfgjfjhgghhgjghjhjkkkkgjghghuiiiulkkjlkklKKgfgjhjjghgjhhjhjhhhhhghhgtyytojjjjjuyiyuuint}
should be rewritten in the enlarged form, due to
%\er{MaxVacFullPPNhjjghjjkjhhoujiiint},\er{MaxVacFullPPNhjjghjjkjhhoujiiikjjihjhiuiuint}
\er{MaxVacFullPPNhjjghjjkjhhoujiiikjjihjhiuiuinthhint}:
\begin{multline}\label{khjhhkfgjfjhgghhgjghjhjkkkkgjghghuiiiulkkjlkklKKgfgjhjjghgjhhjhjhhhhhghhgtyytojjjjjuyiyuughgfgint}
\sum_{j=0}^{3}\frac{\partial}{\partial
x^j}\left(\sum_{m=0}^{3}\sum_{n=0}^{3}g^{km}g^{jn}\left(\frac{\partial
A_n}{\partial x^m}-\frac{\partial A_m}{\partial
x^n}\right)\right)+\\
\sum_{j=0}^{3}\frac{1}{\sqrt{|\text{det}\,G|}}\frac{\partial}{\partial
x^j}\left(\sqrt{|\text{det}\,G|}\right)\left(\sum_{m=0}^{3}\sum_{n=0}^{3}g^{km}g^{jn}\left(\frac{\partial
A_n}{\partial x^m}-\frac{\partial A_m}{\partial x^n}\right)\right)
=-4\pi j^k\quad\quad\forall\, k=0,1,2,3.
\end{multline}
Note also that we can rewrite
\er{khjhhkfgjfjhgghhgjghjhjkkkkgjghghuiiiulkkjlkklKKgfgjhjjghgjhhjhjhhhhhghhgtyytojjjjjuyiyuughgfgint}
as:
\begin{equation}\label{khjhhkfgjfjhgghhgjghjhjkkkkgjghghuiiiulkkjlkklKKgfgjhjjghgjhhjhjhhhhhghhgtyytojjjjjuyiyuughgfghhjint}
\sum_{j=0}^{3}\frac{\partial}{\partial
x^j}\left(\sum_{m=0}^{3}\sum_{n=0}^{3}\sqrt{|\text{det}\,G|}\,g^{km}g^{jn}\left(\frac{\partial
A_n}{\partial x^m}-\frac{\partial A_m}{\partial x^n}\right)\right)
=-4\pi \sqrt{|\text{det}\,G|}\, j^k\quad\quad\forall\, k=0,1,2,3.
\end{equation}








Next by
%\er{MaxVacFullPPNhjjghjjkjhhint} and
\er{MaxVacFullPPNhjjghjjkjhh1int} we have
\begin{equation}\label{MaxVacFullPPNhjjghjjkjhhiuyyint}
\frac{1}{2}|\vec D|^2-\frac{1}{2}|\vec
B|^2=-\sum_{j=0}^{3}\sum_{k=0}^{3}\frac{1}{4}F^{jk}F_{jk}.
\end{equation}
Therefore, by
\er{fgjfjhgghhgjghjhjijhojihjhjjijhjjjjjuiijjjklihhojjjoouuoiuiuint},
\er{fgjfjhgghhgjghjhjijhojihjhjjijhjjjljljpkyuuyyuint} and
\er{MaxVacFullPPNhjjghjjkjhhiuyyint}, we can rewrite the density of
the Lagrangian of the electromagnetic field, defined in
\er{vhfffngghkjgghPPNint} as
\begin{equation}\label{vhfffngghkjgghPPNggjgjjkgjoiuiint}
L_1\left(\vec A,\Psi,\vec
x,t\right):=\frac{1}{4\pi}\left(\frac{1}{2}\left|\vec
D\right|^2-\frac{1}{2}\left|\vec
B\right|^2-4\pi\left(\rho\Psi-\frac{1}{c}\vec A\cdot\vec
j\right)\right),
\end{equation}
in the equivalent covariant form:
\begin{multline}\label{vhfffngghkjgghPPNggjgjjkgjoiuioigghint}
L_1=\frac{1}{4\pi}\left(-\sum_{n=0}^{3}\sum_{k=0}^{3}\frac{1}{4}F^{nk}F_{nk}-\sum_{k=0}^{3}4\pi
j^k A_k\right)=\\
\frac{1}{4\pi}\left(-\sum_{n=0}^{3}\sum_{k=0}^{3}\sum_{m=0}^{3}\sum_{p=0}^{3}\frac{1}{4}g^{mn}g^{pk}\left(\frac{\partial
A_p}{\partial x^m}-\frac{\partial A_m}{\partial
x^p}\right)\left(\frac{\partial A_k}{\partial x^n}-\frac{\partial
A_n}{\partial x^k}\right)-\sum_{k=0}^{3}4\pi j^k A_k\right).
\end{multline}
The density of Lagrangian in
\er{vhfffngghkjgghPPNggjgjjkgjoiuioigghint} is also fully analogous
to the covariant formulation of the Lagrangian density of the
electromagnetic field in Special and General Relativity and the only
difference is the choice of the pseudo-metric tensor
$\{g^{ij}\}_{0\leq i,j\leq 3}$.



\subsubsection{Covariant formulation of Lagrangian of motion of a
classical charged particle in the external gravitational and
electromagnetic fields} Given a classical charged particle with
inertial mass $m$, charge $\sigma$, three-dimensional place $\vec
r(t)$ and three-dimensional velocity $\frac{d\vec r}{dt}$ in the
outer gravitational field with three-dimensional vectorial potential
$\vec v(\vec x,t)$, the outer electromagnetical field with
three-dimensional vectorial potential $\vec A(\vec x,t)$ and scalar
potential $\vec \Psi(\vec x,t)$, consider a usual Lagrangian that is
a particular case of \er{vhfffngghkjgghfjjint}:
\begin{equation}\label{vhfffngghkjgghfjjSYSPNkoijjhpoiint}
L_0\left(\frac{d\vec r}{dt},t\right):=
\left\{\frac{m}{2}\left|\frac{d\vec r}{dt}-\vec v(\vec
r,t)\right|^2-\sigma\left(\Psi(\vec r,t)-\frac{1}{c}\vec A(\vec
r,t)\cdot\frac{d\vec r}{dt}\right)\right\}.
\end{equation}
Then, since we are interesting in critical points of the functional
\begin{equation}\label{btfffygtgyggyijhhkkSYSPNuhuygyygyggyint}
J_0=\int_0^T L_0\left(\frac{d\vec r}{dt},\vec r,t\right)dt,
\end{equation}
adding a constant does not changes the physical meaning of the
Lagrangian and we can rewrite
\er{vhfffngghkjgghfjjSYSPNkoijjhpoiint} as:
\begin{equation}\label{vhfffngghkjgghfjjSYSPNkoijjhpoiuuiint}
L'_0\left(\frac{d\vec r}{dt},t\right):=
\left\{\left(\frac{m}{2}\left|\frac{d\vec r}{dt}-\vec v(\vec
r,t)\right|^2-\frac{mc^2}{2}\right)-\sigma\left(\Psi(\vec
r,t)-\frac{1}{c}\vec A(\vec r,t)\cdot\frac{d\vec
r}{dt}\right)\right\}.
\end{equation}
and \er{btfffygtgyggyijhhkkSYSPNuhuygyygyggyint} as
\begin{multline}\label{btfffygtgyggyijhhkkSYSPNuhuygyygyggyuyyint}
J'_0:=J_0-\frac{Tmc^2}{2}=\int_0^T L'_0\left(\frac{d\vec r}{dt},\vec
r,t\right)dt=\\ \int_0^T\left\{\left(\frac{m}{2}\left|\frac{d\vec
r}{dt}-\vec v(\vec
r,t)\right|^2-\frac{mc^2}{2}\right)-\sigma\left(\Psi(\vec
r,t)-\frac{1}{c}\vec A(\vec r,t)\cdot\frac{d\vec
r}{dt}\right)\right\}dt,
\end{multline}
Next consider the four-vector field of the momentum on the group
$\mathcal{S}_0$: $\left(p^0(t),p^1(t),p^2(t),p^3(t)\right)$, defined
by \er{fgjfjhgghhgjghjhjijhojihjhjjijhjjjjjuiijhjhhint} and
\er{fgjfjhgghhgjghjhjijhojihjhjjijhjjjjjuiijjjklihhint} as:
\begin{equation}\label{fgjfjhgghhgjghjhjijhojihjhjjijhjjjjjuiijhjhhioiint}
\left(
p^0(t),p^1(t),p^2(t),p^3(t)\right):=\left(m,\frac{m}{c}\frac{d\vec
r}{dt}(t)\right)=
\left(m,\frac{m}{c}\frac{dr_1}{dt}(t),\frac{m}{c}\frac{dr_2}{dt}(t),\frac{m}{c}\frac{dr_3}{dt}(t)\right)
\end{equation}
Then by
%\er{fgjfjhgghhgjghjhjkkkkgjghghuiiiulkkjKKyuyyu0ioioiogghghghgghghhgint} and
\er{fgjfjhgghhgjghjhjkkkkgjghghuiiiulkkjKKyuyyu0ioioiogghghghgghghint}
we have
\begin{multline}\label{fgjfjhgghhgjghjhjkkkkgjghghuiiiulkkjKKyuyyu0ioioiogghghghgghghhghgvint}
\frac{mc^2}{2}\left(\frac{1}{c^2}\left|\frac{d\vec r}{dt}-\vec
v(\vec r,t)\right|^2-1\right)=\frac{m}{2}\left|\frac{d\vec
r}{dt}-\vec v(\vec
r,t)\right|^2-\frac{mc^2}{2}\\=-\frac{c^2}{2m}\left(\sum_{k=0}^{3}p^kp_k\right)=-\frac{mc^2}{2}\left(\sum_{j=0}^{3}\sum_{k=0}^{3}g_{jk}(\vec
r,t)\,\frac{p^j}{m}\,\frac{p^k}{m}\right).
\end{multline}
On the other hand if we consider the four-covector of the
electromagnetic potential on the group $\mathcal{S}_0$:
$(A_0,A_1,A_2,A_3)$, defined by
\er{fgjfjhgghhgjghjhjijhojihjhjjijhjjjljljpkint} as:
\begin{equation}\label{fgjfjhgghhgjghjhjijhojihjhjjijhjjjljljpkyuuyyuhhhhjint}
(A_0,A_1,A_2,A_3)=(\Psi,-\vec A),
%\quad\text{where}\quad A_0=\Psi\;\;\text{and}\;\;(A_1,A_2,A_3)=-\vec A,
\end{equation}
then we can write,
\begin{equation}\label{btfffygtgyggyijhhkkSYSPNuhuygyygyggyuyyioiooiihhhjint}
\sigma\left(\Psi(\vec r,t)-\frac{1}{c}\vec A(\vec
r,t)\cdot\frac{d\vec r}{dt}\right)=\sum_{k=0}^{3}\sigma A_k(\vec
r,t)\,\frac{p^k}{m}.
\end{equation}
Thus by
\er{fgjfjhgghhgjghjhjkkkkgjghghuiiiulkkjKKyuyyu0ioioiogghghghgghghhghgvint}
and \er{btfffygtgyggyijhhkkSYSPNuhuygyygyggyuyyioiooiihhhjint} we
rewrite \er{btfffygtgyggyijhhkkSYSPNuhuygyygyggyuyyint} in a
covariant form:
\begin{multline}\label{btfffygtgyggyijhhkkSYSPNuhuygyygyggyuyyuyuyint}
J'_0=\int_0^T L'_0\left(\frac{d\vec r}{dt},\vec r,t\right)dt=
\int_0^T\left\{-\frac{mc^2}{2}\left(\sum_{j=0}^{3}\sum_{k=0}^{3}g_{jk}(\vec
r,t)\,\frac{p^j}{m}\,\frac{p^k}{m}\right)-\sum_{k=0}^{3}\sigma
A_k(\vec r,t)\,\frac{p^k}{m}\right\}dt.
\end{multline}
Thus if we consider the four-dimensional space-time trajectory of
the particle:
\begin{equation}\label{btfffjhgjghghint}
\left(\chi^0(t),\chi^1(t),\chi^2(t),\chi^3(t)\right)=\left(t,\frac{1}{c}r_1(t),\frac{1}{c}r_2(t),\frac{1}{c}r_3(t)\right),
\end{equation}
then we rewrite \er{btfffygtgyggyijhhkkSYSPNuhuygyygyggyuyyuyuyint}
as:
\begin{equation}\label{btfffygtgyggyijhhkkSYSPNuhuygyygyggyuyyuyuykuhghgint}
J'_0=
\int_0^T\left\{-\frac{mc^2}{2}\left(\sum_{j=0}^{3}\sum_{k=0}^{3}g_{jk}\left(\chi(t)\right)\,\frac{d\chi^j}{dt}\,\frac{d\chi^k}{dt}\right)-\sum_{k=0}^{3}\sigma
A_k\left(\chi(t)\right)\,\frac{d\chi^k}{dt}\right\}dt.
\end{equation}
Moreover, $\left(\frac{d\chi^0}{dt}, \frac{d\chi^1}{dt},
\frac{d\chi^2}{dt}, \frac{d\chi^3}{dt}\right)$ is a four-vector on
the group $\mathcal{S}_0$ and the global non-relativistic time $t$
is the scalar on the group $\mathcal{S}_0$.


Next we also can consider a more general Lagrangian than
\er{btfffygtgyggyijhhkkSYSPNuhuygyygyggyuyyuyuykuhghgint}: given a
function $\mathcal{G}(\tau):\mathbb{R}\to\mathbb{R}$ define:
\begin{equation}\label{btfffygtgyggyijhhkkSYSPNuhuygyygyggyuyyuyuykuhghgjjojint}
J_{\mathcal{G}}(\chi)=
\int_0^T\left\{-mc^2\;\mathcal{G}\left(\sum_{j=0}^{3}\sum_{k=0}^{3}g_{jk}\left(\chi(t)\right)\,\frac{d\chi^j}{dt}\,\frac{d\chi^k}{dt}\right)-\sum_{k=0}^{3}\sigma
A_k\left(\chi(t)\right)\,\frac{d\chi^k}{dt}\right\}dt.
\end{equation}
Clearly,
\er{btfffygtgyggyijhhkkSYSPNuhuygyygyggyuyyuyuykuhghgjjojint} is
written in covariant form, and in particular,
\er{btfffygtgyggyijhhkkSYSPNuhuygyygyggyuyyuyuykuhghgjjojint} is
invariant under the change of non-inertial cartesian coordinate
systems. In particular, for $\mathcal{G}(\tau):=\frac{1}{2}\tau$ we
obtain \er{btfffygtgyggyijhhkkSYSPNuhuygyygyggyuyyuyuykuhghgint}.


Another important particular case is the following choice:
$\mathcal{G}(\tau):=\sqrt{\tau}$. Then we deduce:
\begin{equation}\label{btfffygtgyggyijhhkkSYSPNuhuygyygyggyuyyuyuykuhghgjjojiyyint}
J_{rl}(\chi)=
\int_0^T\left\{-mc^2\;\sqrt{\left(\sum_{j=0}^{3}\sum_{k=0}^{3}g_{jk}\left(\chi(t)\right)\,\frac{d\chi^j}{dt}\,\frac{d\chi^k}{dt}\right)}-\sum_{k=0}^{3}\sigma
A_k\left(\chi(t)\right)\,\frac{d\chi^k}{dt}\right\}dt,
\end{equation}
that is in somewhat analogous to the relativistic Lagrangian of the
motion of charged particle. Due to \er{btfffjhgjghghint} we rewrite
\er{btfffygtgyggyijhhkkSYSPNuhuygyygyggyuyyuyuykuhghgjjojiyyint} in
a three-dimensional form as:
\begin{equation}\label{btfffygtgyggyijhhkkSYSPNuhuygyygyggyuyyuyuykuhghgjjojiyyyuyuuyyint}
J_{rl}(\vec r)= \int_0^T\left\{
-mc^2\sqrt{1-\frac{1}{c^2}\left|\frac{d\vec r}{dt}-\vec v(\vec
r,t)\right|^2}-\sigma\left(\Psi(\vec r,t)-\frac{1}{c}\vec A(\vec
r,t)\cdot\frac{d\vec r}{dt}\right)\right\}dt.
\end{equation}
Thus in the case
$$\frac{1}{c^2}\left|\frac{d\vec r}{dt}-\vec v(\vec
r,t)\right|^2\ll 1,$$ up to additive constant,
\er{btfffygtgyggyijhhkkSYSPNuhuygyygyggyuyyuyuykuhghgjjojiyyyuyuuyyint}
becomes to be \er{btfffygtgyggyijhhkkSYSPNuhuygyygyggyint}, where
$L_0$ is given by \er{vhfffngghkjgghfjjSYSPNkoijjhpoiint}. Note that
the Lagrangian in
\er{btfffygtgyggyijhhkkSYSPNuhuygyygyggyuyyuyuykuhghgjjojiyyint} has
the following advantage with respect to
\er{btfffygtgyggyijhhkkSYSPNuhuygyygyggyuyyuyuykuhghgint}: if we
parameterize the curve in \er{btfffjhgjghghint} by some arbitrary
parameter $s$ that is different from the global time $t$, then
changing variables of integration in
\er{btfffygtgyggyijhhkkSYSPNuhuygyygyggyuyyuyuykuhghgjjojiyyint}
from $t$ to $s$ gives:
\begin{equation}\label{btfffygtgyggyijhhkkSYSPNuhuygyygyggyuyyuyuykuhghgjjojiyyjljlghint}
J_{rl}(\chi)=
\int_a^b\left\{-mc^2\;\sqrt{\left(\sum_{j=0}^{3}\sum_{k=0}^{3}g_{jk}\left(\chi(s)\right)\,\frac{d\chi^j}{ds}\,\frac{d\chi^k}{ds}\right)}-\sum_{k=0}^{3}\sigma
A_k\left(\chi(s)\right)\,\frac{d\chi^k}{ds}\right\}ds,
\end{equation}
that has exactly the same form as
\er{btfffygtgyggyijhhkkSYSPNuhuygyygyggyuyyuyuykuhghgjjojiyyint},
however $s$ in
\er{btfffygtgyggyijhhkkSYSPNuhuygyygyggyuyyuyuykuhghgjjojiyyjljlghint}
can be \underline{arbitrary} parameter of the curve.

Finally, we would like to note that if the motion of some particle
is ruled by the relativistic-like Lagrangian in
\er{btfffygtgyggyijhhkkSYSPNuhuygyygyggyuyyuyuykuhghgjjojiyyyuyuuyyint},
then, although the absolute value of the velocity of the particle
$\left|\frac{d\vec r}{dt}\right|$ can be arbitrary large, the
absolute value of the difference between the velocity of the
particle and the local gravitational potential cannot exceed the
value $c$, i.e.:
\begin{equation}\label{btfffygtgyggyijhhkkSYSPNuhuygyygyggyuyyuyuykuhghgjjojiyyyuyuuyyijyyuyuint}
\left|\vec u(t)-\vec v(\vec r,t)\right|:=\left|\frac{d\vec
r}{dt}-\vec v(\vec r,t)\right|\,<\,c\quad\quad\quad\quad\forall\, t,
\end{equation}
provided that
\er{btfffygtgyggyijhhkkSYSPNuhuygyygyggyuyyuyuykuhghgjjojiyyyuyuuyyijyyuyuint}
is satisfied in some initial instant of time. Note also that the
quantity in the right hand side of
\er{btfffygtgyggyijhhkkSYSPNuhuygyygyggyuyyuyuykuhghgjjojiyyyuyuuyyijyyuyuint}
is invariant under the change of inertial or non-inertial cartesian
coordinate system.



\subsubsection{Physical laws in curvilinear coordinate systems in the
non-relativistic space-time}
%$\mathbb{R}^4$
Let $\mathcal{S}$ be the group of all smooth non-degenerate
invertible transformations from $\mathbb{R}^4$ onto $\mathbb{R}^4$
having the form \er{fgjfjhgghyuyyuint}:
\begin{equation}\label{fgjfjhgghyuyyuyughgint}
\begin{cases}
x'^0=f^{(0)}(x^0,x^1,x^2,x^3),\\
x'^1=f^{(1)}(x^0,x^1,x^2,x^3),\\
x'^2=f^{(2)}(x^0,x^1,x^2,x^3),\\
x'^3=f^{(3)}(x^0,x^1,x^2,x^3),
\end{cases}
\end{equation}
and let $\mathcal{S}_0$ be a subgroup of transformations of the form
\er{noninchgravortbstrjgghguittu2intrrrZZygjyghhjint}. Then, it is
clear, that given any object that is a scalar, four-vector,
four-covector, two-times covariant tensor or two-times contravariant
tensor on the group $\mathcal{S}_0$, defined in every cartesian
non-inertial coordinate system, we can uniquely extend the
definition of this object, in such a way that it will be defined
also in every curvilinear coordinate systems in $\mathbb{R}^4$ and
will be respectively a scalar, four-vector, four-covector, two-times
covariant tensor or two-times contravariant tensor on the wider
group $\mathcal{S}$. Thus all the physical laws that have a
covariant form preserve their form also in transformations of the
form \er{fgjfjhgghyuyyuyughgint} i.e. in curvilinear coordinate
systems. In particular, the Maxwell Equations in every curvilinear
coordinate system have the form of
\er{khjhhkfgjfjhgghhgjghjhjkkkkgjghghuiiiulkkjlkklKKgfgjhjjghgjhhjhjhhhhhghhgtyytojjjjjuyiyuughgfgint}
or equivalently of
\er{khjhhkfgjfjhgghhgjghjhjkkkkgjghghuiiiulkkjlkklKKgfgjhjjghgjhhjhjhhhhhghhgtyytojjjjjuyiyuughgfghhjint}:
\begin{multline}\label{khjhhkfgjfjhgghhgjghjhjkkkkgjghghuiiiulkkjlkklKKgfgjhjjghgjhhjhjhhhhhghhgtyytojjjjjuyiyuughgfghjjhkpkint}
\sum_{j=0}^{3}\frac{\partial}{\partial
x^j}\left(\sum_{m=0}^{3}\sum_{n=0}^{3}g^{km}g^{jn}\left(\frac{\partial
A_n}{\partial x^m}-\frac{\partial A_m}{\partial
x^n}\right)\right)+\\
\sum_{j=0}^{3}\frac{1}{\sqrt{|\text{det}\,G|}}\frac{\partial}{\partial
x^j}\left(\sqrt{|\text{det}\,G|}\right)\left(\sum_{m=0}^{3}\sum_{n=0}^{3}g^{km}g^{jn}\left(\frac{\partial
A_n}{\partial x^m}-\frac{\partial A_m}{\partial x^n}\right)\right)
=-4\pi j^k\quad\quad\forall\, k=0,1,2,3,
\end{multline}
or equivalently:
\begin{equation}\label{khjhhkfgjfjhgghhgjghjhjkkkkgjghghuiiiulkkjlkklKKgfgjhjjghgjhhjhjhhhhhghhgtyytojjjjjuyiyuughgfghhjkkhjint}
\sum_{j=0}^{3}\frac{\partial}{\partial
x^j}\left(\sum_{m=0}^{3}\sum_{n=0}^{3}\sqrt{|\text{det}\,G|}\,g^{km}g^{jn}\left(\frac{\partial
A_n}{\partial x^m}-\frac{\partial A_m}{\partial x^n}\right)\right)
=-4\pi \sqrt{|\text{det}\,G|}\, j^k\quad\quad\forall\, k=0,1,2,3.
\end{equation}
Here $\{A_k\}_{k=0,1,2,3}$ is the four-covector of the
electromagnetic potential, $\{j^k\}_{k=0,1,2,3}$ is the four-vector
of the current and $G:=\{g_{kj}\}_{k,j=0,1,2,3}$,
$\{g^{kj}\}_{k,j=0,1,2,3}$ are pseudo-metric covariant and
contravariant tensors. Note, that in curvilinear coordinate system
we can have $\text{det}\,G\neq Const$ and thus we need to consider
the enlarged form
\er{khjhhkfgjfjhgghhgjghjhjkkkkgjghghuiiiulkkjlkklKKgfgjhjjghgjhhjhjhhhhhghhgtyytojjjjjuyiyuughgfgint}
instead of
\er{khjhhkfgjfjhgghhgjghjhjkkkkgjghghuiiiulkkjlkklKKgfgjhjjghgjhhjhjhhhhhghhgtyytojjjjjuyiyuuint}.
Moreover, the density of the Lagrangian of the electromagnetic field
in every curvilinear coordinate system in $\mathbb{R}^4$ also has a
form of \er{vhfffngghkjgghPPNggjgjjkgjoiuioigghint}:
\begin{multline}\label{vhfffngghkjgghPPNggjgjjkgjoiuioigghjhhhint}
L_1=\frac{1}{4\pi}\left(-\sum_{n=0}^{3}\sum_{k=0}^{3}\frac{1}{4}F^{nk}F_{nk}-\sum_{k=0}^{3}4\pi
j^k A_k\right)=\\
\frac{1}{4\pi}\left(-\sum_{n=0}^{3}\sum_{k=0}^{3}\sum_{m=0}^{3}\sum_{p=0}^{3}\frac{1}{4}g^{mn}g^{pk}\left(\frac{\partial
A_p}{\partial x^m}-\frac{\partial A_m}{\partial
x^p}\right)\left(\frac{\partial A_k}{\partial x^n}-\frac{\partial
A_n}{\partial x^k}\right)-\sum_{k=0}^{3}4\pi j^k A_k\right),
\end{multline}
where
%\er{huohuioy89gjjhjffffff3478zzrrZZZhjhhjhhjjhhffGGhjjh}
\begin{equation}\label{huohuioy89gjjhjffffff3478zzrrZZZhjhhjhhjjhhffGGhjjhjhhjhjint}
F_{ij}:=\frac{\partial A_j}{\partial x^i}-\frac{\partial
A_i}{\partial x^j}\quad\quad\forall\, i,j=0,1,2,3\,.
\end{equation}


Next the general Lagrangian of motion of the charged particle in the
gravitational and electromagnetic field
\er{btfffygtgyggyijhhkkSYSPNuhuygyygyggyuyyuyuykuhghgjjojint}
preserve its form in every curvilinear coordinate system:
\begin{equation}\label{btfffygtgyggyijhhkkSYSPNuhuygyygyggyuyyuyuykuhghgjjojhjfgint}
J_{\mathcal{G}}(\chi)=
\int_0^T\left\{-mc^2\;\mathcal{G}\left(\sum_{j=0}^{3}\sum_{k=0}^{3}g_{jk}\left(\chi(t)\right)\,\frac{d\chi^j}{dt}\,\frac{d\chi^k}{dt}\right)-\sum_{k=0}^{3}\sigma
A_k\left(\chi(t)\right)\,\frac{d\chi^k}{dt}\right\}dt.
\end{equation}
where $t$ is the global time, which is a scalar on the group
$\mathcal{S}$,
\begin{equation}\label{btfffjhgjghghijhhint}
\left(\chi^0(t),\chi^1(t),\chi^2(t),\chi^3(t)\right):=\left(\frac{1}{c}x^0(t),\frac{1}{c}x_1(t),\frac{1}{c}x_2(t),\frac{1}{c}x_3(t)\right),
\end{equation}
and $\left(x^0(t),x^1(t),x^2(t),x^3(t)\right)\in\mathbb{R}^4$ is a
four-dimensional space-time trajectory of the particle,
parameterized by the global time. Note that if we denote by $t$ the
scalar of global time, then in a general curvilinear coordinate
system the coordinate $x^0$ can differ from $ct$, and the equality
$x^0=ct$ valid, in general, only in cartesian non-inertial
coordinate systems. However, since the equality in
\er{fgjfjhgghhgjghjhjijhojihjhjjijhjjjjjuiijjjkhjhjhjuiiuuuyuint}
has a covariant form, the scalar of the global time $t$ satisfies
the following Eikonal-type equation in every curvilinear coordinate
system:
\begin{equation}\label{fgjfjhgghhgjghjhjijhojihjhjjijhjjjjjuiijjjkhjhjhjuiiuuuyuuoiuuiioint}
\sum_{j=0}^{n}\sum_{k=0}^{n}\,g^{jk}
%(x^0,x^1,x^2,x^3)
\,\frac{\partial
t}{\partial x^j}
%(x^0,x^1,x^2,x^3)
\,\frac{\partial t}{\partial
x^k}
%(x^0,x^1,x^2,x^3)
\,=\, \frac{1}{c^2}.
\end{equation}



Next, in the particular case of the relativistic-like Lagrangian
where $\mathcal{G}(\tau):=\sqrt{\tau}$, the Lagrangian in
\er{btfffygtgyggyijhhkkSYSPNuhuygyygyggyuyyuyuykuhghgjjojiyyjljlghint}
also preserve their form in every curvilinear coordinate system:
\begin{equation}\label{btfffygtgyggyijhhkkSYSPNuhuygyygyggyuyyuyuykuhghgjjojiyyjljlghhhhjhjint}
J_{rl}(\chi)=
\int_a^b\left\{-mc^2\;\sqrt{\left(\sum_{j=0}^{3}\sum_{k=0}^{3}g_{jk}\left(\chi(s)\right)\,\frac{d\chi^j}{ds}\,\frac{d\chi^k}{ds}\right)}-\sum_{k=0}^{3}\sigma
A_k\left(\chi(s)\right)\,\frac{d\chi^k}{ds}\right\}ds,
\end{equation}
where $s$ is the arbitrary parameter of the trajectory:
\begin{equation}\label{btfffjhgjghghijhhhuint}
\left(\chi^0(s),\chi^1(s),\chi^2(s),\chi^3(s)\right):=\left(\frac{1}{c}x^0(s),\frac{1}{c}x_1(s),\frac{1}{c}x_2(s),\frac{1}{c}x_3(s)\right).
\end{equation}
In particular we can take $s:=\chi^0$ in
\er{btfffygtgyggyijhhkkSYSPNuhuygyygyggyuyyuyuykuhghgjjojiyyjljlghhhhjhjint}.



Finally we would like to note the following fact: since in the
absence of essential gravitational masses, in every inertial
coordinate system the three-dimensional vectorial gravitational
potential $\vec v$ is a constant,
% and therefore,
there exists a unique inertial coordinate system where $\vec v=0$
everywhere. In this particular system by
\er{hoyuiouigyfg3478zzrrZZffGGhhjhjint} and the fact that $\vec v=0$
we have:
\begin{equation}\label{hoyuiouigyfg3478zzrrZZffGGhhjhjiuiint}
\begin{cases}
g_{00}=1\\
g_{ij}=-\delta_{ij}\quad\forall 1\leq i,j\leq 3\\
g_{0j}=g_{j0}=0\quad\forall 1\leq j\leq 3.
\end{cases}
\end{equation}
and thus the Maxwell equations are the same as in the Special
Relativity. Moreover, in this system the Lagrangian of the motion of
the particle of the form
\er{btfffygtgyggyijhhkkSYSPNuhuygyygyggyuyyuyuykuhghgjjojiyyjljlghhhhjhjint}
is also the same as in the Special Relativity. Thus, since Maxwell
equations
\er{khjhhkfgjfjhgghhgjghjhjkkkkgjghghuiiiulkkjlkklKKgfgjhjjghgjhhjhjhhhhhghhgtyytojjjjjuyiyuughgfghjjhkpkint}
and the Lagrangian of the motion of particles
\er{btfffygtgyggyijhhkkSYSPNuhuygyygyggyuyyuyuykuhghgjjojiyyjljlghhhhjhjint}
preserve their form in every curvilinear coordinate system of the
group $\mathcal{S}$, they stay the same as in Special Relativity
also in the case of every curvilinear coordinate system. Thus in the
particular case of $\mathcal{G}(\tau):=\sqrt{\tau}$ in
\er{btfffygtgyggyijhhkkSYSPNuhuygyygyggyuyyuyuykuhghgjjojhjfgint}
and in the absence of essential gravitational masses, the unique
formal mathematical difference between our model and the Special
Relativity is that in the frames of our model we consider the
Galilean Transformations as transformations of the change of
inertial coordinate systems and
\er{noninchgravortbstrjgghguittu2int} as transformations of the
change of non-inertial cartesian coordinate system, however the
Lorenz transformations lead to non-inertial \underline{curvilinear}
coordinate system. In contrast, in the Special Relativity the
fundamental role of the Lorenz transformations, i.e. the
transformations that preserve the form
\er{hoyuiouigyfg3478zzrrZZffGGhhjhjiuiint} of the pseudo-metric
tensor, is postulated as the role of transformations of the change
of inertial coordinate systems, and at the same time the Galilean
Transformations and transformations
\er{noninchgravortbstrjgghguittu2int} lead to
\underline{curvilinear} non-inertial coordinate system.





















































































































































\subsection{Macroscopic Electrodynamics in the presence of dielectric and/or magnetic
mediums} Consider system  \er{MaxVacFull1bjkgjhjhgjaaaint}
%\er{MaxVacFull1ninshtrhjkk}
in some inertial or non-inertial cartesian coordinate system inside
a dielectric and/or magnetic medium:
\begin{equation}\label{MaxVacFullnnnnGGint}
\begin{cases}
curl_{\vec x} \vec H_0= \frac{4\pi}{c}\left(\vec j+\vec j_m+\vec
j_p\right)+ \frac{1}{c}\frac{\partial \vec D_0}{\partial t}
%\quad\text{for}\;\;(\vec x,t)\in\R^3\times[0,+\infty),
\\
div_{\vec x} \vec D_0= 4\pi\left(\rho+\rho_p\right)
%\quad\quad\text{for}\;\;(\vec x,t)\in\R^3\times[0,+\infty),
\\
curl_{\vec x} \vec E+\frac{1}{c}\frac{\partial \vec B}{\partial t}=0
%\quad\quad\text{for}\;\;(\vec x,t)\in\R^3\times[0,+\infty),
\\
div_{\vec x} \vec B= 0,
%\quad\quad\text{for}\;\;(\vec x,t)\in\R^3\times[0,+\infty),
%D_0:=E+\frac{1}{c}\,v\times
%B\quad\quad\text{for}\;\;(\vec x,t)\in\R^3\times[0,+\infty)\\
%H:=B+\frac{1}{c}\,v\times
%D_0\quad\quad\text{for}\;\;(\vec x,t)\in\R^3\times[0,+\infty).
\end{cases}
\end{equation}
where $\vec E$ is the electric field, $\vec B$ is the magnetic
field, $\vec v:=\vec v(\vec x,t)$ is the vectorial gravitational
potential, $\rho$ is the average (macroscopic) charge density,
$\rho_p$ is the density of the charge of polarization, $\vec j$ is
the average (macroscopic) current density, $\vec j_m$ is the density
of the current of magnetization, $\vec j_p$ is the density of the
current of polarization and
\begin{equation}\label{MaxVacFullnnnngkggjkklhGGint}
%\begin{cases}
\vec D_0:=\vec E+\frac{1}{c}\,\vec v\times \vec
B\quad\text{and}\quad
%\quad\quad\text{for}\;\;(\vec x,t)\in\R^3\times[0,+\infty)\\
\vec H_0:=\vec B+\frac{1}{c}\,\vec v\times \vec D_0.
%\quad\quad\text{for}\;\;(\vec x,t)\in\R^3\times[0,+\infty).
%\end{cases}
\end{equation}
It is well known from the Lorentz theory that in the case of a
moving dielectric/magnetic medium
\begin{equation}\label{PolarGGint}
%\begin{cases}
\rho_p=-div_{\vec x} \vec P\quad\text{and}\quad
%\quad\quad\text{for}\;\;(\vec x,t)\in\R^3\times[0,+\infty)\\
\vec j_p=\frac{\partial \vec P}{\partial t}-curl_{\vec x}\left(
%\frac{1}{c}\,
\vec u\times \vec P\right),
%\quad\quad\text{for}\;\;(\vec x,t)\in\R^3\times[0,+\infty).
%\end{cases}
\end{equation}
where $\vec P:\R^3\times[0,+\infty)\to\R^3$ is the field of
polarization and $\vec u:=\vec u(\vec x,t)$ is the field of
velocities of the dielectric medium (see also \cite{PC}, page 610).
Furthermore,
\begin{equation}\label{MagnetGGint}
\vec j_m=c\, curl_{\vec x} \vec M,
\end{equation}
where $\vec M:\R^3\times[0,+\infty)\to\R^3$ is the field of
magnetization. Thus if we consider
\begin{equation}\label{OprdddGGint}
\vec D:=\vec D_0+4\pi \vec P=\vec E+\frac{1}{c}\,\vec v\times \vec
B+4\pi \vec P,
\end{equation}
and
\begin{multline}\label{Oprddd1GGint}
%\quad\quad\text{for}\;\;(\vec x,t)\in\R^3\times[0,+\infty)\\
\vec H:=\vec H_0-4\pi \vec M+\frac{4\pi}{c}\,\vec u\times \vec P
%=\vec B+\frac{1}{c}\,\vec v\times \vec D_0+\frac{4\pi}{c}\,\vec u\times \vec P-4\pi \vec M\\
%%%B+\frac{1}{c}\,v\times D-4\pi M,
=\vec B+\frac{4\pi}{c}\,\vec u\times \vec P+\frac{1}{c}\,\vec
v\times \vec E+\frac{1}{c}\,\vec v\times\left(\frac{1}{c}\,\vec
v\times \vec B\right)-4\pi \vec M,
%\quad\quad\text{for}\;\;(\vec x,t)\in\R^3\times[0,+\infty).
\end{multline}
we obtain the usual Maxwell equations of the form:
\begin{equation}\label{MaxMedFullGGint}
\begin{cases}
curl_{\vec x} \vec H= \frac{4\pi}{c}\vec j+
\frac{1}{c}\frac{\partial \vec D}{\partial t}
%\quad\text{for}\;\;(\vec x,t)\in\R^3\times[0,+\infty),
\\
div_{\vec x} \vec D= 4\pi\rho
%\quad\quad\text{for}\;\;(\vec x,t)\in\R^3\times[0,+\infty),
\\
curl_{\vec x} \vec E+\frac{1}{c}\frac{\partial \vec B}{\partial t}=
0
%\quad\quad\text{for}\;\;(\vec x,t)\in\R^3\times[0,+\infty),
\\
div_{\vec x} \vec B=0.
%\quad\quad\text{for}\;\;(\vec x,t)\in\R^3\times[0,+\infty),
%
%D_0:=E+\frac{1}{c}\,v\times
%B\quad\quad\text{for}\;\;(\vec x,t)\in\R^3\times[0,+\infty)\\
%H:=B+\frac{1}{c}\,v\times
%D_0\quad\quad\text{for}\;\;(\vec x,t)\in\R^3\times[0,+\infty).
\end{cases}
\end{equation}
We call $\vec D$ by the electric displacement field and $\vec H$ by
the $\vec H$-magnetic field in a medium.

Next,
%consider the change of certain non-inertial cartesian coordinate system $(*)$ to another cartesian coordinate system $(**)$ of the form \er{noninchgravortbstrjgghguittu2int}.
in section \ref{DMPGG} we prove that the laws of transformation of
electromagnetic fields in dielectric/magnetic medium, under the
change of non-inertial cartesian coordinate system of the form
\er{noninchgravortbstrjgghguittu2int}, are exactly the same as
\er{yuythfgfyftydtydtydtyddyyyhhddhhhredPPN111hgghjgint} in the
vacuum, i.e. having the form of
\begin{equation}\label{guigikvbvbggjklhjkkgjgGGGGint}
\begin{cases}
\vec D'=A(t)\cdot \vec D\\
\vec B'=A(t)\cdot\vec B\\
\vec E'=A(t)\cdot\vec E-\frac{1}{c}\,\left(\frac{dA}{dt}(t)\cdot\vec
x+\frac{d\vec
z}{dt}(t)\right)\times \left(A(t)\cdot\vec B\right)\\
\vec H'=A(t)\cdot\vec H+\frac{1}{c}\,\left(\frac{dA}{dt}(t)\cdot\vec
x+\frac{d\vec z}{dt}(t)\right)\times \left(A(t)\cdot\vec D\right),
\end{cases}
\end{equation}
provided that
\begin{equation}\label{guigikvbvbGGint}
\begin{cases}
%\vec E'=A(t)\cdot\vec E-\frac{1}{c}\,\left(A'(t)\cdot\vec x+\frac{d\vec z}{dt}(t)\right)\times \left(A(t)\cdot\vec B\right),\\
%\vec B'=A(t)\cdot\vec B,\\
\vec P'=A(t)\cdot\vec P,\\
\vec M'=A(t)\cdot\vec M,\\
%\vec v'=A(t)\cdot \vec v+A'(t)\cdot\vec x+\vec w(t)\\
\vec u'=A(t)\cdot \vec u+\frac{dA}{dt}(t)\cdot\vec x+\frac{d\vec z}{dt}(t)\\
\vec v'=A(t)\cdot \vec v+\frac{dA}{dt}(t)\cdot\vec x+\frac{d\vec
z}{dt}(t).
\end{cases}
\end{equation}




Next it is well known that in the case of simplest homogenous
isotropic dielectrics and/or magnetics we have
\begin{equation}\label{EBDHTrans444knbuihuigGGint}
\begin{cases}
\vec P=\gamma\left(\vec E+\frac{1}{c}\,\vec u\times \vec B\right),\\
\vec M=\kappa\vec B,
\end{cases}
\end{equation}
where $\gamma$ and $\kappa$ are material coefficients.
%are invariant under the following transformations
%\begin{equation}\label{Gal444}
%\begin{cases}
%x'=x+wt,\\
%t'=t.
%\end{cases}
%\end{equation}
%and
Using \er{guigikvbvbGGint}, it can be easily seen that the laws in
\er{EBDHTrans444knbuihuigGGint} are invariant under the changes of
inertial or non-inertial cartesian coordinate system. Next denoting
$\gamma_0=\frac{1}{1+4\pi\gamma}$ and $\kappa_0=1-4\pi \kappa$ and
defining the speed-like vector field
\begin{equation}\label{OprdddsimGGffyhjyhhzzkk}
\vec {\tilde u}:=\left(\gamma_0\vec v+(1-\gamma_0)\vec
u\right)=\frac{1}{1+4\pi\gamma}\left(\vec v+4\pi\gamma\vec u\right),
\end{equation}
by plugging \er{EBDHTrans444knbuihuigGGint} into \er{OprdddGGint}
and \er{Oprddd1GGint} we deduce
\begin{equation}\label{OprdddsimGGint}
\vec E=\gamma_0\vec D-\frac{1}{c}\,\vec {\tilde u}\times \vec B,
\end{equation}
and
\begin{equation}\label{Oprddd1simGGint}
%\quad\quad\text{for}\;\;(\vec x,t)\in\R^3\times[0,+\infty)\\
\vec H=\kappa_0\vec B+\frac{1}{c}\,\vec {\tilde u}\times \vec
D+\frac{(1-\gamma_0)}{c^2}\,(\vec u-\vec v)\times\left(\left(\vec
u-\vec v\right)\times \vec B\right),
%\quad\quad\text{for}\;\;(\vec x,t)\in\R^3\times[0,+\infty).
\end{equation}
where we call $\gamma_0$ and $\kappa_0$ dielectric and magnetic
permeability of the medium. Thus by \er{MaxMedFullGGint},
\er{OprdddsimGGffyhjyhhzzkk} \er{OprdddsimGGint} and
\er{Oprddd1simGGint} we have
\begin{equation}\label{MaxMedFullGGffykkzzkk}
\begin{cases}
curl_{\vec x} \vec H=\frac{4\pi}{c}\vec j+
\frac{1}{c}\frac{\partial \vec D}{\partial t},\\
div_{\vec x} \vec D=4\pi\rho,\\
curl_{\vec x} \vec E+\frac{1}{c}\frac{\partial \vec B}{\partial t}=0,\\
div_{\vec x} \vec B=0,\\
\vec E=\gamma_0\vec D-\frac{1}{c}\,\vec {\tilde u}\times \vec B,\\
\vec H=\kappa_0\vec B+\frac{1}{c}\,\vec {\tilde u}\times \vec
D+\frac{(1-\gamma_0)}{c^2}\,(\vec u-\vec v)\times\left(\left(\vec
u-\vec v\right)\times \vec B\right),\\
\vec {\tilde u}:=\left(\gamma_0\vec v+(1-\gamma_0)\vec u\right),
%D_0:=E+\frac{1}{c}\,v\times
%B\quad\quad\text{for}\;\;(\vec x,t)\in\R^3\times[0,+\infty)\\
%H:=B+\frac{1}{c}\,v\times
%D_0\quad\quad\text{for}\;\;(\vec x,t)\in\R^3\times[0,+\infty).
\end{cases}
\end{equation}
where $\vec {\tilde u}$ is a speed-like vector field that we call
the optical displacement of the moving medium. Note that for the
case $\gamma_0=1$ and $\kappa_0=1$, the system
\er{MaxMedFullGGffykkzzkk} is exactly the same as the corresponding
system in the vacuum. The equations in \er{MaxMedFullGGffykkzzkk}
take much simpler forms in the case where the quantity
\begin{equation}\label{OprdddsimGGffyhjyhhtygrffgfzzkk}
\frac{|1-\gamma_0|\cdot|\vec u-\vec v|^2}{c^2}\ll 1
\end{equation} is
negligible, that happens if the absolute value of the difference
between the medium velocity and vectorial gravitational potential is
much less then the constant $c$ or/and $\gamma_0$ is close to the
value $1$. Indeed, in this case, instead of \er{OprdddsimGGint} and
\er{Oprddd1simGGint} we obtain the following relations:
\begin{align}\label{OprdddsimsimsimGGint}
\vec E=\gamma_0\vec D-\frac{1}{c}\,\vec {\tilde u}\times \vec B,\\
\label{Oprddd1simsimsimGGint}
%\quad\quad\text{for}\;\;(\vec x,t)\in\R^3\times[0,+\infty)\\
\vec H=\kappa_0\vec B+\frac{1}{c}\,\vec {\tilde u}\times \vec D.
%\quad\quad\text{for}\;\;(\vec x,t)\in\R^3\times[0,+\infty).
\end{align}
As a consequence we obtain the full system of Maxwell equations in
the medium:
\begin{equation}\label{MaxMedFullGGffykkhjhhzzkk}
\begin{cases}
curl_{\vec x} \vec H=\frac{4\pi}{c}\vec j+
\frac{1}{c}\frac{\partial \vec D}{\partial t},\\
div_{\vec x} \vec D=4\pi\rho,\\
curl_{\vec x} \vec E+\frac{1}{c}\frac{\partial \vec B}{\partial t}=0,\\
div_{\vec x} \vec B=0\\
\vec E=\gamma_0\vec D-\frac{1}{c}\,\vec {\tilde u}\times \vec B\\
\vec H=\kappa_0\vec B+\frac{1}{c}\,\vec {\tilde u}\times \vec D,\\
\vec {\tilde u}=\left(\gamma_0\vec v+(1-\gamma_0)\vec u\right),
%D_0:=E+\frac{1}{c}\,v\times
%B\quad\quad\text{for}\;\;(\vec x,t)\in\R^3\times[0,+\infty)\\
%H:=B+\frac{1}{c}\,v\times
%D_0\quad\quad\text{for}\;\;(\vec x,t)\in\R^3\times[0,+\infty).
\end{cases}
\end{equation}
where $\vec {\tilde u}$ is the speed-like vector field and
$\gamma_0$ and $\kappa_0$ are dielectric and magnetic permeability
of the medium. Note that \er{MaxMedFullGGffykkhjhhzzkk} is analogous
to the system of Maxwell equations in the vacuum and it is also
invariant under the change of inertial or non-inertial cartesian
coordinate system, provided that under this transformation we have
\er{guigikvbvbGGint}.






Next, it is well known that the Ohm's Law in a conducting medium has
the form
\begin{equation}\label{vjhfhjtjhjhuyyiyGGint}
\vec j-\rho\,\vec u=\varepsilon\left(\vec E+\frac{1}{c}\,\vec
u\times \vec B\right),
\end{equation}
where $\vec u$ is the velocity of the medium and $\varepsilon$ is a
material coefficient. As before, using
%\er{yuythfgfyftydtydtydtyddyyyhhddhhhredPPN111hgghjgint}
\er{guigikvbvbggjklhjkkgjgGGGGint}, it can be easily seen that the
Ohm's Law is invariant under the changes of inertial or non-inertial
cartesian coordinate system.





























































































\subsection{Some further consequences of Maxwell equations}
%\label{CM}
%\subsubsection{General case}
%\label{gcCM}
Again consider the system of Maxwell equations in the vacuum or in a
medium of the form \er{MaxMedFullGGffykkhjhhzzkk}:
%\er{MaxVacFull1bjkgjhjhgjaaaPPN}:
\begin{equation}\label{MaxVacFullPPNffGGint}
\begin{cases}
curl_{\vec x} \vec H=\frac{4\pi}{c}\vec j+
\frac{1}{c}\frac{\partial \vec D}{\partial t},\\
div_{\vec x} \vec D=4\pi\rho,\\
curl_{\vec x} \vec E+\frac{1}{c}\frac{\partial \vec B}{\partial t}=0,\\
div_{\vec x} \vec B=0\\
\vec E=\gamma_0\vec D-\frac{1}{c}\,\vec {\tilde u}\times \vec B\\
\vec H=\kappa_0\vec B+\frac{1}{c}\,\vec {\tilde u}\times \vec D,\\
\vec {\tilde u}=\left(\gamma_0\vec v+(1-\gamma_0)\vec u\right),
%D_0:=E+\frac{1}{c}\,v\times
%B\quad\quad\text{for}\;\;(\vec x,t)\in\R^3\times[0,+\infty)\\
%H:=B+\frac{1}{c}\,v\times
%D_0\quad\quad\text{for}\;\;(\vec x,t)\in\R^3\times[0,+\infty).
\end{cases}
\end{equation}
%
%
%
\begin{comment}
\begin{equation}\label{MaxVacFullPPNffGGint}
\begin{cases}
curl_{\vec x} \vec H\equiv \frac{4\pi}{c}\vec
j+\frac{1}{c}\frac{\partial
\vec D}{\partial t},\\
%\quad\text{for}\;\;(\vec x,t)\in\R^3\times[0,+\infty),\\
div_{\vec x} \vec D\equiv 4\pi\rho,\\
%\quad\quad\text{for}\;\;(\vec x,t)\in\R^3\times[0,+\infty),\\
curl_{\vec x} \vec E+\frac{1}{c}\frac{\partial \vec B}{\partial t}\equiv 0,\\
%\quad\quad\text{for}\;\;(\vec x,t)\in\R^3\times[0,+\infty),\\
div_{\vec x} \vec B\equiv 0,\\
%\quad\quad\text{for}\;\;(\vec x,t)\in\R^3\times[0,+\infty),\\
\vec E=\vec D-\frac{1}{c}\,\vec v\times \vec B,\\
%\quad\quad\text{for}\;\;(\vec x,t)\in\R^3\times[0,+\infty)\\
\vec H=\vec B+\frac{1}{c}\,\vec v\times \vec D,
%\quad\quad\text{for}\;\;(\vec x,t)\in\R^3\times[0,+\infty).
\end{cases}
\end{equation}
\end{comment}
%
%
%
where $\gamma_0\neq 0$ and $\kappa_0\neq 0$ are material
coefficients, $\vec v$ is the vectorial gravitational potential $u$
is the medium velocity and $\vec {\tilde u}=\left(\gamma_0\vec
v+(1-\gamma_0)\vec u\right)$ is the speed-like vector field. Remind
that in the case of the vacuum we have $\gamma_0=\kappa_0=1$, $\vec
{\tilde u}=\vec v$ and equations \er{MaxVacFullPPNffGGint} are
precise (in the frames of our model). Otherwise, in the case
$\gamma_0\neq 1$ equations \er{MaxVacFullPPNffGGint} are just an
approximation that is good enough for the case:
\begin{equation}\label{OprdddsimGGffyhjyhhtygrffgfzzjjjint}
\frac{|1-\gamma_0|\cdot|\vec u-\vec v|^2}{c^2}\ll 1.
\end{equation}
Throughout this section we study equation \er{MaxVacFullPPNffGGint}
in domains where we assume that the coefficients $\gamma_0\neq 0$
and $\kappa_0\neq 0$ vary sufficiently slow on the place and time
and thus their spatial and temporal derivatives are negligible. Next
again by the third and the fourth equations in
\er{MaxVacFullPPNffGGint} we can write
\begin{equation}\label{MaxVacFull1bjkgjhjhgjgjgkjfhjfdghghligioiuittrPPNggint}
\begin{cases}
\vec B\equiv curl_{\vec x} \vec A,\\
\vec E\equiv-\nabla_{\vec x}\Psi-\frac{1}{c}\frac{\partial\vec
A}{\partial t},
%,\\ div_{\vec x}\vec A\equiv 0,
\end{cases}
\end{equation}
where $\Psi$ and $\vec A$ are the usual scalar and the vectorial
electromagnetic potentials. Then by
\er{MaxVacFull1bjkgjhjhgjgjgkjfhjfdghghligioiuittrPPNggint} and
\er{MaxVacFullPPNffGGint} we have
\begin{equation}\label{vhfffngghPPN333yuyuGGint}
\begin{cases}
\vec B= curl_{\vec x} \vec A\\
\vec E=-\nabla_{\vec x}\Psi-\frac{1}{c}\frac{\partial\vec
A}{\partial t}\\
 \vec D=-\frac{1}{\gamma_0}\nabla_{\vec
x}\Psi-\frac{1}{\gamma_0 c}\frac{\partial\vec A}{\partial t}+\frac{1}{c\gamma_0}\vec {\tilde u}\times curl_{\vec x}\vec A\\
\vec H= \kappa_0 \,curl_{\vec x} \vec A+\frac{1}{c}\,\vec {\tilde
u}\times\left(-\frac{1}{\gamma_0}\nabla_{\vec
x}\Psi-\frac{1}{\gamma_0 c}\frac{\partial\vec A}{\partial
t}+\frac{1}{\gamma_0 c}\vec {\tilde u}\times curl_{\vec x}\vec
A\right).
%\\ div_{\vec x}\vec A\equiv 0.
\end{cases}
\end{equation}
Next we remind the definition of the proper scalar electromagnetic
potential:
\begin{equation}\label{vhfffngghhjghhgPPNghghghutghffugghjhjkjjklggkkkint}
\Psi_0:=\Psi-\frac{1}{c}\vec A\cdot\vec v,
\end{equation}
and remind also that $\vec A$ is a proper vector field and $\Psi_0$
is a proper scalar field. Then in the case of the medium we also
define an additional scalar electromagnetic potential:
\begin{equation}\label{vhfffngghhjghhgPPNghghghutghffugghjhjkjjklggint}
\Psi_1:=\Psi-\frac{1}{c}\vec A\cdot\vec {\tilde u}.
\end{equation}
Then, since $\vec A$ is a proper vector field, we deduce that
$\Psi_1$ is also a proper scalar field. Moreover, in the case of the
vacuum or more generally in the case where $\gamma_0\approx 1$ we
have $\Psi_1=\Psi_0$. Thus by
\er{vhfffngghhjghhgPPNghghghutghffugghjhjkjjklggint} we rewrite
\er{vhfffngghPPN333yuyuGGint} as:
\begin{equation}\label{vhfffngghPPNffGGint}
\begin{cases}
\vec B= curl_{\vec x} \vec A\\
\vec E=-\nabla_{\vec x}\Psi_1-\frac{1}{c}\frac{\partial\vec
A}{\partial t}-\frac{1}{c}\nabla_{\vec x}\left(\vec A\cdot\vec
{\tilde u}\right)
%=-\nabla_{\vec x}\Psi-\frac{1}{c}\frac{\partial\vec A}{\partial t}
\\
 \vec D=-\frac{1}{\gamma_0}\nabla_{\vec
x}\Psi_1-\frac{1}{\gamma_0 c}\left(\frac{\partial\vec A}{\partial
t}-\vec {\tilde u}\times curl_{\vec x}\vec A+\nabla_{\vec
x}\left(\vec A\cdot\vec {\tilde u}\right)\right)
%=-\nabla_{\vec x}\Psi-\frac{1}{c}\frac{\partial\vec A}{\partial t}+\frac{1}{c}\vec v\times curl_{\vec x}\vec A
\\
\vec H= \kappa_0\,curl_{\vec x} \vec A-\frac{1}{c}\,\vec {\tilde
u}\times
%\left(-\nabla_{\vec x}\Psi-\frac{1}{c}\frac{\partial\vec A}{\partial t}+\frac{1}{c}\vec v\times curl_{\vec x}\vec A\right).
\left(\frac{1}{\gamma_0}\nabla_{\vec x}\Psi_1+\frac{1}{\gamma_0
c}\left(\frac{\partial\vec A}{\partial t}-\vec {\tilde u}\times
curl_{\vec x}\vec A+\nabla_{\vec x}\left(\vec A\cdot\vec {\tilde
u}\right)\right)\right).
%\\ div_{\vec x}\vec A\equiv 0.
\end{cases}
\end{equation}
%%%%%
%
%
%
\begin{comment}
Using Proposition \ref{yghgjtgyrtrtint} we rewrite the third
equation in \er{vhfffngghPPNffGGint} as
\begin{equation}\label{vhfffngghPPNffGG1int}
\vec D=-\frac{1}{\gamma_0}\nabla_{\vec x}\Psi_1-\frac{1}{\gamma_0
c}\left(\frac{\partial\vec A}{\partial t}-curl_{\vec x}\left(\vec
{\tilde u}\times\vec A\right)+\left(div_{\vec x}\vec A\right)\vec
{\tilde u}+\left(d_{\vec x}\vec {\tilde u}+\left\{d_{\vec x}\vec
{\tilde u}\right\}^T\right)\cdot\vec A-\left(div_{\vec x}\vec
{\tilde u}\right)\vec A\right).
\end{equation}
Then by \er{vhfffngghPPNffGG1int}, \er{vhfffngghPPNffGGint} and
\er{MaxVacFullPPNffGGint} we have
\begin{multline}\label{MaxVacFullPPNmmmffGGint}
\frac{1}{\gamma_0 c}\left(\frac{\partial}{\partial t}\left(div_{\vec
x}\vec A\right)+div_{\vec x} \left\{\left(div_{\vec x}\vec
A\right)\vec {\tilde u}\right\}\right)\\+\frac{1}{\gamma_0
c}\,div_{\vec x} \left\{\left(d_{\vec x}\vec {\tilde
u}+\left\{d_{\vec x}\vec {\tilde u}\right\}^T\right)\cdot\vec
A-\left(div_{\vec x}\vec {\tilde u}\right)\vec
A\right\}+\frac{1}{\gamma_0}\,\Delta_{\vec x}\Psi_1=-4\pi\rho
\end{multline}
and
\begin{multline}\label{MaxVacFullPPNnnnffGGint}
curl_{\vec x} \left\{\kappa_0\,curl_{\vec x} \vec
A-\frac{1}{\gamma_0 c}\,\vec {\tilde u}\times
%\left(-\nabla_{\vec x}\Psi-\frac{1}{c}\frac{\partial\vec A}{\partial t}+\frac{1}{c}\vec v\times curl_{\vec x}\vec A\right).
\left(\nabla_{\vec x}\Psi_1+\frac{1}{c}\left(\frac{\partial\vec
A}{\partial t}-\vec {\tilde u}\times curl_{\vec x}\vec
A+\nabla_{\vec x}\left(\vec A\cdot\vec {\tilde
u}\right)\right)\right)\right\}=\\
\frac{4\pi}{c}\vec j+\frac{1}{\gamma_0 c}\frac{\partial}{\partial
t}\left\{-\nabla_{\vec x}\Psi_1-\frac{1}{c}\left(\frac{\partial\vec
A}{\partial t}-\vec {\tilde u}\times curl_{\vec x}\vec
A+\nabla_{\vec x}\left(\vec A\cdot\vec {\tilde
u}\right)\right)\right\}.
\end{multline}
Then, we rewrite \er{MaxVacFullPPNmmmffGGint} as:
\begin{multline}\label{MaxVacFullPPNmmmffffffhhGGint}
-\frac{1}{\gamma_0 c}\left(\frac{\partial}{\partial
t}\left(div_{\vec x}\vec A\right)+div_{\vec x}
\left\{\left(div_{\vec x}\vec A\right)\vec {\tilde
u}\right\}\right)-\frac{1}{\gamma_0}\,\Delta_{\vec
x}\Psi_1\\=4\pi\rho+\frac{1}{\gamma_0 c}\,div_{\vec x}
\left\{\left(d_{\vec x}\vec {\tilde u}+\left\{d_{\vec x}\vec {\tilde
u}\right\}^T\right)\cdot\vec A-\left(div_{\vec x}\vec {\tilde
u}\right)\vec A\right\},
\end{multline}
and \er{MaxVacFullPPNnnnffGGint} as:
\begin{multline}\label{MaxVacFullPPNnnnffffffhhGGint}
-\kappa_0\,\Delta_{\vec x}\vec A-\frac{1}{\gamma_0 c^2}curl_{\vec x}
\left\{\vec {\tilde u}\times
%\left(-\nabla_{\vec x}\Psi-\frac{1}{c}\frac{\partial\vec A}{\partial t}+\frac{1}{c}\vec v\times curl_{\vec x}\vec A\right).
\left(\frac{\partial\vec A}{\partial t}-\vec {\tilde u}\times
curl_{\vec x}\vec A+\nabla_{\vec x}\left(\vec A\cdot\vec {\tilde
u}\right)\right)\right\}= \frac{4\pi}{c}\vec j\\-\frac{1}{\gamma_0
c^2}\frac{\partial}{\partial t}\left(\frac{\partial\vec A}{\partial
t}-\vec {\tilde u}\times curl_{\vec x}\vec A+\nabla_{\vec
x}\left(\vec A\cdot\vec {\tilde u}\right)\right)-\left(\nabla_{\vec
x}\left(\frac{1}{\gamma_0 c}\frac{\partial}{\partial
t}\Psi_1+\kappa_0\,div_{\vec x} \vec A\right)-\frac{1}{\gamma_0
c}curl_{\vec x} \left(\vec {\tilde u}\times
%\left(-\nabla_{\vec x}\Psi-\frac{1}{c}\frac{\partial\vec A}{\partial t}+\frac{1}{c}\vec v\times curl_{\vec x}\vec A\right).
\nabla_{\vec x}\Psi_1\right)\right).
\end{multline}
Then by \er{MaxVacFullPPNnnnffffffhhGGint},
\er{MaxVacFullPPNmmmffffffhhGGint} and \er{apfrm6} we deduce:
\begin{multline}\label{MaxVacFullPPNnnnffffffyuughhhGGint}
-\kappa_0\,\Delta_{\vec x}\vec A-\frac{1}{\gamma_0 c^2}curl_{\vec x}
\left\{\vec {\tilde u}\times
%\left(-\nabla_{\vec x}\Psi-\frac{1}{c}\frac{\partial\vec A}{\partial t}+\frac{1}{c}\vec v\times curl_{\vec x}\vec A\right).
\left(\frac{\partial\vec A}{\partial t}-\vec {\tilde u}\times
curl_{\vec x}\vec A+\nabla_{\vec x}\left(\vec A\cdot\vec {\tilde
u}\right)\right)\right\}=\frac{4\pi}{c}\vec j\\-\nabla_{\vec
x}\left(\frac{1}{\gamma_0 c}\frac{\partial}{\partial
t}\Psi_1+\frac{1}{\gamma_0 c}\vec {\tilde u}\cdot\nabla_{\vec
x}\Psi_1+\kappa_0\,div_{\vec x} \vec A\right)-\frac{1}{\gamma_0
c^2}\frac{\partial}{\partial t}\left(\frac{\partial\vec A}{\partial
t}-\vec {\tilde u}\times curl_{\vec x}\vec A+\nabla_{\vec
x}\left(\vec A\cdot\vec {\tilde u}\right)\right)\\+\frac{1}{\gamma_0
c}\left(\nabla_{\vec x}\left(\vec {\tilde u}\cdot\nabla_{\vec
x}\Psi_1\right)+curl_{\vec x} \left(\vec {\tilde u}\times
%\left(-\nabla_{\vec x}\Psi-\frac{1}{c}\frac{\partial\vec A}{\partial t}+\frac{1}{c}\vec v\times curl_{\vec x}\vec A\right).
\nabla_{\vec x}\Psi_1\right)\right) =\frac{4\pi}{c}\vec
j\\-\nabla_{\vec x}\left(\frac{1}{\gamma_0
c}\frac{\partial}{\partial t}\Psi_1+\frac{1}{\gamma_0 c}\vec {\tilde
u}\cdot\nabla_{\vec x}\Psi_1+\kappa_0\,div_{\vec x} \vec
A\right)-\frac{1}{\gamma_0 c^2}\frac{\partial}{\partial
t}\left(\frac{\partial\vec A}{\partial t}-\vec {\tilde u}\times
curl_{\vec x}\vec A+\nabla_{\vec x}\left(\vec A\cdot\vec {\tilde
u}\right)\right)\\+\frac{1}{\gamma_0 c}\left(\left(d_{\vec x}\vec
{\tilde u}+\left\{d_{\vec x}\vec {\tilde u}\right\}^T\right)\cdot
\nabla_{\vec x}\Psi_1-\left(div_{\vec x}\vec {\tilde
u}\right)\nabla_{\vec x}\Psi_1\right)+\frac{1}{\gamma_0
c}\left(\Delta_{\vec x}\Psi_1\right)\vec {\tilde
u}=\frac{4\pi}{c}\left(\vec j-\rho\vec {\tilde
u}\right)\\-\nabla_{\vec x}\left(\frac{1}{\gamma_0
c}\frac{\partial}{\partial t}\Psi_1+\frac{1}{\gamma_0 c}\vec {\tilde
u}\cdot\nabla_{\vec x}\Psi_1+\kappa_0\,div_{\vec x} \vec A\right)
-\frac{1}{\gamma_0 c^2}\frac{\partial}{\partial
t}\left(\frac{\partial\vec A}{\partial t}-\vec {\tilde u}\times
curl_{\vec x}\vec A+\nabla_{\vec x}\left(\vec A\cdot\vec {\tilde
u}\right)\right)\\+\frac{1}{\gamma_0 c}\left(\left(d_{\vec x}\vec
{\tilde u}+\left\{d_{\vec x}\vec {\tilde u}\right\}^T\right)\cdot
\nabla_{\vec x}\Psi_1-\left(div_{\vec x}\vec {\tilde
u}\right)\nabla_{\vec x}\Psi_1\right)\\-\frac{1}{\gamma_0
c^2}\left(\left(\frac{\partial}{\partial t}\left(div_{\vec x}\vec
A\right)+div_{\vec x} \left\{\left(div_{\vec x}\vec A\right)\vec
{\tilde u}\right\}\right)+div_{\vec x} \left\{\left(d_{\vec x}\vec
{\tilde u}+\left\{d_{\vec x}\vec {\tilde
u}\right\}^T\right)\cdot\vec A-\left(div_{\vec x}\vec {\tilde
u}\right)\vec A\right\}\right)\vec {\tilde u}.
\end{multline}
So we have
\begin{multline}\label{MaxVacFullPPNnnnffffffyuughjhjhjhhjhhGGint}
-\kappa_0\,\Delta_{\vec x}\vec A-\frac{1}{\gamma_0 c^2}curl_{\vec x}
\left\{\vec {\tilde u}\times
%\left(-\nabla_{\vec x}\Psi-\frac{1}{c}\frac{\partial\vec A}{\partial t}+\frac{1}{c}\vec v\times curl_{\vec x}\vec A\right).
\left(\frac{\partial\vec A}{\partial t}-\vec {\tilde u}\times
curl_{\vec x}\vec A+\nabla_{\vec x}\left(\vec A\cdot\vec {\tilde
u}\right)\right)\right\}=\frac{4\pi}{c}\left(\vec j-\rho\vec {\tilde
u}\right)\\-\nabla_{\vec x}\left(\frac{1}{\gamma_0
c}\frac{\partial}{\partial t}\Psi_1+\frac{1}{\gamma_0 c}\vec {\tilde
u}\cdot\nabla_{\vec x}\Psi_1+\kappa_0\,div_{\vec x} \vec
A\right)-\frac{1}{\gamma_0 c^2}\frac{\partial}{\partial
t}\left(\frac{\partial\vec A}{\partial t}-\vec {\tilde u}\times
curl_{\vec x}\vec A+\nabla_{\vec x}\left(\vec A\cdot\vec {\tilde
u}\right)\right)\\+\frac{1}{\gamma_0 c}\left(\left(d_{\vec x}\vec
{\tilde u}+\left\{d_{\vec x}\vec {\tilde u}\right\}^T\right)\cdot
\nabla_{\vec x}\Psi_1-\left(div_{\vec x}\vec {\tilde
u}\right)\nabla_{\vec x}\Psi_1\right)\\-\frac{1}{\gamma_0
c^2}\left(\left(\frac{\partial}{\partial t}\left(div_{\vec x}\vec
A\right)+div_{\vec x} \left\{\left(div_{\vec x}\vec A\right)\vec
{\tilde u}\right\}\right)+div_{\vec x} \left\{\left(d_{\vec x}\vec
{\tilde u}+\left\{d_{\vec x}\vec {\tilde
u}\right\}^T\right)\cdot\vec A-\left(div_{\vec x}\vec {\tilde
u}\right)\vec A\right\}\right)\vec {\tilde u},
\end{multline}
that we rewrite as
\begin{multline}\label{MaxVacFullPPNnnnffffffyuughjhjhjhhjjkjhkkjhhGGint}
-\kappa_0\,\Delta_{\vec x}\vec A= \frac{4\pi}{c}\left(\vec
j-\rho\vec {\tilde u}\right)+\frac{1}{\gamma_0 c}\left(\left(d_{\vec
x}\vec {\tilde u}+\left\{d_{\vec x}\vec {\tilde
u}\right\}^T\right)\cdot \nabla_{\vec x}\Psi_1-\left(div_{\vec
x}\vec {\tilde u}\right)\nabla_{\vec x}\Psi_1\right)\\-\nabla_{\vec
x}\left(\frac{1}{\gamma_0 c}\frac{\partial}{\partial
t}\Psi_1+\frac{1}{\gamma_0 c}\vec {\tilde u}\cdot\nabla_{\vec
x}\Psi_1+\kappa_0\,div_{\vec x} \vec A\right)-\frac{1}{\gamma_0
c^2}\frac{\partial}{\partial t}\left(\frac{\partial\vec A}{\partial
t}-\vec {\tilde u}\times curl_{\vec x}\vec A+\nabla_{\vec
x}\left(\vec A\cdot\vec {\tilde u}\right)\right)\\+\frac{1}{\gamma_0
c^2}curl_{\vec x} \left\{\vec {\tilde u}\times
%\left(-\nabla_{\vec x}\Psi-\frac{1}{c}\frac{\partial\vec A}{\partial t}+\frac{1}{c}\vec v\times curl_{\vec x}\vec A\right).
\left(\frac{\partial\vec A}{\partial t}-\vec {\tilde u}\times
curl_{\vec x}\vec A+\nabla_{\vec x}\left(\vec A\cdot\vec {\tilde
u}\right)\right)\right\}\\-\frac{1}{\gamma_0 c^2}\left(div_{\vec
x}\left(\frac{\partial\vec A}{\partial t}-\vec {\tilde u}\times
curl_{\vec x}\vec A+\nabla_{\vec x}\left(\vec A\cdot\vec {\tilde
u}\right)\right)\right)\vec {\tilde u},
\end{multline}
Thus by \er{MaxVacFullPPNmmmffffffhhGGint} we have
\end{comment}
%
%
%
%
Then by inserting \er{vhfffngghPPNffGGint} into
\er{MaxVacFullPPNffGGint} straightforward calculations presented in
subsection \ref{gcCM} lead to the following equations:
\begin{multline}\label{MaxVacFullPPNmmmffffffhhtygghGGint}
-\frac{1}{c}\left(\frac{\partial}{\partial t}\left(div_{\vec x}\vec
A\right)+div_{\vec x} \left\{\left(div_{\vec x}\vec A\right)\vec
{\tilde u}\right\}\right)-\Delta_{\vec
x}\Psi_1\\=4\pi\gamma_0\rho+\frac{1}{c}\,div_{\vec x}
\left\{\left(d_{\vec x}\vec {\tilde u}+\left\{d_{\vec x}\vec {\tilde
u}\right\}^T\right)\cdot\vec A-\left(div_{\vec x}\vec {\tilde
u}\right)\vec A\right\},
\end{multline}
and
%by \er{MaxVacFullPPNnnnffffffyuughjhjhjhhjjkjhkkjhhGGint} we have
\begin{multline}\label{MaxVacFullPPNnnnffffffyuughjhjhjhhjjkjhkkjhhjhghGGint}
-\Delta_{\vec x}\vec A= \frac{4\pi}{\kappa_0 c}\left(\vec j-\rho\vec
{\tilde u}\right)+\frac{1}{\kappa_0\gamma_0 c}\left(\left(d_{\vec
x}\vec {\tilde u}+\left\{d_{\vec x}\vec {\tilde
u}\right\}^T\right)\cdot \nabla_{\vec x}\Psi_1-\left(div_{\vec
x}\vec {\tilde u}\right)\nabla_{\vec x}\Psi_1\right)\\-\nabla_{\vec
x}\left(\frac{1}{\kappa_0\gamma_0 c}\left(\frac{\partial}{\partial
t}\Psi_1+\vec {\tilde u}\cdot\nabla_{\vec x}\Psi_1\right)+div_{\vec
x} \vec A\right)-\frac{1}{\kappa_0\gamma_0
c^2}\frac{\partial}{\partial t}\left(\frac{\partial\vec A}{\partial
t}-\vec {\tilde u}\times curl_{\vec x}\vec A+\nabla_{\vec
x}\left(\vec A\cdot\vec {\tilde
u}\right)\right)\\+\frac{1}{\kappa_0\gamma_0 c^2}curl_{\vec x}
\left\{\vec {\tilde u}\times
%\left(-\nabla_{\vec x}\Psi-\frac{1}{c}\frac{\partial\vec A}{\partial t}+\frac{1}{c}\vec v\times curl_{\vec x}\vec A\right).
\left(\frac{\partial\vec A}{\partial t}-\vec {\tilde u}\times
curl_{\vec x}\vec A+\nabla_{\vec x}\left(\vec A\cdot\vec {\tilde
u}\right)\right)\right\}\\-\frac{1}{\kappa_0\gamma_0
c^2}\left(div_{\vec x}\left(\frac{\partial\vec A}{\partial t}-\vec
{\tilde u}\times curl_{\vec x}\vec A+\nabla_{\vec x}\left(\vec
A\cdot\vec {\tilde u}\right)\right)\right)\vec {\tilde u}.
\end{multline}


Next if we assume the following calibration of the potentials:
\begin{equation}\label{MaxVacFullPPNjjjjffhhGGGGGGint}
div_{\vec x}\vec A=0,
\end{equation}
then by \er{MaxVacFullPPNjjjjffhhGGGGGGint},
\er{MaxVacFullPPNmmmffffffhhtygghGGint},
\er{MaxVacFullPPNnnnffffffyuughjhjhjhhjjkjhkkjhhjhghGGint}
% and \er{apfrm6} we have
%in subsection \ref{gcCM}
we obtain:
\begin{equation}\label{MaxVacFullPPNmmmffffffhhtygghGGGGint}
-\Delta_{\vec x}\Psi_1=4\pi\gamma_0\rho+\frac{1}{c}\,div_{\vec x}
\left\{\left(d_{\vec x}\vec {\tilde u}+\left\{d_{\vec x}\vec {\tilde
u}\right\}^T\right)\cdot\vec A-\left(div_{\vec x}\vec {\tilde
u}\right)\vec A\right\},
\end{equation}
and
\begin{multline}\label{MaxVacFullPPNnnnffffffyuughjhjhjhhjjkjhkkjhhjhghGGGGint}
-\Delta_{\vec x}\vec A= \frac{4\pi}{\kappa_0 c}\left(\vec j-\rho\vec
{\tilde u}\right)+\frac{1}{\kappa_0\gamma_0 c}\left(\left(d_{\vec
x}\vec {\tilde u}+\left\{d_{\vec x}\vec {\tilde
u}\right\}^T\right)\cdot \nabla_{\vec x}\Psi_1-\left(div_{\vec
x}\vec {\tilde u}\right)\nabla_{\vec
x}\Psi_1\right)\\-\frac{1}{\kappa_0\gamma_0 c}\,\nabla_{\vec
x}\left(\frac{\partial}{\partial t}\Psi_1+\vec {\tilde
u}\cdot\nabla_{\vec x}\Psi_1\right)-\frac{1}{\kappa_0\gamma_0
c^2}\frac{\partial}{\partial t}\left(\frac{\partial\vec A}{\partial
t}-\vec {\tilde u}\times curl_{\vec x}\vec A+\nabla_{\vec
x}\left(\vec A\cdot\vec {\tilde
u}\right)\right)\\+\frac{1}{\kappa_0\gamma_0 c^2}curl_{\vec x}
\left\{\vec {\tilde u}\times
%\left(-\nabla_{\vec x}\Psi-\frac{1}{c}\frac{\partial\vec A}{\partial t}+\frac{1}{c}\vec v\times curl_{\vec x}\vec A\right).
\left(\frac{\partial\vec A}{\partial t}-\vec {\tilde u}\times
curl_{\vec x}\vec A+\nabla_{\vec x}\left(\vec A\cdot\vec {\tilde
u}\right)\right)\right\}\\-\frac{1}{\kappa_0\gamma_0
c^2}\left(div_{\vec x}\left(\frac{\partial\vec A}{\partial t}-\vec
{\tilde u}\times curl_{\vec x}\vec A+\nabla_{\vec x}\left(\vec
A\cdot\vec {\tilde u}\right)\right)\right)\vec {\tilde u}.
%=
%\\
%
%
%
%\frac{4\pi}{\kappa_0 c}\left(\vec j-\rho\vec {\tilde u}\right)-\frac{1}{\kappa_0\gamma_0 c}\left(\frac{\partial}{\partial
%t}\left(\nabla_{\vec x}\Psi_1\right)-curl_{\vec x}\left(\vec {\tilde u}\times\nabla_{\vec x}\Psi_1\right)+\left(\Delta_{\vec
%x}\Psi_1\right)\vec {\tilde u}\right)\\-\frac{1}{\kappa_0\gamma_0 c^2}\frac{\partial}{\partial t}\left(\frac{\partial\vec A}{\partial
%t}-\vec {\tilde u}\times curl_{\vec x}\vec A+\nabla_{\vec x}\left(\vec A\cdot\vec {\tilde
%u}\right)\right)\\+\frac{1}{\kappa_0\gamma_0 c^2}curl_{\vec x} \left\{\vec {\tilde u}\times
%\left(\frac{\partial\vec A}{\partial t}-\vec {\tilde u}\times curl_{\vec x}\vec A+\nabla_{\vec x}\left(\vec A\cdot\vec {\tilde
%u}\right)\right)\right\}\\-\frac{1}{\kappa_0\gamma_0 c^2}\left(div_{\vec x}\left(\frac{\partial\vec A}{\partial t}-\vec
%{\tilde u}\times curl_{\vec x}\vec A+\nabla_{\vec x}\left(\vec A\cdot\vec {\tilde u}\right)\right)\right)\vec {\tilde u}.
\end{multline}
On the other hand, if we assume the following alternative
calibration of the potentials:
\begin{equation}\label{MaxVacFullPPNjjjjffhhGGint}
\frac{1}{\kappa_0\gamma_0 c}\left(\frac{\partial\Psi_1}{\partial
t}+\vec {\tilde u}\cdot\nabla_{\vec x}\Psi_1\right)+div_{\vec x}\vec
A=0,
\end{equation}
then by \er{MaxVacFullPPNjjjjffhhGGint},
\er{MaxVacFullPPNmmmffffffhhtygghGGint} and
\er{MaxVacFullPPNnnnffffffyuughjhjhjhhjjkjhkkjhhjhghGGint} we have
\begin{multline}\label{MaxVacFullPPNmmmffffffiuiuhjuGGint}
\frac{1}{\kappa_0\gamma_0 c^2}\left(\frac{\partial}{\partial
t}\left(\frac{\partial\Psi_1}{\partial t}+\vec {\tilde
u}\cdot\nabla_{\vec x}\Psi_1\right)+div_{\vec x}
\left\{\left(\frac{\partial\Psi_1}{\partial t}+\vec {\tilde
u}\cdot\nabla_{\vec x}\Psi_1\right)\vec {\tilde
u}\right\}\right)-\Delta_{\vec
x}\Psi_1\\=4\pi\gamma_0\rho+\frac{1}{c}\,div_{\vec x}
\left\{\left(d_{\vec x}\vec {\tilde u}+\left\{d_{\vec x}\vec {\tilde
u}\right\}^T\right)\cdot\vec A-\left(div_{\vec x}\vec {\tilde
u}\right)\vec A\right\},
\end{multline}
and
\begin{multline}\label{MaxVacFullPPNnnnffffffyuughjhjhjhhjjkjhkkjhujgGGint}
-\Delta_{\vec x}\vec A= \frac{4\pi}{\kappa_0 c}\left(\vec j-\rho\vec
{\tilde u}\right)+\frac{1}{\kappa_0\gamma_0 c}\left(\left(d_{\vec
x}\vec {\tilde u}+\left\{d_{\vec x}\vec {\tilde
u}\right\}^T\right)\cdot \nabla_{\vec x}\Psi_1-\left(div_{\vec
x}\vec {\tilde u}\right)\nabla_{\vec
x}\Psi_1\right)\\-\frac{1}{\kappa_0\gamma_0
c^2}\frac{\partial}{\partial t}\left(\frac{\partial\vec A}{\partial
t}-\vec {\tilde u}\times curl_{\vec x}\vec A+\nabla_{\vec
x}\left(\vec A\cdot\vec {\tilde
u}\right)\right)\\+\frac{1}{\kappa_0\gamma_0 c^2}curl_{\vec x}
\left\{\vec {\tilde u}\times
%\left(-\nabla_{\vec x}\Psi-\frac{1}{c}\frac{\partial\vec A}{\partial t}+\frac{1}{c}\vec v\times curl_{\vec x}\vec A\right).
\left(\frac{\partial\vec A}{\partial t}-\vec {\tilde u}\times
curl_{\vec x}\vec A+\nabla_{\vec x}\left(\vec A\cdot\vec {\tilde
u}\right)\right)\right\}\\-\frac{1}{\kappa_0\gamma_0
c^2}\left(div_{\vec x}\left(\frac{\partial\vec A}{\partial t}-\vec
{\tilde u}\times curl_{\vec x}\vec A+\nabla_{\vec x}\left(\vec
A\cdot\vec {\tilde u}\right)\right)\right)\vec {\tilde u}.
\end{multline}



In particular, assume that we have the following approximation: if
the changes in space of the physical characteristics of the
electromagnetic fields become essential in the spatial landscape
$L_e$ and the changes in space of the field $\vec {\tilde u}$
becomes essential in the spatial landscape $L_{u}$, then we assume
\begin{equation}\label{MaxVacFullPPNmmmffffffhhtygghGGGGyuhggghghint}
L_e\ll L_u,\quad\text{or equivalently:}\quad \frac{|d_{\vec x}\vec
{\tilde u}|}{|\vec {\tilde u}|}\ll\frac{|d_{\vec x}\vec A|}{|\vec
A|}\quad\text{and}\quad\frac{|d_{\vec x}\vec {\tilde u}|}{|\vec
{\tilde u}|}\ll\frac{|\nabla_{\vec x}\Psi_1|}{|\Psi_1|}.
\end{equation}
i.e. the field $\vec {\tilde u}$ vary in space much weaker then
$\vec A$ and $\Psi_1$. Estimation
\er{MaxVacFullPPNmmmffffffhhtygghGGGGyuhggghghint} holds especially
good for the electromagnetic waves of high frequency for example for
the visible light. However,
\er{MaxVacFullPPNmmmffffffhhtygghGGGGyuhggghghint} is still well for
almost every electromagnetic field we meet in the common life,
except probably the magnetic field of the Earth. Then, taking into
the account \er{MaxVacFullPPNmmmffffffhhtygghGGGGyuhggghghint},
under the calibration \er{MaxVacFullPPNjjjjffhhGGGGGGint}, we
rewrite \er{MaxVacFullPPNmmmffffffhhtygghGGGGint} and
\er{MaxVacFullPPNnnnffffffyuughjhjhjhhjjkjhkkjhhjhghGGGGint} as
\begin{equation}\label{MaxVacFullPPNmmmffffffhhtygghGGGGggint}
-\Delta_{\vec x}\Psi_1\approx 4\pi\gamma_0\rho,
\end{equation}
and
\begin{multline}\label{MaxVacFullPPNnnnffffffyuughjhjhjhhjjkjhkkjhhjhghGGGGggint}
-\Delta_{\vec x}\vec A\approx
%
%
%
%\frac{4\pi}{\kappa_0 c}\left(\vec j-\rho\vec {\tilde u}\right)-\frac{1}{\kappa_0\gamma_0 c}\left(\frac{\partial}{\partial t}\left(\nabla_{\vec x}\Psi_1\right)-curl_{\vec x}\left(\vec {\tilde
%u}\times\nabla_{\vec x}\Psi_1\right)+\left(\Delta_{\vec x}\Psi_1\right)\vec {\tilde u}\right)\\-\frac{1}{\kappa_0\gamma_0 c^2}\frac{\partial}{\partial t}\left(\frac{\partial\vec A}{\partial
%t}-\vec {\tilde u}\times curl_{\vec x}\vec A+\nabla_{\vec x}\left(\vec A\cdot\vec {\tilde u}\right)\right)\\+\frac{1}{\kappa_0\gamma_0 c^2}curl_{\vec x}
%\left\{\vec {\tilde u}\times\left(\frac{\partial\vec A}{\partial t}-\vec {\tilde u}\times curl_{\vec x}\vec A+\nabla_{\vec
%x}\left(\vec A\cdot\vec {\tilde u}\right)\right)\right\}\\-\frac{1}{\kappa_0\gamma_0 c^2}\left(div_{\vec x}\left(\frac{\partial\vec A}{\partial t}-\vec
%{\tilde u}\times curl_{\vec x}\vec A+\nabla_{\vec x}\left(\vec A\cdot\vec {\tilde u}\right)\right)\right)\vec {\tilde u}=\\
\frac{4\pi}{\kappa_0 c}\vec j-\frac{1}{\kappa_0\gamma_0
c}\left(\frac{\partial}{\partial t}\left(\nabla_{\vec
x}\Psi_1\right)-curl_{\vec x}\left(\vec {\tilde u}\times\nabla_{\vec
x}\Psi_1\right)\right)\\-\frac{1}{\kappa_0\gamma_0
c^2}\frac{\partial}{\partial t}\left(\frac{\partial\vec A}{\partial
t}-\vec {\tilde u}\times curl_{\vec x}\vec A+\nabla_{\vec
x}\left(\vec A\cdot\vec {\tilde
u}\right)\right)\\+\frac{1}{\kappa_0\gamma_0 c^2}curl_{\vec x}
\left\{\vec {\tilde u}\times
%\left(-\nabla_{\vec x}\Psi-\frac{1}{c}\frac{\partial\vec A}{\partial t}+\frac{1}{c}\vec v\times curl_{\vec x}\vec A\right).
\left(\frac{\partial\vec A}{\partial t}-\vec {\tilde u}\times
curl_{\vec x}\vec A+\nabla_{\vec x}\left(\vec A\cdot\vec {\tilde
u}\right)\right)\right\}\\-\frac{1}{\kappa_0\gamma_0
c^2}\left(div_{\vec x}\left(\frac{\partial\vec A}{\partial t}-\vec
{\tilde u}\times curl_{\vec x}\vec A+\nabla_{\vec x}\left(\vec
A\cdot\vec {\tilde u}\right)\right)\right)\vec {\tilde u},
\end{multline}
(See subsection \ref{gcCM} for details). Note that, using
Proposition \ref{yghgjtgyrtrtint} we deduce that the approximate
equations \er{MaxVacFullPPNmmmffffffhhtygghGGGGggint} and
\er{MaxVacFullPPNnnnffffffyuughjhjhjhhjjkjhkkjhhjhghGGGGggint} are
still invariant under the change of inertial or non-inertial
cartesian coordinate system, provided that $\vec A$ is a proper
vector field and $\Psi_1$ is a proper scalar field. So we can use
approximate equations \er{MaxVacFullPPNmmmffffffhhtygghGGGGggint}
and \er{MaxVacFullPPNnnnffffffyuughjhjhjhhjjkjhkkjhhjhghGGGGggint}
in the coordinate system $(*)$ even if
\er{MaxVacFullPPNmmmffffffhhtygghGGGGyuhggghghint} is not satisfied
in the system $(*)$, provided that
\er{MaxVacFullPPNmmmffffffhhtygghGGGGyuhggghghint} is satisfied in
another system $(**)$.


On the other hand, taking into the account
\er{MaxVacFullPPNmmmffffffhhtygghGGGGyuhggghghint}, under the
calibration \er{MaxVacFullPPNjjjjffhhGGint}, we rewrite
\er{MaxVacFullPPNmmmffffffiuiuhjuGGint} and
\er{MaxVacFullPPNnnnffffffyuughjhjhjhhjjkjhkkjhujgGGint} as
\begin{equation}\label{MaxVacFullPPNmmmffffffiuiuhjuGGggint}
\frac{1}{\kappa_0\gamma_0 c^2}\left(\frac{\partial}{\partial
t}\left(\frac{\partial\Psi_1}{\partial t}+\vec {\tilde
u}\cdot\nabla_{\vec x}\Psi_1\right)+div_{\vec x}
\left\{\left(\frac{\partial\Psi_1}{\partial t}+\vec {\tilde
u}\cdot\nabla_{\vec x}\Psi_1\right)\vec {\tilde
u}\right\}\right)-\Delta_{\vec x}\Psi_1\approx 4\pi\gamma_0\rho.
\end{equation}
and
\begin{multline}\label{MaxVacFullPPNnnnffffffyuughjhjhjhhjjkjhkkjhujgGGggint}
-\Delta_{\vec x}\vec A\approx \frac{4\pi}{\kappa_0 c}\left(\vec
j-\rho\vec {\tilde u}\right)-\frac{1}{\kappa_0\gamma_0
c^2}\frac{\partial}{\partial t}\left(\frac{\partial\vec A}{\partial
t}-\vec {\tilde u}\times curl_{\vec x}\vec A+\nabla_{\vec
x}\left(\vec A\cdot\vec {\tilde
u}\right)\right)\\+\frac{1}{\kappa_0\gamma_0 c^2}curl_{\vec x}
\left\{\vec {\tilde u}\times
%\left(-\nabla_{\vec x}\Psi-\frac{1}{c}\frac{\partial\vec A}{\partial t}+\frac{1}{c}\vec v\times curl_{\vec x}\vec A\right).
\left(\frac{\partial\vec A}{\partial t}-\vec {\tilde u}\times
curl_{\vec x}\vec A+\nabla_{\vec x}\left(\vec A\cdot\vec {\tilde
u}\right)\right)\right\}\\-\frac{1}{\kappa_0\gamma_0
c^2}\left(div_{\vec x}\left(\frac{\partial\vec A}{\partial t}-\vec
{\tilde u}\times curl_{\vec x}\vec A+\nabla_{\vec x}\left(\vec
A\cdot\vec {\tilde u}\right)\right)\right)\vec {\tilde u}.
\end{multline}
Again note that, using Proposition \ref{yghgjtgyrtrtint} we deduce
that the approximate equations
\er{MaxVacFullPPNmmmffffffiuiuhjuGGggint} and
\er{MaxVacFullPPNnnnffffffyuughjhjhjhhjjkjhkkjhujgGGggint} are still
invariant under the change of inertial or non-inertial cartesian
coordinate system, provided that $\vec A$ is a proper vector field
and $\Psi_1$ is a proper scalar field. So we can use approximate
equations \er{MaxVacFullPPNmmmffffffiuiuhjuGGggint} and
\er{MaxVacFullPPNnnnffffffyuughjhjhjhhjjkjhkkjhujgGGggint} in the
coordinate system $(*)$ even if
\er{MaxVacFullPPNmmmffffffhhtygghGGGGyuhggghghint} is not satisfied
in the system $(*)$, provided that
\er{MaxVacFullPPNmmmffffffhhtygghGGGGyuhggghghint} is satisfied in
another system $(**)$.



Finally note that by \er{MaxVacFullPPNmmmffffffiuiuhjuGGggint},
\er{MaxVacFullPPNnnnffffffyuughjhjhjhhjjkjhkkjhujgGGggint} and
\er{MaxVacFullPPNmmmffffffhhtygghGGGGyuhggghghint} we can write the
further approximating equations:
\begin{equation}\label{MaxVacFullPPNmmmffffffiuiuhjuGGggFGint}
\frac{1}{c^2_0}\left(\frac{\partial}{\partial
t}\left(\frac{\partial\Psi_1}{\partial t}+\vec {\tilde
u}\cdot\nabla_{\vec x}\Psi_1\right)+div_{\vec x}
\left\{\left(\frac{\partial\Psi_1}{\partial t}+\vec {\tilde
u}\cdot\nabla_{\vec x}\Psi_1\right)\vec {\tilde
u}\right\}\right)-\Delta_{\vec x}\Psi_1 \approx 4\pi\gamma_0\rho,
\end{equation}
and
\begin{equation}\label{MaxVacFullPPNnnnffffffyuughjhjhjhhjjkjhkkjhujgGGggFGint}
\frac{1}{c^2_0}\left(\frac{\partial}{\partial
t}\left(\frac{\partial\vec A}{\partial t}+d_{\vec x}\vec A\cdot\vec
{\tilde u}\right)+div_{\vec x} \left\{\left(\frac{\partial\vec
A}{\partial t}+d_{\vec x}\vec A\cdot\vec {\tilde
u}\right)\otimes\vec {\tilde u}\right\}\right)-\Delta_{\vec x}\vec
A\approx \frac{4\pi}{\kappa_0 c}\left(\vec j-\rho\vec {\tilde
u}\right),
\end{equation}
where the scalar quantity $c_0$, defined by:
\begin{equation}\label{gughhghfbvnbvint}
c_0=c\sqrt{\kappa_0\gamma_0},
\end{equation}
is called speed of light in the medium. Note that, although the
approximate equations \er{MaxVacFullPPNmmmffffffiuiuhjuGGggFGint}
and \er{MaxVacFullPPNnnnffffffyuughjhjhjhhjjkjhkkjhujgGGggFGint} are
invariant under the Galilean Transformation, they are not invariant
under the more general change of non-inertial cartesian coordinate
system. However, \er{MaxVacFullPPNmmmffffffiuiuhjuGGggFGint} and
\er{MaxVacFullPPNnnnffffffyuughjhjhjhhjjkjhkkjhujgGGggFGint} are
more convenient then \er{MaxVacFullPPNmmmffffffiuiuhjuGGggint} and
\er{MaxVacFullPPNnnnffffffyuughjhjhjhhjjkjhkkjhujgGGggint}, since
the scalar potential $\Psi_1$ and every of the three scalar
components of the vector potential $\vec A$ in
\er{MaxVacFullPPNmmmffffffiuiuhjuGGggFGint} and
\er{MaxVacFullPPNnnnffffffyuughjhjhjhhjjkjhkkjhujgGGggFGint}
satisfies four decoupled equations of the same type, that differ
only by the right parts.

In the absence of charges and currents (for example for
electromagnetic waves) equations
\er{MaxVacFullPPNmmmffffffiuiuhjuGGggFGint} and
\er{MaxVacFullPPNnnnffffffyuughjhjhjhhjjkjhkkjhujgGGggFGint} become:
\begin{equation}\label{MaxVacFullPPNmmmffffffiuiuhjuGGggFGelint}
\frac{1}{c^2_0}\left(\frac{\partial}{\partial
t}\left(\frac{\partial\Psi_1}{\partial t}+\vec {\tilde
u}\cdot\nabla_{\vec x}\Psi_1\right)+div_{\vec x}
\left\{\left(\frac{\partial\Psi_1}{\partial t}+\vec {\tilde
u}\cdot\nabla_{\vec x}\Psi_1\right)\vec {\tilde
u}\right\}\right)-\Delta_{\vec x}\Psi_1=0,
\end{equation}
and
\begin{equation}\label{MaxVacFullPPNnnnffffffyuughjhjhjhhjjkjhkkjhujgGGggFGelint}
\frac{1}{c^2_0}\left(\frac{\partial}{\partial
t}\left(\frac{\partial\vec A}{\partial t}+d_{\vec x}\vec A\cdot\vec
{\tilde u}\right)+div_{\vec x} \left\{\left(\frac{\partial\vec
A}{\partial t}+d_{\vec x}\vec A\cdot\vec {\tilde
u}\right)\otimes\vec {\tilde u}\right\}\right)-\Delta_{\vec x}\vec
A=0.
\end{equation}
Therefore, by \er{vhfffngghPPNffGGint}, differentiating
\er{MaxVacFullPPNmmmffffffiuiuhjuGGggFGelint} and
\er{MaxVacFullPPNnnnffffffyuughjhjhjhhjjkjhkkjhujgGGggFGelint} and
further usage of \er{MaxVacFullPPNmmmffffffhhtygghGGGGyuhggghghint}
gives that if the scalar field $U:=U(\vec x,t)$ is one of the three
scalar components of every of the fields $\vec E$, $\vec B$, $\vec
D$ or $\vec H$, then $U$ satisfies the following approximate scalar
equation of the wave type:
\begin{equation}\label{MaxVacFullPPNmmmffffffiuiuhjuGGggFGelGHGHGHGGint}
\frac{1}{c^2_0}\left(\frac{\partial}{\partial t}\left(\frac{\partial
U}{\partial t}+\vec {\tilde u}\cdot\nabla_{\vec x}U\right)+div_{\vec
x} \left\{\left(\frac{\partial U}{\partial t}+\vec {\tilde
u}\cdot\nabla_{\vec x} U\right)\vec {\tilde
u}\right\}\right)-\Delta_{\vec x}U \approx 0,
\end{equation}
where,
\begin{equation}\label{uyuyuyyint}
\vec {\tilde u}=\left(\gamma_0\vec v+(1-\gamma_0)\vec u\right)\,.
\end{equation}














\subsubsection{The case of quasistationary electromagnetic fields
inside a slowly moving medium in a weak gravitational
field}
%\label{qfCM}
Assume that in the given inertial or non-inertial
cartesian coordinate system $(*)$ the field $\vec {\tilde u}$ is
weak, meaning that at any instant on every point:
\begin{equation}\label{slowaetherGGaaint}
\frac{1}{\kappa_0\gamma_0}\frac{|\vec {\tilde u}|^2}{c^2}\,\ll\, 1.
%\frac{|\vec v|^2}{c^2}\ll 1
\end{equation}
Here $\vec {\tilde u}=\left(\gamma_0\vec v+(1-\gamma_0)\vec
u\right)$ is the speed-like vector field, where $\vec v$ is a
vectorial gravitational potential in the system $(*)$ and $\vec u$
is the medium velocity. Furthermore, consider quasistationary
electromagnetic fields. This means the following: assume that the
changes in time of the physical characteristics of the
electromagnetic fields become essential after certain interval of
time $T_e$ and the changes in space of the physical characteristics
of the fields become essential in the spatial landscape $L_e$. Then
we assume that
\begin{equation}\label{slochangGGaaint}
(\kappa_0\gamma_0)\frac{c^2T^2_e}{L^2_e}\,\gg\, 1.
\end{equation}
Next assume that we are under the calibration
\er{MaxVacFullPPNjjjjffhhGGGGGGint}. Then by \er{slowaetherGGaaint}
and \er{slochangGGaaint} we rewrite
\er{MaxVacFullPPNmmmffffffhhtygghGGGGint} and
\er{MaxVacFullPPNnnnffffffyuughjhjhjhhjjkjhkkjhhjhghGGGGint} as
\begin{equation}\label{MaxVacFullPPNmmmffffffhhtygghGGGGaaint}
-\Delta_{\vec x}\Psi_1=4\pi\gamma_0\rho+\frac{1}{c}\,div_{\vec x}
\left\{\left(d_{\vec x}\vec {\tilde u}+\left\{d_{\vec x}\vec {\tilde
u}\right\}^T\right)\cdot\vec A-\left(div_{\vec x}\vec {\tilde
u}\right)\vec A\right\},
\end{equation}
and
\begin{equation}\label{MaxVacFullPPNnnnffffffyuughjhjhjhhjjkjhkkjhhjhghGGGGaa1int}
-\Delta_{\vec x}\vec A\approx \frac{4\pi}{\kappa_0 c}\left(\vec
j-\rho\vec {\tilde u}\right)-\frac{1}{\kappa_0\gamma_0
c}\left(\frac{\partial}{\partial t}\left(\nabla_{\vec
x}\Psi_1\right)-curl_{\vec x}\left(\vec {\tilde u}\times\nabla_{\vec
x}\Psi_1\right)+\left(\Delta_{\vec x}\Psi_1\right)\vec {\tilde
u}\right).
%
%
%
\end{equation}
Moreover, by \er{slowaetherGGaaint} and \er{slochangGGaaint} we can
perform further approximation of
\er{MaxVacFullPPNnnnffffffyuughjhjhjhhjjkjhkkjhhjhghGGGGaa1int} and
we get
\begin{equation}\label{MaxVacFullPPNnnnffffffyuughjhjhjhhjjkjhkkjhhjhghGGGGaaint}
-\Delta_{\vec x}\vec A
%\approx
%\frac{4\pi}{\kappa_0 c}\left(\vec j-\rho\vec {\tilde u}\right)-\frac{1}{\kappa_0\gamma_0 c}\left(\frac{\partial}{\partial t}\left(\nabla_{\vec
%x}\Psi_1\right)-curl_{\vec x}\left(\vec {\tilde u}\times\nabla_{\vec x}\Psi_1\right)+\left(\Delta_{\vec x}\Psi_1\right)\vec {\tilde u}\right)\\
\approx\frac{4\pi}{\kappa_0 c}\,\vec
j-\frac{1}{\kappa_0 c}\left(\frac{\partial}{\partial
t}\left(\nabla_{\vec x}\psi_0\right)-curl_{\vec x}\left(\vec
v\times\nabla_{\vec x}\psi_0\right)\right),
\end{equation}
where $\psi_0(\vec x,t)$ is the classical Coulomb's potential which
satisfies
\begin{equation}\label{columbPPNaaint}
-\Delta_{\vec x}\psi_0\equiv 4\pi\rho.
\end{equation}
So we rewrite \er{MaxVacFullPPNmmmffffffhhtygghGGGGaaint} and
\er{MaxVacFullPPNnnnffffffyuughjhjhjhhjjkjhkkjhhjhghGGGGaaint} as
\begin{equation}\label{MaxVacFull1bjkgjhjhgjgjgkjfhjfdghcgjhhjgkgkgugyyurhjfffhfjklhhhgkjgGGaaKKint}
\begin{cases}
-\Delta_{\vec x}\vec A \approx\frac{4\pi}{\kappa_0 c}\vec {\widetilde j},\\
%\quad\text{for}\;\;(\vec x,t)\in\R^3\times[0,+\infty),\\
-\Delta_{\vec x}\Psi_1=4\pi\gamma_0\rho+\frac{1}{c}\,div_{\vec x}
\left\{\left(d_{\vec x}\vec {\tilde u}+\left\{d_{\vec x}\vec {\tilde
u}\right\}^T\right)\cdot\vec A-\left(div_{\vec x}\vec {\tilde
u}\right)\vec A\right\},
\end{cases}
\end{equation}
where we set the reduced current:
\begin{equation}\label{reducedcurrentfhfhjfhjGGaaint}
\begin{cases}
\vec {\widetilde j}:=\vec j-\frac{1}{4\pi}\frac{\partial}{\partial
t} \left(\nabla_{\vec x}\psi_0\right)+\frac{1}{4\pi}curl_{\vec
x}\left(\vec {\tilde u}\times \nabla_{\vec x}\psi_0\right),\\
-\Delta_{\vec x}\psi_0= 4\pi\rho.
\end{cases}
\end{equation}
Note that by the Continuum Equation of the Conservation of Charges:
\begin{equation}\label{toksohraneniezarjadaPPNaaint}
\frac{\partial\rho}{\partial t}+div_{\vec x}\vec j\equiv 0,
\end{equation}
the reduced current clearly satisfies:
\begin{equation}\label{divreducedcurrentPPNaaint}
div_{\vec x}\vec {\widetilde j}\equiv 0.
\end{equation}
Moreover,
%by \er{reducedcurrentfhfhjfhjGGaaint} we clearly have
%\begin{equation}\label{reducedcurrentPPNuighjhjaaint}\vec {\widetilde j}:=(\vec j-\rho\vec {\tilde u})-\frac{1}{4\pi}\left(\frac{\partial}{\partial t}
%\left(\nabla_{\vec x}\psi_0\right)-curl_{\vec x}\left(\vec {\tilde u}\times \nabla_{\vec x}\psi_0\right)+\left(div_{\vec x}\left\{\nabla_{\vec x}\psi_0\right\}\right)\vec {\tilde u}\right),
%\end{equation}
%and thus, by \er{reducedcurrentPPNuighjhjaaint},
using Proposition
\ref{yghgjtgyrtrtint}
%from the Appendix
we can easily deduce that $\vec{\widetilde j}$ is a proper vector
field (see subsection \ref{qfCM} for detatls). Finally, the
approximate vectorial electromagnetic potential $\vec A$ from
\er{MaxVacFull1bjkgjhjhgjgjgkjfhjfdghcgjhhjgkgkgugyyurhjfffhfjklhhhgkjgGGaaKKint}
clearly satisfies:
\begin{equation}\label{MaxVacFull1bjkgjhjhgjgjgkjfhjfdghcgjhhjgkgkgugyyurkkkGGGGGaaint}
div_{\vec x}\vec A=0.
\end{equation}
Next, since by \er{vhfffngghhjghhgPPNghghghutghffugghjhjkjjklggint}
we have
\begin{equation}\label{vhfffngghhjghhgPPNghghghutghffugghjhjkjjklgghkhhhint}
\Psi_1:=\Psi-\frac{1}{c}\vec A\cdot\vec {\tilde u},
\end{equation}
%
%
%
\begin{comment}
and since by
\er{MaxVacFull1bjkgjhjhgjgjgkjfhjfdghcgjhhjgkgkgugyyurkkkGGGGGaaint},
\er{apfrm6} and \er{apfrm9} we have
\begin{multline}\label{MaxVacFullPPNnnnffffffyuughjhjhjhhjjkjhkkjhhjhghGGGGaajjkkjkjljint}
div_{\vec x} \left\{\left(d_{\vec x}\vec {\tilde u}+\left\{d_{\vec
x}\vec {\tilde u}\right\}^T\right)\cdot\vec A-\left(div_{\vec x}\vec
{\tilde u}\right)\vec A\right\}-\Delta_{\vec x}\left(\vec A\cdot\vec
{\tilde u}\right)=\\div_{\vec x} \left\{\left(d_{\vec x}\vec {\tilde
u}+\left\{d_{\vec x}\vec {\tilde u}\right\}^T\right)\cdot\vec
A-\left(div_{\vec x}\vec {\tilde u}\right)\vec A-\nabla_{\vec
x}\left(\vec A\cdot\vec {\tilde u}\right)\right\}=\\ div_{\vec x}
\left\{d_{\vec x}\vec {\tilde u}\cdot\vec A-d_{\vec x}\vec
A\cdot\vec {\tilde u}+\left(div_{\vec x}\vec A\right)\vec {\tilde
u}-\left(div_{\vec x}\vec {\tilde u}\right)\vec A-\vec {\tilde
u}\times curl_{\vec x}\vec A\right\}\\= div_{\vec x}
\left\{curl_{\vec x}\left(\vec {\tilde u}\times\vec A\right)-\vec
{\tilde u}\times curl_{\vec x}\vec A\right\}= -div_{\vec x}
\left\{\vec {\tilde u}\times curl_{\vec x}\vec A\right\},
\end{multline}
\end{comment}
%
%
%
we rewrite
\er{MaxVacFull1bjkgjhjhgjgjgkjfhjfdghcgjhhjgkgkgugyyurhjfffhfjklhhhgkjgGGaaKKint}
as:
\begin{equation}\label{MaxVacFull1bjkgjhjhgjgjgkjfhjfdghcgjhhjgkgkgugyyurhjfffhfjklhhhgkjgGGaaint}
\begin{cases}
-\Delta_{\vec x}\vec A \approx\frac{4\pi}{\kappa_0 c}\vec {\widetilde j},\\
%\quad\text{for}\;\;(\vec x,t)\in\R^3\times[0,+\infty),\\
-\Delta_{\vec x}\Psi= 4\pi\gamma_0\rho-\frac{1}{c}\,div_{\vec
x}\left(\vec {\tilde u}\times curl_{\vec x} \vec A\right).
\end{cases}
\end{equation}
where
\begin{equation}\label{reducedcurrentfhfhjfhjGGaakklklint}
\begin{cases}
\vec {\widetilde j}:=\vec j-\frac{1}{4\pi}\frac{\partial}{\partial
t} \left(\nabla_{\vec x}\psi_0\right)+\frac{1}{4\pi}curl_{\vec
x}\left(\vec {\tilde u}\times \nabla_{\vec x}\psi_0\right),\\
-\Delta_{\vec x}\psi_0= 4\pi\rho,
\end{cases}
\end{equation}
(see subsection \ref{qfCM} for details). So in order to find the
scalar and the vectorial electromagnetic potentials we just need to
solve Laplace equations. Knowing the approximate electromagnetic
potentials by \er{vhfffngghPPN333yuyuGGint} we can find the
approximations of of the electromagnetic fields:
\begin{equation}\label{MaxVacFull1bjkgjhjhgjgjgkjfhjfdghghligioiuittrhiguffGGaaint}
\begin{cases}
\vec B= curl_{\vec x} \vec A\\
\vec E=-\nabla_{\vec x}\Psi-\frac{1}{c}\frac{\partial\vec
A}{\partial t}\\
 \vec D=-\frac{1}{\gamma_0}\nabla_{\vec
x}\Psi-\frac{1}{\gamma_0 c}\frac{\partial\vec A}{\partial t}+\frac{1}{c\gamma_0}\vec {\tilde u}\times curl_{\vec x}\vec A\\
\vec H=\kappa_0 \,curl_{\vec x} \vec A+\frac{1}{c}\,\vec {\tilde
u}\times\left(-\frac{1}{\gamma_0}\nabla_{\vec
x}\Psi-\frac{1}{\gamma_0 c}\frac{\partial\vec A}{\partial
t}+\frac{1}{\gamma_0 c}\vec {\tilde u}\times curl_{\vec x}\vec
A\right),
\end{cases}
\end{equation}
where $\Psi$ and $\vec A$ are given by
\er{MaxVacFull1bjkgjhjhgjgjgkjfhjfdghcgjhhjgkgkgugyyurhjfffhfjklhhhgkjgGGaaint}.
Note also that, since $\vec {\widetilde j}$ is a proper vector
field, by Proposition \ref{yghgjtgyrtrtint} we deduce that equations
\er{MaxVacFull1bjkgjhjhgjgjgkjfhjfdghcgjhhjgkgkgugyyurhjfffhfjklhhhgkjgGGaaKKint}
and thus also equations
\er{MaxVacFull1bjkgjhjhgjgjgkjfhjfdghcgjhhjgkgkgugyyurhjfffhfjklhhhgkjgGGaaint}
are invariant under the change of non-inertial cartesian coordinate
system, provided that $\vec A$ is a proper vector field and
$\Psi_1=\Psi-\frac{1}{c}\vec A\cdot\vec {\tilde u}$ is a proper
scalar field. So the approximate solutions in the case of
quasistationary fields in a weak gravitational field satisfy the
same transformation as the exact solutions of Maxwell Equations.
%(see \er{yuythfgfyftydtydtydtyddyyyhhddhhh}).
Therefore, if in coordinate system $(*)$ we can use the approximate
equations, given by
\er{MaxVacFull1bjkgjhjhgjgjgkjfhjfdghcgjhhjgkgkgugyyurhjfffhfjklhhhgkjgGGaaint}
and
\er{MaxVacFull1bjkgjhjhgjgjgkjfhjfdghghligioiuittrhiguffGGaaint},
%, given by \er{MaxVacFull1bjkgjhjhgjgjgkjfhjfdghcgjhhjgkgkgugyyurhjfffhfjklhhhgkjggjgjuiyuuijk}
%and \er{MaxVacFull1bjkgjhjhgjgjgkjfhjfdghghligioiuittrhiguffgjggjgkhkk},
then we can use the similar approximation
%, given by \er{MaxVacFull1bjkgjhjhgjgjgkjfhjfdghcgjhhjgkgkgugyyurhjfffhfjklhhhgkjggjgj}
%and \er{MaxVacFull1bjkgjhjhgjgjgkjfhjfdghghligioiuittrhiguffgjghlkk}
also in coordinate system $(**)$, even in the case when in system
$(**)$ \er{slowaetherGGaaint} or \er{slochangGGaaint} are not
satisfied.
\begin{remark}
The solutions of
\er{MaxVacFull1bjkgjhjhgjgjgkjfhjfdghcgjhhjgkgkgugyyurhjfffhfjklhhhgkjgGGaaint}
and \er{MaxVacFull1bjkgjhjhgjgjgkjfhjfdghghligioiuittrhiguffGGaaint}
satisfy the following equations:
\begin{equation}\label{MaxVacFull1bjkgjhjhgjaaajhfghhgGGaaint}
\begin{cases}
curl_{\vec x} \left(\kappa_0\vec B+\frac{1}{c}\,\vec {\tilde
u}\times \left(- \nabla_{\vec x}\psi_0\right)\right)\equiv
\frac{4\pi}{c}\vec j+\frac{1}{c}\frac{\partial (-
\nabla_{\vec x}\psi_0)}{\partial t},\\
div_{\vec x} \vec D=4\pi\rho,\\
curl_{\vec x} \vec E+\frac{1}{c}\frac{\partial \vec B}{\partial t}=0,\\
div_{\vec x} \vec B=0\\
\vec E=\gamma_0\vec D-\frac{1}{c}\,\vec {\tilde u}\times \vec B\\
\vec H=\kappa_0\vec B+\frac{1}{c}\,\vec {\tilde u}\times \vec D,\\
\vec {\tilde u}=\left(\gamma_0\vec v+(1-\gamma_0)\vec u\right),
\end{cases}
\end{equation}
where $\psi_0$ was defined by \er{columbPPNaaint}. Equations
\er{MaxVacFull1bjkgjhjhgjaaajhfghhgGGaaint} differ from the original
Maxwell equations \er{MaxVacFullPPNffGGint} only by neglecting the
divergence-free part of the vector field $\vec D$ on the first
equation.
\end{remark}


Next, assume that, in addition to the validity of approximation
\er{slowaetherGGaaint} and \er{slochangGGaaint}, the approximation
\er{MaxVacFullPPNmmmffffffhhtygghGGGGyuhggghghint} also holds. Then
we further approximate
\er{MaxVacFull1bjkgjhjhgjgjgkjfhjfdghcgjhhjgkgkgugyyurhjfffhfjklhhhgkjgGGaaKKint}
as:
\begin{equation}\label{MaxVacFull1bjkgjhjhgjgjgkjfhjfdghcgjhhjgkgkgugyyurhjfffhfjklhhhgkjgGGaaKKjkjjint}
\begin{cases}
-\Delta_{\vec x}\Psi_1=4\pi\gamma_0\rho,\\
-\Delta_{\vec x}\vec A \approx \frac{4\pi}{\kappa_0 c}\,\vec
j-\frac{1}{\kappa_0\gamma_0 c}\left(\frac{\partial}{\partial
t}\left(\nabla_{\vec
x}\Psi_1\right)-curl_{\vec x}\left(\vec {\tilde u}\times\nabla_{\vec x}\Psi_1\right)\right)\\
\Psi=\Psi_1+\frac{1}{c}\vec A\cdot\vec {\tilde u}.
\end{cases}
\end{equation}
Moreover, as before, we deduce that equations
\er{MaxVacFull1bjkgjhjhgjgjgkjfhjfdghcgjhhjgkgkgugyyurhjfffhfjklhhhgkjgGGaaKKjkjjint}
are also invariant under the change of non-inertial cartesian
coordinate system. Therefore, as before, if in coordinate system
$(*)$ we can use the approximation equations, given by
\er{MaxVacFull1bjkgjhjhgjgjgkjfhjfdghcgjhhjgkgkgugyyurhjfffhfjklhhhgkjgGGaaKKjkjjint}
%, given by \er{MaxVacFull1bjkgjhjhgjgjgkjfhjfdghcgjhhjgkgkgugyyurhjfffhfjklhhhgkjggjgjuiyuuijk}
%and \er{MaxVacFull1bjkgjhjhgjgjgkjfhjfdghghligioiuittrhiguffgjggjgkhkk},
then we can use the similar equations
%, given by \er{MaxVacFull1bjkgjhjhgjgjgkjfhjfdghcgjhhjgkgkgugyyurhjfffhfjklhhhgkjggjgj}
%and \er{MaxVacFull1bjkgjhjhgjgjgkjfhjfdghghligioiuittrhiguffgjghlkk}
also in coordinate system $(**)$, even in the case when in system
$(**)$ \er{slowaetherGGaaint}, \er{slochangGGaaint} or
\er{MaxVacFullPPNmmmffffffhhtygghGGGGyuhggghghint} are not
satisfied.







Finally, assume that we are under the alternative calibration
\er{MaxVacFullPPNjjjjffhhGGint}. Then by \er{slowaetherGGaaint} and
\er{slochangGGaaint} we rewrite
\er{MaxVacFullPPNmmmffffffiuiuhjuGGint} and
\er{MaxVacFullPPNnnnffffffyuughjhjhjhhjjkjhkkjhujgGGint} as:
\begin{equation}\label{MaxVacFullPPNmmmffffffiuiuhjuGGaaint}
-\Delta_{\vec x}\Psi_1\approx
4\pi\gamma_0\rho+\frac{1}{c}\,div_{\vec x} \left\{\left(d_{\vec
x}\vec {\tilde u}+\left\{d_{\vec x}\vec {\tilde
u}\right\}^T\right)\cdot\vec A-\left(div_{\vec x}\vec {\tilde
u}\right)\vec A\right\},
\end{equation}
and
\begin{equation}\label{MaxVacFullPPNnnnffffffyuughjhjhjhhjjkjhkkjhujgGGaaint}
-\Delta_{\vec x}\vec A\approx \frac{4\pi}{\kappa_0 c}\left(\vec
j-\rho\vec {\tilde u}\right)+\frac{1}{\kappa_0\gamma_0
c}\left(\left(d_{\vec x}\vec {\tilde u}+\left\{d_{\vec x}\vec
{\tilde u}\right\}^T\right)\cdot \nabla_{\vec
x}\Psi_1-\left(div_{\vec x}\vec {\tilde u}\right)\nabla_{\vec
x}\Psi_1\right).
\end{equation}
Thus if we assume that in addition to the approximation
\er{slowaetherGGaaint} and \er{slochangGGaaint} the approximation
\er{MaxVacFullPPNmmmffffffhhtygghGGGGyuhggghghint} also holds, we
further approximate \er{MaxVacFullPPNmmmffffffiuiuhjuGGaaint} and
\er{MaxVacFullPPNnnnffffffyuughjhjhjhhjjkjhkkjhujgGGaaint} as:
\begin{equation}\label{MaxVacFullPPNmmmffffffiuiuhjuGGaaghgghghint}
\begin{cases}
-\Delta_{\vec x}\Psi_1\approx 4\pi\gamma_0\rho,
\\
-\Delta_{\vec x}\vec A\approx \frac{4\pi}{\kappa_0 c}\left(\vec
j-\rho\vec {\tilde u}\right)\\
\Psi=\Psi_1+\frac{1}{c}\vec A\cdot\vec {\tilde u}.
\end{cases}
\end{equation}
Moreover, as before, we deduce that equations
\er{MaxVacFullPPNmmmffffffiuiuhjuGGaaghgghghint} are also invariant
under the change of non-inertial cartesian coordinate system.
Therefore, as before, if in coordinate system $(*)$ we can use the
approximation equations, given by
\er{MaxVacFullPPNmmmffffffiuiuhjuGGaaghgghghint}
%, given by \er{MaxVacFull1bjkgjhjhgjgjgkjfhjfdghcgjhhjgkgkgugyyurhjfffhfjklhhhgkjggjgjuiyuuijk}
%and \er{MaxVacFull1bjkgjhjhgjgjgkjfhjfdghghligioiuittrhiguffgjggjgkhkk},
then we can use the similar equations
%, given by \er{MaxVacFull1bjkgjhjhgjgjgkjfhjfdghcgjhhjgkgkgugyyurhjfffhfjklhhhgkjggjgj}
%and \er{MaxVacFull1bjkgjhjhgjgjgkjfhjfdghghligioiuittrhiguffgjghlkk}
also in coordinate system $(**)$, even in the case when in system
$(**)$ \er{slowaetherGGaaint}, \er{slochangGGaaint} or
\er{MaxVacFullPPNmmmffffffhhtygghGGGGyuhggghghint} are not
satisfied.








































































































































\subsection{Geometric optics inside a moving medium and/or in the presence of gravitational
field}
%\label{GO}
%\subsubsection{Derivation of the Eikonal equation}\label{ekGO}
Assume that in some inertial or non-inertial cartesian coordinate
system a scalar field $U:=U(\vec x,t)$, characterizing some wave,
satisfies the following wave equation
\begin{equation}\label{MaxVacFullPPNmmmffffffiuiuhjuughbghhjjint}
\frac{1}{c^2_0}\left(\frac{\partial}{\partial t}\left(\frac{\partial
U}{\partial t}+\vec {\tilde u}\cdot\nabla_{\vec x}U\right)+div_{\vec
x} \left\{\left(\frac{\partial U}{\partial t}+\vec {\tilde
u}\cdot\nabla_{\vec x} U\right)\vec {\tilde
u}\right\}\right)-\Delta_{\vec x}U=0,
\end{equation}
where $\vec {\tilde u}:=\vec {\tilde u}(\vec x,t)$ is some
moderately changing (in space and in time) speed-like vector field
and $c_0:=c_0(\vec x,t)>0$ is a moderately changing (in space and in
time) scalar quantity, that we call wave propagation speed. Note
that \er{MaxVacFullPPNmmmffffffiuiuhjuughbghhjjint} coincides with
\er{MaxVacFullPPNmmmffffffiuiuhjuGGggFGelGHGHGHGGint} and thus, in
particular, $U$ can represent one of the scalar components of the
electromagnetic field.

Next if we assume that the fields $\vec {\tilde u}$ and $c_0$ are
independent on the time variable, then we can write the field $U$ as
a Furier's Transform on the time variable:
\begin{equation}\label{MaxVacFullPPNmmmffffffhhtygghGGGGyuhggghghghffggggint}
U(\vec x,t)=\int \hat U(\vec x,\omega)e^{i\omega
t}d\omega\quad\text{where}\quad\hat U(\vec
x,\omega):=\frac{1}{2\pi}\int U(\vec x,t)e^{-i\omega t}dt\,.
\end{equation}
Moreover, by \er{MaxVacFullPPNmmmffffffiuiuhjuughbghhjjint} we
obtain that the Furier's Transform $\hat U(\vec x,\omega)$
satisfies:
\begin{equation}\label{MaxVacFullPPNmmmffffffiuiuhjuughbghhjjuggint}
\frac{1}{c^2_0}\left(i\omega\left(i\omega \hat U+\vec {\tilde
u}\cdot\nabla_{\vec x}\hat U\right)+div_{\vec x}
\left\{\left(i\omega \hat U+\vec {\tilde u}\cdot\nabla_{\vec x} \hat
U\right)\vec {\tilde u}\right\}\right)-\Delta_{\vec x}\hat U=0.
\end{equation}
Thus by \er{MaxVacFullPPNmmmffffffiuiuhjuughbghhjjuggint}, for every
given $\omega$ the monochromatic wave type function
\begin{equation}\label{MaxVacFullPPNmmmffffffhhtygghGGGGyuhggghghghffgggguiuint}
U_\omega(\vec x,t):=\hat U(\vec x,\omega)e^{i\omega t}
\end{equation}
is a complex solution of
\begin{equation}\label{MaxVacFullPPNmmmffffffiuiuhjuughbghhjjghghhint}
\frac{1}{c^2_0}\left(\frac{\partial}{\partial t}\left(\frac{\partial
U_\omega}{\partial t}+\vec {\tilde u}\cdot\nabla_{\vec
x}U_\omega\right)+div_{\vec x} \left\{\left(\frac{\partial
U_\omega}{\partial t}+\vec {\tilde u}\cdot\nabla_{\vec x}
U_\omega\right)\vec {\tilde u}\right\}\right)-\Delta_{\vec
x}U_\omega=0.
\end{equation}
Note that equation
\er{MaxVacFullPPNmmmffffffiuiuhjuughbghhjjghghhint} coincides with
\er{MaxVacFullPPNmmmffffffiuiuhjuughbghhjjint}. Moreover, by
\er{MaxVacFullPPNmmmffffffhhtygghGGGGyuhggghghghffggggint} a general
solution of \er{MaxVacFullPPNmmmffffffiuiuhjuughbghhjjint} can be
represented as a superposition of monochromatic waves of type
$U_\omega=f(\vec x,\omega)e^{i\omega t}$ that satisfy
\er{MaxVacFullPPNmmmffffffiuiuhjuughbghhjjghghhint} for every
$\omega$.

Next assume that a scalar \underline{complex} field $U:=U(\vec x,t)$
satisfies \er{MaxVacFullPPNmmmffffffiuiuhjuughbghhjjint}. In
particular, $U$ can be a monochromatic solution of
\er{MaxVacFullPPNmmmffffffiuiuhjuughbghhjjghghhint}. Although from
now we consider that the fields $\vec {\tilde u}$ and $c_0$ can
depend on the time variable, assume however, that we have the
following approximation, analogous to
\er{MaxVacFullPPNmmmffffffhhtygghGGGGyuhggghghint}: if the changes
of the physical characteristics of the field $U$ become essential in
the spatial landscape $L_e$ and the temporal landscape $T_e$, and
the changes of the field $\vec {\tilde u}$ becomes essential in the
spatial landscape $L_{u}$ and the temporal landscape $T_{u}$ , then
we assume
\begin{equation}\label{MaxVacFullPPNmmmffffffhhtygghGGGGyuhggghghghffgint}
\left(c_0T_e+L_e\right)\ll \left(c_0T_u+L_u\right),\quad\text{or
equivalently:}\quad \frac{\left(|\partial_t\vec {\tilde u}|+c_0|
d_{\vec x}\vec {\tilde u}|\right)}{|\vec {\tilde
u}|}\ll\frac{\left(|\partial_t U|+c_0| d_{\vec x}U|\right)}{|U|}.
\end{equation}
Furthermore, we represent the complex field $U$ as:
\begin{equation}\label{MaxVacFullPPNmmmffffffiuiuhjuughbghhuiiujjjjjjjjint}
U(\vec x,t)=A(\vec x,t)e^{iT(\vec x,t)},
\end{equation}
where $A:=A(\vec x,t)$ and $T:=T(\vec x,t)$ are real scalar fields.
Then define
\begin{equation}\label{MaxVacFullPPNmmmffffffiuiuhjuughbghhuiiujjjjjjjjhhhjjjint}
\omega:=\left<\,\left|\frac{\partial T}{\partial t}\right|\,\right>,
\end{equation}
where the sign $\left<\cdot\right>$ means the spatial and temporal
averaging. Next define $k_0$ and a scalar field $S:=S(\vec x,t)$ by
\begin{equation}\label{MaxVacFullPPNmmmffffffiuiuhjuughbghhuiiujjjjjjjjhhhjjjkkint}
k_0:=\frac{\omega}{c}\quad\quad\text{and}\quad\quad S(\vec
x,t)=\frac{1}{k_0}T(\vec x,t),
\end{equation}
where $c$ is a constant in the Maxwell equations for the vacuum. So
we clearly have
\begin{equation}\label{MaxVacFullPPNmmmffffffiuiuhjuughbghhuiiujjint}
U(\vec x,t)=A(\vec x,t)e^{ik_0S(\vec x,t)}\,.
\end{equation}
%Moreover, assume that the spatial and time derivatives of $\vec {\tilde u}$ of first and second order are negligible with respect to $k_0$.
Then, by \er{MaxVacFullPPNmmmffffffhhtygghGGGGyuhggghghghffgint} we
approximate equation \er{MaxVacFullPPNmmmffffffiuiuhjuughbghhjjint}
as:
\begin{equation}\label{MaxVacFullPPNmmmffffffiuiuhjuughbghhiuijuyuggkkjjint}
\frac{1}{c^2_0}\left(\frac{\partial^2 U}{\partial t^2}+2\vec {\tilde
u}\cdot\nabla_{\vec x}\left(\frac{\partial U}{\partial
t}\right)+\left(\nabla^2_{\vec x}U\cdot\vec {\tilde
u}\right)\cdot\vec {\tilde u}\right)-\Delta_{\vec x}U=0.
\end{equation}
Thus, inserting \er{MaxVacFullPPNmmmffffffiuiuhjuughbghhuiiujjint}
into \er{MaxVacFullPPNmmmffffffiuiuhjuughbghhiuijuyuggkkjjint},
%
%
%
\begin{comment}
we deduce:
\begin{multline}\label{MaxVacFullPPNmmmffffffiuiuhjuughbghhiuijjjint}
%\frac{1}{c^2_0}\frac{\partial^2}{\partial t^2}\left(Ae^{ik_0S}\right)
-\frac{k^2_0}{c^2_0}\left(\frac{\partial S}{\partial
t}\right)^2Ae^{ik_0S}+\frac{ik_0}{c^2_0}\left(\frac{\partial^2
S}{\partial t^2}\right)Ae^{ik_0S}+\frac{2ik_0}{c^2_0}\frac{\partial
A}{\partial t}\frac{\partial S}{\partial
t}e^{ik_0S}+\frac{1}{c^2_0}\frac{\partial^2 A}{\partial
t^2}e^{ik_0S}
\\-\frac{2k^2_0}{c^2_0}\frac{\partial S}{\partial
t}\left(\vec {\tilde u}\cdot\nabla_{\vec
x}S\right)Ae^{ik_0S}+\frac{2ik_0 }{c^2_0}\left(\vec {\tilde u}\cdot
\nabla_{\vec x}\left(\frac{\partial S}{\partial
t}\right)\right)Ae^{ik_0S}+\frac{2ik_0}{c^2_0}\left(\vec {\tilde
u}\cdot\nabla_{\vec x}A\right)\frac{\partial S}{\partial
t}e^{ik_0S}\\+\frac{2ik_0}{c^2_0}\left(\vec {\tilde
u}\cdot\nabla_{\vec x}S\right)\frac{\partial A}{\partial
t}e^{ik_0S}+\frac{2}{c^2_0}\vec {\tilde u}\cdot\nabla_{\vec
x}\left(\frac{\partial A}{\partial
t}\right)e^{ik_0S}-\frac{k^2_0}{c^2}\left|\vec {\tilde
u}\cdot\nabla_{\vec
x}S\right|^2Ae^{ik_0S}\\+\frac{ik_0}{c^2_0}\left(\left(\nabla^2_{\vec
x}S\cdot\vec {\tilde u}\right)\cdot\vec {\tilde
u}\right)Ae^{ik_0S}+\frac{2ik_0}{c^2_0}\left(\vec {\tilde
u}\cdot\nabla_{\vec x}A\right)\left(\vec {\tilde u}\cdot\nabla_{\vec
x}S\right)e^{ik_0S}+\frac{1}{c^2_0}\left(\left(\nabla^2_{\vec
x}A\cdot\vec {\tilde u}\right)\cdot\vec {\tilde
u}\right)e^{ik_0S}\\+k^2_0\left|\nabla_{\vec
x}S\right|^2Ae^{ik_0S}-ik_0\left(\Delta_{\vec
x}S\right)Ae^{ik_0S}-2ik_0\left(\nabla_{\vec x}A\cdot\nabla_{\vec
x}S\right)e^{ik_0S}-\left(\Delta_{\vec x}A\right)e^{ik_0S}=0.
\end{multline}
Then:
\begin{multline}\label{MaxVacFullPPNmmmffffffiuiuhjuughbghhiuijghghghjjint}
%\frac{1}{c^2_0}\frac{\partial^2}{\partial t^2}\left(Ae^{ik_0S}\right)
-\frac{k^2_0}{c^2_0}\left(\frac{\partial S}{\partial
t}\right)^2A+\frac{ik_0}{c^2_0}\left(\frac{\partial^2 S}{\partial
t^2}\right)A+\frac{2ik_0}{c^2_0}\frac{\partial A}{\partial
t}\frac{\partial S}{\partial t}+\frac{1}{c^2_0}\frac{\partial^2
A}{\partial t^2}-\frac{2k^2_0}{c^2_0}\frac{\partial S}{\partial
t}\left(\vec {\tilde u}\cdot\nabla_{\vec x}S\right)A\\+\frac{2ik_0
}{c^2_0}\left(\vec {\tilde u}\cdot \nabla_{\vec
x}\left(\frac{\partial S}{\partial
t}\right)\right)A+\frac{2ik_0}{c^2_0}\left(\vec {\tilde
u}\cdot\nabla_{\vec x}A\right)\frac{\partial S}{\partial
t}+\frac{2ik_0}{c^2_0}\left(\vec {\tilde u}\cdot\nabla_{\vec
x}S\right)\frac{\partial A}{\partial t}+\frac{2}{c^2_0}\vec {\tilde
u}\cdot\nabla_{\vec x}\left(\frac{\partial A}{\partial
t}\right)\\-\frac{k^2_0}{c^2}\left|\vec {\tilde u}\cdot\nabla_{\vec
x}S\right|^2A+\frac{ik_0}{c^2_0}\left(\left(\nabla^2_{\vec
x}S\cdot\vec {\tilde u}\right)\cdot\vec {\tilde
u}\right)A+\frac{2ik_0}{c^2_0}\left(\vec {\tilde u}\cdot\nabla_{\vec
x}A\right)\left(\vec {\tilde u}\cdot\nabla_{\vec
x}S\right)+\frac{1}{c^2_0}\left(\left(\nabla^2_{\vec x}A\cdot\vec
{\tilde u}\right)\cdot\vec {\tilde
u}\right)\\+k^2_0\left|\nabla_{\vec
x}S\right|^2A-ik_0\left(\Delta_{\vec
x}S\right)A-2ik_0\left(\nabla_{\vec x}A\cdot\nabla_{\vec
x}S\right)-\left(\Delta_{\vec x}A\right)=0.
\end{multline}
\end{comment}
%
%
%
and comparing both real and imaginary part of
\er{MaxVacFullPPNmmmffffffiuiuhjuughbghhiuijuyuggkkjjint} to zero we
obtain two equations:
%Thus, since the zero complex number has both real and imaginary part
%equal to zero, by \er{MaxVacFullPPNmmmffffffiuiuhjuughbghhiuijghghghjjint} we have:
\begin{multline}\label{MaxVacFullPPNmmmffffffiuiuhjuughbghhiuijghghghhjhjhjjint}
k^2_0\left(\left|\nabla_{\vec
x}S\right|^2-\frac{1}{c^2_0}\left(\frac{\partial S}{\partial t}+\vec
{\tilde u}\cdot\nabla_\vec x
S\right)^2\right)A+\frac{1}{c^2_0}\left(\frac{\partial^2 A}{\partial
t^2}+2\vec {\tilde u}\cdot\nabla_{\vec x}\left(\frac{\partial
A}{\partial t}\right)+\left(\left(\nabla^2_{\vec x}A\cdot\vec
{\tilde u}\right)\cdot\vec {\tilde u}\right)\right)-\Delta_{\vec
x}A=0,
%
%
%
%\\-\frac{k^2_0}{c^2_0}\left(\frac{\partial S}{\partial t}\right)^2A+\frac{1}{c^2_0}\frac{\partial^2 A}{\partial t^2}
%-\frac{2k^2_0}{c^2_0}\frac{\partial S}{\partial t}\left(\vec {\tilde u}\cdot\nabla_{\vec x}S\right)A+\frac{2}{c^2_0}\vec {\tilde
%u}\cdot\nabla_{\vec x}\left(\frac{\partial A}{\partial t}\right)-\frac{k^2_0}{c^2_0}\left|\vec {\tilde u}\cdot\nabla_{\vec
%x}S\right|^2A+\frac{1}{c^2_0}\left(\left(\nabla^2_{\vec x}A\cdot\vec{\tilde u}\right)\cdot\vec {\tilde
%u}\right)\\+k^2_0\left|\nabla_{\vec x}S\right|^2A-\Delta_{\vec x}A=0,
\end{multline}
and
\begin{multline}\label{MaxVacFullPPNmmmffffffiuiuhjuughbghhiuijghghghhhfhjjint}
\frac{1}{c^2_0}\left(\frac{\partial^2 S}{\partial t^2}+2\vec {\tilde
u}\cdot \nabla_{\vec x}\left(\frac{\partial S}{\partial
t}\right)+\left(\nabla^2_{\vec x}S\cdot\vec {\tilde
u}\right)\cdot\vec {\tilde
u}\right)A\\+\frac{2}{c^2_0}\left(\frac{\partial S}{\partial t}+\vec
{\tilde u}\cdot\nabla_{\vec x}S\right)\left(\frac{\partial
A}{\partial t}+\vec {\tilde u}\cdot\nabla_{\vec
x}A\right)-\left(\Delta_{\vec x}S\right)A-2\nabla_{\vec
x}A\cdot\nabla_{\vec x}S=0,
%=\\ \frac{1}{c^2_0}\left(\frac{\partial^2 S}{\partial t^2}\right)A+\frac{2}{c^2_0}\frac{\partial A}{\partial
%t}\frac{\partial S}{\partial t}+\frac{2 }{c^2_0}\left(\vec {\tilde u}\cdot \nabla_{\vec x}\left(\frac{\partial S}{\partial
%t}\right)\right)A+\frac{2}{c^2_0}\left(\vec {\tilde u}\cdot\nabla_{\vec x}A\right)\frac{\partial S}{\partial
%t}+\frac{2}{c^2_0}\left(\vec {\tilde u}\cdot\nabla_{\vec x}S\right)\frac{\partial A}{\partial
%t}\\+\frac{1}{c^2_0}\left(\left(\nabla^2_{\vec x}S\cdot\vec {\tilde u}\right)\cdot\vec {\tilde u}\right)A+\frac{2}{c^2_0}\left(\vec
%{\tilde u}\cdot\nabla_{\vec x}A\right)\left(\vec {\tilde u}\cdot\nabla_{\vec x}S\right)-\left(\Delta_{\vec x}S\right)A-2\nabla_{\vec x}A\cdot\nabla_{\vec x}S=0.
\end{multline}
(see subsection \ref{GO} for details).

Next assume the Geometric Optics approximation that is good for the
electromagnetic wave of high frequency for example for the visible
light. The Geometric Optics approximation means the following:
assume that the changes in time of $c_0$, $A$ and $S$ become
essential after certain interval of time $T_e$ and the changes in
space of $c_0$, $A$ and $S$ become essential in the spatial
landscape $L_e$. Then we assume that
\begin{equation}\label{slochangGGaaffgfgjhjgghint}
k^2_0c^2_0T^2_e\,\gg\, 1\quad\text{and}\quad k^2_0L^2_e\,\gg\, 1.
\end{equation}
%together with the assumption \er{MaxVacFullPPNmmmffffffhhtygghGGGGyuhggghghghffg}.
Thus, by \er{slochangGGaaffgfgjhjgghint} we approximate
\er{MaxVacFullPPNmmmffffffiuiuhjuughbghhiuijghghghhjhjhjjint} as the
Eikonal-type equation:
\begin{equation}\label{MaxVacFullPPNmmmffffffiuiuhjuughbghhiuijghghghhjhjhhghyuyjjjhhjhjffint}
\frac{1}{c^2_0}\left(\frac{\partial S}{\partial t}+\vec {\tilde
u}\cdot\nabla_\vec x S\right)^2=\left|\nabla_\vec x S\right|^2,
\end{equation}
and using \er{MaxVacFullPPNmmmffffffhhtygghGGGGyuhggghghghffgint} we
approximate
%rewrite
\er{MaxVacFullPPNmmmffffffiuiuhjuughbghhiuijghghghhhfhjjint} as:
%
%
%
\begin{comment}
\begin{multline}\label{MaxVacFullPPNmmmffffffiuiuhjuughbghhiuijghghghhhfhhghghguygtjjint}
\frac{1}{c^2_0}\left(\frac{\partial^2 S}{\partial t^2}+2\vec {\tilde
u}\cdot \nabla_{\vec x}\left(\frac{\partial S}{\partial
t}\right)+\left(\nabla^2_{\vec x}S\cdot\vec {\tilde
u}\right)\cdot\vec {\tilde
u}\right)A\\+\frac{2}{c^2_0}\left(\frac{\partial S}{\partial t}+\vec
{\tilde u}\cdot\nabla_{\vec x}S\right)\left(\frac{\partial
A}{\partial t}+\vec {\tilde u}\cdot\nabla_{\vec
x}A\right)-\left(\Delta_{\vec x}S\right)A-2\nabla_{\vec
x}A\cdot\nabla_{\vec x}S=0\,.
\end{multline}
\end{comment}
%
%
%
%
%
%
\begin{comment}
\begin{multline}\label{MaxVacFullPPNmmmffffffiuiuhjuughbghhiuijghghghhhfhhghghguygtjjint}
\\
%
%
%
%\frac{1}{c^2_0}\frac{\partial^2}{\partial t^2}\left(Ae^{ik_0S}\right)
\frac{1}{c^2_0}\left(\frac{\partial^2 S}{\partial
t^2}\right)A+\frac{2}{c^2_0}\frac{\partial A}{\partial
t}\frac{\partial S}{\partial t}+\frac{2 }{c^2_0}\left(\vec {\tilde
u}\cdot \nabla_{\vec x}\left(\frac{\partial S}{\partial
t}\right)\right)A+\frac{2}{c^2_0}\left(\vec {\tilde
u}\cdot\nabla_{\vec x}A\right)\frac{\partial S}{\partial
t}+\frac{2}{c^2_0}\left(\vec {\tilde u}\cdot\nabla_{\vec
x}S\right)\frac{\partial A}{\partial
t}\\+\frac{1}{c^2_0}\left(\left(\nabla^2_{\vec x}S\cdot\vec {\tilde
u}\right)\cdot\vec {\tilde u}\right)A+\frac{2}{c^2_0}\left(\vec
{\tilde u}\cdot\nabla_{\vec x}A\right)\left(\vec {\tilde
u}\cdot\nabla_{\vec x}S\right)-\left(\Delta_{\vec
x}S\right)A-2\nabla_{\vec x}A\cdot\nabla_{\vec x}S=0\,.
\end{multline}
\end{comment}
%
%
%
%Further approximation of
%\er{MaxVacFullPPNmmmffffffiuiuhjuughbghhiuijghghghhhfhhghghguygtjjint},
%due to \er{MaxVacFullPPNmmmffffffhhtygghGGGGyuhggghghghffgint}
%gives:
\begin{multline}\label{MaxVacFullPPNmmmffffffiuiuhjuughbghhiuijghghghhhfhhghghguygtjjjkkjjkkhhint}
%\left(
\frac{1}{c^2_0}\left(\frac{\partial}{\partial t}\left(\frac{\partial
S}{\partial t}+\vec {\tilde u}\cdot\nabla_{\vec x}S\right)+div_{\vec
x} \left\{\left(\frac{\partial S}{\partial t}+\vec {\tilde
u}\cdot\nabla_{\vec x} S\right)\vec {\tilde
u}\right\}\right)A-\left(\Delta_{\vec x}S\right)A
%\right)
\\+\frac{2}{c^2_0}\left(\frac{\partial S}{\partial
t}+\vec {\tilde u}\cdot\nabla_{\vec x}S\right)\left(\frac{\partial
A}{\partial t}+\vec {\tilde u}\cdot\nabla_{\vec x}A\right)
%-\left(\Delta_{\vec x}S\right)A
-2\nabla_{\vec x}S\cdot\nabla_{\vec x}A=0\,,
\end{multline}
%and we write again the Eikonal type equation
%\er{MaxVacFullPPNmmmffffffiuiuhjuughbghhiuijghghghhjhjhhghyuyjjint}:
%\begin{equation}\label{MaxVacFullPPNmmmffffffiuiuhjuughbghhiuijghghghhjhjhhghyuyjjjhhjhjffint}
%\frac{1}{c^2_0}\left(\frac{\partial S}{\partial t}+\vec {\tilde
%u}\cdot\nabla_\vec x S\right)^2=\left|\nabla_\vec x S\right|^2.
%\end{equation}
Then, as before, we deduce that equation
\er{MaxVacFullPPNmmmffffffiuiuhjuughbghhiuijghghghhjhjhhghyuyjjjhhjhjffint}
is invariant under the change of non-inertial cartesian coordinate
system, provided that under such change we have $S'=S$. Moreover,
\er{MaxVacFullPPNmmmffffffiuiuhjuughbghhiuijghghghhhfhhghghguygtjjjkkjjkkhhint}
is also invariant under the change of non-inertial cartesian
coordinate system, in the case that under such change we have
$A'=A$, provided that $S'=S$. So if the approximations
\er{MaxVacFullPPNmmmffffffhhtygghGGGGyuhggghghghffgint} and
\er{slochangGGaaffgfgjhjgghint} are valid in some cartesian
coordinate system $(*)$, then we can use
\er{MaxVacFullPPNmmmffffffiuiuhjuughbghhiuijghghghhjhjhhghyuyjjjhhjhjffint}
and
\er{MaxVacFullPPNmmmffffffiuiuhjuughbghhiuijghghghhhfhhghghguygtjjjkkjjkkhhint}
also in any other inertial or non-inertial cartesian coordinate
system $(**)$ even in the case when
\er{MaxVacFullPPNmmmffffffhhtygghGGGGyuhggghghghffgint} and
\er{slochangGGaaffgfgjhjgghint} are not valid in the system $(**)$,
provided that under the change of coordinate system we have $A'=A$
and $S'=S$.

\subsubsection{The case of the monochromatic wave}
%\label{mnGO}
Next, up to the end of this subsection, consider the case of
monochromatic wave of the constant frequency
$\nu=\frac{\omega}{2\pi}$ where the fields $\vec {\tilde u}$ and
$c_0$ are independent on the time variable i.e. the case of
\er{MaxVacFullPPNmmmffffffiuiuhjuughbghhuiiujjjjjjjjint} where we
have
\begin{equation}\label{MaxVacFullPPNmmmffffffiuiuhjuughbghhiuijghghghhjhjhhghyuyhhjhjihikjjjjkjljklint}
\begin{cases}
\frac{\partial T}{\partial t}=\omega\\
\frac{\partial A}{\partial t}=0\\
\frac{\partial \vec {\tilde u}}{\partial t}=0\\
\frac{\partial c_0}{\partial t}=0.
\end{cases}
\end{equation}
Then, by
\er{MaxVacFullPPNmmmffffffiuiuhjuughbghhuiiujjjjjjjjhhhjjjint} and
\er{MaxVacFullPPNmmmffffffiuiuhjuughbghhuiiujjjjjjjjhhhjjjkkint} we
rewrite
\er{MaxVacFullPPNmmmffffffiuiuhjuughbghhiuijghghghhjhjhhghyuyhhjhjihikjjjjkjljklint}
as
\begin{equation}\label{MaxVacFullPPNmmmffffffiuiuhjuughbghhiuijghghghhjhjhhghyuyhhjhjihikjjjint}
\begin{cases}
\frac{\partial S}{\partial t}=c\\
\frac{\partial A}{\partial t}=0.
\end{cases}
\end{equation}
Thus $\nabla_{\vec x}S$ is independent on $t$ and moreover, by
\er{MaxVacFullPPNmmmffffffiuiuhjuughbghhiuijghghghhjhjhhghyuyhhjhjihikjjjint}
we rewrite
\er{MaxVacFullPPNmmmffffffiuiuhjuughbghhiuijghghghhjhjhhghyuyjjjhhjhjffint}
as:
\begin{equation}\label{MaxVacFullPPNmmmffffffiuiuhjuughbghhiuijghghghhjhjhhghyuyiyyujjint}
\frac{c^2}{c^2_0}\left(1+\frac{1}{c}\vec {\tilde u}\cdot\nabla_\vec
x S\right)^2=\left|\nabla_\vec x S\right|^2,
\end{equation}
and, using
\er{MaxVacFullPPNmmmffffffiuiuhjuughbghhiuijghghghhjhjhhghyuyhhjhjihikjjjint}
and \er{MaxVacFullPPNmmmffffffhhtygghGGGGyuhggghghghffgint} we
rewrite
\er{MaxVacFullPPNmmmffffffiuiuhjuughbghhiuijghghghhhfhhghghguygtjjjkkjjkkhhint}
as:
\begin{equation}\label{MaxVacFullPPNmmmffffffiuiuhjuughbghhiuijghghghhhfhhghghguygtjuuujjint}
%\frac{1}{c^2_0}\frac{\partial^2}{\partial t^2}\left(Ae^{ik_0S}\right)
2\left(\nabla_{\vec x}S-\frac{c}{c_0}\left(1+\frac{1}{c}\left(\vec
{\tilde u}\cdot\nabla_{\vec x}S\right)\right)\frac{\vec {\tilde
u}}{c_0}\right)\cdot\nabla_{\vec
x}A=\left(\frac{1}{c^2_0}\left(\left(\nabla^2_{\vec x}S\cdot\vec
{\tilde u}\right)\cdot\vec {\tilde u}\right)-\left(\Delta_{\vec
x}S\right)\right)A.
\end{equation}
%\end{comment*}
%
%
%
%
%
%
%
%
\begin{comment}
\begin{multline}\label{MaxMedFullGGffgggyyojjhhjkhjyyiuhggjhhjhuyytytyuuytrrtghjtyuggyuighjuyioyyfgffhyuhhghzzrrhhkkkhhjjh}
\int\limits_a^b\sqrt{\left(1-\frac{\left|\vec {\tilde u}\left(\vec
r(s)\right)\right|^2}{c^2_0}\right)\left|\vec h\left(\vec
r(s)\right)-\frac{1}{\left|\vec {\tilde u}\left(\vec
r(s)\right)\right|^2}\left(\vec {\tilde u}\left(\vec
r(s)\right)\cdot\vec h\left(\vec r(s)\right)\right)\vec {\tilde
u}\left(\vec r(s)\right)\right|^2+\left|\frac{1}{\left|\vec {\tilde
u}\right|}\vec {\tilde u}\cdot\vec
h\right|^2}\cdot\\
\cdot\sqrt{\left(1-\frac{\left|\vec {\tilde u}\left(\vec
r(s)\right)\right|^2}{c^2_0}\right)^{-1}\left|\vec
r'(s)-\frac{1}{\left|\vec {\tilde u}\left(\vec
r(s)\right)\right|^2}\left(\vec {\tilde u}\left(\vec
r(s)\right)\cdot\vec r'(s)\right)\vec {\tilde u}\left(\vec
r(s)\right)\right|^2+\left|\frac{1}{\left|\vec {\tilde u}\left(\vec
r(s)\right)\right|}\vec {\tilde u}\left(\vec r(s)\right)\cdot\vec
r'(s)\right|^2}ds\\ =\int\limits_a^b\frac{c}{c_0\left(\vec
r(s)\right)}\cdot\\ \cdot\sqrt{\left(1-\frac{\left|\vec {\tilde
u}\left(\vec r(s)\right)\right|^2}{c^2_0}\right)^{-1}\left|\vec
r'(s)-\frac{1}{\left|\vec {\tilde u}\left(\vec
r(s)\right)\right|^2}\left(\vec {\tilde u}\left(\vec
r(s)\right)\cdot\vec r'(s)\right)\vec {\tilde u}\left(\vec
r(s)\right)\right|^2+\left|\frac{1}{\left|\vec {\tilde u}\left(\vec
r(s)\right)\right|}\vec {\tilde u}\left(\vec r(s)\right)\cdot\vec
r'(s)\right|^2}ds.
\end{multline}


\begin{multline}\label{MaxMedFullGGffgggyyojjhhjkhjyyiuhggjhhjhuyytytyuuytrrtghjtyuggyuighjuyioyyfgffhyuhhghzzrrkkhhkkkhhhjhkjhhghhgghhhihiy}
J\left(\vec r(\cdot)\right):= -\int_a^b\frac{1}{c}n^2\left(\vec
r(s)\right)\left(1-\frac{\left|\vec {\tilde u}\left(\vec
r(s)\right)\right|^2}{c^2_0}\right)^{-1}\vec {\tilde u}\left(\vec
r(s)\right)\cdot\vec r'(s)ds\\+\int\limits_a^bn\left(\vec
r(s)\right)\left(1-\frac{\left|\vec {\tilde u}\left(\vec
r(s)\right)\right|^2}{c^2_0}\right)^{-\frac{1}{2}}
\cdot\\
\cdot\sqrt{\left|\vec r'(s)-\frac{1}{\left|\vec {\tilde u}\left(\vec
r(s)\right)\right|^2}\left(\vec {\tilde u}\left(\vec
r(s)\right)\cdot\vec r'(s)\right)\vec {\tilde u}\left(\vec
r(s)\right)\right|^2+\left(1-\frac{\left|\vec {\tilde u}\left(\vec
r(s)\right)\right|^2}{c^2_0}\right)^{-1}\left|\frac{1}{\left|\vec
{\tilde u}\left(\vec r(s)\right)\right|}\vec {\tilde u}\left(\vec
r(s)\right)\cdot\vec r'(s)\right|^2}ds,
\end{multline}



\begin{equation}\label{MaxVacFullPPNmmmffffffiuiuhjuughbghhiuijghghghhhfhhghghguygtjuuujjjkyuuyhhijhhhhjihhj}
%\frac{1}{c^2_0}\frac{\partial^2}{\partial t^2}\left(Ae^{ik_0S}\right)
\vec h:=\frac{c}{c^2_0}\left(1+\frac{1}{c}\left(\vec {\tilde
u}\cdot\nabla_{\vec x}S\right)\right)\vec {\tilde u}-\nabla_{\vec
x}S,
\end{equation}

\begin{equation}\label{MaxVacFullPPNmmmffffffiuiuhjuughbghhiuijghghghhjhjhhghyuyiyyujjkhgggghggutggjkkjklkj}
\left(1-\frac{\left|\vec {\tilde
u}\right|^2}{c^2_0}\right)\left|\vec h-\frac{1}{\left|\vec {\tilde
u}\right|^2}\left(\vec {\tilde u}\cdot\vec h\right)\vec {\tilde
u}\right|^2+\left|\frac{1}{\left|\vec {\tilde u}\right|}\vec {\tilde
u}\cdot\vec h\right|^2=\left|\vec h\right|^2\left(1-\frac{\left|\vec
{\tilde u}\right|^2}{c^2_0}\right)+\frac{1}{c^2_0}\left|\vec {\tilde
u}\cdot\vec h\right|^2= \frac{c^2}{c^2_0},
\end{equation}

\begin{equation}\label{MaxVacFullPPNmmmffffffiuiuhjuughbghhiuijghghghhhfhhghghguygtjuuujjjkyuuyhhijhhhhj}
%\frac{1}{c^2_0}\frac{\partial^2}{\partial t^2}\left(Ae^{ik_0S}\right)
\frac{1}{|\vec {\tilde u}|}\vec {\tilde u}\cdot\vec
h=\frac{c}{c^2_0}|\vec {\tilde u}|-\left(\frac{1}{|\vec {\tilde
u}|}\vec {\tilde u}\cdot\nabla_{\vec x}S\right)\left(1-\frac{|\vec
{\tilde u}|^2}{c^2_0}\right),
\end{equation}
\end{comment}
%
%
%
%







































In particular, in the case of the region of the space where the
following approximation is valid:
\begin{equation}\label{ojhkkint}
\frac{|\vec {\tilde u}|^2}{c^2_0}\ll 1,
\end{equation}
up to order $O\left(\frac{|\vec {\tilde u}|^2}{c^2_0}\right)$, we
rewrite
\er{MaxVacFullPPNmmmffffffiuiuhjuughbghhiuijghghghhjhjhhghyuyiyyujjint}
as:
\begin{equation}\label{MaxVacFullPPNmmmffffffiuiuhjuughbghhiuijghghghhjhjhhghyuyiyyujjljkint}
\left|\frac{c\vec {\tilde u}}{c^2_0}-\nabla_\vec x
S\right|^2=\frac{c^2}{c^2_0},
\end{equation}
and
\er{MaxVacFullPPNmmmffffffiuiuhjuughbghhiuijghghghhhfhhghghguygtjuuujjint}
as:
\begin{equation}\label{MaxVacFullPPNmmmffffffiuiuhjuughbghhiuijghghghhhfhhghghguygtjuuujjjkint}
%\frac{1}{c^2_0}\frac{\partial^2}{\partial t^2}\left(Ae^{ik_0S}\right)
\left(\frac{c\vec {\tilde u}}{c^2_0}-\nabla_{\vec
x}S\right)\cdot\nabla_{\vec x}A+\frac{1}{2}\left(-\Delta_{\vec
x}S\right)A=0.
\end{equation}
The Eikonal equation
\er{MaxVacFullPPNmmmffffffiuiuhjuughbghhiuijghghghhjhjhhghyuyiyyujjljkint}
and equation of the beam propagation
\er{MaxVacFullPPNmmmffffffiuiuhjuughbghhiuijghghghhhfhhghghguygtjuuujjjkint}
are two basic equations of propagation of monochromatic light in the
Geometric Optics approximation inside a moving medium or/and in the
presence of non-trivial gravitational field, provided that the field
$\vec {\tilde u}$ satisfies \er{ojhkkint}.

Next if we consider an arbitrary characteristic curve $\vec
r(s):[a,b]\to\mathbb{R}^3$ of equation
\er{MaxVacFullPPNmmmffffffiuiuhjuughbghhiuijghghghhhfhhghghguygtjuuujjjkint}
defined as a solution of ordinary differential equation
\begin{equation}\label{MaxVacFullPPNmmmffffffiuiuhjuughbghhiuijghghghhhfhhghghguygtjuuujjjkkint}
\begin{cases}
\frac{d\vec r}{ds}(s)=\frac{c}{c^2_0\left(\vec r(s)\right)}\vec
{\tilde u}\left(\vec
r(s)\right)-\nabla_{\vec x}S\left(\vec r(s)\right)\\
\vec r(a)=\vec x_0,
\end{cases}
\end{equation}
then, as before, by
\er{MaxVacFullPPNmmmffffffiuiuhjuughbghhiuijghghghhhfhhghghguygtjuuujjjkint}
and
\er{MaxVacFullPPNmmmffffffiuiuhjuughbghhiuijghghghhhfhhghghguygtjuuujjjkkint}
we have
\begin{equation}\label{MaxVacFullPPNmmmffffffiuiuhjuughbghhiuijghghghhhfhhghghguygtjuuujjjkhjhjhint}
%\frac{1}{c^2_0}\frac{\partial^2}{\partial t^2}\left(Ae^{ik_0S}\right)
\frac{d}{ds}\left(A\left(\vec r(s)\right)\right)=\nabla_{\vec
x}A\left(\vec r(s)\right)\cdot\frac{d\vec
r}{ds}(s)=\frac{1}{2}\left(\Delta_{\vec x}S\left(\vec
r(s)\right)\right)A\left(\vec r(s)\right),
\end{equation}
that implies
\begin{equation}\label{MaxVacFullPPNmmmffffffiuiuhjuughbghhiuijghghghhhfhhghghguygtjuuujjjkhjhjhffggint}
A\left(\vec r(s)\right)=A\left(\vec
x_0\right)e^{\frac{1}{2}\int_a^{s}\left(\Delta_{\vec x}S\left(\vec
r(\tau)\right)\right)d\tau}\quad\quad\forall s\in[a,b].
\end{equation}
In particular,
\begin{equation}\label{MaxVacFullPPNmmmffffffiuiuhjuughbghhiuijghghghhhfhhghghguygtjuuujjjkhjhjhffggiouuiiuint}
A\left(\vec x_0\right)=0\;\;\text{implies}\;\; A\left(\vec
r(s)\right)=0\quad\forall s\in[a,b],\quad\text{and}\quad A\left(\vec
x_0\right)\neq 0\;\;\text{implies}\;\; A\left(\vec r(s)\right)\neq
0\quad\forall s\in[a,b].
\end{equation}
Therefore, by
\er{MaxVacFullPPNmmmffffffiuiuhjuughbghhiuijghghghhhfhhghghguygtjuuujjjkhjhjhffggiouuiiuint}
we deduce that in the case of \er{ojhkkint} the curve that satisfies
\er{MaxVacFullPPNmmmffffffiuiuhjuughbghhiuijghghghhhfhhghghguygtjuuujjjkkint}
coincides with the beam of light that passes through the point $\vec
x_0$. So in the case of \er{ojhkkint}, equality
\er{MaxVacFullPPNmmmffffffiuiuhjuughbghhiuijghghghhhfhhghghguygtjuuujjjkkint}
is the equation of a beam and the vector field $\vec h$ defined for
every $\vec x$ by:
\begin{equation}\label{MaxVacFullPPNmmmffffffiuiuhjuughbghhiuijghghghhhfhhghghguygtjuuujjjkyuuyint}
%\frac{1}{c^2_0}\frac{\partial^2}{\partial t^2}\left(Ae^{ik_0S}\right)
\vec h(\vec x):=\frac{c}{c^2_0(\vec x)}\vec {\tilde u}(\vec
x)-\nabla_{\vec x}S(\vec x),
\end{equation}
is the direction of the propagation of the beam that passes through
point $\vec x$. Moreover, by
\er{MaxVacFullPPNmmmffffffiuiuhjuughbghhiuijghghghhjhjhhghyuyiyyujjljkint}
$\vec h$ satisfies
\begin{equation}\label{MaxVacFullPPNmmmffffffiuiuhjuughbghhiuijghghghhjhjhhghyuyiyyujjljkgghhgint}
|\vec h|^2=\frac{c^2}{c^2_0}.
\end{equation}
%
%
%
%
\begin{comment}
Next, under the approximation \er{ojhkkint}
%again assume that the approximate equations in \er{MaxVacFullPPNmmmffffffiuiuhjuughbghhiuijghghghhjhjhhghyuyiyyujjljk}
%and \er{MaxVacFullPPNmmmffffffiuiuhjuughbghhiuijghghghhhfhhghghguygtjuuujjjk} are valid in the given region and
consider a curve $\vec r(s):[a,b]\to\mathbb{R}^3$ with endpoints
$\vec r(a)=N$ and $\vec r(b)=M$. Then integrating the square root of
both sides of
\er{MaxVacFullPPNmmmffffffiuiuhjuughbghhiuijghghghhjhjhhghyuyiyyujjljkint}
over the curve $\vec r(s)$ we deduce
\begin{equation}\label{MaxMedFullGGffgggyyojjhhjkhjyyiuhggjhhjhuyytytyuuytrrtghjtyuggyuighjuyioyyfgffhyuhhghzzrrhhkkkint}
\int_a^b\left|\frac{c}{c^2_0\left(\vec r(s)\right)}\vec {\tilde
u}\left(\vec r(s)\right)-\nabla_{\vec x}S\left(\vec
r(s)\right)\right|\,\left|\vec
r'(s)\right|ds=\int_a^b\frac{c}{c_0\left(\vec r(s)\right)}\left|\vec
r'(s)\right|ds.
\end{equation}
Thus in particular,
\begin{equation}\label{MaxMedFullGGffgggyyojjhhjkhjyyiuhggjhhjhuyytytyuuytrrtghjtyuggyuighjuyioyyfgffhyuhhghzzrriuihhkkkint}
\int_a^b\left(\frac{c}{c^2_0\left(\vec r(s)\right)}\vec {\tilde
u}\left(\vec r(s)\right)-\nabla_{\vec x}S\left(\vec
r(s)\right)\right)\cdot\vec
r'(s)ds\leq\int_a^b\frac{c}{c_0\left(\vec r(s)\right)}\left|\vec
r'(s)\right|ds,
\end{equation}
i.e.
\begin{equation}\label{MaxMedFullGGffgggyyojjhhjkhjyyiuhggjhhjhuyytytyuuytrrtghjtyuggyuighjuyioyyfgffhyuhhghzzrrkkijjhjhhkkkint}
\left(-S(M)\right)-
\left(-S(N)\right)\leq\int_a^b\frac{c}{c_0\left(\vec
r(s)\right)}\left|\vec
r'(s)\right|ds-\int_a^b\frac{c}{c^2_0\left(\vec r(s)\right)}\vec
{\tilde u}\left(\vec r(s)\right)\cdot\vec r'(s)ds.
\end{equation}
Moreover, if
\begin{equation}\label{MaxMedFullGGffgggyyojjyugggjhhjzzrrhhkkkint}
\frac{d\vec r}{ds}(s)=\sigma(s)\vec h\left(\vec
r(s)\right):=\sigma(s)\left(\frac{c}{c^2_0\left(\vec
r(s)\right)}\vec {\tilde u}\left(\vec r(s)\right)-\nabla_{\vec
x}S\left(\vec r(s)\right)\right),
\end{equation}
for some nonnegative scalar factor $\sigma=\sigma(s)$ then by
\er{MaxMedFullGGffgggyyojjyugggjhhjzzrrhhkkkint} we rewrite
\er{MaxMedFullGGffgggyyojjhhjkhjyyiuhggjhhjhuyytytyuuytrrtghjtyuggyuighjuyioyyfgffhyuhhghzzrrhhkkkint}
as
\begin{equation}\label{MaxMedFullGGffgggyyojjhhjkhjyyiuhggjhhjhuyytytyuuytrrtghjtyuggyuighjuyioyyfgffhyuhhghzzrrkkhhkkkint}
\left(-S(M)\right)-
\left(-S(N)\right)=\int_a^b\frac{c}{c_0\left(\vec
r(s)\right)}\left|\vec
r'(s)\right|ds-\int_a^b\frac{c}{c^2_0\left(\vec r(s)\right)}\vec
{\tilde u}\left(\vec r(s)\right)\cdot\vec r'(s)ds.
\end{equation}
Thus, by comparing
\er{MaxVacFullPPNmmmffffffiuiuhjuughbghhiuijghghghhhfhhghghguygtjuuujjjkkint}
with \er{MaxMedFullGGffgggyyojjyugggjhhjzzrrhhkkkint} and using
\er{MaxMedFullGGffgggyyojjhhjkhjyyiuhggjhhjhuyytytyuuytrrtghjtyuggyuighjuyioyyfgffhyuhhghzzrrkkijjhjhhkkkint}
and
\er{MaxMedFullGGffgggyyojjhhjkhjyyiuhggjhhjhuyytytyuuytrrtghjtyuggyuighjuyioyyfgffhyuhhghzzrrkkhhkkkint},
we deduce that if we assume that the light travel from the point $N$
to the point $M$ across the curve $\vec {\tilde
r}(s):[a,b]\to\mathbb{R}^3$ such that $\vec{\tilde r(a)}=N$ and
$\vec {\tilde r}(b)=M$, then
\begin{equation}\label{MaxMedFullGGffgggyyojjhhjkhjyyiuhggjhhjhuyytytyuuytrrtghjtyuggyuighjuyioyyfgffhyuhhghzzrrkkhhkkkhhhint}
\left(-S(M)\right)-
\left(-S(N)\right)=\int_a^b\frac{c}{c_0\left(\vec{\tilde
r}(s)\right)}\left|\vec {\tilde
r}'(s)\right|ds-\int_a^b\frac{c}{c^2_0\left(\vec{\tilde
r}(s)\right)}\vec {\tilde u}\left(\vec{\tilde r}(s)\right)\cdot\vec
{\tilde r}'(s)ds,
\end{equation}
and for every other curve $\vec r(s):[a,b]\to\mathbb{R}^3$ with
endpoints $\vec r(a)=N$ and $\vec r(b)=M$ we have
%\er{MaxMedFullGGffgggyyojjhhjkhjyyiuhggjhhjhuyytytyuuytrrtghjtyuggyuighjuyioyyfgffhyuhhghzzrrkkhhkkk}.
\begin{multline}\label{MaxMedFullGGffgggyyojjhhjkhjyyiuhggjhhjhuyytytyuuytrrtghjtyuggyuighjuyioyyfgffhyuhhghzzrrkkhhkkkhhhjhkjhhint}
\int_a^b\frac{c}{c_0\left(\vec r(s)\right)}\left|\vec
r'(s)\right|ds-\int_a^b\frac{c}{c^2_0\left(\vec r(s)\right)}\vec
{\tilde u}\left(\vec r(s)\right)\cdot\vec r'(s)ds\geq\\
\int_a^b\frac{c}{c_0\left(\vec{\tilde r}(s)\right)}\left|\vec
{\tilde r}'(s)\right|ds-\int_a^b\frac{c}{c^2_0\left(\vec{\tilde
r}(s)\right)}\vec {\tilde u}\left(\vec{\tilde r}(s)\right)\cdot\vec
{\tilde r}'(s)ds.
\end{multline}
\end{comment}
%
%
%
%
Next by
\er{MaxVacFullPPNmmmffffffiuiuhjuughbghhiuijghghghhjhjhhghyuyiyyujjljkint}
and
\er{MaxVacFullPPNmmmffffffiuiuhjuughbghhiuijghghghhhfhhghghguygtjuuujjjkint}
in subsection \ref{GO} we prove the following Fermat Principle:
\begin{proposition}\label{gughghfint}
Assume Geometric Optics approximation together with \er{ojhkkint}.
Then the light that travels from point $N$ to point $M$ chooses the
path $\vec r(s):[a,b]\to\mathbb{R}^3$ with endpoints $\vec r(a)=N$
and $\vec r(b)=M$ which minimizes the quantity:
\begin{equation}\label{MaxMedFullGGffgggyyojjhhjkhjyyiuhggjhhjhuyytytyuuytrrtghjtyuggyuighjuyioyyfgffhyuhhghzzrrkkhhkkkhhhjhkjhhghhgghint}
J\left(\vec r(\cdot)\right):=\int_a^bn\left(\vec
r(s)\right)\left|\vec r'(s)\right|ds-\int_a^b
\frac{1}{c}n^2\left(\vec r(s)\right)\vec {\tilde u}\left(\vec
r(s)\right)\cdot\vec r'(s)ds,
\end{equation}
where we set the refraction index:
\begin{equation}\label{MaxMedFullGGffgggyyojjhhjkhjyyiuhggjhhjhuyytytyuuytrrtghjtyuggyuighjuyioyyfgffhyuhhghzzrrkkhhkkkhhhjhkjhhghhgghiuiu1int}
n\left(\vec x\right):=\frac{c}{c_0\left(\vec x\right)}.
\end{equation}
Moreover, if the path $\vec r(s):[a,b]\to\mathbb{R}^3$ with
endpoints $\vec r(a)=N$ and $\vec r(b)=M$ is the real path of the
light, then:
\begin{equation}\label{MaxMedFullGGffgggyyojjhhjkhjyyiuhggjhhjhuyytytyuuytrrtghjtyuggyuighjuyioyyfgffhyuhhghzzrrkkhhkkkint}
\left(-S(M)\right)- \left(-S(N)\right)=\int_a^bn\left(\vec
r(s)\right)\left|\vec r'(s)\right|ds-\int_a^b
\frac{1}{c}n^2\left(\vec r(s)\right)\vec {\tilde u}\left(\vec
r(s)\right)\cdot\vec r'(s)ds.
\end{equation}
\end{proposition}
See also subsection \ref{GO} for the generalization of the Fermat
Principle to the case where we cannot take \er{ojhkkint} into
account.


In particular, by Proposition \ref{gughghfint} the path of travel of
the light satisfies the Euler-Lagrange equation for the functional
$J\left(\vec r(\cdot)\right)$,
%
%
%
\begin{comment}
\begin{multline}\label{MaxMedFullGGffgggyyojjhhjkhjyyiuhggjhhjhuyytytyuuytrrtghjtyuggyuighjuyioyyfgffhyuhhghzzrrkkhhkkkhhhjhkjhhghhgghiuiu2int}
\frac{d}{ds}\left(n\left(\vec r(s)\right)\frac{1}{\left|\vec
r'(s)\right|}\vec r'(s)-\frac{1}{c}n^2\left(\vec r(s)\right)\vec
{\tilde u}\left(\vec r(s)\right)\right)=\\
\left|\vec r'(s)\right|\nabla_{\vec x}n\left(\vec
r(s)\right)-\frac{2}{c}\left(\vec {\tilde u}\left(\vec
r(s)\right)\cdot\vec r'(s)\right)n\left(\vec r(s)\right)\nabla_{\vec
x}n\left(\vec r(s)\right)-\frac{1}{c}n^2\left(\vec
r(s)\right)\left\{d_{\vec x}\vec {\tilde u}\left(\vec
r(s)\right)\right\}^T\cdot\vec r'(s),
\end{multline}
that we rewrite as:
\begin{multline}\label{MaxMedFullGGffgggyyojjhhjkhjyyiuhggjhhjhuyytytyuuytrrtghjtyuggyuighjuyioyyfgffhyuhhghzzrrkkhhkkkhhhjhkjhhghhgghiuiuhhjhj1int}
\frac{1}{\left|\vec r'(s)\right|}\frac{d}{ds}\left(n\left(\vec
r(s)\right)\frac{1}{\left|\vec r'(s)\right|}\vec r'(s)\right)=\\
\nabla_{\vec x}n\left(\vec r(s)\right)+\frac{1}{c}n^2\left(\vec
r(s)\right)\left(d_{\vec x}\vec {\tilde u}\left(\vec
r(s)\right)-\left\{d_{\vec x}\vec {\tilde u}\left(\vec
r(s)\right)\right\}^T\right)\cdot\left(\frac{1}{\left|\vec
r'(s)\right|}\vec r'(s)\right)\\
+\frac{2}{c}n\left(\vec r(s)\right)\left\{\vec {\tilde u}\left(\vec
r(s)\right)\otimes\nabla_{\vec x}n\left(\vec
r(s)\right)-\nabla_{\vec x}n\left(\vec r(s)\right)\otimes\vec
{\tilde u}\left(\vec
r(s)\right)\right\}\cdot\left(\frac{1}{\left|\vec r'(s)\right|}\vec
r'(s)\right).
\end{multline}
Therefor by \er{apfrm9} and
\er{MaxMedFullGGffgggyyojjhhjkhjyyiuhggjhhjhuyytytyuuytrrtghjtyuggyuighjuyioyyfgffhyuhhghzzrrkkhhkkkhhhjhkjhhghhgghiuiuhhjhj1int}
we deduce
\end{comment}
%
%
%
that is the differential equation of the path of light:
\begin{multline}\label{MaxMedFullGGffgggyyojjhhjkhjyyiuhggjhhjhuyytytyuuytrrtghjtyuggyuighjuyioyyfgffhyuhhghzzrrkkhhkkkhhhjhkjhhghhgghiuiuhhjhjint}
\frac{d}{d\lambda}\left(n\left(\vec r\right)\frac{d\vec
r}{d\lambda}\right)=\frac{1}{c}n^2\left(\vec
r\right)\left(curl_{\vec x}\vec {\tilde u}\left(\vec
r\right)\right)\times\frac{d\vec r}{d\lambda}\\ +\nabla_{\vec
x}n\left(\vec r\right) +\frac{2}{c}n\left(\vec r\right)\left\{\vec
{\tilde u}\left(\vec r\right)\otimes\nabla_{\vec x}n\left(\vec
r\right)-\nabla_{\vec x}n\left(\vec r\right)\otimes\vec {\tilde
u}\left(\vec r\right)\right\}\cdot\frac{d\vec r}{d\lambda},
\end{multline}
where
\begin{equation}\label{MaxMedFullGGffgggyyojjhhjkhjyyiuhggjhhjhuyytytyuuytrrtghjtyuggyuighjuyioyyfgffhyuhhghzzrrkkhhkkkhhhjhkjhhghhgghiuiuint}
\lambda:=\int_a^s\left|\vec r'(\tau)\right|d\tau,
\end{equation}
is the natural parameter of the curve (see subsection \ref{GO} for
details).


Next, assume that the wave we consider has an electromagnetic
nature. Then by \er{gughhghfbvnbvint} and \er{uyuyuyyint} we have
\begin{equation}\label{gughhghfbvnbvyyuuyrint}
c_0=c\sqrt{\kappa_0\gamma_0}\quad\text{and}\quad\vec {\tilde
u}=\left(\gamma_0\vec v+(1-\gamma_0)\vec u\right),
\end{equation}
where, $\vec u$ is the medium velocity and $\vec v$ is the local
vectorial gravitational potential. Moreover, assume that we consider
light traveling in some region either filled with the resting medium
of constant dielectric permeability $\gamma_0$ and magnetic
permeability $\kappa_0$ or in the vacuum. Then by
\er{gughhghfbvnbvyyuuyrint} and
\er{MaxMedFullGGffgggyyojjhhjkhjyyiuhggjhhjhuyytytyuuytrrtghjtyuggyuighjuyioyyfgffhyuhhghzzrrkkhhkkkhhhjhkjhhghhgghiuiu1int}
we have:
\begin{equation}\label{gughhghfbvnbvyyuuyrhiyyuiuuint}
n=\frac{1}{\sqrt{\kappa_0\gamma_0}}\;\;\;\text{is a
constant,}\quad\text{and}\quad\vec {\tilde u}=\gamma_0\vec v,
\end{equation}
Then by \er{gughhghfbvnbvyyuuyrhiyyuiuuint} we rewrite
\er{MaxMedFullGGffgggyyojjhhjkhjyyiuhggjhhjhuyytytyuuytrrtghjtyuggyuighjuyioyyfgffhyuhhghzzrrkkhhkkkhhhjhkjhhghhgghiuiuhhjhjint}
as:
\begin{equation}\label{MaxMedFullGGffgggyyojjhhjkhjyyiuhggjhhjhuyytytyuuytrrtghjtyuggyuighjuyioyyfgffhyuhhghzzrrkkhhkkkhhhjhkjhhghhgghiuiuhhjhjiuiuyuint}
\frac{d^2\vec
r}{d\lambda^2}=\frac{1}{c}\sqrt{\frac{\gamma_0}{\kappa_0}}\left(curl_{\vec
x}\vec v\left(\vec r\right)\right)\times\frac{d\vec r}{d\lambda}.
\end{equation}
In particular, if our coordinate system is inertial, or more
generally non-rotating, then $curl_{\vec x}\vec v=0$ and we deduce
that the path of the light from the point $N$ to the point $M$ is
the direct line connecting these points, provided we take in the
account estimation \er{ojhkkint}.

On the other hand, if our system is rotating, then, since $\vec v$
is a speed-like vector field, we clearly deduce:
\begin{equation}\label{MaxMedFullGGffgggyyojjhhjkhjyyiuhggjhhjhuyytytyuuytrrtghjtyuggyuighjuyioyyfgffhyuhhghzzrrkkhhkkkhhhjhkjhhghhgghiuiuhhjhjiuiuyujjkint}
curl_{\vec x}\vec v=-2\vec w,
\end{equation}
where $\vec w$ is the vector of the angular speed of rotation of our
coordinate system. Thus by inserting
\er{MaxMedFullGGffgggyyojjhhjkhjyyiuhggjhhjhuyytytyuuytrrtghjtyuggyuighjuyioyyfgffhyuhhghzzrrkkhhkkkhhhjhkjhhghhgghiuiuhhjhjiuiuyujjkint}
into
\er{MaxMedFullGGffgggyyojjhhjkhjyyiuhggjhhjhuyytytyuuytrrtghjtyuggyuighjuyioyyfgffhyuhhghzzrrkkhhkkkhhhjhkjhhghhgghiuiuhhjhjiuiuyuint}
we deduce:
\begin{equation}\label{MaxMedFullGGffgggyyojjhhjkhjyyiuhggjhhjhuyytytyuuytrrtghjtyuggyuighjuyioyyfgffhyuhhghzzrrkkhhkkkhhhjhkjhhghhgghiuiuhhjhjiuiuyuyyuint}
\frac{d^2\vec
r}{d\lambda^2}=-\frac{2}{c}\sqrt{\frac{\gamma_0}{\kappa_0}}\vec
w\times\frac{d\vec r}{d\lambda}.
\end{equation}
%
%
%
\begin{comment}
In particular, by
\er{MaxMedFullGGffgggyyojjhhjkhjyyiuhggjhhjhuyytytyuuytrrtghjtyuggyuighjuyioyyfgffhyuhhghzzrrkkhhkkkhhhjhkjhhghhgghiuiuhhjhjiuiuyuyyuint}
if we consider that $\vec w=(0,0,w)$ and $\vec r=(x,y,z)$, then
there exist three dimensionless constants $C_1$, $C_2$ and $C_3$
such that
\begin{equation}\label{MaxMedFullGGffgggyyojjhhjkhjyyiuhggjhhjhuyytytyuuytrrtghjtyuggyuighjuyioyyfgffhyuhhghzzrrkkhhkkkhhhjhkjhhghhgghiuiuhhjhjiuiuyuyyukklint}
\begin{cases}
\frac{dx}{d\lambda}=-C_1\sin{\left(\frac{2w}{c}\sqrt{\frac{\gamma_0}{\kappa_0}}\lambda\right)}+C_2\cos{\left(\frac{2w}{c}\sqrt{\frac{\gamma_0}{\kappa_0}}\lambda\right)}
\\
\frac{dy}{d\lambda}=-C_1\cos{\left(\frac{2w}{c}\sqrt{\frac{\gamma_0}{\kappa_0}}\lambda\right)}-C_2\sin{\left(\frac{2w}{c}\sqrt{\frac{\gamma_0}{\kappa_0}}\lambda\right)}
\\
\frac{dz}{d\lambda}=C_3,
\end{cases}
\end{equation}
and moreover, since $\lambda$ is a natural parameter, the constants
satisfy:
\begin{equation}\label{MaxMedFullGGffgggyyojjhhjkhjyyiuhggjhhjhuyytytyuuytrrtghjtyuggyuighjuyioyyfgffhyuhhghzzrrkkhhkkkhhhjhkjhhghhgghiuiuhhjhjiuiuyuyyuojkint}
C^2_1+C^2_2+C^2_3=1.
\end{equation}
Then by
\er{MaxMedFullGGffgggyyojjhhjkhjyyiuhggjhhjhuyytytyuuytrrtghjtyuggyuighjuyioyyfgffhyuhhghzzrrkkhhkkkhhhjhkjhhghhgghiuiuhhjhjiuiuyuyyukklint}
there exist three additional constants $D_1$, $D_2$ and $D_3$ such
that
\end{comment}
%
%
%
The solution of
\er{MaxMedFullGGffgggyyojjhhjkhjyyiuhggjhhjhuyytytyuuytrrtghjtyuggyuighjuyioyyfgffhyuhhghzzrrkkhhkkkhhhjhkjhhghhgghiuiuhhjhjiuiuyuyyuint}
is the following:
\begin{equation}\label{MaxMedFullGGffgggyyojjhhjkhjyyiuhggjhhjhuyytytyuuytrrtghjtyuggyuighjuyioyyfgffhyuhhghzzrrkkhhkkkhhhjhkjhhghhgghiuiuhhjhjiuiuyuyyukklghhgint}
\begin{cases}
x(\lambda)=C_1\frac{c}{2w}\sqrt{\frac{\kappa_0}{\gamma_0}}\left(\cos{\left(\frac{2w}{c}\sqrt{\frac{\gamma_0}{\kappa_0}}\lambda\right)}-1\right)+C_2\frac{c}{2w}\sqrt{\frac{\kappa_0}{\gamma_0}}\sin{\left(\frac{2w}{c}\sqrt{\frac{\gamma_0}{\kappa_0}}\lambda\right)}+D_1
\\
y(\lambda)=-C_1\frac{c}{2w}\sqrt{\frac{\kappa_0}{\gamma_0}}\sin{\left(\frac{2w}{c}\sqrt{\frac{\gamma_0}{\kappa_0}}\lambda\right)}+C_2\frac{c}{2w}\sqrt{\frac{\kappa_0}{\gamma_0}}\left(\cos{\left(\frac{2w}{c}\sqrt{\frac{\gamma_0}{\kappa_0}}\lambda\right)}-1\right)+D_2
\\
z(\lambda)=C_3\lambda+D_3,
\end{cases}
\end{equation}
where, since $\lambda$ is a natural parameter of the curve, we have:
\begin{equation}\label{MaxMedFullGGffgggyyojjhhjkhjyyiuhggjhhjhuyytytyuuytrrtghjtyuggyuighjuyioyyfgffhyuhhghzzrrkkhhkkkhhhjhkjhhghhgghiuiuhhjhjiuiuyuyyuojkint}
C^2_1+C^2_2+C^2_3=1.
\end{equation}
So, the curve in
\er{MaxMedFullGGffgggyyojjhhjkhjyyiuhggjhhjhuyytytyuuytrrtghjtyuggyuighjuyioyyfgffhyuhhghzzrrkkhhkkkhhhjhkjhhghhgghiuiuhhjhjiuiuyuyyukklghhgint}
is the trajectory of the light in the rotating coordinate system,
provided we assume \er{ojhkkint}. In particular, by
\er{MaxMedFullGGffgggyyojjhhjkhjyyiuhggjhhjhuyytytyuuytrrtghjtyuggyuighjuyioyyfgffhyuhhghzzrrkkhhkkkhhhjhkjhhghhgghiuiuhhjhjiuiuyuyyukklghhgint}
%and \er{MaxMedFullGGffgggyyojjhhjkhjyyiuhggjhhjhuyytytyuuytrrtghjtyuggyuighjuyioyyfgffhyuhhghzzrrkkhhkkkhhhjhkjhhghhgghiuiuhhjhjiuiuyuyyukklint}
we have:
\begin{equation}\label{MaxMedFullGGffgggyyojjhhjkhjyyiuhggjhhjhuyytytyuuytrrtghjtyuggyuighjuyioyyfgffhyuhhghzzrrkkhhkkkhhhjhkjhhghhgghiuiuhhjhjiuiuyuyyuojkjkkkhguggint}
\begin{cases}
x(0)=D_1,\quad y(0)=D_2,\quad z(0)=D_3,\\
%\quad\text{and}\quad
\frac{dx}{d\lambda}(0)=C_2,\quad\frac{dy}{d\lambda}(0)=-C_1,\quad\frac{dz}{d\lambda}(0)=C_3.
\end{cases}
\end{equation}
%
%
%
\begin{comment} Moreover, by
\er{MaxMedFullGGffgggyyojjhhjkhjyyiuhggjhhjhuyytytyuuytrrtghjtyuggyuighjuyioyyfgffhyuhhghzzrrkkhhkkkhhhjhkjhhghhgghiuiuhhjhjiuiuyuyyukkl}
the radius of curvature of the curve satisfies:
\begin{equation}\label{MaxMedFullGGffgggyyojjhhjkhjyyiuhggjhhjhuyytytyuuytrrtghjtyuggyuighjuyioyyfgffhyuhhghzzrrkkhhkkkhhhjhkjhhghhgghiuiuhhjhjiuiuyuyyukklklihi}
R:=\frac{1}{\left|\frac{d^2\vec
r}{d\lambda^2}\right|}=\frac{c}{2w}\sqrt{\frac{\kappa_0}{\gamma_0}}\frac{1}{C^2_1+C^2_2}\,.
\end{equation}
\end{comment}
%
%
%
The constants $C_1,C_2,C_3,D_1,D_2,D_3$ can be determined either by
the initial data
\er{MaxMedFullGGffgggyyojjhhjkhjyyiuhggjhhjhuyytytyuuytrrtghjtyuggyuighjuyioyyfgffhyuhhghzzrrkkhhkkkhhhjhkjhhghhgghiuiuhhjhjiuiuyuyyuojkjkkkhguggint}
or by the beginning and the ending points $N$ and $M$ of the curve.
%
%
%
%
\begin{comment}
Consider the coordinates of the endpoints of the curve as
$N=(N_1,N_2,N_3)$ and $M=(M_1,M_2,M_3)$. Then for $\lambda=0$ we
must have $(x,y,z)=N$ i.e.
\begin{equation}\label{MaxMedFullGGffgggyyojjhhjkhjyyiuhggjhhjhuyytytyuuytrrtghjtyuggyuighjuyioyyfgffhyuhhghzzrrkkhhkkkhhhjhkjhhghhgghiuiuhhjhjiuiuyuyyuojkjkkkh}
\begin{cases}
D_1=N_1\\
D_2=N_2,\\
D_3=N_3.
\end{cases}
\end{equation}
Moreover we must have
\er{MaxMedFullGGffgggyyojjhhjkhjyyiuhggjhhjhuyytytyuuytrrtghjtyuggyuighjuyioyyfgffhyuhhghzzrrkkhhkkkhhhjhkjhhghhgghiuiuhhjhjiuiuyuyyuojk}.
Finally there must exist $\lambda_0>0$, which turns to be the length
of the curve, such that
\begin{equation}\label{MaxMedFullGGffgggyyojjhhjkhjyyiuhggjhhjhuyytytyuuytrrtghjtyuggyuighjuyioyyfgffhyuhhghzzrrkkhhkkkhhhjhkjhhghhgghiuiuhhjhjiuiuyuyyukklghhghyhjhjh}
\begin{cases}
M_1=C_1\frac{c}{2w}\sqrt{\frac{\kappa_0}{\gamma_0}}\left(\cos{\left(\frac{2w}{c}\sqrt{\frac{\gamma_0}{\kappa_0}}\lambda_0\right)}-1\right)+C_2\frac{c}{2w}\sqrt{\frac{\kappa_0}{\gamma_0}}\sin{\left(\frac{2w}{c}\sqrt{\frac{\gamma_0}{\kappa_0}}\lambda_0\right)}+D_1
\\
M_2=-C_1\frac{c}{2w}\sqrt{\frac{\kappa_0}{\gamma_0}}\sin{\left(\frac{2w}{c}\sqrt{\frac{\gamma_0}{\kappa_0}}\lambda_0\right)}+C_2\frac{c}{2w}\sqrt{\frac{\kappa_0}{\gamma_0}}\left(\cos{\left(\frac{2w}{c}\sqrt{\frac{\gamma_0}{\kappa_0}}\lambda_0\right)}-1\right)+D_2
\\
M_3=C_3\lambda_0+D_3.
\end{cases}
\end{equation}
\end{comment}
%
%
%
%
%
%
%
%
%
\begin{comment}
Moreover, by
\er{MaxMedFullGGffgggyyojjhhjkhjyyiuhggjhhjhuyytytyuuytrrtghjtyuggyuighjuyioyyfgffhyuhhghzzrrkkhhkkkhhhjhkjhhghhgghiuiuhhjhjiuiuyuyyukkl}
the radius of curvature of the curve satisfies:
\begin{equation}\label{MaxMedFullGGffgggyyojjhhjkhjyyiuhggjhhjhuyytytyuuytrrtghjtyuggyuighjuyioyyfgffhyuhhghzzrrkkhhkkkhhhjhkjhhghhgghiuiuhhjhjiuiuyuyyukklklihi}
R:=\frac{1}{\left|\frac{d^2\vec
r}{d\lambda^2}\right|}=\frac{c}{2w}\sqrt{\frac{\kappa_0}{\gamma_0}}\frac{1}{C^2_1+C^2_2}\,.
\end{equation}




Next consider the direct line, parameterized by the natural
parameter, passing through the same initial point and having the
same initial tangent vector as the curve in
\er{MaxMedFullGGffgggyyojjhhjkhjyyiuhggjhhjhuyytytyuuytrrtghjtyuggyuighjuyioyyfgffhyuhhghzzrrkkhhkkkhhhjhkjhhghhgghiuiuhhjhjiuiuyuyyukklghhg}
and
\er{MaxMedFullGGffgggyyojjhhjkhjyyiuhggjhhjhuyytytyuuytrrtghjtyuggyuighjuyioyyfgffhyuhhghzzrrkkhhkkkhhhjhkjhhghhgghiuiuhhjhjiuiuyuyyuojkjkkkhgugg}.
Then we can write this line as:
\begin{equation}\label{MaxMedFullGGffgggyyojjhhjkhjyyiuhggjhhjhuyytytyuuytrrtghjtyuggyuighjuyioyyfgffhyuhhghzzrrkkhhkkkhhhjhkjhhghhgghiuiuhhjhjiuiuyuyyukklghhgjkjk}
\begin{cases}
x_0(\lambda)=C_2\lambda+D_1
\\
y_0(\lambda)=-C_1\lambda+D_2
\\
z_0(\lambda)=C_3\lambda+D_3.
\end{cases}
\end{equation}
In particular by comparing
\er{MaxMedFullGGffgggyyojjhhjkhjyyiuhggjhhjhuyytytyuuytrrtghjtyuggyuighjuyioyyfgffhyuhhghzzrrkkhhkkkhhhjhkjhhghhgghiuiuhhjhjiuiuyuyyukklghhg}
and
\er{MaxMedFullGGffgggyyojjhhjkhjyyiuhggjhhjhuyytytyuuytrrtghjtyuggyuighjuyioyyfgffhyuhhghzzrrkkhhkkkhhhjhkjhhghhgghiuiuhhjhjiuiuyuyyukklghhgjkjk}
and using Teylor formula from the basic calculus, together with
\er{MaxMedFullGGffgggyyojjhhjkhjyyiuhggjhhjhuyytytyuuytrrtghjtyuggyuighjuyioyyfgffhyuhhghzzrrkkhhkkkhhhjhkjhhghhgghiuiuhhjhjiuiuyuyyuojk},
we deduce:
\begin{equation}\label{MaxMedFullGGffgggyyojjhhjkhjyyiuhggjhhjhuyytytyuuytrrtghjtyuggyuighjuyioyyfgffhyuhhghzzrrkkhhkkkhhhjhkjhhghhgghiuiuhhjhjiuiuyuyyukklghhgjkjkiuuui}
\begin{cases}
\left|x(\lambda)-x_0(\lambda)\right|\leq
%(C_1+C_2)
\frac{2w}{c}\sqrt{\frac{\gamma_0}{\kappa_0}}
\left(1+\frac{2w}{c}\sqrt{\frac{\gamma_0}{\kappa_0}}\left|\lambda\right|\right)\lambda^2
\\
\left|y(\lambda)-y_0(\lambda)\right|\leq
%(C_1+C_2)
\frac{2w}{c}\sqrt{\frac{\gamma_0}{\kappa_0}}
\left(1+\frac{2w}{c}\sqrt{\frac{\gamma_0}{\kappa_0}}\left|\lambda\right|\right)\lambda^2
\\
z(\lambda)=z_0(\lambda).
\end{cases}
\end{equation}
In particular we obtain the following estimate of deviation of the
path of light for the direct line with the same initial point and
initial tangent vector in a rotating coordinate system:
\begin{equation}\label{MaxMedFullGGffgggyyojjhhjkhjyyiuhggjhhjhuyytytyuuytrrtghjtyuggyuighjuyioyyfgffhyuhhghzzrrkkhhkkkhhhjhkjhhghhgghiuiuhhjhjiuiuyuyyukklghhgjkjkiuuuioiojo}
\left(\left|x(\lambda)-x_0(\lambda)\right|
+\left|y(\lambda)-y_0(\lambda)\right|+\left|z(\lambda)-z_0(\lambda)\right|\right)\,=\,|\lambda|\;O\left(\frac{2w\lambda}{c}\right),
\end{equation}
where $\lambda$ is a natural parameter of the curve.
\end{comment}
%
%
%
%
%








\subsubsection{The laws of reflection and refraction}
%\label{rrGO}
Next consider a monochromatic wave of the frequency
$\nu=\omega/(2\pi)$ characterized by:
\begin{equation}\label{MaxVacFullPPNmmmffffffiuiuhjuughbghhuiiujjhhjjhjhhjint}
U(\vec x,t)=A(\vec x,t)e^{ik_0S(\vec x,t)},\quad\text{where}\quad
k_0=\frac{\omega}{c}\quad\text{and}\quad\frac{\partial S}{\partial
t}=c\,,
\end{equation}
and, consistently with
\er{MaxVacFullPPNmmmffffffiuiuhjuughbghhiuijghghghhhfhhghghguygtjuuujjjkyuuyint}
consider a direction field:
\begin{equation}\label{MaxVacFullPPNmmmffffffiuiuhjuughbghhiuijghghghhhfhhghghguygtjuuujjjkyuuykjkjjint}
%\frac{1}{c^2_0}\frac{\partial^2}{\partial t^2}\left(Ae^{ik_0S}\right)
\vec h(\vec x)=\frac{c}{c^2_0(\vec x)}\vec {\tilde u}(\vec
x)-\nabla_{\vec x}S(\vec x).
\end{equation}
Furthermore, assume that this wave undergoes reflection and/or
refraction on the surface $\mathcal{T}$ with the outcoming unit
normal $\vec n$, separating two regions characterized respectively
by $c_0=c^{(1)}_0$ and $\vec {\tilde u}=\vec {\tilde u}_1$ and by
$c^{(2)}_0$ and $\vec {\tilde u}_2$, with the formation of the
reflected wave (of the same frequency), characterized by:
\begin{equation}\label{MaxVacFullPPNmmmffffffiuiuhjuughbghhuiiujjhhjjhjhhjughint}
U_1(\vec x,t)=A_1(\vec x,t)e^{ik_0S_1(\vec
x,t)},\quad\text{where}\quad\frac{\partial S_1}{\partial t}=c\,,
\end{equation}
and by a direction field:
\begin{equation}\label{MaxVacFullPPNmmmffffffiuiuhjuughbghhiuijghghghhhfhhghghguygtjuuujjjkyuuykjkjjhjhhjint}
%\frac{1}{c^2_0}\frac{\partial^2}{\partial t^2}\left(Ae^{ik_0S}\right)
\vec h_1(\vec x)=\frac{c}{c^2_0(\vec x)}\vec {\tilde u}(\vec
x)-\nabla_{\vec x}S_1(\vec x),
\end{equation}
and formation of the refracted wave (of the same frequency),
characterized by:
\begin{equation}\label{MaxVacFullPPNmmmffffffiuiuhjuughbghhuiiujjhhjjhjhhj1int}
U_2(\vec x,t)=A_2(\vec x,t)e^{ik_0S_2(\vec
x,t)},\quad\text{where}\quad\frac{\partial S_2}{\partial t}=c\,.
\end{equation}
and by a direction field:
\begin{equation}\label{MaxVacFullPPNmmmffffffiuiuhjuughbghhiuijghghghhhfhhghghguygtjuuujjjkyuuykjkjj1int}
%\frac{1}{c^2_0}\frac{\partial^2}{\partial t^2}\left(Ae^{ik_0S}\right)
\vec h_2(\vec x)=\frac{c}{\left(c^{(2)}_0(\vec x)\right)^2}\vec
{\tilde u_2}(\vec x)-\nabla_{\vec x}S_2(\vec x).
\end{equation}
Then the boundary conditions of $U$, $U_1$ and $U_2$ depend on the
physical meaning of these fields. However, one of the
\underline{necessary} conditions should be that
\begin{equation}\label{MaxMedFullGGffgggyyojjyugggjhhjiiuuiyuyuyyuyuzzhhkkkint}
%\omega^{(1)}=\omega\quad\text{and}\quad
S_1(\vec x,t)=S_2(\vec x,t)+C_2=S(\vec x,t)\quad\quad\forall\vec
x\in\mathcal{T},
%\quad\text{and thus}\quad
\end{equation}
where $C_2$ is a real constant. In particular
\er{MaxMedFullGGffgggyyojjyugggjhhjiiuuiyuyuyyuyuzzhhkkkint}
implies:
%
%
%
%\begin{comment*}
\begin{equation}\label{MaxMedFullGGffgggyyojjyugggjhhjiiuuiyuyuyyuyuzzhhkkkgdfgint}
\nabla_{\vec x} S_1-\left(\vec n\cdot \nabla_{\vec x} S_1\right)\vec
n
%=\nabla_{\vec x} S-\left(\vec n\cdot \nabla_{\vec x} S\right)\vec n
=\nabla_{\vec x} S_2-\left(\vec n\cdot \nabla_{\vec x}
S_2\right)\vec n=\nabla_{\vec x} S-\left(\vec n\cdot \nabla_{\vec x}
S\right)\vec n\quad\quad\forall\vec x\in\mathcal{T}.
\end{equation}
In particular, for every point on the surface $\mathcal{T}$ vectors
$\nabla_{\vec x} S_1$ and $\nabla_{\vec x} S_2$ lie in the plane
formed by vectors $\vec n$ and $\nabla_{\vec x} S$. Moreover, by
\er{MaxVacFullPPNmmmffffffiuiuhjuughbghhiuijghghghhhfhhghghguygtjuuujjjkyuuykjkjjint},
\er{MaxVacFullPPNmmmffffffiuiuhjuughbghhiuijghghghhhfhhghghguygtjuuujjjkyuuykjkjjhjhhjint}
and \er{MaxMedFullGGffgggyyojjyugggjhhjiiuuiyuyuyyuyuzzhhkkkgdfgint}
we have
%\end{comment*}
%
%
%
\begin{equation}\label{MaxMedFullGGffgggyyojjyugggjhhjiiuuiyuyuyyuyuzzhhkkkgdfghgghghint}
\vec h_1-\left(\vec n\cdot \vec h_1\right)\vec n=\vec h-\left(\vec
n\cdot \vec h\right)\vec n\quad\quad\forall\vec x\in\mathcal{T},
\end{equation}
and in particular, for every point on the surface $\mathcal{T}$
vector $\vec h_1$ lies in the plane formed by vectors $\vec n$ and
$\vec h$.
%\begin{equation}\label{MaxMedFullGGffgggyyojjyugggjhhjiiuuiyuyuyyuyuzzhhkkkgdfg}
%\left(\nabla_{\vec x} S^{(1)}-\nabla_{\vec x} S\right)\times\vec n=\left(\nabla_{\vec x} S^{(2)}-\nabla_{\vec x} S\right)\times\vec n=0\quad\forall\vec x\in\mathcal{T}.
%\end{equation}
Next, assume that the approximate equations in
\er{MaxVacFullPPNmmmffffffiuiuhjuughbghhiuijghghghhjhjhhghyuyiyyujjljkint}
and
\er{MaxVacFullPPNmmmffffffiuiuhjuughbghhiuijghghghhhfhhghghguygtjuuujjjkint}
are valid in every of two regions on the both sides of
$\mathcal{T}$. Then by
\er{MaxVacFullPPNmmmffffffiuiuhjuughbghhiuijghghghhjhjhhghyuyiyyujjljkgghhgint}
we have
\begin{equation}\label{MaxVacFullPPNmmmffffffiuiuhjuughbghhiuijghghghhjhjhhghyuyiyyujjljkgghhguyuyint}
|\vec h_1|=|\vec h|=\frac{c}{c_0}.
\end{equation}
Then, since $\vec h_1\neq\vec h$, by
\er{MaxMedFullGGffgggyyojjyugggjhhjiiuuiyuyuyyuyuzzhhkkkgdfghgghghint}
and
\er{MaxVacFullPPNmmmffffffiuiuhjuughbghhiuijghghghhjhjhhghyuyiyyujjljkgghhguyuyint}
we deduce
\begin{equation}\label{MaxMedFullGGffgggyyojjyugggjhhjiiuuiyuyuyyuyuzzhhkkkgdfghgghghijhjhhkhint}
\vec n\cdot \vec h_1=-\vec n\cdot \vec h\quad\quad\forall\vec
x\in\mathcal{T}.
\end{equation}
So, by
\er{MaxVacFullPPNmmmffffffiuiuhjuughbghhiuijghghghhjhjhhghyuyiyyujjljkgghhguyuyint}
and
\er{MaxMedFullGGffgggyyojjyugggjhhjiiuuiyuyuyyuyuzzhhkkkgdfghgghghijhjhhkhint}
we obtain the law of reflection: vector $\vec h_1$ lies in the plane
formed by vectors $\vec n$ and $\vec h$, and we have:
\begin{equation}\label{MaxMedFulljhhjjjjjint}
\theta\left(\vec h,-\vec n\right)=\theta_1\left(\vec h_1,\vec
n\right)
\end{equation}
where $\theta\left(\vec h,-\vec n\right)$ is the angle between the
incoming beam direction $\vec h$ and the incoming normal to the
surface $-\vec n$ and $\theta_1\left(\vec h_1,\vec n\right)$ is the
angle between the reflected beam direction $\vec h_1$ and the
outcoming normal $\vec n$.

Next assume that the wave we consider in
\er{MaxVacFullPPNmmmffffffiuiuhjuughbghhuiiujjhhjjhjhhjint} has an
electromagnetic nature. Then by \er{gughhghfbvnbvyyuuyrint} we have
\begin{equation}\label{gughhghfbvnbvyyuuyint}
c_0=c\sqrt{\kappa_0\gamma_0}\quad\text{and}\quad\vec {\tilde
u}=\left(\gamma_0\vec v+(1-\gamma_0)\vec u\right),
\end{equation}
where, $\vec u$ is the medium velocity and $\vec v$ is the local
vectorial gravitational potential. Similarly, on the second side of
surface $\mathcal{T}$ we have
\begin{equation}\label{gughhghfbvnbvyyuuyGGHHGint}
c^{(2)}_0=c\sqrt{\kappa^{(2)}_0\gamma^{(2)}_0}\quad\text{and}\quad\vec
{\tilde u}^{(2)}=\left(\gamma^{(2)}_0\vec v+(1-\gamma^{(2)}_0)\vec
u^{(2)}\right),
\end{equation}
where, $\vec u^{(2)}$ is the medium velocity on the second side of
surface $\mathcal{T}$. Furthermore, assume that the medium rests on
the both sides of surface $\mathcal{T}$ and the magnetic
permeability is the same on both sides of surface $\mathcal{T}$.
I.e. we have
\begin{equation}\label{gughhghfbvnbvyyuuyllint}
\kappa^{(2)}_0=\kappa_0\quad\text{and}\quad\vec u^{(2)}=\vec u=0,
\end{equation}
however $\gamma^{(2)}_0$ can differ from $\gamma_0$. Then in this
particular case we rewrite \er{gughhghfbvnbvyyuuyint} and
\er{gughhghfbvnbvyyuuyGGHHGint} as
\begin{equation}\label{gughhghfbvnbvyyuuygggint}
c_0=c\sqrt{\kappa_0\gamma_0}\quad\text{and}\quad\vec {\tilde
u}=\gamma_0\vec v,
\end{equation}
and
\begin{equation}\label{gughhghfbvnbvyyuuyGGHHGhjhjhint}
c^{(2)}_0=c\sqrt{\kappa_0\gamma^{(2)}_0}\quad\text{and}\quad\vec
{\tilde u}^{(2)}=\gamma^{(2)}_0\vec v,
\end{equation}
Then in particular, by \er{gughhghfbvnbvyyuuygggint} and
\er{gughhghfbvnbvyyuuyGGHHGhjhjhint} we deduce
\begin{equation}\label{gughhghfbvnbvyyuuyGGHHGhjhjhhhjhjint}
\frac{c}{\left(c^{(2)}_0\right)^2}\vec {\tilde
u}^{(2)}=\frac{c}{c^2_0}\vec {\tilde u}=\frac{1}{\kappa_0 c}\vec v.
\end{equation}
Thus, by inserting
\er{MaxVacFullPPNmmmffffffiuiuhjuughbghhiuijghghghhhfhhghghguygtjuuujjjkyuuykjkjjint}
and \er{gughhghfbvnbvyyuuyGGHHGhjhjhhhjhjint} into
\er{MaxMedFullGGffgggyyojjyugggjhhjiiuuiyuyuyyuyuzzhhkkkgdfgint}, we
deduce:
\begin{equation}\label{MaxMedFullGGffgggyyojjyugggjhhjiiuuiyuyuyyuyuzzhhkkkgdfg1int}
\vec h_2-\left(\vec n\cdot \vec h_2\right)\vec n=\vec h-\left(\vec
n\cdot \vec h\right)\vec n\quad\quad\forall\vec x\in\mathcal{T},
\end{equation}
and in particular, for every point on the surface $\mathcal{T}$
vector $\vec h_2$ lies in the plane formed by vectors $\vec n$ and
$\vec h$. On the other hand by
\er{MaxVacFullPPNmmmffffffiuiuhjuughbghhiuijghghghhjhjhhghyuyiyyujjljkgghhgint}
we have:
\begin{equation}\label{MaxVacFullPPNmmmffffffiuiuhjuughbghhiuijghghghhjhjhhghyuyiyyujjljkgghhglkjlkjkjljljhhjjint}
|\vec h|=\frac{c}{c_0}\quad\text{and}\quad|\vec
h_2|=\frac{c}{c^{(2)}_0}.
\end{equation}
So, by
\er{MaxMedFullGGffgggyyojjyugggjhhjiiuuiyuyuyyuyuzzhhkkkgdfg1int}
and
\er{MaxVacFullPPNmmmffffffiuiuhjuughbghhiuijghghghhjhjhhghyuyiyyujjljkgghhglkjlkjkjljljhhjjint},
in the cases when \er{gughhghfbvnbvyyuuyllint} holds, we have the
Snell's law of refraction: vector $\vec h_2$ lies in the plane
formed by vectors $\vec n$ and $\vec h$, and we have:
\begin{equation}\label{MaxMedFulljhhjjjjjhhhjint}
n\sin{\left(\theta\left(\vec h,\vec
n\right)\right)}=n_2\sin{\left(\theta_2\left(\vec h_2,\vec
n\right)\right)}
\end{equation}
where $\theta\left(\vec h,\vec n\right)$ is the angle between the
incoming beam direction $\vec h$ and the normal to the surface $\vec
n$, $\theta_2\left(\vec h_2,\vec n\right)$ is the angle between the
refracted beam direction $\vec h_2$ and the normal $\vec n$ and as
in
\er{MaxMedFullGGffgggyyojjhhjkhjyyiuhggjhhjhuyytytyuuytrrtghjtyuggyuighjuyioyyfgffhyuhhghzzrrkkhhkkkhhhjhkjhhghhgghiuiu1int}
we set refraction indexes:
\begin{equation}\label{MaxMedFullGGffgggyyojjhhjkhjyyiuhggjhhjhuyytytyuuytrrtghjtyuggyuighjuyioyyfgffhyuhhghzzrrkkhhkkkhhhjhkjhhghhgghiuiujjkjkint}
n:=\frac{c}{c_0}\quad\text{and}\quad n_2:=\frac{c}{c^{(2)}_0}.
\end{equation}
Note, that in the case when \er{gughhghfbvnbvyyuuyllint} dose not
hold, however we have $\vec {\tilde u}^{(2)}=\vec {\tilde u}=0$
instead, the Snell's law still holds. However, in the frames of our
model, in contrast to the law of reflection, the Snell's law dose
not hold exactly in the case where the magnetic permeability
$\kappa_0$ on the one side of surface $\mathcal{T}$ differ from
$\kappa^{(2)}_0$ on the another side of the surface and at the same
time the field $\vec v\neq 0$ is nontrivial.
%\begin{equation}\label{MaxVacFullPPNmmmffffffiuiuhjuughbghhiuijghghghhhfhhghghguygtjuuujjjk}
%\left(-\nabla_{\vec x}S+\frac{c\vec {\tilde u}}{c^2_0}\right)\cdot\nabla_{\vec x}A+\frac{1}{2}\left(-\Delta_{\vec x}S\right)A=0.
%\end{equation}
%
%
%
%
\begin{comment}
\begin{multline}\label{MaxVacFullPPNmmmffffffiuiuhjuughbghhiuijghghghhjhjhhghyuyiyyujjkhgggghggiuiuiu}
\frac{c^2}{c^2_0}\left(1+\frac{1}{c^2}\left|\nabla_\vec x
S\right|^2\left|\vec {\tilde u}\right|^2-\frac{1}{c^2}\left|\vec
{\tilde u}\cdot\nabla_\vec x S\right|^2\right)=\left|\nabla_{\vec
x}S-\frac{c}{c_0}\left(1+\frac{1}{c}\left(\vec {\tilde
u}\cdot\nabla_{\vec x}S\right)\right)\frac{\vec {\tilde
u}}{c_0}\right|^2=\\ \left|\nabla_{\vec
x}S\right|^2+\frac{c^2}{c^4_0}\left(1+\frac{1}{c}\left(\vec {\tilde
u}\cdot\nabla_{\vec x}S\right)\right)^2\left|\vec {\tilde
u}\right|^2-\frac{2c}{c^2_0}\left(1+\frac{1}{c}\left(\vec {\tilde
u}\cdot\nabla_{\vec x}S\right)\right)\left(\nabla_{\vec x}S\cdot\vec
{\tilde u}\right)=\\ \left|\nabla_{\vec
x}S\right|^2+\frac{c^2}{c^4_0}\left(1+\frac{2}{c}\left(\vec {\tilde
u}\cdot\nabla_{\vec x}S\right)+\frac{1}{c^2}\left(\vec {\tilde
u}\cdot\nabla_{\vec x}S\right)^2\right)\left|\vec {\tilde
u}\right|^2-\frac{2c}{c^2_0}\left(1+\frac{1}{c}\left(\vec {\tilde
u}\cdot\nabla_{\vec x}S\right)\right)\left(\nabla_{\vec x}S\cdot\vec
{\tilde u}\right).
\end{multline}
\begin{multline}\label{MaxVacFullPPNmmmffffffiuiuhjuughbghhiuijghghghhjhjhhghyuyiyyujjkhgggghggiuiuiupkkkll}
0=\left(\left|\nabla_{\vec
x}S\right|^2-\frac{c^2}{c^2_0}\right)\left(1-\frac{\left|\vec
{\tilde
u}\right|^2}{c^2_0}\right)+\frac{c^2}{c^4_0}\left(\frac{2}{c}\left(\vec
{\tilde u}\cdot\nabla_{\vec x}S\right)+\frac{1}{c^2}\left(\vec
{\tilde u}\cdot\nabla_{\vec x}S\right)^2\right)\left|\vec {\tilde
u}\right|^2-\frac{c}{c^2_0}\left(2+\frac{1}{c}\left(\vec {\tilde
u}\cdot\nabla_{\vec x}S\right)\right)\left(\nabla_{\vec x}S\cdot\vec
{\tilde u}\right).
\end{multline}
\begin{multline}\label{MaxVacFullPPNmmmffffffiuiuhjuughbghhiuijghghghhjhjhhghyuyiyyujjkhgggghggiuiuiupkkkllokljk;ljk}
0=\left(\left|\nabla_{\vec
x}S\right|^2-\frac{c^2}{c^2_0}\right)\left(1-\frac{\left|\vec
{\tilde
u}\right|^2}{c^2_0}\right)-\frac{c^2}{c^2_0}\left(\frac{2}{c}\left(\nabla_{\vec
x}S\cdot\vec {\tilde u}\right)+\frac{1}{c^2}\left|\vec {\tilde
u}\cdot\nabla_{\vec x}S\right|^2\right)\left(1-\frac{\left|\vec
{\tilde u}\right|^2}{c^2_0}\right).
\end{multline}
\end{comment}
%
%
%
%











\subsubsection{Sagnac effect}
%\label{seGO}
Assume again the monochromatic electromagnetic wave of the frequency
$\nu=\omega/(2\pi)$ characterized by:
\begin{equation}\label{MaxVacFullPPNmmmffffffiuiuhjuughbghhuiiujjhhjjhjhhjhjjhhjhjjhint}
U(\vec x,t)=A(\vec x,t)e^{iT(\vec x,t)}=A(\vec x,t)e^{ik_0S(\vec
x,t)},\quad\text{where}\quad
k_0=\frac{\omega}{c}\quad\text{and}\quad\frac{\partial S}{\partial
t}=c\,.
\end{equation}
Then by \er{gughhghfbvnbvyyuuyrint} we have
\begin{equation}\label{gughhghfbvnbvyyuuyrhjhjhjint}
c_0=c\sqrt{\kappa_0\gamma_0}\quad\text{and}\quad\vec {\tilde
u}=\left(\gamma_0\vec v+(1-\gamma_0)\vec u\right),
\end{equation}
where, $\vec u$ is the medium velocity and $\vec v$ is the local
vectorial gravitational potential. Moreover, assume again that we
consider light traveling in some region either filled with the
resting medium of constant dielectric permeability $\gamma_0$ and
magnetic permeability $\kappa_0$ or in the vacuum. Then by
\er{gughhghfbvnbvyyuuyrhjhjhjint} and
\er{MaxMedFullGGffgggyyojjhhjkhjyyiuhggjhhjhuyytytyuuytrrtghjtyuggyuighjuyioyyfgffhyuhhghzzrrkkhhkkkhhhjhkjhhghhgghiuiu1int}
we have
\begin{equation}\label{gughhghfbvnbvyyuuyrhiyyuiuuhjhint}
n=\frac{1}{\sqrt{\kappa_0\gamma_0}}\;\;\;\text{is a
constant,}\quad\text{and}\quad\vec {\tilde u}=\gamma_0\vec v.
\end{equation}
Next, assume that the light travels from point $N$ to point $M$
across the curve $\vec r(s):[a,b]\to\mathbb{R}^3$ with endpoints
$\vec r(a)=N$ and $\vec r(b)=M$ undergoing possibly certain number
of reflections from mirrors during its travel. Then by
\er{MaxMedFullGGffgggyyojjhhjkhjyyiuhggjhhjhuyytytyuuytrrtghjtyuggyuighjuyioyyfgffhyuhhghzzrrkkhhkkkint},
\er{gughhghfbvnbvyyuuyrhiyyuiuuhjhint} and
\er{MaxMedFullGGffgggyyojjyugggjhhjiiuuiyuyuyyuyuzzhhkkkint} we
have:
\begin{equation}\label{MaxMedFullGGffgggyyojjhhjkhjyyiuhggjhhjhuyytytyuuytrrtghjtyuggyuighjuyioyyfgffhyuhhghzzrrkkhhkkkhjjhint}
\delta(-S):=\left(-S(M^-)\right)-
\left(-S(N^+)\right)=\frac{1}{\sqrt{\kappa_0\gamma_0}}\int_a^b
\left|\vec r'(s)\right|ds-\frac{1}{\kappa_0 c}\int_a^b\vec
v\left(\vec r(s)\right)\cdot\vec r'(s)ds.
\end{equation}
In particular, if we assume that $M=N$ i.e. our curve is closed and
moreover, our curve is the boundary of some surface $\mathcal{S}_0$,
then by Stokes Theorem we have:
\begin{multline}\label{MaxMedFullGGffgggyyojjhhjkhjyyiuhggjhhjhuyytytyuuytrrtghjtyuggyuighjuyioyyfgffhyuhhghzzrrkkhhkkkhjjhjjint}
\delta(-S)=\left(-S(M^-)\right)-
\left(-S(M^+)\right)=\frac{1}{\sqrt{\kappa_0\gamma_0}}\int_a^b
\left|\vec r'(s)\right|ds-\frac{1}{\kappa_0 c}\iint \left(curl_{\vec
x}\vec v\right)\cdot\vec n
\,d\mathcal{S}_0\\=\frac{1}{\sqrt{\kappa_0\gamma_0}}\left|\partial
\mathcal{S}_0\right|-\frac{1}{\kappa_0 c}\iint \left(curl_{\vec
x}\vec v\right)\cdot\vec n \,d\mathcal{S}_0,
\end{multline}
where $\vec n$ is the unit normal to the surface.
%Note that although we assume $M=N$, $\delta(-S)$ can differ from zero since the light can undergo reflections during its travel.
In particular, if our coordinate system is inertial, or more
generally non-rotating, then $curl_{\vec x}\vec v=0$ and by
\er{MaxMedFullGGffgggyyojjhhjkhjyyiuhggjhhjhuyytytyuuytrrtghjtyuggyuighjuyioyyfgffhyuhhghzzrrkkhhkkkhjjhjjint}
we deduce
\begin{equation}\label{MaxMedFullGGffgggyyojjhhjkhjyyiuhggjhhjhuyytytyuuytrrtghjtyuggyuighjuyioyyfgffhyuhhghzzrrkkhhkkkhjjhjjhjhhjjhjjhint}
\delta(-S)=\frac{1}{\sqrt{\kappa_0\gamma_0}}\left|\partial
\mathcal{S}_0\right|.
\end{equation}
On the other hand, if our system is rotating, then as in
\er{MaxMedFullGGffgggyyojjhhjkhjyyiuhggjhhjhuyytytyuuytrrtghjtyuggyuighjuyioyyfgffhyuhhghzzrrkkhhkkkhhhjhkjhhghhgghiuiuhhjhjiuiuyujjkint}
we clearly deduce:
\begin{equation}\label{MaxMedFullGGffgggyyojjhhjkhjyyiuhggjhhjhuyytytyuuytrrtghjtyuggyuighjuyioyyfgffhyuhhghzzrrkkhhkkkhhhjhkjhhghhgghiuiuhhjhjiuiuyujjkjjkint}
curl_{\vec x}\vec v=-2\vec w,
\end{equation}
where $\vec w$ is the vector of the angular speed of rotation of our
coordinate system. Then by
\er{MaxMedFullGGffgggyyojjhhjkhjyyiuhggjhhjhuyytytyuuytrrtghjtyuggyuighjuyioyyfgffhyuhhghzzrrkkhhkkkhhhjhkjhhghhgghiuiuhhjhjiuiuyujjkjjkint}
and
\er{MaxMedFullGGffgggyyojjhhjkhjyyiuhggjhhjhuyytytyuuytrrtghjtyuggyuighjuyioyyfgffhyuhhghzzrrkkhhkkkhjjhjjint}
we deduce
\begin{equation}\label{MaxMedFullGGffgggyyojjhhjkhjyyiuhggjhhjhuyytytyuuytrrtghjtyuggyuighjuyioyyfgffhyuhhghzzrrkkhhkkkhjjhjjjjkint}
\delta(-S)=\frac{1}{\sqrt{\kappa_0\gamma_0}}\left|\partial
\mathcal{S}_0\right|+\frac{2}{\kappa_0 c}\iint \vec w\cdot\vec n
\,d\mathcal{S}_0.
\end{equation}
In particular, if the surface $\mathcal{S}_0$ is a part of some
plain then we rewrite
\er{MaxMedFullGGffgggyyojjhhjkhjyyiuhggjhhjhuyytytyuuytrrtghjtyuggyuighjuyioyyfgffhyuhhghzzrrkkhhkkkhjjhjjjjkint}
as
\begin{equation}\label{MaxMedFullGGffgggyyojjhhjkhjyyiuhggjhhjhuyytytyuuytrrtghjtyuggyuighjuyioyyfgffhyuhhghzzrrkkhhkkkhjjhjjjjkjjkjljklint}
\delta(-S)=\frac{1}{\sqrt{\kappa_0\gamma_0}}\left|\partial
\mathcal{S}_0\right|+\frac{2}{\kappa_0 c} \left(\vec w\cdot\vec
n\right)\left|\mathcal{S}_0\right|.
\end{equation}
On the other hand, if the light travels across the same curve in the
opposite direction, then we must have:
\begin{equation}\label{MaxMedFullGGffgggyyojjhhjkhjyyiuhggjhhjhuyytytyuuytrrtghjtyuggyuighjuyioyyfgffhyuhhghzzrrkkhhkkkhjjhjjjjkjjkjljklpooint}
\delta(-S^-)=\frac{1}{\sqrt{\kappa_0\gamma_0}}\left|\partial
\mathcal{S}_0\right|-\frac{2}{\kappa_0 c} \left(\vec w\cdot\vec
n\right)\left|\mathcal{S}_0\right|.
\end{equation}
Thus, by taking the difference in two cases and using
\er{MaxVacFullPPNmmmffffffiuiuhjuughbghhuiiujjhhjjhjhhjhjjhhjhjjhint},
we deduce:
\begin{equation}\label{MaxMedFullGGffgggyyojjhhjkhjyyiuhggjhhjhuyytytyuuytrrtghjtyuggyuighjuyioyyfgffhyuhhghzzrrkkhhkkkhjjhjjjjkjjkjljklkkkhjhjint}
\left(\delta(-T)-\delta(-T^-)\right)=k_0\left(\delta(-S)-\delta(-S^-)\right)=\frac{4\omega}{\kappa_0
c^2}\,. \left(\vec w\cdot\vec n\right)\left|\mathcal{S}_0\right|.
\end{equation}
Here, $\gamma_0$ and $\kappa_0$ are the dielectric and the magnetic
permeability of the medium, $T$ is given in
\er{MaxVacFullPPNmmmffffffiuiuhjuughbghhuiiujjhhjjhjhhjhjjhhjhjjhint},
$\left|\mathcal{S}_0\right|$ is the area of the flat surface bounded
by the closed path of the light, $\vec n$ is the unit normal to the
surface, $\omega$ is the frequency of the light and $\vec w$ is the
angular speed vector of the rotation of our coordinate system.


































































































































\subsubsection{Fizeau experiment}\label{seGOfzint}
Assume again the monochromatic electromagnetic wave of the frequency
$\nu=\omega/(2\pi)$ characterized by:
\begin{equation}\label{MaxVacFullPPNmmmffffffiuiuhjuughbghhuiiujjhhjjhjhhjhjjhhjhjjhfzint}
U(\vec x,t)=A(\vec x,t)e^{iT(\vec x,t)}=A(\vec x,t)e^{ik_0S(\vec
x,t)},\quad\text{where}\quad
k_0=\frac{\omega}{c}\quad\text{and}\quad\frac{\partial S}{\partial
t}=c\,.
\end{equation}
Then by \er{gughhghfbvnbvyyuuyrint} we have
\begin{equation}\label{gughhghfbvnbvyyuuyrhjhjhjfzint}
c_0=c\sqrt{\kappa_0\gamma_0}\quad\text{and}\quad\vec {\tilde
u}=\left(\gamma_0\vec v+(1-\gamma_0)\vec u\right),
\end{equation}
where, $\vec u$ is the medium velocity and $\vec v$ is the local
vectorial gravitational potential. Moreover, assume that we consider
light traveling in some region filled with the moving medium of
constant dielectric permeability $\gamma_0$ and magnetic
permeability $\kappa_0$. Then by \er{gughhghfbvnbvyyuuyrhjhjhjfzint}
and
\er{MaxMedFullGGffgggyyojjhhjkhjyyiuhggjhhjhuyytytyuuytrrtghjtyuggyuighjuyioyyfgffhyuhhghzzrrkkhhkkkhhhjhkjhhghhgghiuiu1int}
we have
\begin{equation}\label{gughhghfbvnbvyyuuyrhiyyuiuuhjhfzint}
n=\frac{c}{c_0}=\frac{1}{\sqrt{\kappa_0\gamma_0}}\;\;\;\text{is a
constant,}\quad\text{and}\quad\vec {\tilde u}=\gamma_0\vec
v+\left(1-\frac{1}{\kappa_0n^2}\right)\vec u.
\end{equation}
Next, assume that the light travels from point $N$ to point $M$
across the curve $\vec r(s):[a,b]\to\mathbb{R}^3$ with endpoints
$\vec r(a)=N$ and $\vec r(b)=M$ undergoing possibly certain number
of reflections from mirrors during its travel. Then, as before, by
\er{MaxMedFullGGffgggyyojjhhjkhjyyiuhggjhhjhuyytytyuuytrrtghjtyuggyuighjuyioyyfgffhyuhhghzzrrkkhhkkkint},
\er{gughhghfbvnbvyyuuyrhiyyuiuuhjhfzint} and
\er{MaxMedFullGGffgggyyojjyugggjhhjiiuuiyuyuyyuyuzzhhkkkint} we
have:
\begin{multline}\label{MaxMedFullGGffgggyyojjhhjkhjyyiuhggjhhjhuyytytyuuytrrtghjtyuggyuighjuyioyyfgffhyuhhghzzrrkkhhkkkhjjhfzint}
\delta(-S):=\left(-S(M^-)\right)- \left(-S(N^+)\right)=\\
n\int_a^b \left|\vec r'(s)\right|ds-\frac{1}{\kappa_0 c}\int_a^b\vec
v\left(\vec r(s)\right)\cdot\vec
r'(s)ds-\frac{n^2}{c}\left(1-\frac{1}{\kappa_0n^2}\right)\int_a^b\vec
u\left(\vec r(s)\right)\cdot\vec r'(s)ds.
\end{multline}
Next assume that, either our curve is perpendicular to the direction
of the vectorial gravitational potential $\vec v$, that happens, for
example, if our path of the light is tangent to the Earth surface,
or assume that our curve is closed, i.e. $M=N$ and moreover, our
coordinate system is inertial, or more generally non-rotating. In
particular, if we assume that $M=N$ i.e. our curve is closed and
moreover, our coordinate system is inertial, or more generally
non-rotating, then, as before, by Stokes Theorem we have:
\begin{equation}\label{MaxMedFullGGffgggyyojjhhjkhjyyiuhggjhhjhuyytytyuuytrrtghjtyuggyuighjuyioyyfgffhyuhhghzzrrkkhhkkkhjjhjjfzint}
\int_a^b\vec v\left(\vec r(s)\right)\cdot\vec r'(s)ds=0.
\end{equation}
On the other hand in the case that our curve is perpendicular to the
direction of the vectorial gravitational potential $\vec v$,
\er{MaxMedFullGGffgggyyojjhhjkhjyyiuhggjhhjhuyytytyuuytrrtghjtyuggyuighjuyioyyfgffhyuhhghzzrrkkhhkkkhjjhjjfzint}
also trivially follows. Therefore, by inserting
\er{MaxMedFullGGffgggyyojjhhjkhjyyiuhggjhhjhuyytytyuuytrrtghjtyuggyuighjuyioyyfgffhyuhhghzzrrkkhhkkkhjjhjjfzint}
into
\er{MaxMedFullGGffgggyyojjhhjkhjyyiuhggjhhjhuyytytyuuytrrtghjtyuggyuighjuyioyyfgffhyuhhghzzrrkkhhkkkhjjhfzint}
in both cases we obtain:
\begin{multline}\label{MaxMedFullGGffgggyyojjhhjkhjyyiuhggjhhjhuyytytyuuytrrtghjtyuggyuighjuyioyyfgffhyuhhghzzrrkkhhkkkhjjhfzfzfzint}
\delta(-S)=\left(-S(M^-)\right)- \left(-S(N^+)\right)= n\int_a^b
\left|\vec
r'(s)\right|ds-\frac{n^2}{c}\left(1-\frac{1}{\kappa_0n^2}\right)\int_a^b\vec
u\left(\vec r(s)\right)\cdot\vec r'(s)ds.
\end{multline}
Then by
\er{MaxMedFullGGffgggyyojjhhjkhjyyiuhggjhhjhuyytytyuuytrrtghjtyuggyuighjuyioyyfgffhyuhhghzzrrkkhhkkkhjjhfzfzfzint}
and
\er{MaxVacFullPPNmmmffffffiuiuhjuughbghhuiiujjhhjjhjhhjhjjhhjhjjhfzint}
we deduce
\begin{multline}\label{MaxMedFullGGffgggyyojjhhjkhjyyiuhggjhhjhuyytytyuuytrrtghjtyuggyuighjuyioyyfgffhyuhhghzzrrkkhhkkkhjjhfzfzfzfzffzint}
\delta(-T):=\left(-T(M^-)\right)-
\left(-T(N^+)\right)=k_0\delta(-S)\\= \frac{n\omega}{c}\int_a^b
\left|\vec
r'(s)\right|ds-\frac{n^2\omega}{c^2}\left(1-\frac{1}{\kappa_0n^2}\right)\int_a^b\vec
u\left(\vec r(s)\right)\cdot\vec r'(s)ds\\
\frac{n^2\omega}{c^2}\left(c_0\int_a^b \left|\vec
r'(s)\right|ds-\left(1-\frac{1}{\kappa_0n^2}\right)\int_a^b\vec
u\left(\vec r(s)\right)\cdot\vec r'(s)ds\right).
\end{multline}
In particular, if the absolute value $\left|\vec u\left(\vec
r(s)\right)\right|$ is a constant across the curve and if the angle
between $r'(s)$ and $\vec u\left(\vec r(s)\right)$ is a constant
across the curve and equals to the value $\theta$ then denoting the
length of the path by $L$:
\begin{equation}\label{MaxMedFullGGffgggyyojjhhjkhjyyiuhggjhhjhuyytytyuuytrrtghjtyuggyuighjuyioyyfgffhyuhhghzzrrkkhhkkkhjjhjjfzuyyint}
L:=\int_a^b \left|\vec r'(s)\right|ds,
\end{equation}
by
\er{MaxMedFullGGffgggyyojjhhjkhjyyiuhggjhhjhuyytytyuuytrrtghjtyuggyuighjuyioyyfgffhyuhhghzzrrkkhhkkkhjjhfzfzfzfzffzint}
we deduce:
\begin{equation}\label{MaxMedFullGGffgggyyojjhhjkhjyyiuhggjhhjhuyytytyuuytrrtghjtyuggyuighjuyioyyfgffhyuhhghzzrrkkhhkkkhjjhfzfzfzfzffzghghgfzint}
\delta(-T)=k_0\delta(-S)=\frac{\omega L
n^2}{c^2}\left(c_0-\left(1-\frac{1}{\kappa_0n^2}\right)|\vec
u|\cos{(\theta)}\right).
\end{equation}
Thus, if the direction of $\vec u$ coincides with the direction of
the light i.e. $\theta=0$ then
\begin{equation}\label{MaxMedFullGGffgggyyojjhhjkhjyyiuhggjhhjhuyytytyuuytrrtghjtyuggyuighjuyioyyfgffhyuhhghzzrrkkhhkkkhjjhfzfzfzfzffzghghgfzytyfzint}
\delta(-T)=k_0\delta(-S)=\frac{\omega L
n^2}{c^2}\left(c_0-\left(1-\frac{1}{\kappa_0n^2}\right)|\vec
u|\right)\approx\frac{\omega
L}{\left(c_0+\left(1-\frac{1}{\kappa_0n^2}\right)|\vec u|\right)}.
\end{equation}
On the other hand, if the direction of $\vec u$ is opposite to the
direction of the light i.e. $\theta=\pi$ then
\begin{equation}\label{MaxMedFullGGffgggyyojjhhjkhjyyiuhggjhhjhuyytytyuuytrrtghjtyuggyuighjuyioyyfgffhyuhhghzzrrkkhhkkkhjjhfzfzfzfzffzghghgfzuyuyhffzint}
\delta(-T)=k_0\delta(-S)=\frac{\omega L
n^2}{c^2}\left(c_0+\left(1-\frac{1}{\kappa_0n^2}\right)|\vec
u|\right)\approx\frac{\omega
L}{\left(c_0-\left(1-\frac{1}{\kappa_0n^2}\right)|\vec u|\right)}.
\end{equation}
So, in the case where the magnetic permeability is close to one,
i.e. $\kappa_0=1$, in the frames of our model we explain the results
of the Fizeau experiment.




































































































\section{Notations and preliminaries} \noindent$\bullet$ By
$\R^{p\times q}$ we denote the set of $p\times q$-matrixes with real
coefficients.

%\item
%For a $p\times q$ matrix $A$ with $ij$-th entry $a_{ij}$ we denote
%by $|A|=\bigl(\Sigma_{i=1}^{p}\Sigma_{j=1}^{q}a_{ij}^2\bigr)^{1/2}$
%the Frobenius norm of $A$.

%%\item For two matrices $A,B\in\R^{p\times q}$  with $ij$-th entries
%%$a_{ij}$ and $b_{ij}$ respectively, we write\\
%%$A:B\,:=\,\sum\limits_{i=1}^{p}\sum\limits_{j=1}^{q}a_{ij}b_{ij}$.

\noindent$\bullet$ For a $p\times q$ matrix $A$ with $ij$-th entry
$a_{ij}$ and for a $q\times d$ matrix $B$ with $ij$-th entry
$b_{ij}$ we denote by $AB:=A\cdot B$ their product, i.e. the
$p\times d$ matrix, with $ij$-th entry
$\sum\limits_{k=1}^{q}a_{ik}b_{kj}$.

\noindent$\bullet$ We identify a vector $\vec
u=(u_1,\ldots,u_q)\in\R^q$ with the $q\times 1$ matrix
%$A$ with
having $i1$-th entry $u_i$, so that for the $p\times q$ matrix $A$
with $ij$-th entry $a_{ij}$ and for $\vec
v=(v_1,v_2,\ldots,v_q)\in\R^q$ we denote by $A\,\vec v :=A\cdot\vec
v$ the $p$-dimensional vector $\vec u=(u_1,\ldots,u_p)\in\R^p$,
given by $u_i=\sum\limits_{k=1}^{q}a_{ik}v_k$ for every $1\leq i\leq
p$.

\noindent$\bullet$ For a $p\times q$ matrix $A$ with $ij$-th entry
$a_{ij}$ denote by $A^T$ the transpose $q\times p$ matrix with
$ij$-th entry $a_{ji}$.

\noindent$\bullet$ For a $p\times p$ matrix $A$ with $ij$-th entry
$a_{ij}$ denote $tr(A):=\sum_{k=1}^{p}a_{kk}$ (the trace of the
matrix $A$).

\noindent$\bullet$ For $\vec u=(u_1,\ldots,u_p)\in\R^p$ and $\vec
v=(v_1,\ldots,v_p)\in\R^p$ we denote by $\vec u\vec v:=\vec
u\cdot\vec v:=\sum\limits_{k=1}^{p}u_k v_k$ the standard scalar
product. We also note that $\vec u\vec v=\vec u^T\vec v=\vec v^T\vec
u$ as products of matrices.

\noindent$\bullet$ For $\vec u=(u_1,u_2,u_3)\in\R^3$ and $\vec
v=(v_1,v_2,v_3)\in\R^3$ we denote
%by
$$\vec u\times\vec v:=
%$ the vector product $
\left(u_2 v_3-u_3 v_2, u_3 v_1-u_1 v_3, u_1 v_2-u_2
v_1\right)\in\R^3.$$

\noindent$\bullet$ For $\vec u=(u_1,\ldots,u_p)\in\R^p$ and $\vec
v=(v_1,\ldots,v_q)\in\R^q$ we denote by $\vec u\otimes\vec v$ the
$p\times q$ matrix with $ij$-th entry $u_i v_j$ (i.e. $\vec
u\otimes\vec v=\vec u\,\vec v^T$ as a product of matrices).
%\item For
%any $p\times q$ matrix $A$ with $ij$-th entry $a_{ij}$ and $\vec
%v=(v_1,v_2,\ldots,v_d)\in\R^d$ we denote by $A\otimes\vec v$ the
%$p\times q\times d$ tensor with $ijk$-th entry $a_{ij}v_k$.

\noindent$\bullet$
%For any $p\times p$ matrix $A$ with $ij$-th entry $a_{ij}$ and any
%$q\times q$ matrix $B$ with $ij$-th entry $b_{ij}$ we denote by
%$A\oplus B$ the $(p+q)\times (p+q)$ matrix with $ij$-th entry
%$c_{ij}$, satisfying $c_{ij}=a_{ij}$ for every $1\leq i,j\leq p$,
%$c_{ij}=b_{(i-p)(j-p)}$ for every $p< i,j\leq (p+q)$, $c_{ij}=0$ for
%$1\leq i\leq p$ and $p<j\leq (p+q)$ and $c_{ij}=0$
%for $p<i\leq (p+q)$ and $1\leq j\leq p$.\\
Given a vector valued function $\vec f(\vec x)=\left(f_1(\vec
x),\ldots,f_k(\vec x)\right):\O\to\R^k$ ($\O\subset\R^N$) we denote
by $D\vec f$ the $k\times N$ matrix with $ij$-th entry
$\frac{\partial f_i}{\partial x_j}$. In the case of a scalar valued
function $\psi(\vec x):\O\subset\R^N\to\R$ we associate with $D
\psi$ (which, by definition, belongs to $\R^{1\times N}$) the
corresponding vector $\nabla \psi:=\left(\frac{\partial
\psi}{\partial x_1},\ldots,\frac{\partial \psi}{\partial
x_N}\right)$.

\noindent$\bullet$ Given a matrix valued function $F(\vec
x):=\{F_{ij}(\vec x)\}:\O\subset\R^N\to\R^{k\times N}$,
%($\O\subset\R^N$)
we denote by ${div}\, F$ the $\R^k$-valued vector field defined by
$div\,F(\vec x):=(l_1,\ldots,l_k)(\vec x)$ where $l_i(\vec
x)=\sum\limits_{j=1}^{N}\frac{\partial F_{ij}}{\partial x_j}(\vec
x)$. Given a vector valued function $\vec f(\vec
x):=\left(f_{1}(\vec x),\ldots, f_{N}(\vec
x)\right):\O\subset\R^N\to\R^{N}$ we denote
%($\O\subset\R^N$)
${div}\, \vec f:=\sum\limits_{j=1}^{N}\frac{\partial f_j}{\partial
x_j}$.

\noindent$\bullet$ Given a scalar or vector valued function $\vec
f(\vec x):\O\subset\R^N\to\R^{k}$ we denote by $\Delta \vec f$ the
Laplacian of $\vec f$ defined by $\Delta \vec
f:=\sum\limits_{j=1}^{N}\frac{\partial^2\vec  f}{\partial x^2_j}$.

\noindent$\bullet$ Given a vector valued function $\vec f(\vec
x)=\left(f_1(\vec x),f_2(\vec x),f_3(\vec
x)\right):G\subset\R^3\to\R^3$
%($G\subset\R^3$)
we denote
$$curl\,\vec f(\vec x):=\left(\frac{\partial f_3}{\partial x_2}-\frac{\partial f_2}{\partial x_3}\,,
\frac{\partial f_1}{\partial x_3}-\frac{\partial f_3}{\partial
x_1}\,,\frac{\partial f_2}{\partial x_1}-\frac{\partial
f_1}{\partial x_2}\right)(\vec x).$$
%
%
%
%
%
\begin{comment}
\item Given a
matrix valued function $F(\vec x)=\left\{f_{ij}(\vec
x)\right\}(1\leq i\leq p,\,1\leq j\leq q):\O\to\R^{p\times q}$
($\O\subset\R^N$) we denote by $DF$ or by $\nabla_{\vec x} F$ the
$p\times q\times N$ tensor with $ijk$-th entry $\frac{\partial
f_{ij}}{\partial x_k}$.
\item For every dimension $d$
we denote by $I$ the unit $d\times d$-matrix and by $O$ the null
$d\times d$-matrix.
\item Given a vector valued
measure $\mu=(\mu_1,\ldots,\mu_k)$ (where for any $1\leq j\leq k$,
$\mu_j$ is a finite signed measure) we denote by $\|\mu\|(E)$ its
total variation measure of the set $E$.
\item For any $\mu$-measurable function $f$, we define the product measure
$f\cdot\mu$ by: $f\cdot\mu(E)=\int_E f\,d\mu$, for every
$\mu$-measurable set $E$.
\item Throughout this paper we assume that
$\O\subset\R^N$ is an open set.
%a Lipschitz domain.
\end{comment}
%
%
%
%
%
%\end{itemize}
We have the following trivial identities:
%
%
%
%
\begin{comment}
\begin{align}
\vec a\times \vec b=-\vec b\times\vec a\quad\text{and}\quad \vec
a\cdot(\vec b\times\vec c)=(\vec a\times\vec b)\cdot\vec
c\quad\quad\forall\,\vec a,\vec b,\vec c\in\R^3,\label{frm1}\\
\vec a\times(\vec b\times\vec c)=(\vec a\cdot\vec c)\,\vec b-(\vec
a\cdot\vec b)\,\vec c\quad\quad\forall\,\vec a,\vec b,\vec
c\in\R^3,\label{frm2}\\
(A\cdot \vec b)\times \vec c-(A\cdot \vec c)\times \vec
b=tr(A)\,(\vec b\times \vec c)-A^T\cdot(\vec b\times \vec
c)\quad\quad\forall A\in\R^{3\times 3},\;\forall\,\vec b,\vec
c\in\R^3,\label{frm10}\\
A^T\cdot\left((A\cdot \vec b)\times (A\cdot \vec c)\right)=(det\,
A)\,(\vec b\times \vec c)\quad\quad\forall A\in\R^{3\times
3},\;\forall\,\vec b,\vec
c\in\R^3,\label{frm90}\\
{div}(\vec f\times \vec g)=\vec g\cdot curl\,\vec f-\vec f\cdot
curl\,\vec g\quad\quad
%\text{for vector valued functions}\;\;
\forall\,\vec f,\vec g:G\subset\R^3\to\R^3,\label{frm3}\\
{div}(\psi \vec f)=\psi\,{div}\, \vec f+\nabla\psi\cdot \vec
f\quad\quad
%\text{for vector valued functions}\;\;
\forall\,\psi:G\subset\R^3\to\R,\;\;
\forall\,\vec f:G\subset\R^3\to\R^3,\label{frm4}\\
curl\,(\psi \vec f)=\psi\,curl\, \vec f+\nabla\psi\times \vec
f\quad\quad \forall\,\psi:G\subset\R^3\to\R,\;\;
\forall\,\vec f:G\subset\R^3\to\R^3,\label{frm5}\\
{div}\left(curl\,\vec f\right)=0\quad\quad\forall\,\vec
f:G\subset\R^3\to\R^3,\label{frm11}\\
{curl}\left(\nabla\psi\right)=0\quad\quad\forall\,\vec
\psi:G\subset\R^3\to\R,\label{frm100}
\\ curl\,\left(curl\,\vec f\right)=\nabla\left({div} \vec f\right)-\Delta\vec
f\quad\quad\forall\,\vec f:G\subset\R^3\to\R^3,\label{frm12}
\\ curl\,(\vec f\times \vec g)=({div}\, \vec g)\,\vec f -({div}\, \vec
f)\,\vec g +(D \vec f)\cdot \vec g-(D \vec g)\cdot \vec f\quad\quad
\forall\,\vec f,\vec g:G\subset\R^3\to\R^3,\label{frm6}\\
curl\,(\vec f\times \vec g)=({div}\, \vec g)\,\vec f -({div}\, \vec
f)\,\vec g +(D \vec f)\cdot \vec g-(D \vec g)\cdot \vec f\quad\quad
\forall\,\vec f,\vec g:G\subset\R^3\to\R^3,\label{frm6kkk}\\
\nabla(\vec f\cdot \vec g)=(D \vec f)^T\cdot \vec g+(D
\vec g)^T\cdot \vec f\quad\quad \forall\,\vec f,\vec g:G\subset\R^3\to\R^3,\label{frm7}\\
\vec f\times(curl\,\vec g)=(D \vec g)^T\cdot \vec f-(D
\vec g)\cdot \vec f\quad\quad \forall\,\vec f,\vec g:G\subset\R^3\to\R^3,\label{frm9}\\
\nabla(\vec f\cdot \vec g)=\vec f\times(curl\,\vec g)+\vec
g\times(curl\,\vec f)+(D \vec f)\cdot \vec g+(D \vec g)\cdot \vec
f\quad\quad \forall\,\vec f,\vec g:G\subset\R^3\to\R^3,\label{frm8}
\end{align}
We have the following identities:
\end{comment}
%
%
%
%
\begin{align}
\vec a\times \vec b=-\vec b\times\vec a\quad\text{and}\quad \vec
a\cdot(\vec b\times\vec c)=(\vec a\times\vec b)\cdot\vec
c\quad\quad\forall\,\vec a,\vec b,\vec c\in\R^3,\label{apfrm1}\\
\vec a\times(\vec b\times\vec c)=(\vec a\cdot\vec c)\,\vec b-(\vec
a\cdot\vec b)\,\vec c\quad\quad\forall\,\vec a,\vec b,\vec
c\in\R^3,\label{apfrm2}\\
(A\cdot \vec b)\times \vec c-(A\cdot \vec c)\times \vec
b=tr(A)\,(\vec b\times \vec c)-A^T\cdot(\vec b\times \vec
c)\quad\quad\forall A\in\R^{3\times 3},\;\forall\,\vec b,\vec
c\in\R^3,\label{apfrm10}\\
A^T\cdot\left((A\cdot \vec b)\times (A\cdot \vec c)\right)=(det\,
A)\,(\vec b\times \vec c)\quad\quad\forall A\in\R^{3\times
3},\;\forall\,\vec b,\vec
c\in\R^3,\label{apfrm90}\\
{div}(\vec f\times \vec g)=\vec g\cdot curl\,\vec f-\vec f\cdot
curl\,\vec g\quad\quad
%\text{for vector valued functions}\;\;
\forall\,\vec f,\vec g:G\subset\R^3\to\R^3,\label{apfrm3}\\
{div}(\psi \vec f)=\psi\,{div}\, \vec f+\nabla\psi\cdot \vec
f\quad\quad
%\text{for vector valued functions}\;\;
\forall\,\psi:G\subset\R^3\to\R,\;\;
\forall\,\vec f:G\subset\R^3\to\R^3,\label{apfrm4}\\
curl\,(\psi \vec f)=\psi\,curl\, \vec f+\nabla\psi\times \vec
f\quad\quad \forall\,\psi:G\subset\R^3\to\R,\;\;
\forall\,\vec f:G\subset\R^3\to\R^3,\label{apfrm5}\\
{div}\left(curl\,\vec f\right)=0\quad\quad\forall\,\vec
f:G\subset\R^3\to\R^3,\label{apfrm11}\\
{curl}\left(\nabla\psi\right)=0\quad\quad\forall\,\vec
\psi:G\subset\R^3\to\R,\label{apfrm100}
\\ curl\,\left(curl\,\vec f\right)=\nabla\left({div}\, \vec f\right)-\Delta\vec
f\quad\quad\forall\,\vec f:G\subset\R^3\to\R^3,\label{apfrm12}
\\ curl\,(\vec f\times \vec g)=({div}\, \vec g)\,\vec f -({div}\, \vec
f)\,\vec g +(D \vec f)\cdot \vec g-(D \vec g)\cdot \vec f\quad\quad
\forall\,\vec f,\vec g:G\subset\R^3\to\R^3,\label{apfrm6}\\
curl\,(\vec f\times \vec g)=div\left(\vec f\otimes \vec g-\vec
g\otimes \vec f\right)\quad\quad
\forall\,\vec f,\vec g:G\subset\R^3\to\R^3,\label{apfrm6kkk}\\
div\left(\vec f\otimes \vec g\right)=(D \vec f)\cdot \vec g+({div}\,
\vec g)\,\vec f\quad\quad
\forall\,\vec f,\vec g:G\subset\R^3\to\R^3,\label{apfrm6kkkppp}\\
\nabla(\vec f\cdot \vec g)=(D \vec f)^T\cdot \vec g+(D
\vec g)^T\cdot \vec f\quad\quad \forall\,\vec f,\vec g:G\subset\R^3\to\R^3,\label{apfrm7}\\
\vec f\times(curl\,\vec g)=(D \vec g)^T\cdot \vec f-(D
\vec g)\cdot \vec f\quad\quad \forall\,\vec f,\vec g:G\subset\R^3\to\R^3,\label{apfrm9}\\
\nabla(\vec f\cdot \vec g)=\vec f\times(curl\,\vec g)+\vec
g\times(curl\,\vec f)+(D \vec f)\cdot \vec g+(D \vec g)\cdot \vec
f\quad\quad \forall\,\vec f,\vec
g:G\subset\R^3\to\R^3,\label{apfrm8}
\end{align}
where
%in the last five identities
we mean by $A\cdot \vec l$  the usual product of matrix
$A\in\R^{3\times 3}$ and vector $\vec l\in\R^3$ and by $A^T$ we mean
the transpose of matrix $A$.
%





\section{Transformations of scalar and vector fields under the change of inertial or non-inertial cartesian coordinate
system}\label{fhfgfghdfdfdfd}
\begin{definition}\label{bggghghgj}
Consider the change of some non-inertial cartesian coordinate system
$(*)$ to another cartesian coordinate system $(**)$ of the form:
\begin{equation}\label{noninchgravortbstr}
\begin{cases}
\vec x'=A(t)\cdot\vec x+\vec z(t),\\
t'=t,
\end{cases}
\end{equation}
where $A(t)\in SO(3)$ is a rotation, i.e. $A(t)\in \R^{3\times 3}$,
$det\, A(t)>0$ and $A(t)\cdot A^T(t)=I$.
%and $\vec z(t)$ is given vector depending only on the time $t$.
\begin{itemize}
\item
We say that the scalar field $\psi:=\psi(\vec
x,t):\R^3\times[0,+\infty)\to\R$ is a proper scalar field if, under
every change of coordinate system given by \er{noninchgravortbstr},
this field transforms by the law:
\begin{equation}\label{uguyytfdddd}
\psi'(\vec x',t')=\psi(\vec x,t).
\end{equation}

\item
We say that the vector field $\vec f:=\vec f(\vec
x,t):\R^3\times[0,+\infty)\to\R^3$ is a proper vector field if,
under every change of coordinate system given by
\er{noninchgravortbstr}, this field transforms by the law:
\begin{equation}\label{uguyytfddddgghjjg}
\vec f'(\vec x',t')=A(t)\cdot\vec f(\vec x,t),
\end{equation}


\item
We say that the vector field $\vec v:=\vec v(\vec
x,t):\R^3\times[0,+\infty)\to\R^3$ is a speed-like vector field if,
under every change of coordinate system given by
\er{noninchgravortbstr}, this field transforms by the law:
\begin{equation}
\label{NoIn5redbstr}\vec v'(\vec x',t')=A(t)\cdot \vec v(\vec x,t)+
%A'(t)
\frac{d A}{dt}(t)\cdot\vec x+
%\frac{d\vec z}{dt}(t)
\vec w(t),
\end{equation}
where we set
\begin{equation}\label{buitguihjk}
\vec w(t):=\frac{d\vec z}{dt}(t)\quad\quad\forall\,t.
\end{equation}

\item
We say that the matrix valued field $T:=T(\vec
x,t):\R^3\times[0,+\infty)\to\R^{3\times 3}$ is a proper matrix
field if, under every change of coordinate system given by
\er{noninchgravortbstr}, this field transforms by the law:
\begin{equation}\label{uguyytfddddgghjjghjjj}
T'(\vec x',t')=A(t)\cdot T(\vec x,t)\cdot A^T(t)=A(t)\cdot T(\vec
x,t)\cdot \left\{A(t)\right\}^{-1}.
\end{equation}
\end{itemize}
\end{definition}
\begin{proposition}\label{yghgjtgyrtrt}
If $\psi:\R^3\times[0,+\infty)\to\R$ is a proper scalar field, $\vec
f:\R^3\times[0,+\infty)\to\R^3$ and $\vec
g:\R^3\times[0,+\infty)\to\R^3$ are proper vector fields, $\vec
v:\R^3\times[0,+\infty)\to\R^3$ and $\vec
u:\R^3\times[0,+\infty)\to\R^3$ are speed-like vector fields and
$T:\R^3\times[0,+\infty)\to\R^{3\times 3}$ is a proper matrix field,
then:
\begin{itemize}
\item[{\bf(i)}] scalar fields defined in every coordinate system as $\,\vec f\cdot\vec g$, $div_{\vec x} \vec f$ and $div_{\vec x} \vec
v$ are proper scalar fields;

\item[{\bf(ii)}] vector fields defined in every coordinate system as $\nabla_{\vec x}\psi$, $div_{\vec x} T$, $curl_{\vec x}\vec
f$, $\vec f\times\vec g$,
%$\Delta_{\vec x}\vec v$
$div_{\vec x}\left(d_{\vec x} \vec v+\left\{d_{\vec x} \vec
v\right\}^T\right)$, $\nabla_{\vec x}\left(div_{\vec x}\vec
v\right)$, $\Delta_{\vec x}\vec v$, $curl_{\vec x}\left(curl_{\vec
x}\vec v\right)$ and $(\vec u-\vec v)$ are proper vector fields;

\item[{\bf(iii)}] matrix fields defined in every coordinate system as $\,d_{\vec x} \vec f$ and $\left(d_{\vec x} \vec v+\left\{d_{\vec x} \vec v\right\}^T\right)$ are proper
matrix fields;

\item[{\bf(iv)}] scalar fields $\xi:\R^3\times[0,+\infty)\to\R$ and $\zeta:\R^3\times[0,+\infty)\to\R$, defined
in every coordinate system by
\begin{equation}\label{vfyutuyfffhhgfhgfhgtg}
\xi:=\frac{\partial\psi}{\partial t}+\vec v\cdot\nabla_{\vec x}\vec
\psi\quad \text{and}\quad \zeta:=\frac{\partial\psi}{\partial
t}+div_{\vec x}\left\{ \psi\vec v\right\}
\end{equation}
are proper scalar fields;

\item[{\bf(v)}] vector fields $\vec\Theta:\R^3\times[0,+\infty)\to\R^3$ and $\vec\Xi:\R^3\times[0,+\infty)\to\R^3$, defined
in every coordinate system by
\begin{equation}\label{vfyutuyfffhhgfhgfh}
\vec\Theta:=\frac{\partial \vec f}{\partial t}- curl_{\vec
x}\left(\vec v\times \vec f\right)+\left({div}_{\vec x}\vec
f\right)\vec v\quad\text{and}\quad \vec\Xi:=\frac{\partial \vec
f}{\partial t}- \vec v\times curl_{\vec x}\vec f+\nabla_{\vec
x}\left(\vec v\cdot\vec f\right),
\end{equation}
are proper vector fields and
\begin{equation}
%\begin{multline}
\label{vhfffngghhjghhgjlkhjhkPPP} \vec\Xi
%\frac{\partial\vec f}{\partial t}-\vec v\times curl_{\vec x}\vec f+\nabla_{\vec x}\left(\vec f\cdot\vec v\right)
=
%\left(\frac{\partial\vec f}{\partial t}-curl_{\vec x}\left(\vec v\times\vec f\right)+\left(div_{\vec x}\vec f\right)\vec v\right)
\vec\Theta-\left(div_{\vec x}\vec v\right)\vec f+ \left(d_{\vec
x}\vec v+\left\{d_{\vec x}\vec v\right\}^T\right)\cdot\vec f.
%\end{multline}
\end{equation}
\end{itemize}
\end{proposition}
\begin{proof}
By \er{noninchgravortbstr} and the chain rule for every vector
fields $\vec \Gamma:\R^3\times[0,+\infty)\to\R^3$ and $\vec
\Lambda:\R^3\times[0,+\infty)\to\R^3$ we have
\begin{equation}
\label{vyguiuiujggghjjgredbstr}
\begin{cases}
\left(A(t)\cdot\vec \Gamma\right)\cdot\left(A(t)\cdot\vec
\Lambda\right)=\vec\Gamma\cdot\vec \Lambda\\
\left(A(t)\cdot\vec \Gamma\right)\times\left(A(t)\cdot\vec
\Lambda\right)=A(t)\cdot\left(\vec\Gamma\times\vec \Lambda\right)
\\
%\nabla_{\vec x'}\left(A(t)\cdot\vec F\right)=A(t)\cdot\nabla_{\vec x}\vec F\cdot A^{-1}(t)\\
d_{\vec x'}\vec \Gamma=\left(d_{\vec x}\vec \Gamma\right)\cdot A^{-1}(t)\\
curl_{\vec x'}\left( A(t)\cdot\vec \Gamma\right)=A(t)\cdot
curl_{\vec x}
\vec \Gamma\\
div_{\vec x'}\left( A(t)\cdot\vec \Gamma\right)=div_{\vec x}\vec
\Gamma.
\end{cases}
\end{equation}
Thus, in particular, by \er{vyguiuiujggghjjgredbstr} and
\er{uguyytfddddgghjjg} we have
\begin{equation}\label{uguyytfddddgghjjguyyjyhkkyhhjhjh}
\vec f'\cdot\vec g'=\vec f\cdot\vec g,\quad\quad \vec f'\times\vec
g'=A(t)\left(\vec f\times\vec g\right),
\end{equation}
and
\begin{equation}\label{uguyytfddddgghjjguyyjyhkkyh}
div_{\vec x'}\vec f'=div_{\vec x'}\left(A(t)\cdot\vec
f\right)=div_{\vec x}\vec f,
\end{equation}
and by \er{vyguiuiujggghjjgredbstr} and \er{NoIn5redbstr} we have
\begin{multline}
\label{NoIn5redbstrgfhgfhfh} div_{\vec x'}\vec v'=div_{\vec
x'}\left\{A(t)\cdot \vec v+
%A'(t)
A'(t)\cdot\vec x+\vec w(t)\right\}=div_{\vec x}\left\{\vec
v+A^{-1}(t)\cdot A'(t)\cdot\vec x+A^{-1}(t)\cdot\vec
w(t)\right\}\\=div_{\vec x}\vec v+tr\left(A^{-1}(t)\cdot
A'(t)\right).
\end{multline}
where $tr\left(A^{-1}(t)\cdot A'(t)\right)$ is the trace of the
matrix $A^{-1}(t)\cdot A'(t)$ (sum of diagonal elements). However,
since $A^T(t)\cdot A(t)=I$ we have $A^{-1}(t)=A^T(t)$ and
$A^{-1}(t)\cdot A'(t)=S(t)$, where $S^T(t)=-S(t)$. In particular
$tr\, S(t)=0$ and thus
\begin{equation}
\label{NoIn5redbstrgfhgfhfhuhhjhjj} tr\,\left(A^{-1}(t)\cdot
A'(t)\right)=0.
\end{equation}
Thus by \er{NoIn5redbstrgfhgfhfh} and
\er{NoIn5redbstrgfhgfhfhuhhjhjj} we have
\begin{equation}
\label{NoIn5redbstrgfhgfhfhgffgf} div_{\vec x'}\vec v'=div_{\vec
x}\vec v.
\end{equation}
So by \er{uguyytfddddgghjjguyyjyhkkyhhjhjh},
\er{uguyytfddddgghjjguyyjyhkkyh} and \er{NoIn5redbstrgfhgfhfhgffgf}
we proved {\bf(i)}.


 Next by \er{vyguiuiujggghjjgredbstr} and \er{uguyytfddddgghjjg} we
have
\begin{equation}\label{uguyytfddddgghjjghjg}
d_{\vec x'}\vec f'=d_{\vec x'}\left(A(t)\cdot\vec f\right)=A(t)\cdot
d_{\vec x'}\vec f=A(t)\cdot\left(d_{\vec x}\vec f\right)\cdot
A^{-1}(t)=A(t)\cdot\left(d_{\vec x}\vec f\right)\cdot A^T(t),
\end{equation}
and by \er{vyguiuiujggghjjgredbstr} and \er{NoIn5redbstr} we have
\begin{multline}
\label{NoIn5redbstruiytrr} d_{\vec x'}\vec v'=d_{\vec
x'}\left(A(t)\cdot \vec v+ A'(t) \cdot\vec x+ \vec
w(t)\right)=A(t)\cdot d_{\vec x'}\vec v+d_{\vec x}\left(A'(t)
\cdot\vec x\right)\cdot A^{-1}(t)\\=A(t)\cdot\left(d_{\vec x}\vec
v\right)\cdot A^{-1}(t)+A'(t)A^{-1}(t)=A(t)\cdot\left(d_{\vec x}\vec
v\right)\cdot A^T(t)+A'(t)\cdot A^T(t).
\end{multline}
Then taking the transpose of the both sides of
\er{NoIn5redbstruiytrr} we infer
\begin{equation}
\label{NoIn5redbstruiytrrhjjggh} \left\{d_{\vec x'}\vec
v'\right\}^T=A(t)\cdot\left\{d_{\vec x}\vec v\right\}^T\cdot
A^T(t)+A(t)\cdot \left\{A'(t)\right\}^T.
\end{equation}
However, as before, since $A(t)\cdot A^T(t)=I$ we have $A'(t)\cdot
A^T(t)+A(t)\cdot \left\{A'(t)\right\}^T=0$, by
\er{NoIn5redbstruiytrr} and \er{NoIn5redbstruiytrrhjjggh} we have
\begin{equation}
\label{NoIn5redbstruiytrrhjjgghjkgkgh} \left(d_{\vec x'}\vec
v'+\left\{d_{\vec x'}\vec v'\right\}^T\right)=A(t)\cdot\left(d_{\vec
x}\vec v+\left\{d_{\vec x}\vec v\right\}^T\right)\cdot A^T(t).
\end{equation}
So by \er{uguyytfddddgghjjghjg} and
\er{NoIn5redbstruiytrrhjjgghjkgkgh} we proved {\bf(iii)}.

 Next by the chain rule and \er{uguyytfdddd} we obtain
\begin{equation}\label{uguyytfddddlhhjlk}
\nabla_{\vec x'}\psi'=\nabla_{\vec
x'}\psi=\left\{A^{-1}(t)\right\}^T\cdot\nabla_{\vec
x}\psi=A(t)\cdot\nabla_{\vec x}\psi,
\end{equation}
by \er{uguyytfddddgghjjg} and \er{vyguiuiujggghjjgredbstr} we obtain
\begin{equation}\label{uguyytfddddgghjjgghghhf}
curl_{\vec x'}\vec f'=curl_{\vec x'}\left(A(t)\cdot\vec
f\right)=A(t)\cdot curl_{\vec x}\vec f,
\end{equation}
and by the chain rule and \er{uguyytfddddgghjjghjjj} we have
\begin{equation}\label{uguyytfddddgghjjghjjjjhghghghjjff}
div_{\vec x'}T'=div_{\vec x'}\left(A(t)\cdot T\cdot
A^T(t)\right)=A(t)\cdot\left(div_{\vec x}T\right).
%A(t)\cdot T\cdot \left\{A(t)\right\}^{-1}.
\end{equation}
Thus by \er{uguyytfddddgghjjghjjjjhghghghjjff} and
\er{NoIn5redbstruiytrrhjjgghjkgkgh} we have
\begin{equation}
\label{NoIn5redbstruiytrrhjjgghjkgkghfhfddgh} div_{\vec
x'}\left(d_{\vec x'}\vec v'+\left\{d_{\vec x'}\vec
v'\right\}^T\right)=A(t)\cdot\left\{div_{\vec x}\left(d_{\vec x}\vec
v+\left\{d_{\vec x}\vec v\right\}^T\right)\right\}.
\end{equation}
On the other hand by \er{uguyytfddddgghjjguyyjyhkkyh} and
\er{uguyytfddddlhhjlk} we have
\begin{equation}
\label{NoIn5redbstrgfhgfhfhgffgfghgjhjgjjj} \nabla_{\vec
x'}\left(div_{\vec x'}\vec v'\right)=A(t)\cdot\nabla_{\vec
x}\left(div_{\vec x}\vec v\right).
\end{equation}
Therefore, by \er{NoIn5redbstruiytrrhjjgghjkgkghfhfddgh} and
\er{NoIn5redbstrgfhgfhfhgffgfghgjhjgjjj}, using \er{apfrm12} we
deduce
\begin{equation}
\label{NoIn5redbstrgfhgfhfhgffgfghgjhjgjjjghjjjf} \Delta_{\vec
x'}\vec v'=A(t)\cdot\Delta_{\vec x}\vec v \quad\text{and}\quad
curl_{\vec x'}\left(curl_{\vec x'}\vec v'\right)=A(t)\cdot
curl_{\vec x}\left(curl_{\vec x}\vec v\right).
\end{equation}
Next by \er{NoIn5redbstr} we deduce
\begin{equation}\label{hjhkujgjgyjfhijkyuiu}
%\label{vyguiuiujggghjjg}
(\vec u'-\vec v')=A(t)\cdot(\vec u-\vec v).
\end{equation}
So by \er{uguyytfddddgghjjguyyjyhkkyhhjhjh}, \er{uguyytfddddlhhjlk},
\er{uguyytfddddgghjjgghghhf},
\er{uguyytfddddgghjjghjjjjhghghghjjff},
\er{NoIn5redbstruiytrrhjjgghjkgkghfhfddgh},
\er{NoIn5redbstrgfhgfhfhgffgfghgjhjgjjj},
\er{NoIn5redbstrgfhgfhfhgffgfghgjhjgjjjghjjjf} and
\er{hjhkujgjgyjfhijkyuiu} we deduce {\bf(ii)}.


Furthermore, by the chain rule for every scalar field
$\gamma:\R^3\times[0,+\infty)\to\R$ and for every vector field $\vec
\Gamma:\R^3\times[0,+\infty)\to\R^3$ we obtain
\begin{equation}\label{hjhkujgjgyjfhijk}
%\label{vyguiuiujggghjjg}
\frac{\partial \vec \gamma}{\partial t}=\frac{\partial \vec
\gamma}{\partial t'}+\left(A'(t)\cdot\vec x+\vec
w(t)\right)\cdot\nabla_{\vec x'}\vec \gamma
\end{equation}
and
\begin{equation}\label{hjhkujgjgyjf}
%\label{vyguiuiujggghjjg}
\frac{\partial \vec \Gamma}{\partial t}=\frac{\partial \vec
\Gamma}{\partial t'}+\left(d_{\vec x'}\vec
\Gamma\right)\cdot\left(A'(t)\cdot\vec x+\vec w(t)\right).
\end{equation}
Therefore, by \er{hjhkujgjgyjf} and \er{vyguiuiujggghjjgredbstr}
\begin{equation}
\label{vyguiuiujggghjjgggjredbstr} \frac{\partial \vec
\Gamma}{\partial t'}=\frac{\partial \vec \Gamma}{\partial
t}-\left(d_{\vec x}\vec \Gamma\right)\cdot \left(A^{-1}(t)\cdot
A'(t)\cdot\vec x+A^{-1}(t)\cdot \vec w(t)\right),
\end{equation}
and by \er{vyguiuiujggghjjgredbstr} \er{uguyytfddddlhhjlk} and
\er{hjhkujgjgyjfhijk}
\begin{equation}\label{hjhkujgjgyjfhijkjkhkggjggj}
%\label{vyguiuiujggghjjg}
\frac{\partial \vec \gamma}{\partial t'}+\left(A(t)\cdot\vec
\Gamma+A'(t)\cdot\vec x+\vec w(t)\right)\cdot\nabla_{\vec x'}\vec
\gamma=\frac{\partial \vec \gamma}{\partial
t}+\vec\Gamma\cdot\nabla_{\vec x}\gamma.
\end{equation}
In particular, by \er{uguyytfdddd}, \er{NoIn5redbstr} and
\er{hjhkujgjgyjfhijkjkhkggjggj} we have
\begin{equation}\label{hjhkujgjgyjfhijkjkhkggjggjugiu}
%\label{vyguiuiujggghjjg}
\frac{\partial \vec \psi}{\partial t'}+\vec v'\cdot\nabla_{\vec
x'}\vec \psi=\frac{\partial \vec \psi}{\partial t}+\vec
v\cdot\nabla_{\vec x}\psi
\end{equation}
and then since
\begin{equation}\label{hjhkujgjgyjfhijkjkhkggjggjugiujghj}
\frac{\partial \vec \psi}{\partial t}+div_{\vec x}\left\{\psi\vec
v\right\}=\frac{\partial \vec \psi}{\partial t}+\vec
v\cdot\nabla_{\vec x}\psi+\psi\left(div_{\vec x}\vec v\right),
\end{equation}
by \er{hjhkujgjgyjfhijkjkhkggjggjugiu}, \er{uguyytfdddd} and
\er{NoIn5redbstrgfhgfhfhgffgf} we infer {(\bf iv)}. On the other
hand,
%
%
%
\begin{comment}
using the trivial identity
\begin{equation}\label{cufuyggjjjfredbstr}
A\cdot\left(\vec a\times\vec b\right)=\left(A\cdot\vec
a\right)\times\left(A\cdot\vec b\right)\quad\forall\, \vec
a\in\R^3,\;\;\forall\, \vec b\in\R^3,\;\;\forall\, A\in SO(3),
\end{equation}
\end{comment}
%
%
%
%Moreover,
by \er{vyguiuiujggghjjgredbstr}, \er{vyguiuiujggghjjgggjredbstr} and
\er{NoIn5redbstr}
%%and \er{cufuyggjjjfredbstr}
%and \er{apfrm6}
for every vector field $\vec \Gamma:\R^3\times[0,+\infty)\to\R^3$ we
get:
\begin{multline}\label{bgvfgfhhtgjtggguyfredbstr}
\frac{\partial \left(A(t)\cdot\vec \Gamma\right)}{\partial
t'}-curl_{\vec x'}\left(\vec v'\times\left(A(t)\cdot\vec
\Gamma\right)\right)+\left({div}_{\vec x'}\left(A(t)\cdot\vec
\Gamma\right)\right)\vec v'=\\
\left(A(t)\cdot\frac{\partial \vec \Gamma}{\partial
t}+A'(t)\cdot\vec \Gamma-A(t)\cdot \left(d_{\vec x}\vec
\Gamma\right)\cdot \left(A^{-1}(t)\cdot A'(t)\cdot\vec
x+A^{-1}(t)\cdot \vec w(t)\right)\right)\\-A(t)\cdot curl_{\vec
x}\left(\left(\vec v+A^{-1}(t)\cdot A'(t)\cdot\vec
x+A^{-1}(t)\cdot\vec w(t)\right)\times
\vec\Gamma\right)\\+\left({div}_{\vec x}\vec
\Gamma\right)\left(A(t)\cdot\vec v+A'(t)\cdot\vec x+\vec
w(t)\right)\\= A(t)\cdot\left(\frac{\partial \vec \Gamma}{\partial
t}- curl_{\vec x}\left(\vec v\times
\vec\Gamma\right)+\left({div}_{\vec x}\vec \Gamma\right)\vec
v\right)\\+A(t)\cdot \left(d_{\vec x}\left(A^{-1}(t)\cdot
A'(t)\cdot\vec x+A^{-1}(t)\cdot\vec w(t)\right)\right)\cdot\vec
\Gamma\\-A(t)\cdot \left(d_{\vec x}\vec \Gamma\right)\cdot
\left(A^{-1}(t)\cdot A'(t)\cdot\vec x+A^{-1}(t)\cdot \vec
w(t)\right)\\+A(t)\cdot\left(\left({div}_{\vec x}\vec
\Gamma\right)\left(A^{-1}(t)\cdot A'(t)\cdot\vec
x+A^{-1}(t)\cdot\vec w(t)\right)\right)\\-A(t)\cdot curl_{\vec
x}\left(\left(A^{-1}(t)\cdot A'(t)\cdot\vec x+A^{-1}(t)\cdot\vec
w(t)\right)\times \vec\Gamma\right).
\end{multline}
On the other hand,  by \er{apfrm6} we have,
\begin{multline}\label{bgvfgfhhtgjtgjuiyuijjkredbstr}
\left(d_{\vec x}\left(A^{-1}(t)\cdot A'(t)\cdot\vec
x+A^{-1}(t)\cdot\vec w(t)\right)\right)\cdot\vec
\Gamma\\-\left(d_{\vec x}\vec \Gamma\right)\cdot
\left(A^{-1}(t)\cdot A'(t)\cdot\vec x+A^{-1}(t)\cdot \vec
w(t)\right)\\+\left({div}_{\vec x}\vec
\Gamma\right)\left(A^{-1}(t)\cdot A'(t)\cdot\vec
x+A^{-1}(t)\cdot\vec w(t)\right)\\- curl_{\vec
x}\left(\left(A^{-1}(t)\cdot A'(t)\cdot\vec x+A^{-1}(t)\cdot\vec
w(t)\right)\times \vec\Gamma\right)\\=\left({div}_{\vec
x}\left(A^{-1}(t)\cdot A'(t)\cdot\vec x+A^{-1}(t)\cdot\vec
w(t)\right)\right)\vec \Gamma.
\end{multline}
Therefore, by \er{bgvfgfhhtgjtggguyfredbstr} and
\er{bgvfgfhhtgjtgjuiyuijjkredbstr} we deduce:
\begin{multline}\label{bgvfgfhhtgjtgredbstr}
\frac{\partial \left(A(t)\cdot\vec \Gamma\right)}{\partial
t'}-curl_{\vec x'}\left(\vec v'\times\left(A(t)\cdot\vec
\Gamma\right)\right)+\left({div}_{\vec x'}\left(A(t)\cdot\vec
\Gamma\right)\right)\vec v'=\\ A(t)\cdot\left(\frac{\partial \vec
\Gamma}{\partial t}- curl_{\vec x}\left(\vec v\times
\vec\Gamma\right)+\left({div}_{\vec x}\vec \Gamma\right)\vec
v\right)\\+A(t)\cdot\left(\left({div}_{\vec x}\left(A^{-1}(t)\cdot
A'(t)\cdot\vec x+A^{-1}(t)\cdot\vec w(t)\right)\right)\vec
\Gamma\right)
\\=A(t)\cdot\left(\frac{\partial \vec \Gamma}{\partial t}- curl_{\vec
x}\left(\vec v\times \vec\Gamma\right)+\left({div}_{\vec x}\vec
\Gamma\right)\vec v\right)+\left(tr\left(A^{-1}(t)\cdot
A'(t)\right)\right)A(t)\cdot\vec \Gamma,
\end{multline}
where $tr\left(A^{-1}(t)\cdot A'(t)\right)$ is the trace of the
matrix $A^{-1}(t)\cdot A'(t)$.
%(sum of diagonal elements).
%However, since $A^T(t)\cdot A(t)=I$ we have $A^{-1}(t)=A^T(t)$ and $A^{-1}(t)\cdot A'(t)=S(t)$, where $S^T(t)=-S(t)$.
%In particular $tr S(t)=0$ and thus $tr\,\left(A^{-1}(t)\cdot A'(t)\right)=0$.
Therefore, by \er{bgvfgfhhtgjtgredbstr} and
\er{NoIn5redbstrgfhgfhfhuhhjhjj} for every vector field $\vec
\Gamma:\R^3\times[0,+\infty)\to\R^3$ we have:
\begin{multline}\label{bgvfgfhhtgjtggiuguiuiredbstr}
\frac{\partial \left(A(t)\cdot\vec \Gamma\right)}{\partial
t'}-curl_{\vec x'}\left(\vec v'\times\left(A(t)\cdot\vec
\Gamma\right)\right)+\left({div}_{\vec x'}\left(A(t)\cdot\vec
\Gamma\right)\right)\vec v'\\=A(t)\cdot\left(\frac{\partial \vec
\Gamma}{\partial t}- curl_{\vec x}\left(\vec v\times
\vec\Gamma\right)+\left({div}_{\vec x}\vec \Gamma\right)\vec
v\right).
\end{multline}
Thus, by \er{bgvfgfhhtgjtggiuguiuiredbstr} and
\er{uguyytfddddgghjjg} we infer
\begin{equation}\label{bgvfgfhhtgjtggiuguiuiredbstrkgghkj}
\frac{\partial \vec f'}{\partial t'}-curl_{\vec x'}\left(\vec
v'\times\vec f'\right)+\left({div}_{\vec x'}\vec f'\right)\vec
v'=A(t)\cdot\left(\frac{\partial \vec f}{\partial t}- curl_{\vec
x}\left(\vec v\times \vec f\right)+\left({div}_{\vec x}\vec
f\right)\vec v\right).
\end{equation}
Finally, by \er{apfrm9}, \er{apfrm7} and \er{apfrm6} we deduce
\begin{multline}\label{vhfffngghhjghhgjlkhjhk}
\frac{\partial\vec f}{\partial t}-\vec v\times curl_{\vec x}\vec
f+\nabla_{\vec x}\left(\vec f\cdot\vec v\right)=\nabla_{\vec
x}\left(\vec f\cdot\vec v\right)+\frac{\partial\vec f}{\partial
t}+d_{\vec x}\vec f\cdot\vec v-\left\{d_{\vec x}\vec
f\right\}^T\cdot\vec v\\=\frac{\partial\vec f}{\partial t}+d_{\vec
x}\vec f\cdot\vec v+ \left\{d_{\vec x}\vec v\right\}^T\cdot\vec
f=\frac{\partial\vec f}{\partial t}+d_{\vec x}\vec f\cdot\vec
v-d_{\vec x}\vec v\cdot\vec f+\left(d_{\vec x}\vec v+\left\{d_{\vec
x}\vec v\right\}^T\right)\cdot\vec f\\= \left(\frac{\partial\vec
f}{\partial t}-curl_{\vec x}\left(\vec v\times\vec
f\right)+\left(div_{\vec x}\vec f\right)\vec
v\right)-\left(div_{\vec x}\vec v\right)\vec f+ \left(d_{\vec x}\vec
v+\left\{d_{\vec x}\vec v\right\}^T\right)\cdot\vec f.
\end{multline}
So we get \er{vhfffngghhjghhgjlkhjhkPPP}. Moreover, by
\er{uguyytfddddgghjjg}, \er{NoIn5redbstrgfhgfhfhgffgf},
\er{NoIn5redbstruiytrrhjjgghjkgkgh}, \er{vhfffngghhjghhgjlkhjhk} and
\er{bgvfgfhhtgjtggiuguiuiredbstrkgghkj} we infer
\begin{equation}\label{vhfffngghhjghhgjlkhjhkfghhffhhhhg}
\frac{\partial\vec f'}{\partial t'}-\vec v'\times curl_{\vec x'}\vec
f'+\nabla_{\vec x'}\left(\vec f'\cdot\vec
v'\right)=A(t)\cdot\left(\frac{\partial\vec f}{\partial t}-\vec
v\times curl_{\vec x}\vec f+\nabla_{\vec x}\left(\vec f\cdot\vec
v\right)\right).
\end{equation}
So by \er{bgvfgfhhtgjtggiuguiuiredbstrkgghkj} and
\er{vhfffngghhjghhgjlkhjhkfghhffhhhhg} we finally obtain {(\bf v)}.
\end{proof}














\section{Gravity revised}\label{gugyu} Consider the classical space-time
where the change of some inertial coordinate system $(*)$ to another
inertial coordinate system $(**)$ is given by the Galilean
Transformation:
\begin{equation}\label{noninchgravortbstrjgghguittu1}
\begin{cases}
\vec x'=\vec x+\vec wt,\\
t'=t,
\end{cases}
\end{equation}
and the change of some non-inertial cartesian coordinate system
$(*)$ to another cartesian coordinate system $(**)$ is of the form:
\begin{equation}\label{noninchgravortbstrjgghguittu2}
\begin{cases}
\vec x'=A(t)\cdot\vec x+\vec z(t),\\
t'=t,
\end{cases}
\end{equation}
where $A(t)\in SO(3)$ is a rotation, i.e. $A(t)\in \R^{3\times 3}$,
$det\, A(t)>0$ and $A(t)\cdot A^T(t)=I$, where $A^T$ is the
transpose of the matrix $A$.

 Similarly to the General Theory of Relativity, we assume that
the most general laws of Classical Mechanics should be invariant in
every non-inertial cartesian coordinate system, i.e. they preserve
their form under transformations of the form
\er{noninchgravortbstrjgghguittu2}. Moreover, again as in the
General Theory of Relativity, we assume that the fictitious forces
(inertial forces) in non-inertial coordinate systems and the forces
of Newtonian gravitation have the same nature and represented by
some field in somewhat similar to the Electromagnetic field.

 We begin with some simple observation. Assume that we are away of
essential gravitational masses and strong electromagnetic fields.
Then consider two cartesian coordinate systems $(*)$ and $(**)$,
such that the system $(**)$ is inertial and the change of coordinate
system $(*)$ to coordinate system $(**)$ is given by
\er{noninchgravortbstrjgghguittu2}. Then the
fictitious-gravitational force in the system $(**)$ is trivial $\vec
F'_0=0$. On the other hand, since under the change of coordinate
system of the form \er{noninchgravortbstrjgghguittu2} the velocity
transforms as
\begin{equation}\label{noninchgravortbstrjgghguittu2gjg}
\vec u'=A(t)\cdot\vec u+\frac{dA}{dt}(t)\cdot\vec x+\frac{d\vec
z}{dt}(t)
\end{equation}
and the acceleration
%under the change of coordinate system of the form \er{noninchgravortbstrjgghguittu2}
transforms as
\begin{equation}\label{noninchgravortbstrjgghguittu2gjgghhjhg}
\vec a'=A(t)\cdot\vec a+2\frac{dA}{dt}(t)\cdot\vec
u+\frac{d^2A}{dt^2}(t)\cdot\vec x+\frac{d^2\vec z}{dt^2}(t)
\end{equation}
the fictitious-gravitational force in the system $(*)$ acting on the
particle with inertial mass $m$ is given by
\begin{equation}\label{noninchgravortbstrjgghguittu2gjgghhjhghjhjgg}
\vec F_0=m\left(-2A^T(t)\cdot\frac{dA}{dt}(t)\cdot\vec
u-A^T(t)\cdot\frac{d^2 A}{dt^2}(t)\cdot\vec
x-A^T(t)\cdot\frac{d^2\vec z}{dt^2}(t)\right).
\end{equation}
On the other hand, since $A(t)\cdot A^T(t)=I$ and thus
$A^T(t)\cdot\frac{dA}{dt}(t)+\frac{dA^T}{dt}(t)\cdot A(t)=0$, if we
define a vector field
\begin{equation}\label{noninchgravortbstrjgghguittu2gjgjhjhhklk}
\vec v(\vec x,t):=-A^T(t)\cdot\frac{dA}{dt}(t)\cdot\vec
x-A^T(t)\cdot\frac{d\vec z}{dt}(t),
\end{equation}
then we obviously have
\begin{equation}\label{noninchgravortbstrjgghguittu2gjgjhjhhklkhgffgfg}
\begin{cases}
d_{\vec x}\vec
v=-A^T(t)\cdot\frac{dA}{dt}(t)=\frac{dA^T}{dt}(t)\cdot
A(t)
%=A^T(t)\cdot\frac{dA}{dt}(t)
\\
\left\{d_{\vec x}\vec v\right\}^T=-\frac{dA^T}{dt}(t)\cdot
A(t)=A^T(t)\cdot\frac{dA}{dt}(t)
\\
\frac{\partial\vec v}{\partial
t}=-A^T(t)\cdot\left(\frac{d^2A}{dt^2}(t)\cdot\vec x+\frac{d^2\vec
z}{dt^2}(t)\right)-\frac{dA^T}{dt}(t)\cdot\left(\frac{dA}{dt}(t)\cdot\vec
x+\frac{d\vec z}{dt}(t)\right)
\end{cases}
\end{equation}
Thus by \er{noninchgravortbstrjgghguittu2gjgjhjhhklk} and
\er{noninchgravortbstrjgghguittu2gjgjhjhhklkhgffgfg} we rewrite
\er{noninchgravortbstrjgghguittu2gjgghhjhghjhjgg} as
\begin{equation}\label{noninchgravortbstrjgghguittu2gjgghhjhghjhjgghgghghgh}
\vec F_0=m\left(-2A^T(t)\cdot\frac{dA}{dt}(t)\cdot\vec
u+\frac{\partial\vec v}{\partial t}-\frac{dA^T}{dt}(t)\cdot
A(t)\cdot \vec v\right).
\end{equation}
Then using \er{apfrm9} and
\er{noninchgravortbstrjgghguittu2gjgjhjhhklkhgffgfg} we finally
rewrite \er{noninchgravortbstrjgghguittu2gjgghhjhghjhjgghgghghgh} as
\begin{equation}\label{noninchgravortbstrjgghguittu2gjgghhjhghjhjgghgghghghtytyt}
\vec F_0=m\left(\frac{\partial\vec v}{\partial
t}+\frac{1}{2}\nabla_{\vec x}\left(|\vec v|^2\right)\right)+m\vec
u\times \left(-curl_{\vec x}\vec v\right).
\end{equation}





 Similarly assume that also in the general case of essential
gravitational masses there exists a vector field $\vec v(\vec x,t)$
such that in some inertial or non-inertial cartesian coordinate
system the fictitious-gravitational force is given by
\er{noninchgravortbstrjgghguittu2gjgghhjhghjhjgghgghghghtytyt}. Then
we call the vector field $\vec v$ the vectorial gravitational
potential. We see here the following analogy with Electrodynamics:
denoting
\begin{equation*}
\tilde {\vec E}:=\partial_{t}\vec v+\nabla_{\vec
x}\left(\frac{1}{2}|\vec v|^2\right)\quad\text{and}\quad \tilde
{\vec B}:=-c\, curl_{\vec x}\vec v,
\end{equation*}
we rewrite
\er{noninchgravortbstrjgghguittu2gjgghhjhghjhjgghgghghghtytyt} as
\begin{equation}\label{MaxVacFull1ninshtrgravortjhhjfhfhNewhjh}
%m\frac{d\vec u}{dt}
\vec F_0=m\left(\tilde {\vec E}+\frac{1}{c}\vec u\times\tilde {\vec
B}\right),
%+\vec F,
%\frac{d\vec u}{dt}=-curl_{\vec x}\vec v\times(\vec u-\vec v)+\partial_t\vec v+(\nabla_{\vec x}\vec v)\cdot\vec v+\nabla_{\vec x}\psi_0+\frac{1}{m}\vec F.
\end{equation}
where
\begin{equation*}
curl_{\vec x}\tilde {\vec E}+\frac{1}{c}\frac{\partial}{\partial
t}\tilde {\vec B}=0\quad\text{and}\quad div_{\vec x}\tilde {\vec
B}=0.
\end{equation*}






Next using
\er{noninchgravortbstrjgghguittu2gjgghhjhghjhjgghgghghghtytyt} we
rewrite the Second Law of Newton as
\begin{equation}\label{noninchgravortbstrjgghguittu2gjgghhjhghjhjgghgghghghtytythvfghfgghjgg}
m\frac{d^2\vec x}{dt^2}=m\frac{d\vec u}{dt}=\vec F_0+\vec
F=m\left(\frac{\partial\vec v}{\partial t}(\vec
x,t)+\frac{1}{2}\nabla_{\vec x}\left(\left|\vec
v\right|^2\right)(\vec x,t)\right)+m\vec u\times \left(-curl_{\vec
x}\vec v(\vec x,t)\right)+\vec F,
\end{equation}
where $\vec x:=\vec x(t)$, $\vec u:=\vec u(t)=\frac{d\vec x}{dt}(t)$
and $m$ are the place, the velocity and the inertial mass of some
given particle at the moment of time $t$, $\vec v:=\vec v(\vec x,t)$
is the vectorial gravitational potential and $\vec F$ is the total
non-gravitational force, acting on the given particle.
%APPNEW

















Once we considered the Second Law of Newton in the form
\begin{equation}\label{MaxVacFull1ninshtrgravorthjhjlhhjPPN}
\frac{d\vec u}{dt}=-\vec u\times curl_{\vec x}\vec
v+\partial_{t}\vec v+\nabla_{\vec x}\left(\frac{1}{2}|\vec
v|^2\right)+\frac{1}{m}\vec F,
%\frac{d\vec u}{dt}=-curl_{\vec x}\vec v\times(\vec u-\vec v)+\partial_t\vec v+(\nabla_{\vec x}\vec v)\cdot\vec v+\nabla_{\vec x}\psi_0+\frac{1}{m}\vec F.
\end{equation}
we still need to prove that this law is invariant under the change
of inertial or non-inertial cartesian coordinate system and to
determine the law of transformation for the vectorial-gravitational
potential under the change of coordinate systems. As we will show
above this is indeed the case and moreover, the law of
transformation of the vectorial gravitational potential, under the
change of coordinate system, given by
\er{noninchgravortbstrjgghguittu2}, is:
$$\vec v'=A(t)\cdot \vec
v+\frac{dA}{dt}(t)\cdot\vec x+\frac{d\vec z}{dt}(t)$$ i.e. it is the
same as the transformation of a field of velocities. More precisely
we have the following:
\begin{proposition}\label{gjghghgghg}
Consider the change of some non-inertial cartesian coordinate system
$(*)$ to another cartesian coordinate system $(**)$ of the form:
\begin{equation}\label{noninchgravortPPNnnn}
\begin{cases}
\vec x'=A(t)\cdot\vec x+\vec z(t),\\
t'=t,
\end{cases}
\end{equation}
where $A(t)\in SO(3)$ is a rotation, i.e. $A(t)\in \R^{3\times 3}$,
$det\, A(t)>0$ and $A(t)\cdot A^T(t)=I$. Next, assume that in the
coordinate system $(**)$ we observe a validity of the Second Law of
Newton in the form:
\begin{equation}\label{MaxVacFull1ninshtrgravortPPN}
\frac{d\vec u'}{dt'}=-\vec u'\times curl_{\vec x'}\vec
v'+\partial_{t'}\vec v'+\nabla_{\vec x'}\left(\frac{1}{2}|\vec
v'|^2\right)+\frac{1}{m'}\vec F',
%\frac{d\vec u}{dt}=-curl_{\vec x}\vec v\times(\vec u-\vec v)+\partial_t\vec v+(\nabla_{\vec x}\vec v)\cdot\vec v+\nabla_{\vec x}\psi_0+\frac{1}{m}\vec F.
\end{equation}
where $\vec x':=\vec x'(t')$, $\vec u':=\vec u'(t')=\frac{d\vec
x'}{dt'}(t')$ and $m'$ are the place, the velocity and the mass of
some given particle at the moment of time $t'$, $\vec v':=\vec
v'(\vec x',t')$ is the vectorial gravitational potential and $\vec
F'$ is a total non-gravitational force, acting on the given particle
in the coordinate system $(**)$. Then in the coordinate system $(*)$
we observe a validity of the Second Law of Newton in the (same as
\er{MaxVacFull1ninshtrgravortPPN}) form:
\begin{equation}\label{MaxVacFull1ninshtrgravortjhhjPPNjffjf}
\frac{d\vec u}{dt}=-\vec u\times curl_{\vec x}\vec
v+\partial_{t}\vec v+\nabla_{\vec x}\left(\frac{1}{2}|\vec
v|^2\right)+\frac{1}{m}\vec F,
%\frac{d\vec u}{dt}=-curl_{\vec x}\vec v\times(\vec u-\vec v)+\partial_t\vec v+(\nabla_{\vec x}\vec v)\cdot\vec v+\nabla_{\vec x}\psi_0+\frac{1}{m}\vec F.
\end{equation}
where
\begin{align}
\label{NoIn5gravortPPN11}\vec v'=A(t)\cdot \vec
v+\frac{dA}{dt}(t)\cdot\vec x+\frac{d\vec z}{dt}(t)\\
\label{NoIn1gravortPPN11}\vec F'=A(t)\cdot\vec F,\\
\label{NoIn2gravortPPN11}m'=m,\\
\label{NoIn3gravortPPN11}\vec u'=A(t)\cdot \vec
u+\frac{dA}{dt}(t)\cdot\vec x+\frac{d\vec z}{dt}(t).
\end{align}
\end{proposition}
\begin{proof}
Using \er{apfrm9} we rewrite \er{MaxVacFull1ninshtrgravortPPN} as
\begin{equation}\label{MaxVacFull1ninshtrgravortghhghgPPN}
\frac{d\vec u'}{dt'}=-(\vec u'-\vec v')\times curl_{\vec x'}\vec
v'+\partial_{t'}\vec v'+d_{\vec x'}\vec v'\cdot\vec
v'+\frac{1}{m'}\vec F'.
%\frac{d\vec u}{dt}=-curl_{\vec x}\vec v\times(\vec u-\vec v)+\partial_t\vec v+(\nabla_{\vec x}\vec v)\cdot\vec v+\nabla_{\vec x}\psi_0+\frac{1}{m}\vec F.
\end{equation}
Next define the vector field $\vec v$ in the system $(*)$ in such a
way that it will be related to $\vec v'$ in the system $(**)$ due to
\er{NoIn5gravortPPN11}. I.e. $\vec v$ is given by $$\vec
v:=A^T(t)\cdot\left(\vec v'-\frac{dA}{dt}(t)\cdot\vec x-\frac{d\vec
z}{dt}(t)\right).$$ We are going to prove
\er{MaxVacFull1ninshtrgravortjhhjPPNjffjf} in the system $(*)$ using
the following relations between the physical characteristics in
coordinate systems $(*)$ and $(**)$:
\begin{align}
\label{NoIn1gravortPPN}\vec F'=A(t)\cdot\vec F,\\
\label{NoIn2gravortPPN}m'=m,\\
\label{NoIn3gravortPPN}\vec u'=A(t)\cdot \vec u+A'(t)\cdot\vec x+\vec w(t),\\
\label{NoIn5gravortPPN}\vec v'=A(t)\cdot \vec v+A'(t)\cdot\vec
x+\vec w(t),
\end{align}
where $\vec w(t):=\frac{d\vec z}{dt}(t)$ and
$A'(t)=\frac{dA}{dt}(t)$. Indeed, inserting these relations into
\er{MaxVacFull1ninshtrgravortghhghgPPN} we obtain:
\begin{multline}\label{MaxVacFull1ninshtrgravortghhghgjkggPPN}
\frac{d}{dt}\left(A(t)\cdot \vec u(t)+A'(t)\cdot\vec x(t)+\vec
w(t)\right)=-\left(A(t)\cdot(\vec u-\vec v)\right)\times curl_{\vec
x'}\left(A(t)\cdot \vec v+A'(t)\cdot\vec x+\vec
w(t)\right)\\+\partial_{t'}\left(A(t)\cdot \vec v+A'(t)\cdot\vec
x+\vec w(t)\right)+d_{\vec x'}\left(A(t)\cdot \vec v+A'(t)\cdot\vec
x+\vec w(t)\right)\cdot\left(A(t)\cdot \vec v+A'(t)\cdot\vec x+\vec
w(t)\right)\\+\frac{1}{m}A(t)\cdot\vec F.
%\frac{d\vec u}{dt}=-curl_{\vec x}\vec v\times(\vec u-\vec v)+\partial_t\vec v+(\nabla_{\vec x}\vec v)\cdot\vec v+\nabla_{\vec x}\psi_0+\frac{1}{m}\vec F.
\end{multline}
Next using the chain rule we deduce:
\begin{multline}
\label{vyguiuiujggghjjgggjhgghjortjggghkhjPPN}
\partial_{t'}\left(A(t)\cdot
\vec v+A'(t)\cdot\vec x+\vec w(t)\right)+d_{\vec x'}\left(A(t)\cdot
\vec v+A'(t)\cdot\vec x+\vec w(t)\right)\cdot\left(A'(t)\cdot\vec
x+\vec w(t)\right)=\\
\partial_{t}\left(A(t)\cdot \vec v+A'(t)\cdot\vec x+\vec
w(t)\right).
\end{multline}
Inserting it into \er{MaxVacFull1ninshtrgravortghhghgjkggPPN} we
deduce
\begin{multline}\label{MaxVacFull1ninshtrgravortghhghgjkgghklhPPN}
\frac{d}{dt}\left(A(t)\cdot \vec u(t)+A'(t)\cdot\vec x(t)+\vec
w(t)\right)=-\left(A(t)\cdot(\vec u-\vec v)\right)\times curl_{\vec
x'}\left(A(t)\cdot \vec v+A'(t)\cdot\vec x+\vec
w(t)\right)\\+\partial_{t}\left(A(t)\cdot \vec v+A'(t)\cdot\vec
x+\vec w(t)\right)+d_{\vec x}\left(A(t)\cdot \vec v+A'(t)\cdot\vec
x+\vec w(t)\right)\cdot \vec v+\frac{1}{m}A(t)\cdot\vec F.
%\frac{d\vec u}{dt}=-curl_{\vec x}\vec v\times(\vec u-\vec v)+\partial_t\vec v+(\nabla_{\vec x}\vec v)\cdot\vec v+\nabla_{\vec x}\psi_0+\frac{1}{m}\vec F.
\end{multline}
On the other hand, by \er{noninchgravortPPNnnn} and by Proposition
\ref{yghgjtgyrtrt}
%from the Appendix
we clearly have
\begin{equation}
\label{vyguiuiujggghjjgggorthjjhPPN} curl_{\vec x'}\left(
\left(A(t)\cdot\vec v+A'(t)\cdot\vec x+\vec
w(t)\right)\right)=A(t)\cdot curl_{\vec x}\left(\vec
v+A^{-1}(t)\cdot A'(t)\cdot\vec x+A^{-1}(t)\cdot \vec w(t)\right).
\end{equation}
Inserting it into \er{MaxVacFull1ninshtrgravortghhghgjkgghklhPPN} we
deduce:
\begin{multline}\label{MaxVacFull1ninshtrgravortghhghgjkgghklhjgkghglPPN}
\frac{d}{dt}\left(A(t)\cdot \vec u(t)+A'(t)\cdot\vec x(t)+\vec
w(t)\right)=\\-\left(A(t)\cdot(\vec u-\vec
v)\right)\times\left(A(t)\cdot curl_{\vec x}\left(\vec
v+A^{-1}(t)\cdot A'(t)\cdot\vec x+A^{-1}(t)\cdot
w(t)\right)\right)\\+\partial_{t}\left(A(t)\cdot \vec
v+A'(t)\cdot\vec x+\vec w(t)\right)+d_{\vec x}\left(A(t)\cdot \vec
v+A'(t)\cdot\vec x+\vec w(t)\right)\cdot \vec
v+\frac{1}{m}A(t)\cdot\vec F.
%\frac{d\vec u}{dt}=-curl_{\vec x}\vec v\times(\vec u-\vec v)+\partial_t\vec v+(\nabla_{\vec x}\vec v)\cdot\vec v+\nabla_{\vec x}\psi_0+\frac{1}{m}\vec F.
\end{multline}
Thus by \er{MaxVacFull1ninshtrgravortghhghgjkgghklhjgkghglPPN} and
\er{apfrm90} we have:
\begin{multline}\label{MaxVacFull1ninshtrgravortghhghgjkgghklhjgkghgPPN}
\frac{d}{dt}\left(A(t)\cdot \vec u(t)+A'(t)\cdot\vec x(t)+\vec
w(t)\right)=\\-A(t)\cdot \left((\vec u-\vec v)\times curl_{\vec
x}\left(\vec v+A^{-1}(t)\cdot A'(t)\cdot\vec x+A^{-1}(t)\cdot
w(t)\right)\right)\\+\partial_{t}\left(A(t)\cdot \vec
v+A'(t)\cdot\vec x+\vec w(t)\right)+d_{\vec x}\left(A(t)\cdot \vec
v+A'(t)\cdot\vec x+\vec w(t)\right)\cdot \vec
v+\frac{1}{m}A(t)\cdot\vec F.
%\frac{d\vec u}{dt}=-curl_{\vec x}\vec v\times(\vec u-\vec v)+\partial_t\vec v+(\nabla_{\vec x}\vec v)\cdot\vec v+\nabla_{\vec x}\psi_0+\frac{1}{m}\vec F.
\end{multline}
On the other hand clearly we have
$$\frac{d}{dt}\left(A(t)\cdot \vec u(t)+A'(t)\cdot\vec x(t)+\vec
w(t)\right)=A(t)\cdot\frac{d\vec u}{dt}+2A'(t)\cdot\vec
u+A''(t)\cdot \vec x(t)+\frac{d\vec w}{dt}(t).$$ Inserting it into
\er{MaxVacFull1ninshtrgravortghhghgjkgghklhjgkghgPPN} we deduce:
\begin{multline}\label{MaxVacFull1ninshtrgravortghhghgjkgghklhjgkghghjjkjPPN}
A(t)\cdot\frac{d\vec u}{dt}+2A'(t)\cdot\vec u+A''(t)\cdot \vec
x(t)+\frac{d\vec w}{dt}(t)=\\-A(t)\cdot \left((\vec u-\vec v)\times
curl_{\vec x}\left(\vec v+A^{-1}(t)\cdot A'(t)\cdot\vec
x+A^{-1}(t)\cdot w(t)\right)\right)\\+\partial_{t}\left(A(t)\cdot
\vec v+A'(t)\cdot\vec x+\vec w(t)\right)+d_{\vec x}\left(A(t)\cdot
\vec v+A'(t)\cdot\vec x+\vec
w(t)\right)\cdot \vec v+\frac{1}{m}A(t)\cdot\vec F\\
=-A(t)\cdot \left((\vec u-\vec v)\times curl_{\vec x}\vec
v\right)-A(t)\cdot \left((\vec u-\vec v)\times curl_{\vec
x}\left(A^{-1}(t)\cdot A'(t)\cdot\vec
x\right)\right)\\+A(t)\cdot\partial_t\vec v+2A'(t)\cdot \vec
v+A''(t)\cdot \vec x+\frac{d\vec w}{dt}(t)+A(t)\cdot d_{\vec x}\vec
v\cdot \vec v+\frac{1}{m}A(t)\cdot\vec F.
%\frac{d\vec u}{dt}=-curl_{\vec x}\vec v\times(\vec u-\vec v)+\partial_t\vec v+(\nabla_{\vec x}\vec v)\cdot\vec v+\nabla_{\vec x}\psi_0+\frac{1}{m}\vec F.
\end{multline}
We rewrite
\er{MaxVacFull1ninshtrgravortghhghgjkgghklhjgkghghjjkjPPN} as:
\begin{multline}\label{MaxVacFull1ninshtrgravortghhghgjkgghklhjgkghghjjkjhjkkggjkhjkPPN}
\frac{d\vec u}{dt}=- (\vec u-\vec v)\times curl_{\vec
x}\left(A^{-1}(t)\cdot A'(t)\cdot\vec x\right)-2A^{-1}(t)\cdot
A'(t)\cdot (\vec u-\vec v)\\-(\vec u-\vec v)\times curl_{\vec x}\vec
v +\partial_t\vec v+d_{\vec x}\vec v\cdot \vec v+\frac{1}{m}\vec F.
%\frac{d\vec u}{dt}=-curl_{\vec x}\vec v\times(\vec u-\vec v)+\partial_t\vec v+(\nabla_{\vec x}\vec v)\cdot\vec v+\nabla_{\vec x}\psi_0+\frac{1}{m}\vec F.
\end{multline}
Thus by \er{apfrm9} and
\er{MaxVacFull1ninshtrgravortghhghgjkgghklhjgkghghjjkjhjkkggjkhjkPPN}
we deduce:
\begin{multline}\label{MaxVacFull1ninshtrgravortghhghgjkgghklhjgkghghjjkjhjkkggjkhjkhjjhhPPN}
\frac{d\vec u}{dt}=d_{\vec x}\left(A^{-1}(t)\cdot A'(t)\cdot\vec
x\right)\cdot(\vec u-\vec v)-\left\{d_{\vec x}\left(A^{-1}(t)\cdot
A'(t)\cdot\vec x\right)\right\}^T\cdot(\vec u-\vec
v)-2A^{-1}(t)\cdot A'(t)\cdot (\vec u-\vec v)\\-(\vec u-\vec
v)\times curl_{\vec x}\vec v +\partial_t\vec v+d_{\vec x}\vec v\cdot
\vec v+\frac{1}{m}\vec F\\=\left(A^{-1}(t)\cdot
A'(t)\right)\cdot(\vec u-\vec v)-\left\{A^{-1}(t)\cdot
A'(t)\right\}^T\cdot(\vec u-\vec v)-2A^{-1}(t)\cdot A'(t)\cdot (\vec
u-\vec v)\\-(\vec u-\vec v)\times curl_{\vec x}\vec v
+\partial_t\vec v+d_{\vec x}\vec v\cdot \vec v+\frac{1}{m}\vec F.
%\frac{d\vec u}{dt}=-curl_{\vec x}\vec v\times(\vec u-\vec v)+\partial_t\vec v+(\nabla_{\vec x}\vec v)\cdot\vec v+\nabla_{\vec x}\psi_0+\frac{1}{m}\vec F.
\end{multline}
On the other hand the matrix $A^{-1}(t)\cdot A'(t)$ is antisymmetric
and thus
\begin{equation*}\left\{A^{-1}(t)\cdot
A'(t)\right\}^T=-\left(A^{-1}(t)\cdot A'(t)\right).
\end{equation*}
Inserting it into
\er{MaxVacFull1ninshtrgravortghhghgjkgghklhjgkghghjjkjhjkkggjkhjkhjjhhPPN}
we deduce:
\begin{equation}\label{MaxVacFull1ninshtrgravortghhghgjkgghklhjgkghghjjkjhjkkggjkhjkhjjhhfPPN}
\frac{d\vec u}{dt}=-(\vec u-\vec v)\times curl_{\vec x}\vec v
+\partial_t\vec v+d_{\vec x}\vec v\cdot \vec v+\frac{1}{m}\vec F.
%\frac{d\vec u}{dt}=-curl_{\vec x}\vec v\times(\vec u-\vec v)+\partial_t\vec v+(\nabla_{\vec x}\vec v)\cdot\vec v+\nabla_{\vec x}\psi_0+\frac{1}{m}\vec F.
\end{equation}
Thus again by \er{apfrm9} we finally rewrite
\er{MaxVacFull1ninshtrgravortghhghgjkgghklhjgkghghjjkjhjkkggjkhjkhjjhhfPPN}
as:
\begin{equation}\label{MaxVacFull1ninshtrgravortjhhjPPN}
\frac{d\vec u}{dt}=-\vec u\times curl_{\vec x}\vec
v+\partial_{t}\vec v+\nabla_{\vec x}\left(\frac{1}{2}|\vec
v|^2\right)+\frac{1}{m}\vec F.
%\frac{d\vec u}{dt}=-curl_{\vec x}\vec v\times(\vec u-\vec v)+\partial_t\vec v+(\nabla_{\vec x}\vec v)\cdot\vec v+\nabla_{\vec x}\psi_0+\frac{1}{m}\vec F.
\end{equation}
Therefore in the coordinate system $(*)$ we observe a validity of
Second Law of Newton in the same form as
\er{MaxVacFull1ninshtrgravortPPN}.
\end{proof}
Next, in order to fit the Second Law of Newton in the form
\er{MaxVacFull1ninshtrgravorthjhjlhhjPPN} with the classical Second
Law of Newton and the Newtonian Law of Gravitation we consider that
in \underline{inertial} coordinate system $(*)$, at least in the
first approximation, we should have
\begin{equation}
\label{MaxVacFull1ninshtrgravortghhghgjkgghklhjgkghghjjkjhjkkggjkhjkhjjhhfhjhklkhkhjjklzzzyyyhjggjhgghhjhNWNWBWHWPPN222kgghjghjghj}
\begin{cases}
curl_{\vec x}\vec v= 0,\\
\frac{\partial\vec v}{\partial t}+\frac{1}{2}\nabla_{\vec
x}\left(|\vec v|^2\right)= -\nabla_{\vec x}\Phi,
%\frac{d\vec u}{dt}=-curl_{\vec x}\vec v\times(\vec u-\vec v)+\partial_t\vec v+(\nabla_{\vec x}\vec v)\cdot\vec v+\nabla_{\vec x}\psi_0+\frac{1}{m}\vec F.
\end{cases}
\end{equation}
where $\Phi$ is a scalar Newtonian gravitational potential which
satisfies
\begin{equation}
\label{MaxVacFull1ninshtrgravortghhghgjkgghklhjgkghghjjkjhjkkggjkhjkhjjhhfhjhklkhkhjjklzzzyyyhjggjhgghhjhNWNWNWBWHWPPN222}
\Delta_{\vec x}\Phi=4\pi GM,
\end{equation}
where $M$ is the gravitational mass density and $G$ is the
gravitational constant. Thus, since we require $curl_{\vec x}\vec v=
0$,
\er{MaxVacFull1ninshtrgravortghhghgjkgghklhjgkghghjjkjhjkkggjkhjkhjjhhfhjhklkhkhjjklzzzyyyhjggjhgghhjhNWNWBWHWPPN222kgghjghjghj}
is equivalent to:
\begin{equation}
\label{MaxVacFull1ninshtrgravortghhghgjkgghklhjgkghghjjkjhjkkggjkhjkhjjhhfhjhklkhkhjjklzzzyyyhjggjhgghhjhNWNWBWHWPPN222}
\begin{cases}
curl_{\vec x}\vec v= 0,\\
\frac{\partial\vec v}{\partial t}+d_\vec x\vec v\cdot\vec v=
-\nabla_{\vec x}\Phi,
%\frac{d\vec u}{dt}=-curl_{\vec x}\vec v\times(\vec u-\vec v)+\partial_t\vec v+(\nabla_{\vec x}\vec v)\cdot\vec v+\nabla_{\vec x}\psi_0+\frac{1}{m}\vec F.
\end{cases}
\end{equation}
Clearly the law
\er{MaxVacFull1ninshtrgravortghhghgjkgghklhjgkghghjjkjhjkkggjkhjkhjjhhfhjhklkhkhjjklzzzyyyhjggjhgghhjhNWNWBWHWPPN222}
is invariant under the change of inertial coordinate system given by
\er{noninchgravortbstrjgghguittu1}. Note also that, since in the
system $(*)$ we have $curl_{\vec x}\vec v=0$, we can write
\er{MaxVacFull1ninshtrgravortghhghgjkgghklhjgkghghjjkjhjkkggjkhjkhjjhhfhjhklkhkhjjklzzzyyyhjggjhgghhjhNWNWBWHWPPN222kgghjghjghj}
as the following Hamilton-Jacobi type equation:
\begin{equation}
\label{MaxVacFull1ninshtrgravortghhghgjkgghklhjgkghghjjkjhjkkggjkhjkhjjhhfhjhklkhkhjjklzzzyyyhjggjhgghhjhNWNWNWNWNWBWHWPPN222}
\begin{cases}
\vec v=\nabla_{\vec x}Z,\\
\frac{\partial Z}{\partial t}+\frac{1}{2}\left|\nabla_{\vec
x}Z\right|^2=-\Phi,
%\frac{d\vec u}{dt}=-curl_{\vec x}\vec v\times(\vec u-\vec v)+\partial_t\vec v+(\nabla_{\vec x}\vec v)\cdot\vec v+\nabla_{\vec x}\psi_0+\frac{1}{m}\vec F.
\end{cases}
\end{equation}
where $Z:=Z(\vec x,t)$ is some scalar field. We would like to derive
the law which is invariant in every non-inertial cartesian
coordinate system and is equivalent to
\er{MaxVacFull1ninshtrgravortghhghgjkgghklhjgkghghjjkjhjkkggjkhjkhjjhhfhjhklkhkhjjklzzzyyyhjggjhgghhjhNWNWBWHWPPN222}
in every inertial coordinate system. Note that
\er{MaxVacFull1ninshtrgravortghhghgjkgghklhjgkghghjjkjhjkkggjkhjkhjjhhfhjhklkhkhjjklzzzyyyhjggjhgghhjhNWNWBWHWPPN222}
and
\er{MaxVacFull1ninshtrgravortghhghgjkgghklhjgkghghjjkjhjkkggjkhjkhjjhhfhjhklkhkhjjklzzzyyyhjggjhgghhjhNWNWNWBWHWPPN222}
implies:
\begin{equation}
\label{MaxVacFull1ninshtrgravortghhghgjkgghklhjgkghghjjkjhjkkggjkhjkhjjhhfhjhklkhkhjjklzzzyyyhjfgfkjgjNWBWHWPPN222}
\begin{cases}
curl_{\vec x}\left(curl_{\vec x}\vec v\right)= 0,\\
div_{\vec x}\left\{\frac{\partial\vec v}{\partial t}+d_\vec x\vec
v\cdot\vec v+\frac{1}{2}\vec v\times curl_{\vec x}\vec v\right\}
%+\frac{1}{4}\left|curl_{\vec x}\vec v\right|^2
= -4\pi GM,
\end{cases}
\end{equation}
that we rewrite using \er{apfrm3} as:
\begin{equation}
\label{MaxVacFull1ninshtrgravortghhghgjkgghklhjgkghghjjkjhjkkggjkhjkhjjhhfhjhklkhkhjjklzzzyyyNWBWHWPPN222}
\begin{cases}
curl_{\vec x}\left(curl_{\vec x}\vec v\right)= 0,\\
\frac{\partial}{\partial t}\left(div_{\vec x}\vec v\right)+\vec
v\cdot\nabla_{\vec x}\left(div_{\vec x}\vec
v\right)+\frac{1}{4}\left|d_{\vec x}\vec v+\{d_{\vec x}\vec
v\}^T\right|^2= -4\pi GM,
%\frac{d\vec u}{dt}=-curl_{\vec x}\vec v\times(\vec u-\vec v)+\partial_t\vec v+(\nabla_{\vec x}\vec v)\cdot\vec v+\nabla_{\vec x}\psi_0+\frac{1}{m}\vec F.
\end{cases}
\end{equation}
or, equivalently, as:
\begin{equation}
\label{MaxVacFull1ninshtrgravortghhghgjkgghklhjgkghghjjkjhjkkggjkhjkhjjhhfhjhklkhkhjjklzzzyyyNWBWHWPPN222ll}
\begin{cases}
curl_{\vec x}\left(curl_{\vec x}\vec v\right)= 0,\\
\frac{\partial}{\partial t}\left(div_{\vec x}\vec v\right)+div_{\vec
x}\left\{\left(div_{\vec x}\vec v\right)\vec
v\right\}+\frac{1}{4}\left|d_{\vec x}\vec v+\{d_{\vec x}\vec
v\}^T\right|^2-\left(div_{\vec x}\vec v\right)^2= -4\pi GM.
%\frac{d\vec u}{dt}=-curl_{\vec x}\vec v\times(\vec u-\vec v)+\partial_t\vec v+(\nabla_{\vec x}\vec v)\cdot\vec v+\nabla_{\vec x}\psi_0+\frac{1}{m}\vec F.
\end{cases}
\end{equation}
% and $k\in\mathbb{R}$ is a some constant parameter.
Next observe that using Proposition \ref{yghgjtgyrtrt}
%from the Appendix
we deduce that the laws in
\er{MaxVacFull1ninshtrgravortghhghgjkgghklhjgkghghjjkjhjkkggjkhjkhjjhhfhjhklkhkhjjklzzzyyyNWBWHWPPN222}
and
\er{MaxVacFull1ninshtrgravortghhghgjkgghklhjgkghghjjkjhjkkggjkhjkhjjhhfhjhklkhkhjjklzzzyyyNWBWHWPPN222ll}
are invariant under the change of non-inertial cartesian coordinate
system, given by \er{noninchgravortbstrjgghguittu2}. So, we can
consider
\er{MaxVacFull1ninshtrgravortghhghgjkgghklhjgkghghjjkjhjkkggjkhjkhjjhhfhjhklkhkhjjklzzzyyyNWBWHWPPN222ll}
together with the requirement that $|\vec v|=O(|\vec x|)$ and
$\left|d_{\vec x}\vec v+\{d_{\vec x}\vec v\}^T\right|=o(1)$ as $\vec
x\to\infty$ instead of
\er{MaxVacFull1ninshtrgravortghhghgjkgghklhjgkghghjjkjhjkkggjkhjkhjjhhfhjhklkhkhjjklzzzyyyhjggjhgghhjhNWNWBWHWPPN222}.
Indeed, as we saw
\er{MaxVacFull1ninshtrgravortghhghgjkgghklhjgkghghjjkjhjkkggjkhjkhjjhhfhjhklkhkhjjklzzzyyyhjggjhgghhjhNWNWBWHWPPN222}
implies
\er{MaxVacFull1ninshtrgravortghhghgjkgghklhjgkghghjjkjhjkkggjkhjkhjjhhfhjhklkhkhjjklzzzyyyNWBWHWPPN222ll}.
On the other hand, using
\er{MaxVacFull1ninshtrgravortghhghgjkgghklhjgkghghjjkjhjkkggjkhjkhjjhhfhjhklkhkhjjklzzzyyyNWBWHWPPN222ll}
and the fact that $|\vec v|=O(|\vec x|)$ as $\vec x\to\infty$ we
deduce that there exist cartesian coordinate systems, that we call
non-rotating coordinate systems, such that in these systems we have:
\begin{equation}
\label{MaxVacFull1ninshtrgravortghhghgjkgghklhjgkghghjjkjhjkkggjkhjkhjjhhfhjhklkhkhjjklzzzyyyhjggjhgghhjhNWBWHWPPN222}
\begin{cases}
curl_{\vec x}\vec v= 0,\\
div_{\vec x}\left\{\frac{\partial\vec v}{\partial t}+d_\vec x\vec
v\cdot\vec v\right\}= -4\pi GM\\
curl_{\vec x}\left\{\frac{\partial\vec v}{\partial t}+d_\vec x\vec
v\cdot\vec v\right\}=0.
%\frac{d\vec u}{dt}=-curl_{\vec x}\vec v\times(\vec u-\vec v)+\partial_t\vec v+(\nabla_{\vec x}\vec v)\cdot\vec v+\nabla_{\vec x}\psi_0+\frac{1}{m}\vec F.
\end{cases}
\end{equation}
Furthermore, there exists a non-rotating system where $\vec v\to 0$
as $\vec x\to\infty$. Then in this system
\er{MaxVacFull1ninshtrgravortghhghgjkgghklhjgkghghjjkjhjkkggjkhjkhjjhhfhjhklkhkhjjklzzzyyyhjggjhgghhjhNWBWHWPPN222}
implies
\er{MaxVacFull1ninshtrgravortghhghgjkgghklhjgkghghjjkjhjkkggjkhjkhjjhhfhjhklkhkhjjklzzzyyyhjggjhgghhjhNWNWBWHWPPN222}.
We call the systems where
\er{MaxVacFull1ninshtrgravortghhghgjkgghklhjgkghghjjkjhjkkggjkhjkhjjhhfhjhklkhkhjjklzzzyyyhjggjhgghhjhNWNWBWHWPPN222}
is valid inertial coordinate systems. It is clear that a coordinate
system $(**)$ that we can get from some inertial coordinate system
$(*)$ by the Galilean Transformations also will be inertial.

As a consequence of all mentioned above, the second law of Newton
invariant under the change of non-inertial cartesian coordinate
system is:
\begin{equation}\label{noninchgravortbstrjgghguittu2gjgghhjhghjhjgghgghghghtytythvfghfgghjggghjgjh}
m\frac{d^2\vec x}{dt^2}=m\frac{d\vec
u}{dt}=m\left(\frac{\partial\vec v}{\partial t}(\vec
x,t)+\frac{1}{2}\nabla_{\vec x}\left(\left|\vec
v\right|^2\right)(\vec x,t)\right)-m\vec u\times curl_{\vec x}\vec
v(\vec x,t)+\vec F,
\end{equation}
and the first approximation of the law of gravitation, invariant
under the change of non-inertial cartesian coordinate system is:
%where $\vec v$ satisfies:
\begin{equation}
\label{MaxVacFull1ninshtrgravortghhghgjkgghklhjgkghghjjkjhjkkggjkhjkhjjhhfhjhklkhkhjjklzzzyyyNWBWHWPPN222jkgghgg}
\begin{cases}
curl_{\vec x}\left(curl_{\vec x}\vec v\right)= 0,\\
\frac{\partial}{\partial t}\left(div_{\vec x}\vec v\right)+div_{\vec
x}\left\{\left(div_{\vec x}\vec v\right)\vec
v\right\}+\frac{1}{4}\left|d_{\vec x}\vec v+\{d_{\vec x}\vec
v\}^T\right|^2-\left(div_{\vec x}\vec v\right)^2= -4\pi GM.
%\frac{d\vec u}{dt}=-curl_{\vec x}\vec v\times(\vec u-\vec v)+\partial_t\vec v+(\nabla_{\vec x}\vec v)\cdot\vec v+\nabla_{\vec x}\psi_0+\frac{1}{m}\vec F.
\end{cases}
\end{equation}
%
%
%
\begin{comment}
\begin{equation}
\label{MaxVacFull1ninshtrgravortghhghgjkgghklhjgkghghjjkjhjkkggjkhjkhjjhhfhjhklkhkhjjklzzzyyyNWBWHWPPN222ll3}
\begin{cases}
curl_{\vec x}\left(curl_{\vec x}\vec v\right)= 0,\\
\frac{\partial}{\partial t}\left(div_{\vec x}\vec v\right)+\vec
v\cdot\nabla_{\vec x}\left(div_{\vec x}\vec
v\right)+\frac{1}{4}\left|d_{\vec x}\vec v+\{d_{\vec x}\vec
v\}^T\right|^2= -4\pi GM.
%\frac{d\vec u}{dt}=-curl_{\vec x}\vec v\times(\vec u-\vec v)+\partial_t\vec v+(\nabla_{\vec x}\vec v)\cdot\vec v+\nabla_{\vec x}\psi_0+\frac{1}{m}\vec F.
\end{cases}
\end{equation}
\end{comment}
%
%
%
Here $\vec x:=\vec x(t)$, $\vec u:=\vec u(t)=\frac{d\vec x}{dt}(t)$
and $m$ are the place, the velocity and the inertial mass of some
given particle at the moment of time $t$, $\vec v:=\vec v(\vec x,t)$
is the vectorial gravitational potential, $M$ is the volume density
of gravitational masses and $\vec F$ is the total non-gravitational
force, acting on the given particle. Moreover, the vectorial
gravitational potential $\vec v$ is a speed-like vector field, i.e.
under the changes of inertial or non-inertial cartesian coordinate
system it behaves like a field of velocities of some continuum. Thus
we could introduce the fictitious continuum medium covering all the
space, that we can call Aether, such that $\vec v(\vec x,t)$ is a
fictitious velocity of this medium in the point $\vec x$ at the time
$t$.
%Here $G$ is the gravitational constant.
%SSD

\begin{remark}\label{ugyugg}
The quantity $Z(\vec x,t)$ in
\er{MaxVacFull1ninshtrgravortghhghgjkgghklhjgkghghjjkjhjkkggjkhjkhjjhhfhjhklkhkhjjklzzzyyyhjggjhgghhjhNWNWNWNWNWBWHWPPN222}
is well defined in every non-rotating cartesian coordinate system
and in particular it is well defined in inertial coordinate systems.
It can be easily checked by straightforward calculations that, if
under the change of coordinate system $\bf{(*)}$ to $\bf{(**)}$
given by the Galilean Transformation
\begin{equation}\label{noninchgravortbstrjgghguittu1intmm}
\begin{cases}
\vec x'=\vec x+\vec wt,\\
t'=t,
\end{cases}
\end{equation}
the quantity $Z$ transforms as:
\begin{equation}\label{noninchgravortbstrjgghguittu1intmmjhhj}
Z'(\vec x',t')=Z(\vec x,t)+\vec w\cdot\vec x+\frac{1}{2}|\vec w|^2t,
\end{equation}
then equalities
\begin{equation}
\label{MaxVacFull1ninshtrgravortghhghgjkgghklhjgkghghjjkjhjkkggjkhjkhjjhhfhjhklkhkhjjklzzzyyyhjggjhgghhjhNWNWNWNWNWBWHWPPN222kkk}
\begin{cases}
\vec v+\vec w=\vec v'=\nabla_{\vec x'}Z',\\
\frac{\partial Z'}{\partial t'}+\frac{1}{2}\left|\nabla_{\vec
x'}Z'\right|^2=-\Phi',
%\frac{d\vec u}{dt}=-curl_{\vec x}\vec v\times(\vec u-\vec v)+\partial_t\vec v+(\nabla_{\vec x}\vec v)\cdot\vec v+\nabla_{\vec x}\psi_0+\frac{1}{m}\vec F.
\end{cases}
\end{equation}
in coordinate system $\bf{(**)}$ imply the similar equalities
\begin{equation}
\label{MaxVacFull1ninshtrgravortghhghgjkgghklhjgkghghjjkjhjkkggjkhjkhjjhhfhjhklkhkhjjklzzzyyyhjggjhgghhjhNWNWNWNWNWBWHWPPN222kkkkjj}
\begin{cases}
\vec v=\nabla_{\vec x}Z,\\
\frac{\partial Z}{\partial t}+\frac{1}{2}\left|\nabla_{\vec
x}Z\right|^2=-\Phi,
%\frac{d\vec u}{dt}=-curl_{\vec x}\vec v\times(\vec u-\vec v)+\partial_t\vec v+(\nabla_{\vec x}\vec v)\cdot\vec v+\nabla_{\vec x}\psi_0+\frac{1}{m}\vec F.
\end{cases}
\end{equation}
in coordinate system $\bf{(*)}$, provided that $\Phi'=\Phi$.
\end{remark}
\begin{remark}\label{hhghjggh}
Assume that in some inertial or non-inertial cartesian coordinate
system some particle with the place $\vec r(t)$ and velocity $\vec
u(t)=\frac{d\vec r}{dt}(t)$ moves in the gravitational field, and
all other forces, acting on the particle, except of the
gravitational forces are negligible. Then since, as before, by
\er{MaxVacFull1ninshtrgravorthjhjlhhjPPN} with $\vec F=0$ we have
\begin{multline}\label{MaxVacFull1ninshtrgravorthjhjlhhjPPNhggh}
\frac{d\vec u}{dt}(t)=-\left(\vec u(t)-\vec v\left(\vec
r(t),t\right)\right)\times curl_{\vec x}\vec v\left(\vec
r(t),t\right)+\partial_{t}\vec v\left(\vec r(t),t\right)+d_{\vec
x}\vec v\left(\vec r(t),t\right)\cdot\vec v\left(\vec
r(t),t\right)=\\ \partial_{t}\vec v\left(\vec r(t),t\right)+d_{\vec
x}\vec v\left(\vec r(t),t\right)\cdot\frac{d\vec r}{dt}(t) -d_{\vec
x}\vec v\left(\vec r(t),t\right)\cdot\left(\vec u(t)-\vec
v\left(\vec r(t),t\right)\right)-\left(\vec u(t)-\vec v\left(\vec
r(t),t\right)\right)\times curl_{\vec x}\vec v\left(\vec
r(t),t\right)\\= \frac{d}{dt}\left\{\vec v\left(\vec
r(t),t\right)\right\}-\left\{d_{\vec x}\vec v\left(\vec
r(t),t\right)\right\}^T\cdot\left(\vec u(t)-\vec v\left(\vec
r(t),t\right)\right),
%\frac{d\vec u}{dt}=-curl_{\vec x}\vec v\times(\vec u-\vec v)+\partial_t\vec v+(\nabla_{\vec x}\vec v)\cdot\vec v+\nabla_{\vec x}\psi_0+\frac{1}{m}\vec F.
\end{multline}
we deduce that the vectorial quantity $\left(\frac{d\vec
r}{dt}(t)-\vec v\left(\vec r(t),t\right)\right)=\left(\vec u(t)-\vec
v\left(\vec r(t),t\right)\right)$ satisfies the following first
order homogenous vector linear ordinary differential equation:
\begin{equation}\label{MaxVacFull1ninshtrgravorthjhjlhhjPPNhgghjjjjj}
\frac{d}{dt}\left\{u(t)-\vec v\left(\vec r(t),t\right)\right\}+
\left\{d_{\vec x}\vec v\left(\vec
r(t),t\right)\right\}^T\cdot\left\{\vec u(t)-\vec v\left(\vec
r(t),t\right)\right\}=0.
%\frac{d\vec u}{dt}=-curl_{\vec x}\vec v\times(\vec u-\vec v)+\partial_t\vec v+(\nabla_{\vec x}\vec v)\cdot\vec v+\nabla_{\vec x}\psi_0+\frac{1}{m}\vec F.
\end{equation}
In particular if for some instant of time $t_0$ we have
\begin{equation}\label{hjfgjffgffg}
\vec u(t_0):=\frac{d\vec r}{dt}(t_0)=\vec v\left(\vec
r(t_0),t_0\right)
\end{equation}
then by uniqueness theorem for ordinary differential equations,
\er{MaxVacFull1ninshtrgravorthjhjlhhjPPNhgghjjjjj} and
\er{hjfgjffgffg} together imply
\begin{equation}\label{hjfgjffgffgugh}
\vec u(t):=\frac{d\vec r}{dt}(t)=\vec v\left(\vec
r(t),t\right)\quad\quad\forall t,
\end{equation}
for every instant of time. I.e. if the velocity of the particle for
some initial instant of time coincides with the local vectorial
gravitational potential, then it will coincide with it at any
instant of time and the trajectory of motion will be tangent to the
direction of the local vectorial gravitational potential.
\end{remark}
\begin{remark}\label{gygygygyggjh}
One can wonder: what should be possible values of the vectorial
gravitational potential $\vec v$ in the proximity of the Earth or
another massive body? In order to try to answer this question
consider two cartesian coordinate systems: \underline{non-rotating}
system $\bf(*)$ with the center that coincides with the center of
masses of the Earth and \underline{inertial} system $\bf(**)$
related to some external cosmic bodies. Assume that the center of
masses of the Earth has place $\vec R(t')$ and velocity $\vec
W(t'):=\frac{d\vec R}{dt'}(t')$ in the coordinate system $\bf(**)$.
Thus the change of coordinate system $\bf(*)$ to coordinate system
$\bf(**)$ is given by
\begin{equation}\label{hjjgghghghyu}
\begin{cases}
\vec x'=\vec x+\vec R(t),\\
t'=t,
\end{cases}
\end{equation}
and the vectorial gravitational potential $\vec v$, being a speed
like vector field, transforms as
\begin{equation}\label{hjjgghghghyulll}
\vec v'=\vec v+\vec W(t).
\end{equation}
Next, since the system $\bf(**)$ is inertial, consistently with
\er{MaxVacFull1ninshtrgravortghhghgjkgghklhjgkghghjjkjhjkkggjkhjkhjjhhfhjhklkhkhjjklzzzyyyhjggjhgghhjhNWNWBWHWPPN222}
and
\er{MaxVacFull1ninshtrgravortghhghgjkgghklhjgkghghjjkjhjkkggjkhjkhjjhhfhjhklkhkhjjklzzzyyyhjggjhgghhjhNWNWNWBWHWPPN222}
we have
\begin{equation}
\label{MaxVacFull1ninshtrgravortghhghgjkgghklhjgkghghjjkjhjkkggjkhjkhjjhhfhjhklkhkhjjklzzzyyyhjggjhgghhjhNWNWBWHWPPN222kkk}
\begin{cases}
curl_{\vec x'}\vec v'= 0,\\
\frac{\partial\vec v'}{\partial t'}+d_{\vec x'}\vec v'\cdot\vec v'=
-\nabla_{\vec x'}\Phi'_1-\nabla_{\vec x'}\Phi'_2,
%\frac{d\vec u}{dt}=-curl_{\vec x}\vec v\times(\vec u-\vec v)+\partial_t\vec v+(\nabla_{\vec x}\vec v)\cdot\vec v+\nabla_{\vec x}\psi_0+\frac{1}{m}\vec F.
\end{cases}
\end{equation}
with
\begin{equation}
\label{MaxVacFull1ninshtrgravortghhghgjkgghklhjgkghghjjkjhjkkggjkhjkhjjhhfhjhklkhkhjjklzzzyyyhjggjhgghhjhNWNWNWBWHWPPN222kkk}
\Delta_{\vec x'}\Phi'_1=4\pi GM'_1\quad\text{and}\quad\Delta_{\vec
x'}\Phi'_2=4\pi GM'_2,
\end{equation}
where $M_1$ is the gravitational mass density of the Earth and $M_2$
is the gravitational mass density of all other external cosmic
bodies like sun et.al. Moreover, again since the system $\bf(**)$ is
inertial, we clearly have:
\begin{equation}
\label{MaxVacFull1ninshtrgravortghhghgjkgghklhjgkghghjjkjhjkkggjkhjkhjjhhfhjhklkhkhjjklzzzyyyhjggjhgghhjhNWNWBWHWPPN222kkkjhjhjjhjhjhjhhjhj}
\frac{d\vec W}{\partial t}(t)=\frac{d\vec W}{\partial
t'}(t')=-\nabla_{\vec x'}\Phi'_2\left(\vec R(t),t\right).
\end{equation}
On the other hand inserting \er{hjjgghghghyu} and
\er{hjjgghghghyulll} into
\er{MaxVacFull1ninshtrgravortghhghgjkgghklhjgkghghjjkjhjkkggjkhjkhjjhhfhjhklkhkhjjklzzzyyyhjggjhgghhjhNWNWBWHWPPN222kkk}
and using Proposition \ref{yghgjtgyrtrt}
%from the Appendix
we deduce
\begin{equation}
\label{MaxVacFull1ninshtrgravortghhghgjkgghklhjgkghghjjkjhjkkggjkhjkhjjhhfhjhklkhkhjjklzzzyyyhjggjhgghhjhNWNWBWHWPPN222kkkgtytghjjh}
curl_{\vec x}\vec v=0,
\end{equation}
and
\begin{equation}
\label{MaxVacFull1ninshtrgravortghhghgjkgghklhjgkghghjjkjhjkkggjkhjkhjjhhfhjhklkhkhjjklzzzyyyhjggjhgghhjhNWNWBWHWPPN222kkkgtytghjjhghhg}
\frac{d\vec W}{\partial t}(t)+\frac{\partial\vec v}{\partial
t}+d_{\vec x}\vec v\cdot\vec v=\frac{d\vec W}{\partial
t'}(t')+\frac{\partial\vec v}{\partial t'}+d_{\vec x'}\vec
v\cdot\vec W(t')+d_{\vec x'}\vec v\cdot\vec v= -\nabla_{\vec
x'}\Phi'_1-\nabla_{\vec x'}\Phi'_2= -\nabla_{\vec
x}\Phi_1-\nabla_{\vec x'}\Phi'_2,
%\frac{d\vec u}{dt}=-curl_{\vec x}\vec v\times(\vec u-\vec v)+\partial_t\vec v+(\nabla_{\vec x}\vec v)\cdot\vec v+\nabla_{\vec x}\psi_0+\frac{1}{m}\vec F.
\end{equation}
On the other hand the quantity $\nabla_{\vec x'}\Phi'_2=\nabla_{\vec
x}\Phi_2$, being part of the gravitational field from the far
bodies, varies insignificantly in the space variables in the scale
compatible to the Earth size. Therefore, by
\er{MaxVacFull1ninshtrgravortghhghgjkgghklhjgkghghjjkjhjkkggjkhjkhjjhhfhjhklkhkhjjklzzzyyyhjggjhgghhjhNWNWBWHWPPN222kkkgtytghjjh},
\er{MaxVacFull1ninshtrgravortghhghgjkgghklhjgkghghjjkjhjkkggjkhjkhjjhhfhjhklkhkhjjklzzzyyyhjggjhgghhjhNWNWBWHWPPN222kkkgtytghjjhghhg}
and
\er{MaxVacFull1ninshtrgravortghhghgjkgghklhjgkghghjjkjhjkkggjkhjkhjjhhfhjhklkhkhjjklzzzyyyhjggjhgghhjhNWNWBWHWPPN222kkkjhjhjjhjhjhjhhjhj}
we finally deduce
\begin{equation}
\label{MaxVacFull1ninshtrgravortghhghgjkgghklhjgkghghjjkjhjkkggjkhjkhjjhhfhjhklkhkhjjklzzzyyyhjggjhgghhjhNWNWBWHWPPN222kkkgtytghjjhpop}
\begin{cases}
curl_{\vec x}\vec v=0,\\
\frac{\partial\vec v}{\partial t}+\frac{1}{2}\nabla_{\vec
x}\left(|\vec v|^2\right)=\frac{\partial\vec v}{\partial t}+d_{\vec
x}\vec v\cdot\vec v\approx -\nabla_{\vec x}\Phi_1,
%\frac{d\vec u}{dt}=-curl_{\vec x}\vec v\times(\vec u-\vec v)+\partial_t\vec v+(\nabla_{\vec x}\vec v)\cdot\vec v+\nabla_{\vec x}\psi_0+\frac{1}{m}\vec F.
\end{cases}
\end{equation}
where
\begin{equation}
\label{MaxVacFull1ninshtrgravortghhghgjkgghklhjgkghghjjkjhjkkggjkhjkhjjhhfhjhklkhkhjjklzzzyyyhjggjhgghhjhNWNWNWBWHWPPN222kkkiklklk}
\Delta_{\vec x}\Phi_1=4\pi GM_1.
\end{equation}
Being in the system  $\bf(*)$ which is stationary with respect to
the center of the Earth we look for stationary (i.e. time
independent) solutions of
\er{MaxVacFull1ninshtrgravortghhghgjkgghklhjgkghghjjkjhjkkggjkhjkhjjhhfhjhklkhkhjjklzzzyyyhjggjhgghhjhNWNWBWHWPPN222kkkgtytghjjhpop}.
Thus
\er{MaxVacFull1ninshtrgravortghhghgjkgghklhjgkghghjjkjhjkkggjkhjkhjjhhfhjhklkhkhjjklzzzyyyhjggjhgghhjhNWNWBWHWPPN222kkkgtytghjjhpop}
implies:
\begin{equation}
\label{MaxVacFull1ninshtrgravortghhghgjkgghklhjgkghghjjkjhjkkggjkhjkhjjhhfhjhklkhkhjjklzzzyyyhjggjhgghhjhNWNWBWHWPPN222kkkgtytghjjhpoppkkk}
\begin{cases}
curl_{\vec x}\vec v(\vec x)=0,\\
\left|\vec v(\vec x)\right|^2= -2\Phi_1(\vec x).
%\frac{d\vec u}{dt}=-curl_{\vec x}\vec v\times(\vec u-\vec v)+\partial_t\vec v+(\nabla_{\vec x}\vec v)\cdot\vec v+\nabla_{\vec x}\psi_0+\frac{1}{m}\vec F.
\end{cases}
\end{equation}
On the other hand, the scalar field $\Phi_1$, being the Newtonian
potential of the Earth, is radial and outside of the Earth it is
known that $\Phi_1(\vec x)=-\frac{Gm_0}{|\vec x|}$, where $m_0$ is
the Earth mass. Thus, since there exists a scalar field $Z_0(\vec
x)$ such that $\vec v(\vec x)=\nabla_{\vec x}Z_0(\vec x)$ and since
of symmetry considerations $Z_0(\vec x)=Z_0(|\vec x|)$ should be
radial, by
\er{MaxVacFull1ninshtrgravortghhghgjkgghklhjgkghghjjkjhjkkggjkhjkhjjhhfhjhklkhkhjjklzzzyyyhjggjhgghhjhNWNWBWHWPPN222kkkgtytghjjhpoppkkk}
we obtain
\begin{equation}
\label{MaxVacFull1ninshtrgravortghhghgjkgghklhjgkghghjjkjhjkkggjkhjkhjjhhfhjhklkhkhjjklzzzyyyhjggjhgghhjhNWNWBWHWPPN222kkkgtytghjjhpoppkkkhhhk}
\left|\frac{dZ_0}{d(|\vec x|)}(|\vec x|)\right|= \sqrt{-2\Phi_1(\vec
x)},
\end{equation}
that implies either
\begin{equation}
\label{MaxVacFull1ninshtrgravortghhghgjkgghklhjgkghghjjkjhjkkggjkhjkhjjhhfhjhklkhkhjjklzzzyyyhjggjhgghhjhNWNWBWHWPPN222kkkgtytghjjhpoppkkkhhhkhjhj}
\vec v(\vec x)= \frac{\sqrt{-2\Phi_1(|\vec x|)}}{|\vec x|}\vec x,
\end{equation}
or
\begin{equation}
\label{MaxVacFull1ninshtrgravortghhghgjkgghklhjgkghghjjkjhjkkggjkhjkhjjhhfhjhklkhkhjjklzzzyyyhjggjhgghhjhNWNWBWHWPPN222kkkgtytghjjhpoppkkkhhhkhjhjiuu}
\vec v(\vec x)= -\frac{\sqrt{-2\Phi_1(|\vec x|)}}{|\vec x|}\vec x,
\end{equation}
In particular on the Earth surface we have:
\begin{equation}
\label{MaxVacFull1ninshtrgravortghhghgjkgghklhjgkghghjjkjhjkkggjkhjkhjjhhfhjhklkhkhjjklzzzyyyhjggjhgghhjhNWNWBWHWPPN222kkkgtytghjjhpoppkkkhhhkhjhjiuuokokok}
|\vec v|=\sqrt{\frac{2Gm_0}{r_0}},
\end{equation}
where $r_0$ is the Earth radius and $m_0$ is the Earth mass, i.e.
the absolute value of the vectorial gravitational potential on the
Earth surface approximately equals to the escape velocity and its
direction is normal to the Earth, either downward or upward.
\end{remark}









\section{Maxwell equations revised}\label{MaxRevsPPN} We would like
to make the laws of Electrodynamics in the vacuum to be invariant
under the Galilean transformations. For this purpose we refer to the
analogy with the Maxwell equations in a medium. It is well known
that
%, up to rescaling,
the classical Maxwell equations in a medium have the form of
\begin{equation}\label{MaxVacPPN}
\begin{cases}
curl_{\vec x} \vec H\equiv \frac{4\pi}{c}\vec j+\frac{1}{c}\frac{\partial \vec D}{\partial t}\quad\text{for}\;\;({\vec x},t)\in\R^3\times[0,+\infty),\\
div_{\vec x} \vec D\equiv 4\pi\rho\quad\quad\text{for}\;\;({\vec x},t)\in\R^3\times[0,+\infty),\\
curl_{\vec x} \vec E+\frac{1}{c}\frac{\partial \vec B}{\partial t}\equiv 0\quad\quad\text{for}\;\;({\vec x},t)\in\R^3\times[0,+\infty),\\
div_{\vec x} \vec B\equiv 0\quad\quad\text{for}\;\;({\vec
x},t)\in\R^3\times[0,+\infty).
\end{cases}
\end{equation}
Here $\vec E$ is the electric field, $\vec B$ is the magnetic field,
$\vec D$ is the electric displacement field, $\vec H$ is the $\vec
H$-magnetic field, $\rho$ is the charge density, $\vec j$ is the
current density and $c$ is the universal constant,
%usually
called
%"
speed of light.
%".
%
%
%
It is assumed in the Classical Electrodynamics that for the vacuum
we always have $\vec D\equiv \vec E$ and $\vec H\equiv \vec B$.

 We assume that the Maxwell equations in the vacuum
%(pure aether)
have the usual form \er{MaxVacPPN}, as in any other medium,
however, similarly to the General Theory of Relativity we assume
that the electromagnetic field is influenced by the gravitational
field.
%
%
%
\begin{comment}
i.e.
\begin{equation}\label{MaxVacPPN}
\begin{cases}
curl_{\vec x} \vec H\equiv \frac{4\pi}{c}\vec j+\frac{1}{c}\frac{\partial \vec D}{\partial t}\quad\text{for}\;\;(\vec x,t)\in\R^3\times[0,+\infty),\\
div_{\vec x} \vec D\equiv 4\pi\rho\quad\quad\text{for}\;\;(\vec x,t)\in\R^3\times[0,+\infty),\\
curl_{\vec x} \vec E+\frac{1}{c}\frac{\partial \vec B}{\partial t}\equiv 0\quad\quad\text{for}\;\;(\vec x,t)\in\R^3\times[0,+\infty),\\
div_{\vec x} \vec B\equiv 0\quad\quad\text{for}\;\;(\vec
x,t)\in\R^3\times[0,+\infty),
\end{cases}
\end{equation}
\end{comment}
%
%
%
Then, we assume that for a given inertial coordinate system we have
$\vec D(\vec x,t)= \vec E(\vec x,t)$ and $\vec H(\vec x,t)=\vec
B(\vec x,t)$ for the vacuum only in the case where the vectorial
gravitational potential $\vec v(\vec x,t)$ on the point $\vec x$ at
the time $t$ equals to zero in the given coordinate system i.e.
\begin{equation}\label{IdspEthPPN}
\text{If}\;\;\vec v(\vec x,t)=0\;\;\text{for some}\;\;(\vec
x,t)\in\R^3\times[0,+\infty)\;\;\text{then}\;\;\vec D(\vec x,t)=
\vec E(\vec x,t)\;\;\text{and}\;\;\vec H(\vec x,t)=\vec B(\vec x,t),
\end{equation}
where $\vec v(\vec x,t)$ is the same as in
\er{noninchgravortbstrjgghguittu2gjgghhjhghjhjgghgghghghtytythvfghfgghjggghjgjh}.
In order to obtain the relations $\vec D\sim \vec E$ and $\vec H\sim
\vec B$ in the general case we assume that the equations
\er{MaxVacPPN} and the Lorentz force $\vec F:=\sigma \vec
E+\frac{\sigma}{c}\,\vec u\times \vec B$ (where $\sigma$ is the
charge of the test particle and $\vec u$ is its velocity) are
invariant under the Galilean Transformations:
\begin{equation}\label{Gal2PPN}
\begin{cases}
\vec x'=\vec x+t\vec w,\\
t'=t.
\end{cases}
\end{equation}
First observe that if
%$e$ is charge of the test particle and
$\vec u$ is a velocity of the test particle then $\vec u'=\vec
u+\vec w$. Thus, since we assumed that the Lorentz force $\vec
F:=\sigma \vec E+\frac{\sigma}{c}\vec u\times \vec B$ is invariant
under Galilean transformation we infer $$\sigma \vec
E'+\frac{\sigma}{c}\, (\vec u+\vec w)\times \vec B'=\sigma \vec
E'+\frac{\sigma}{c}\, \vec u'\times \vec B'=\vec F'=\vec F=\sigma
\vec E+\frac{\sigma}{c}\, \vec u\times \vec B.$$ Therefore, we
obtain the following identities:
\begin{equation}\label{EBTransPPN}
\begin{cases}
\vec E'=\vec E-\frac{1}{c}\,\vec w\times \vec B,\\
\vec B'=\vec B.
\end{cases}
\end{equation}
It is easy to check that, under transformations \er{Gal2PPN} and
\er{EBTransPPN}, the last two equations in \er{MaxVacPPN} are
invariant. Next observe that in the absents of currents and charges
the first two equations in \er{MaxVacPPN} for $\vec H$ and $\vec D$
will be the same as the last two for $\vec E$ and $\vec B$ if we
will change the sign of the time there. Therefore, it can be assumed
that the first two equations will stay invariant under the
transformation:
\begin{equation}\label{HDTransPPN}
\begin{cases}
\vec H'=\vec H+\frac{1}{c}\,\vec w\times \vec D,\\
\vec D'=\vec D.
\end{cases}
\end{equation}
Indeed, since $\rho'=\rho$ and $\vec j'=\vec j+\rho \vec w$, it can
be easily checked that under the transformations \er{Gal2PPN} and
\er{HDTransPPN} the first two equations will stay invariant also in
the case of charges and currents. Therefore, we obtained that all
equations in \er{MaxVacPPN} are invariant under the transformations
\er{Gal2PPN}
%
%
%
\begin{comment}
\begin{equation}\label{Gal1}
\begin{cases}
\vec x'=\vec x+t\vec w,\\
t'=t.
\end{cases}
\end{equation}
\begin{equation}\label{EBDHTrans}
\begin{cases}
\vec D'=\vec D,\\
\vec B'=\vec B,\\
\vec E'=\vec E-\frac{1}{c}\,\vec w\times \vec B,\\
\vec H'=\vec H+\frac{1}{c}\,\vec w\times \vec D.
\end{cases}
\end{equation}
\end{comment}
%
%
%
and
\begin{equation}\label{EBDHTrans2PPN}
\begin{cases}
\vec D'=\vec D,\\
\vec B'=\vec B,\\
\vec E'=\vec E-\frac{1}{c}\,\vec w\times \vec B,\\
\vec H'=\vec H+\frac{1}{c}\,\vec w\times \vec D.
\end{cases}
\end{equation}
Next fix some point $(\vec x_0,t_0)\in\R^3\times[0,+\infty)$ and
consider $\vec w:=-\vec v(\vec x_0,t_0)$, where $\vec v$ is the
vectorial gravitational potential. Then, since $\vec v'=\vec v+\vec
w$ (speed-like vector field), we obtain that at the point $(\vec
x'_0,t'_0)$ we have $\vec v'=0$. Therefore, by the assumption
\er{IdspEthPPN} we must have $\vec E'=\vec D'$ and $\vec H'=\vec B'$
at this point. Plugging it into \er{EBDHTrans2PPN}, for this point
we obtain
\begin{multline}
\label{EBDHTranssled1PPN} \vec E(\vec x_0,t_0)+\frac{\vec v(\vec
x_0,t_0)}{c}\times \vec B(\vec x_0,t_0)=\vec E(\vec
x_0,t_0)-\frac{\vec w}{c}\times \vec B(\vec x_0,t_0)\\=\vec E'(\vec
x'_0,t'_0)=\vec D'(\vec x'_0,t'_0)=\vec D(\vec x_0,t_0)
\end{multline}
\begin{multline} \label{EBDHTranssled2PPN}
\vec H(\vec x_0,t_0)-\frac{\vec v(\vec x_0,t_0)}{c}\times \vec
D(\vec x_0,t_0)=\vec H(\vec x_0,t_0)+\frac{\vec w}{c}\times \vec
D(\vec x_0,t_0)\\=\vec H'(\vec x'_0,t'_0)=\vec B'(\vec
x'_0,t'_0)=\vec B(\vec x_0,t_0).
\end{multline}
Thus, since the point $(\vec x_0,t_0)\in\R^3\times[0,+\infty)$ was
arbitrarily chosen, by \er{EBDHTranssled1PPN} and
\er{EBDHTranssled2PPN} we obtain the following relations
\begin{equation}\label{DtoEBtoHPPN}
\begin{cases}
\vec E(\vec x,t)=\vec D(\vec x,t)-\frac{1}{c}\,\vec v(\vec
x,t)\times
\vec B(\vec x,t)\quad\quad\forall(\vec x,t)\in\R^3\times[0,+\infty)\\
\vec H(\vec x,t)=\vec B(\vec x,t)+\frac{1}{c}\,\vec v(\vec
x,t)\times \vec D(\vec x,t)\quad\quad\forall(\vec
x,t)\in\R^3\times[0,+\infty).
\end{cases}
\end{equation}
Plugging \er{DtoEBtoHPPN} into \er{MaxVacPPN} we obtain the full
system of Electrodynamics in the case of an arbitrarily vectorial
gravitational potential:
%%%%%%%%%%%%%%%%%%%%%%%%%%%%%%%%%%%%%%%%%%%%%%%%%%%%%%%%%%%%%%%%%%%%%%%%%%%%%%%%%%%%%%%%%%%%%%%%%%%%%%%%%%%%%%%%%%%%%%%%%%%%%%%%%%%%%%%%%%%%%%%%%%%%%%%%%%%%%%%%%%%%%%%%%%%%%%%%%%%%%%%%%%%%%%%%%%
\begin{equation}\label{MaxVacFullPPN}
\begin{cases}
curl_{\vec x} \vec H\equiv \frac{4\pi}{c}\vec j+\frac{1}{c}\frac{\partial \vec D}{\partial t}\quad\text{for}\;\;(\vec x,t)\in\R^3\times[0,+\infty),\\
div_{\vec x} \vec D\equiv 4\pi\rho\quad\quad\text{for}\;\;(\vec x,t)\in\R^3\times[0,+\infty),\\
curl_{\vec x} \vec E+\frac{1}{c}\frac{\partial \vec B}{\partial t}\equiv 0\quad\quad\text{for}\;\;(\vec x,t)\in\R^3\times[0,+\infty),\\
div_{\vec x} \vec B\equiv
0\quad\quad\text{for}\;\;(\vec x,t)\in\R^3\times[0,+\infty),\\
\vec E=\vec D-\frac{1}{c}\,\vec v\times
\vec B\quad\quad\text{for}\;\;(\vec x,t)\in\R^3\times[0,+\infty)\\
\vec H=\vec B+\frac{1}{c}\,\vec v\times \vec
D\quad\quad\text{for}\;\;(\vec x,t)\in\R^3\times[0,+\infty),
\end{cases}
\end{equation}
where $\vec v$ is the vectorial gravitational potential. It can be
easily checked that system \er{MaxVacFullPPN} and the Lorentz force
$\vec F:=\sigma(\vec E+\frac{\vec u}{c}\times \vec B)$ are invariant
under transformations \er{Gal2PPN} and \er{EBDHTrans2PPN}.
%
%
%
%
\begin{comment}
\begin{equation}\label{Gal2}
\begin{cases}
\vec x'=\vec x+t\vec w,\\
t'=t.
\end{cases}
\end{equation}
and
\begin{equation}\label{EBDHTrans2}
\begin{cases}
\vec D'=\vec D,\\
\vec B'=\vec B,\\
\vec E'=\vec E-\frac{1}{c}\,\vec w\times \vec B,\\
\vec H'=\vec H+\frac{1}{c}\,\vec w\times \vec D.
\end{cases}
\end{equation}
\end{comment}
%
%
%
%
Note here that $\vec D$ and $\vec B$ are invariant under the change
of inertial coordinate system. Moreover, we can write the Lorentz
force as $\vec F:=\sigma(\vec D+\frac{\vec u-\vec v}{c}\times \vec
B)$, where $(\vec u-\vec v)$ is the relative velocity of the test
particle with respect to the fictitious aether.

%SSD



















































\section{Maxwell equations in non-inertial cartesian coordinate
systems}\label{INNONredPPN}
%%%%HERE
Consider the change of certain non-inertial cartesian coordinate
system $(*)$ to another cartesian coordinate system $(**)$:
\begin{equation}\label{noninchredPPN}
\begin{cases}
\vec x'=A(t)\cdot \vec x+\vec z(t),\\
t'=t,
\end{cases}
\end{equation}
where $A(t)\in SO(3)$ is a rotation i.e. $A(t)\in \R^{3\times 3}$,
$det\, A(t)>0$ and $A(t)\cdot A^T(t)=I$ (here $A^T$ is the transpose
matrix of $A$ and $I$ is the identity matrix). Next, assume that in
coordinate system $(**)$ we observe a validity of Maxwell Equations
for the vacuum in the form:
\begin{equation}\label{MaxVacFull1ninshtrredPPN}
\begin{cases}
curl_{\vec x'} \vec H'\equiv \frac{4\pi}{c}\vec
j'+\frac{1}{c}\frac{\partial
\vec D'}{\partial t'},\\
%\quad\text{for}\;\;(\vec x,t)\in\R^3\times[0,+\infty),\\
div_{\vec x'} \vec D'\equiv 4\pi\rho',\\
%\quad\quad\text{for}\;\;(\vec x,t)\in\R^3\times[0,+\infty),\\
curl_{\vec x'} \vec E'+\frac{1}{c}\frac{\partial \vec B'}{\partial t'}\equiv 0,\\
%\quad\quad\text{for}\;\;(\vec x,t)\in\R^3\times[0,+\infty),\\
div_{\vec x'} \vec B'\equiv 0,\\
%\quad\quad\text{for}\;\;(\vec x,t)\in\R^3\times[0,+\infty),\\
\vec E'=\vec D'-\frac{1}{c}\,\vec v'\times \vec B',\\
%\quad\quad\text{for}\;\;(\vec x,t)\in\R^3\times[0,+\infty)\\
\vec H'=\vec B'+\frac{1}{c}\,\vec v'\times \vec D'.
%\quad\quad\text{for}\;\;(\vec x,t)\in\R^3\times[0,+\infty).
\end{cases}
\end{equation}
Moreover, we assume that in coordinate system $(**)$ we observe a
validity of expression for the Lorentz force
\begin{equation}\label{LorenzChredPPN}
\vec F':=\sigma' \vec E'+\frac{\sigma'}{c}\,\vec u'\times \vec B'
\end{equation}
(where $\sigma'$ is the charge of the test particle and $\vec u'$ is
its velocity in coordinate system $(**)$). All above happens, in
particular, if coordinate system $(**)$ is inertial. Observe that if
$\vec F$ is the force in coordinate system $(*)$ which corresponds
to the Lorentz force $\vec F'$ in coordinate system $(**)$, then we
must have $\vec F'=A(t)\cdot\vec F$. Moreover, denoting $\vec
w(t)=\vec z'(t)$, we have the following obvious relations between
the physical characteristics in coordinate systems $(*)$ and $(**)$:
\begin{align}
\label{NoIn1redPPN}\vec F'=A(t)\cdot\vec F,\\
\label{NoIn2redPPN}\sigma'=\sigma,\\
\label{NoIn3redPPN}\vec u'=A(t)\cdot \vec u+A'(t)\cdot\vec x+\vec w(t),\\
\label{NoIn4redPPN}\rho'=\rho,\\
\label{NoIn5redPPN}\vec v'=A(t)\cdot \vec v+A'(t)\cdot\vec x+\vec w(t),\\
\label{NoIn6redPPN}\vec j'=A(t)\cdot \vec j+\rho A'(t)\cdot\vec
x+\rho\vec w(t)
\end{align}
(where $A'(t)$ is a derivative of $A(t)$). We consider the fields
$\vec E$ and $\vec B$ in the coordinate system $(*)$ to be defined
by the expression of Lorentz force:
\begin{equation}\label{LorenzChllredPPN}
\vec F=\sigma \vec E+\frac{\sigma}{c}\,\vec u\times \vec B.
\end{equation}
Plugging it into \er{LorenzChredPPN} and using \er{NoIn1redPPN},
\er{NoIn2redPPN} and \er{NoIn3redPPN} we deduce
\begin{multline}\label{buiguyfttyjredPPN}
\sigma \left(\vec E'+\frac{1}{c}\,\left(A'(t)\cdot\vec x+\vec
w(t)\right)\times \vec B'\right)+\frac{\sigma}{c}\,\left(A(t)\cdot
\vec u\right)\times \vec B'\\=\sigma \vec
E'+\frac{\sigma}{c}\,\left(A(t)\cdot \vec u+A'(t)\cdot\vec x+\vec
w(t)\right)\times \vec B'\\=\sigma' \vec E'+\frac{\sigma'}{c}\,\vec
u'\times \vec B'=\vec F'=A(t)\cdot\vec F=\sigma A(t)\cdot\vec
E+\frac{\sigma}{c}\,A(t)\cdot\left(\vec u\times \vec B\right).
\end{multline}
Thus using the trivial identity
\begin{equation}\label{cufuyggjjjfredPPN}
A\cdot\left(\vec a\times\vec b\right)=\left(A\cdot\vec
a\right)\times\left(A\cdot\vec b\right)\quad\forall\, \vec
a\in\R^3,\;\;\forall\, \vec b\in\R^3,\;\;\forall\, A\in SO(3),
\end{equation}
by \er{buiguyfttyjredPPN} we deduce
\begin{multline}\label{buiguyfttyjuiuihjredPPN}
\sigma \left(\vec E'+\frac{1}{c}\,\left(A'(t)\cdot\vec x+\vec
w(t)\right)\times \vec B'\right)+\frac{\sigma}{c}\,\left(A(t)\cdot
\vec u\right)\times \vec B'\\=\sigma A(t)\cdot\vec
E+\frac{\sigma}{c}\,\left(A(t)\cdot\vec u\right)\times
\left(A(t)\cdot\vec B\right).
\end{multline}
Therefore, since \er{buiguyfttyjuiuihjredPPN} must be valid for
arbitrary choices of $\vec u$ we deduce
\begin{equation*}
\begin{cases}
\vec B'=A(t)\cdot\vec B,
\\
\vec E'+\frac{1}{c}\,\left(A'(t)\cdot\vec x+\vec w(t)\right)\times
\vec B'=A(t)\cdot\vec E.
\end{cases}
\end{equation*}
Therefore,
\begin{equation*}
\vec E'=A(t)\cdot\vec E-\frac{1}{c}\,\left(A'(t)\cdot\vec x+\vec
w(t)\right)\times \vec B'=A(t)\cdot\vec
E-\frac{1}{c}\,\left(A'(t)\cdot\vec x+\vec w(t)\right)\times
\left(A(t)\cdot\vec B\right).
\end{equation*}
So we obtained the following relations linking the fields $\vec
E,\vec B$ in coordinate system $(*)$ and $\vec E',\vec B'$ in
coordinate system $(**)$:
\begin{equation}\label{yuythfgfyftydtydtydtyddredPPN}
\begin{cases}
\vec E'=A(t)\cdot\vec E-\frac{1}{c}\,\left(A'(t)\cdot\vec x+\vec
w(t)\right)\times \left(A(t)\cdot\vec B\right),
\\
\vec B'=A(t)\cdot\vec B.
\end{cases}
\end{equation}
Next, by \er{MaxVacFull1ninshtrredPPN} in coordinate system $(**)$
we have the relations
\begin{equation*}
\begin{cases}
\vec D'=\vec E'+\frac{1}{c}\,\vec v'\times \vec B',\\
%\quad\quad\text{for}\;\;(\vec x,t)\in\R^3\times[0,+\infty)\\
\vec H'=\vec B'+\frac{1}{c}\,\vec v'\times \vec D'.
\end{cases}
\end{equation*}
Analogously we define $\vec D$ and $\vec H$  in coordinate system
$(*)$ by the formulas:
\begin{equation}\label{gjhghfhgdghdredPPN}
\begin{cases}
\vec D=\vec E+\frac{1}{c}\,\vec v\times \vec B,\\
%\quad\quad\text{for}\;\;(\vec x,t)\in\R^3\times[0,+\infty)\\
\vec H=\vec B+\frac{1}{c}\,\vec v\times \vec D.
\end{cases}
\end{equation}
Then with the help of \er{yuythfgfyftydtydtydtyddredPPN},
\er{NoIn5redPPN} and \er{cufuyggjjjfredPPN} we deduce:
\begin{multline*}
\vec D'=\vec E'+\frac{1}{c}\,\vec v'\times \vec B'=\\A(t)\cdot\vec
E-\frac{1}{c}\,\left(A'(t)\cdot\vec x+\vec w(t)\right)\times
\left(A(t)\cdot\vec B\right)+\frac{1}{c}\,\vec v'\times
\left(A(t)\cdot\vec B\right)=\\A(t)\cdot\vec
E-\frac{1}{c}\,\left(A'(t)\cdot\vec x+\vec w(t)\right)\times
\left(A(t)\cdot\vec B\right)+\frac{1}{c}\,\left(A(t)\cdot \vec
v+A'(t)\cdot\vec x+\vec w(t)\right)\times \left(A(t)\cdot\vec
B\right)\\=A(t)\cdot\vec E+\frac{1}{c}\,\left(A(t)\cdot \vec
v\right)\times \left(A(t)\cdot\vec B\right)=A(t)\cdot\left(\vec
E+\frac{1}{c}\,\vec v\times \vec B\right)=A(t)\cdot \vec D,
\end{multline*}
and thus
\begin{multline*}
\vec H'=\vec B'+\frac{1}{c}\,\vec v'\times \vec D'=A(t)\cdot\vec
B+\frac{1}{c}\,\left(A(t)\cdot \vec v+A'(t)\cdot\vec x+\vec
w(t)\right)\times \left(A(t)\cdot\vec D\right)=\\
A(t)\cdot\vec B+\frac{1}{c}\,\left(A(t)\cdot \vec v\right)\times
\left(A(t)\cdot\vec D\right)+\frac{1}{c}\,\left(A'(t)\cdot\vec
x+\vec
w(t)\right)\times \left(A(t)\cdot\vec D\right)=\\
A(t)\cdot\left(\vec B+\frac{1}{c}\,\vec v\times \vec
D\right)+\frac{1}{c}\,\left(A'(t)\cdot\vec x+\vec w(t)\right)\times
\left(A(t)\cdot\vec D\right)\\=A(t)\cdot\vec
H+\frac{1}{c}\,\left(A'(t)\cdot\vec x+\vec w(t)\right)\times
\left(A(t)\cdot\vec D\right).
\end{multline*}
I.e. the following relations are valid:
\begin{equation}\label{yuythfgfyftydtydtydtyddyyyhhddhhhredPPN}
\begin{cases}
\vec D'=A(t)\cdot \vec D,\\
\vec B'=A(t)\cdot\vec B,\\
\vec E'=A(t)\cdot\vec E-\frac{1}{c}\,\left(A'(t)\cdot\vec x+\vec
w(t)\right)\times \left(A(t)\cdot\vec B\right),\\
\vec H'=A(t)\cdot\vec H+\frac{1}{c}\,\left(A'(t)\cdot\vec x+\vec
w(t)\right)\times \left(A(t)\cdot\vec D\right).
\end{cases}
\end{equation}
In particular vector fields $\vec D$ and $\vec B$ are proper vector
fields.

Next, by \er{noninchredPPN} and by Proposition \ref{yghgjtgyrtrt},
%from the Appendix,
for every vector field $\vec
\Gamma:\R^3\times[0,+\infty)\to\R^3$ we have
\begin{equation}
\label{vyguiuiujggghjjgredPPN}
\begin{cases}
%\nabla_{\vec x'}\left(A(t)\cdot\vec F\right)=A(t)\cdot\nabla_{\vec x}\vec F\cdot A^{-1}(t)\\
d_{\vec x'}\vec \Gamma=\left(d_{\vec x}\vec \Gamma\right)\cdot A^{-1}(t)\\
curl_{\vec x'}\left( A(t)\cdot\vec \Gamma\right)=A(t)\cdot
curl_{\vec x}
\vec \Gamma\\
div_{\vec x'}\left( A(t)\cdot\vec \Gamma\right)=div_{\vec x}\vec
\Gamma.
\end{cases}
\end{equation}
Furthermore, by Proposition \ref{yghgjtgyrtrt},
%from the Appendix,
for every vector field $\vec \Gamma:\R^3\times[0,+\infty)\to\R^3$ we
have
\begin{multline}\label{bgvfgfhhtgjtggiuguiuiredPPN}
\frac{\partial \left(A(t)\cdot\vec \Gamma\right)}{\partial
t'}-curl_{\vec x'}\left(\vec v'\times\left(A(t)\cdot\vec
\Gamma\right)\right)+\left({div}_{\vec x'}\left(A(t)\cdot\vec
\Gamma\right)\right)\vec v'\\=A(t)\cdot\left(\frac{\partial \vec
\Gamma}{\partial t}- curl_{\vec x}\left(\vec v\times
\vec\Gamma\right)+\left({div}_{\vec x}\vec \Gamma\right)\vec
v\right).
\end{multline}
On the other hand, by \er{MaxVacFull1ninshtrredPPN} we have
\begin{multline}\label{uiguihjkjkklklklredPPN}
curl_{\vec x'} \vec B'- \frac{4\pi}{c}\left(\vec j'-\rho'\vec
v'\right)-\frac{1}{c}\left(\frac{\partial \vec D'}{\partial
t'}-curl_{\vec x'}\left(\vec v'\times\vec
D'\right)+\left({div}_{\vec x'}\vec D'\right)\vec
v'\right)\\=curl_{\vec x'} \vec H'- \frac{4\pi}{c}\vec
j'-\frac{1}{c}\frac{\partial \vec D'}{\partial t'}=0\end{multline}
and
\begin{equation}\label{uiguihjkjkklklkljhhredPPN}
curl_{\vec x'} \vec D'+\frac{1}{c}\left(\frac{\partial \vec
B'}{\partial t'}-curl_{\vec x'}\left(\vec v'\times\vec
B'\right)+\left({div}_{\vec x'}\vec B'\right)\vec
v'\right)=curl_{\vec x'} \vec E'+\frac{1}{c}\frac{\partial \vec
B'}{\partial t'}=0.
\end{equation}
Thus plugging \er{uiguihjkjkklklklredPPN} and
\er{uiguihjkjkklklkljhhredPPN} into \er{bgvfgfhhtgjtggiuguiuiredPPN}
and using \er{gjhghfhgdghdredPPN}, \er{NoIn4redPPN},
\er{NoIn5redPPN}, \er{NoIn6redPPN} and \er{vyguiuiujggghjjgredPPN}
gives
\begin{multline}\label{uiguihjkjkklklklhkkredPPN}
A(t)\cdot\left(curl_{\vec x} \vec H-\frac{4\pi}{c}\,\vec
j-\frac{1}{c}\frac{\partial \vec D}{\partial
t}+\frac{1}{c}\left(4\pi\rho-{div}_{\vec x}\vec
D\right)\vec v\right)=\\
A(t)\cdot\left(curl_{\vec x} \vec B- \frac{4\pi}{c}\left(\vec
j-\rho\vec v\right)-\frac{1}{c}\left(\frac{\partial \vec D}{\partial
t}-curl_{\vec x}\left(\vec v\times\vec D\right)+\left({div}_{\vec
x}\vec D\right)\vec v\right)\right)=\\curl_{\vec x'} \vec B'-
\frac{4\pi}{c}\left(\vec j'-\rho'\vec
v'\right)-\frac{1}{c}\left(\frac{\partial \vec D'}{\partial
t'}-curl_{\vec x'}\left(\vec v'\times\vec
D'\right)+\left({div}_{\vec x'}\vec D\right)\vec v'\right)=0.
\end{multline}
Similarly
\begin{multline}\label{uiguihjkjkklklkljhhjkggkjredPPN}
A(t)\cdot\left(curl_{\vec x} \vec E+\frac{1}{c}\frac{\partial \vec
B}{\partial t}+\frac{1}{c}\left({div}_{\vec x}\vec B\right)\vec
v\right)=\\A(t)\cdot\left(curl_{\vec x} \vec
D+\frac{1}{c}\left(\frac{\partial \vec B}{\partial t}-curl_{\vec
x}\left(\vec v\times\vec B\right)+\left({div}_{\vec x}\vec
B\right)\vec v\right)\right)\\= curl_{\vec x'} \vec
D'+\frac{1}{c}\left(\frac{\partial \vec B'}{\partial t'}-curl_{\vec
x'}\left(\vec v'\times\vec B'\right)+\left({div}_{\vec x'}\vec
B\right)\vec v'\right)=0.
\end{multline}
On the other hand, by \er{yuythfgfyftydtydtydtyddyyyhhddhhhredPPN},
\er{MaxVacFull1ninshtrredPPN}, \er{vyguiuiujggghjjgredPPN} and
\er{NoIn4redPPN} we obtain:
\begin{equation}
\label{vyguiuiujggkkkhjhjjppopredPPN} 4\pi\rho=4\pi\rho'=div_{\vec
x'} \vec D'=div_{\vec x} \vec D\quad\text{and}\quad 0=div_{\vec x'}
\vec B'=div_{\vec x} \vec B.
\end{equation}
%
%
%
%
\begin{comment}
Thus by \er{vyguiuiujggghjjg}, \er{vyguiuiujggghjjgggj},
\er{yuythfgfyftydtydtydtyddyyyhhddhhh} and \er{cufuyggjjjf} we
deduce
\begin{equation}
\label{vyguiuiujggkkk}
\begin{cases}
curl_{\vec x'} \vec E'=A(t)\cdot curl_{\vec x} \vec
E-\frac{1}{c}\,A(t)\cdot curl_{\vec x}\left(\left(A^{-1}(t)\cdot
A'(t)\cdot\vec x+
A^{-1}(t)\cdot\vec w(t)\right)\times\vec B\right)\\
curl_{\vec x'} \vec H'=A(t)\cdot curl_{\vec x} \vec
H+\frac{1}{c}\,A(t)\cdot curl_{\vec x}\left(\left(A^{-1}(t)\cdot
A'(t)\cdot\vec x+A^{-1}(t)\cdot\vec w(t)\right)\times \vec D\right)\\
\frac{\partial \vec D'}{\partial t'}=A(t)\cdot\frac{\partial \vec
D}{\partial t}+A'(t)\cdot\vec D-A(t)\cdot \left(d_{\vec x}\vec
D\right)\cdot
\left(A^{-1}(t)\cdot A'(t)\cdot\vec x+A^{-1}(t)\cdot \vec w(t)\right)\\
\frac{\partial \vec B'}{\partial t'}=A(t)\cdot\frac{\partial \vec
B}{\partial t}+A'(t)\cdot\vec B-A(t)\cdot \left(d_{\vec x}\vec
B\right)\cdot
\left(A^{-1}(t)\cdot A'(t)\cdot\vec x+A^{-1}(t)\cdot \vec w(t)\right)\\
div_{\vec x'} \vec D'=div_{\vec x} \vec D\\
div_{\vec x'} \vec B'=div_{\vec x} \vec B.
\end{cases}
\end{equation}
Then plugging \er{vyguiuiujggkkk} into \er{MaxVacFull1ninshtr} and
using \er{NoIn4} and \er{NoIn6} gives
\begin{equation}\label{MaxVacFull1ninshtrjhjgfhfh}
\begin{cases}
div_{\vec x} \vec D\equiv 4\pi\rho,\\
div_{\vec x} \vec B\equiv 0,
\end{cases}
\end{equation}
\begin{multline}\label{MaxVacFull1ninshtrjhjg}
A(t)\cdot curl_{\vec x} \vec H+\frac{1}{c}\,A(t)\cdot curl_{\vec
x}\left(\left(A^{-1}(t)\cdot A'(t)\cdot\vec x+A^{-1}(t)\cdot\vec
w(t)\right)\times \vec D\right)=\\ \frac{4\pi}{c}A(t)\cdot \vec
j+\frac{1}{c}\,\left(div_{\vec x} \vec D\right) \left(A'(t)\cdot\vec
x+\vec w(t)\right)\\+\frac{1}{c}\left(A(t)\cdot\frac{\partial \vec
D}{\partial t}+A'(t)\cdot\vec D-A(t)\cdot \left(d_{\vec x}\vec
D\right)\cdot \left(A^{-1}(t)\cdot A'(t)\cdot\vec x+A^{-1}(t)\cdot
\vec w(t)\right)\right)
\end{multline}
and
\begin{multline}\label{MaxVacFull1ninshtrjhjgghgkh}
A(t)\cdot curl_{\vec x} \vec E-\frac{1}{c}\,A(t)\cdot curl_{\vec
x}\left(\left(A^{-1}(t)\cdot A'(t)\cdot\vec x+ A^{-1}(t)\cdot \vec
w(t)\right)\times\vec
B\right)\\+\frac{1}{c}\left(A(t)\cdot\frac{\partial \vec B}{\partial
t}+A'(t)\cdot\vec B-A(t)\cdot \left(d_{\vec x}\vec B\right)\cdot
\left(A^{-1}(t)\cdot A'(t)\cdot\vec x+A^{-1}(t)\cdot \vec
w(t)\right)\right)=0
\end{multline}
On the other hand, by \er{apfrm6} we infer
\begin{multline}\label{MaxVacFull1ninshtrjhjgjhfjfj}
%curl_{\vec x}\,(\vec f\times \vec g)
curl_{\vec x}\left(\left(A^{-1}(t)\cdot A'(t)\cdot\vec
x+A^{-1}(t)\cdot\vec w(t)\right)\times \vec D\right)
=\left({div}_{\vec x} \vec D\right)\,\left(A^{-1}(t)\cdot
A'(t)\cdot\vec x+A^{-1}(t)\cdot\vec w(t)\right) \\-\left({div}_{\vec
x} \left(A^{-1}(t)\cdot A'(t)\cdot\vec x+A^{-1}(t)\cdot\vec
w(t)\right)\right)\,\vec D +\left(d_{\vec x} \left(A^{-1}(t)\cdot
A'(t)\cdot\vec x+A^{-1}(t)\cdot\vec w(t)\right)\right)\cdot \vec
D\\-\left(d_{\vec x} \vec D\right)\cdot \left(A^{-1}(t)\cdot
A'(t)\cdot\vec x+A^{-1}(t)\cdot\vec w(t)\right)=\left({div}_{\vec x}
\vec D\right)\,\left(A^{-1}(t)\cdot A'(t)\cdot\vec
x+A^{-1}(t)\cdot\vec w(t)\right) \\-\left(tr\left(A^{-1}(t)\cdot
A'(t)\right)\right)\,\vec D + A^{-1}(t)\cdot A'(t)\cdot \vec
D-\left(d_{\vec x} \vec D\right)\cdot \left(A^{-1}(t)\cdot
A'(t)\cdot\vec x+A^{-1}(t)\cdot\vec w(t)\right),
\end{multline}
and
\begin{multline}\label{MaxVacFull1ninshtrjhjgjhfjfjyutjk}
%curl_{\vec x}\,(\vec f\times \vec g)
curl_{\vec x}\left(\left(A^{-1}(t)\cdot A'(t)\cdot\vec
x+A^{-1}(t)\cdot\vec w(t)\right)\times \vec B\right)
=\left({div}_{\vec x} \vec B\right)\,\left(A^{-1}(t)\cdot
A'(t)\cdot\vec x+A^{-1}(t)\cdot\vec w(t)\right) \\-\left({div}_{\vec
x} \left(A^{-1}(t)\cdot A'(t)\cdot\vec x+A^{-1}(t)\cdot\vec
w(t)\right)\right)\,\vec B +\left(d_{\vec x} \left(A^{-1}(t)\cdot
A'(t)\cdot\vec x+A^{-1}(t)\cdot\vec w(t)\right)\right)\cdot \vec
B\\-\left(d_{\vec x} \vec B\right)\cdot \left(A^{-1}(t)\cdot
A'(t)\cdot\vec x+A^{-1}(t)\cdot\vec w(t)\right)=
\\-\left(tr\left(A^{-1}(t)\cdot A'(t)\right)\right)\,\vec B +
A^{-1}(t)\cdot A'(t)\cdot \vec B-\left(d_{\vec x} \vec B\right)\cdot
\left(A^{-1}(t)\cdot A'(t)\cdot\vec x+A^{-1}(t)\cdot\vec
w(t)\right),
\end{multline}
where $tr\left(A^{-1}(t)\cdot A'(t)\right)$ is a trace of the matrix
$A^{-1}(t)\cdot A'(t)$ (sum of diagonal elements). However, since
$A^T(t)\cdot A(t)=I$ we have $A^{-1}(t)=A^T(t)$ and $A^{-1}(t)\cdot
A'(t)=S(t)$, where $S^T(t)=-S(t)$. In particular $tr S(t)=0$ and
thus $tr\left(A^{-1}(t)\cdot A'(t)\right)=0$. Therefore, by
\er{MaxVacFull1ninshtrjhjgjhfjfj} and
\er{MaxVacFull1ninshtrjhjgjhfjfjyutjk} we obtain
\begin{multline}\label{MaxVacFull1ninshtrjhjgjhfjfjigyut}
%curl_{\vec x}\,(\vec f\times \vec g)
curl_{\vec x}\left(\left(A^{-1}(t)\cdot A'(t)\cdot\vec
x+A^{-1}(t)\cdot\vec w(t)\right)\times \vec D\right) =\\
\left({div}_{\vec x} \vec D\right)\,\left(A^{-1}(t)\cdot
A'(t)\cdot\vec x+A^{-1}(t)\cdot\vec w(t)\right)+ A^{-1}(t)\cdot
A'(t)\cdot \vec D-\left(d_{\vec x} \vec D\right)\cdot
\left(A^{-1}(t)\cdot A'(t)\cdot\vec x+A^{-1}(t)\cdot\vec
w(t)\right),
\end{multline}
and
\begin{multline}\label{MaxVacFull1ninshtrjhjgjhfjfjyutjkfuygij}
%curl_{\vec x}\,(\vec f\times \vec g)
curl_{\vec x}\left(\left(A^{-1}(t)\cdot A'(t)\cdot\vec
x+A^{-1}(t)\cdot\vec w(t)\right)\times \vec B\right) =\\
A^{-1}(t)\cdot A'(t)\cdot \vec B-\left(d_{\vec x} \vec B\right)\cdot
\left(A^{-1}(t)\cdot A'(t)\cdot\vec x+A^{-1}(t)\cdot\vec
w(t)\right),
\end{multline}
Thus plugging \er{MaxVacFull1ninshtrjhjgjhfjfjigyut} and
\er{MaxVacFull1ninshtrjhjgjhfjfjyutjkfuygij} into
\er{MaxVacFull1ninshtrjhjg} and \er{MaxVacFull1ninshtrjhjgghgkh} we
deduce
\end{comment}
%
%
%
%
Thus plugging \er{uiguihjkjkklklklhkkredPPN},
\er{uiguihjkjkklklkljhhjkggkjredPPN} and
\er{vyguiuiujggkkkhjhjjppopredPPN} we obtain
\begin{equation}\label{fhgjhkhkredPPN}
\begin{cases}
curl_{\vec x} \vec H= \frac{4\pi}{c}\vec
j+\frac{1}{c}\frac{\partial \vec D}{\partial t},\\
div_{\vec x} \vec D=4\pi\rho,
\\
curl_{\vec x} \vec E+\frac{1}{c}\frac{\partial \vec B}{\partial
t}=0,
\\
div_{\vec x} \vec B=0.
\end{cases}
\end{equation}
Then, plugging \er{fhgjhkhkredPPN} into
%\er{MaxVacFull1ninshtrjhjgfhfh} and
\er{gjhghfhgdghdredPPN}, we finally obtain that in coordinate system
$(*)$ the Maxwell equations have the same form as in system $(**)$
i.e.
\begin{equation}\label{MaxVacFull1ninshtrhjkkredPPN}
\begin{cases}
curl_{\vec x} \vec H\equiv \frac{4\pi}{c}\vec
j+\frac{1}{c}\frac{\partial
\vec D}{\partial t},\\
%\quad\text{for}\;\;(\vec x,t)\in\R^3\times[0,+\infty),\\
div_{\vec x} \vec D\equiv 4\pi\rho,\\
%\quad\quad\text{for}\;\;(\vec x,t)\in\R^3\times[0,+\infty),\\
curl_{\vec x} \vec E+\frac{1}{c}\frac{\partial \vec B}{\partial t}\equiv 0,\\
%\quad\quad\text{for}\;\;(\vec x,t)\in\R^3\times[0,+\infty),\\
div_{\vec x} \vec B\equiv 0,\\
%\quad\quad\text{for}\;\;(\vec x,t)\in\R^3\times[0,+\infty),\\
\vec E=\vec D-\frac{1}{c}\,\vec v\times \vec B,\\
%\quad\quad\text{for}\;\;(\vec x,t)\in\R^3\times[0,+\infty)\\
\vec H=\vec B+\frac{1}{c}\,\vec v\times \vec D.
%\quad\quad\text{for}\;\;(\vec x,t)\in\R^3\times[0,+\infty).
\end{cases}
\end{equation}
Therefore, since the assumption, that coordinate system $(**)$ is
inertial, implies the relations of \er{MaxVacFull1ninshtrredPPN}, we
deduce that the expressions of Maxwell equations in the form
\er{MaxVacFull1ninshtrhjkkredPPN} and of the Lorentz force in the
form \er{LorenzChllredPPN}
%
%
%
\begin{comment}
\begin{equation}\label{LorenzChlljklljk}
\vec F:=\sigma \vec E+\frac{\sigma}{c}\,\vec u\times \vec B
\end{equation}
\end{comment}
%
%
%
are valid in every non-inertial cartesian coordinate system.
Moreover, under the change of the
%coordinate
system, given by \er{noninchredPPN}, the transformations of the
electromagnetic fields are given by
\er{yuythfgfyftydtydtydtyddyyyhhddhhhredPPN} i.e.
\begin{equation}\label{yuythfgfyftydtydtydtyddyyyhhddhhhredPPN111hgghjg}
\begin{cases}
\vec D'=A(t)\cdot \vec D,\\
\vec B'=A(t)\cdot\vec B,\\
\vec E'=A(t)\cdot\vec E-\frac{1}{c}\,\left(\frac{dA}{dt}(t)\cdot\vec
x+\frac{d\vec z}{dt}(t)\right)\times \left(A(t)\cdot\vec B\right),\\
\vec H'=A(t)\cdot\vec H+\frac{1}{c}\,\left(\frac{dA}{dt}(t)\cdot\vec
x+\frac{d\vec z}{dt}(t)\right)\times \left(A(t)\cdot\vec D\right).
\end{cases}
\end{equation}
%
%
%
%
\begin{comment}
\begin{equation}\label{noninchtht}
\begin{cases}
\vec x'=A(t)\cdot \vec x+\vec z(t),\\
t'=t,
\end{cases}
\end{equation}
\begin{equation}\label{yuythfgfyftydtydtydtyddyyyhhddhhh}
\begin{cases}
\vec D'=A(t)\cdot \vec D\\
\vec B'=A(t)\cdot\vec B\\
\vec E'=A(t)\cdot\vec E-\frac{1}{c}\,\left(A'(t)\cdot\vec x+\vec
w(t)\right)\times \left(A(t)\cdot\vec B\right)\\
\vec H'=A(t)\cdot\vec H+\frac{1}{c}\,\left(A'(t)\cdot\vec x+\vec
w(t)\right)\times \left(A(t)\cdot\vec D\right),
\end{cases}
\end{equation}
\begin{equation}\label{yuythfgfyftydtydtydtyddyyyhhddhhh}
\begin{cases}
\vec D'=A(t)\cdot \vec D,\\
\vec B'=A(t)\cdot\vec B,\\
\vec E'=A(t)\cdot\vec E-\frac{1}{c}\,\left(A'(t)\cdot\vec x+\vec
w(t)\right)\times \left(A(t)\cdot\vec B\right),\\
\vec H'=A(t)\cdot\vec H+\frac{1}{c}\,\left(A'(t)\cdot\vec x+\vec
w(t)\right)\times \left(A(t)\cdot\vec D\right).
\end{cases}
\end{equation}
where $\vec w(t)=\vec z'(t)$.
\end{comment}
%
%
%
%






So the laws of Electrodynamics are also invariant in non-inertial
coordinate systems.
%SSD1





\begin{remark}\label{gyugfjyhgghg}
Since as already mentioned before, the direction of the local
vectorial gravitational potential is normal to the Earth surface, in
the frames of our model, we provide a non-relativistic explanation
of the classical Michelson-Morley experiment. Indeed in this
experiment the axes of the apparatus are tangent to the Earth
surface and thus the null result cannot be affected by the vectorial
gravitational potential. Since, the value of the local vectorial
gravitational potential equals to the escape velocity, if we
consider the vertical Michelson-Morley experiment, where one of the
axes of the apparatus is normal to the Earth surface, then in the
frames of our model the expected result should be analogous to the
positive result of Aether drift with the speed equal to the escape
velocity. However, regarding the vertical Michelson-Morley
experiment i.e. the modification of Michelson-Morley experiment,
where at least one of the axes of the apparatus is not tangent to
the Earth surface, we found only very scarce and contradictory
information.
\end{remark}

















\section{Scalar and vectorial electromagnetic potentials}\label{ghfvdgfdjfg}
Consider
the system of Maxwell equations in the vacuum of the form
\begin{equation}\label{MaxVacFull1bjkgjhjhgjaaaPPN}
\begin{cases}
curl_{\vec x} \vec H\equiv \frac{4\pi}{c}\vec
j+\frac{1}{c}\frac{\partial
\vec D}{\partial t},\\
%\quad\text{for}\;\;(\vec x,t)\in\R^3\times[0,+\infty),\\
div_{\vec x} \vec D\equiv 4\pi\rho,\\
%\quad\quad\text{for}\;\;(\vec x,t)\in\R^3\times[0,+\infty),\\
curl_{\vec x} \vec E+\frac{1}{c}\frac{\partial \vec B}{\partial t}\equiv 0,\\
%\quad\quad\text{for}\;\;(\vec x,t)\in\R^3\times[0,+\infty),\\
div_{\vec x} \vec B\equiv 0,\\
%\quad\quad\text{for}\;\;(\vec x,t)\in\R^3\times[0,+\infty),\\
\vec E=\vec D-\frac{1}{c}\,\vec v\times \vec B,\\
%\quad\quad\text{for}\;\;(\vec x,t)\in\R^3\times[0,+\infty)\\
\vec H=\vec B+\frac{1}{c}\,\vec v\times \vec D,
%\quad\quad\text{for}\;\;(\vec x,t)\in\R^3\times[0,+\infty).
\end{cases}
\end{equation}
where $\vec v$ is the vectorial gravitational potential. Then by the
third and the fourth equations in \er{MaxVacFull1bjkgjhjhgjaaaPPN}
we can write:
\begin{equation}\label{MaxVacFull1bjkgjhjhgjgjgkjfhjfdghghligioiuittrPPN}
\begin{cases}
\vec B\equiv curl_{\vec x} \vec A,\\
\vec E\equiv-\nabla_{\vec x}\Psi-\frac{1}{c}\frac{\partial\vec
A}{\partial t},
%,\\ div_{\vec x}\vec A\equiv 0,
\end{cases}
\end{equation}
where we call $\Psi$ and $\vec A$ the scalar and the vectorial
electromagnetic potentials. Then by
\er{MaxVacFull1bjkgjhjhgjgjgkjfhjfdghghligioiuittrPPN} and
\er{MaxVacFull1bjkgjhjhgjaaaPPN} we have
\begin{equation}\label{vhfffngghPPN333yuyu}
\begin{cases}
\vec B= curl_{\vec x} \vec A\\
\vec E=-\nabla_{\vec x}\Psi-\frac{1}{c}\frac{\partial\vec
A}{\partial t}\\
 \vec D=-\nabla_{\vec
x}\Psi-\frac{1}{c}\frac{\partial\vec A}{\partial t}+\frac{1}{c}\vec
v\times curl_{\vec x}\vec A\\
\vec H\equiv curl_{\vec x} \vec A+\frac{1}{c}\,\vec
v\times\left(-\nabla_{\vec x}\Psi-\frac{1}{c}\frac{\partial\vec
A}{\partial t}+\frac{1}{c}\vec v\times curl_{\vec x}\vec A\right).
%\\ div_{\vec x}\vec A\equiv 0.
\end{cases}
\end{equation}
We also define the proper scalar electromagnetic potential
$\Psi_0=\Psi_0(\vec x,t)$ by
\begin{equation}\label{vhfffngghhjghhgPPNghghghutghffugghjhjkjjkl}
\Psi_0:=\Psi-\frac{1}{c}\vec A\cdot\vec v.
\end{equation}
The name "proper" will be clarified bellow. Then, by
\er{vhfffngghPPN333yuyu} and
\er{vhfffngghhjghhgPPNghghghutghffugghjhjkjjkl} we have
\begin{equation}\label{vhfffngghPPN}
\begin{cases}
\vec B= curl_{\vec x} \vec A\\
\vec E=-\nabla_{\vec x}\Psi_0-\frac{1}{c}\frac{\partial\vec
A}{\partial t}-\frac{1}{c}\nabla_{\vec x}\left(\vec A\cdot\vec
v\right)
%=-\nabla_{\vec x}\Psi-\frac{1}{c}\frac{\partial\vec A}{\partial t}
\\
 \vec D=-\nabla_{\vec
x}\Psi_0-\frac{1}{c}\left(\frac{\partial\vec A}{\partial t}-\vec
v\times curl_{\vec x}\vec A+\nabla_{\vec x}\left(\vec A\cdot\vec
v\right)\right)
%=-\nabla_{\vec x}\Psi-\frac{1}{c}\frac{\partial\vec A}{\partial t}+\frac{1}{c}\vec v\times curl_{\vec x}\vec A
\\
\vec H\equiv curl_{\vec x} \vec A-\frac{1}{c}\,\vec v\times
%\left(-\nabla_{\vec x}\Psi-\frac{1}{c}\frac{\partial\vec A}{\partial t}+\frac{1}{c}\vec v\times curl_{\vec x}\vec A\right).
\left(-\nabla_{\vec x}\Psi_0+\frac{1}{c}\left(\frac{\partial\vec
A}{\partial t}-\vec v\times curl_{\vec x}\vec A+\nabla_{\vec
x}\left(\vec A\cdot\vec v\right)\right)\right).
%\\ div_{\vec x}\vec A\equiv 0.
\end{cases}
\end{equation}
The electromagnetic potentials are not uniquely defined and thus we
need to choose a calibration. For definiteness we can take $\vec A$
to satisfy
\begin{equation}\label{MaxVacFull1bjkgjhjhgjgjgkjfhjfdghghligioiuittrPPN22}
div_{\vec x}\vec A\equiv 0.
\end{equation}
%
%
%
\begin{comment}
It is clear that if $(\tilde\Psi,\tilde{\vec A})$ is another choice
of electromagnetic potentials with a different calibration then
there exists a scalar field $w:=w(\vec x,t)$ such that
\begin{equation}\label{MaxVacFull1bjkgjhjhgjgjgkjfhjfdghghligioiuittrPPNhjkjhkj}
\begin{cases}
\tilde\Psi=\Psi+\frac{\partial w}{\partial t}\\
\tilde{\vec A}=\vec A-\nabla_{\vec x}w,
\end{cases}
\end{equation}
\end{comment}
%
%
%
It is clear that if $(\tilde\Psi,\tilde\Psi_0,\tilde{\vec A})$ is
another choice of electromagnetic potentials with a different
calibration then there exists a scalar field $w:=w(\vec x,t)$ such
that we have
\begin{equation}\label{MaxVacFull1bjkgjhjhgjgjgkjfhjfdghghligioiuittrPPNhjkjhkjgghhjjhj}
\begin{cases}
\tilde\Psi=\Psi+\frac{1}{c}\frac{\partial w}{\partial t}\\
\tilde{\vec A}=\vec A-\nabla_{\vec x}w\\
\tilde\Psi_0=\Psi_0+\frac{1}{c}\left(\frac{\partial w}{\partial
t}+\vec v\cdot\nabla_{\vec x}w\right).
\end{cases}
\end{equation}







Next consider the change of certain non-inertial cartesian
coordinate system $(*)$ to another cartesian coordinate system
$(**)$:
\begin{equation}\label{noninchredPPN111}
\begin{cases}
\vec x'=A(t)\cdot \vec x+\vec z(t),\\
t'=t,
\end{cases}
\end{equation}
where $A(t)\in SO(3)$ is a rotation i.e. $A(t)\in \R^{3\times 3}$,
$det\, A(t)>0$ and $A(t)\cdot A^T(t)=I$ (here $A^T$ is the transpose
matrix of $A$ and $I$ is the identity matrix). We are going to
investigate, what are the transformations of $(\Psi,\Psi_0,\vec
A)\sim(\Psi',\Psi'_0,\vec A')$ under the change of coordinates,
given by \er{noninchredPPN111}. Since, by
\er{yuythfgfyftydtydtydtyddyyyhhddhhhredPPN} the following relations
are valid
\begin{equation}\label{yuythfgfyftydtydtydtyddyyyhhddhhhredPPN111}
\begin{cases}
\vec D'=A(t)\cdot \vec D,\\
\vec B'=A(t)\cdot\vec B,\\
\vec E'=A(t)\cdot\vec E-\frac{1}{c}\,\left(\frac{dA}{dt}(t)\cdot\vec
x+\frac{d\vec z}{dt}(t)\right)\times \left(A(t)\cdot\vec B\right),\\
\vec H'=A(t)\cdot\vec H+\frac{1}{c}\,\left(\frac{dA}{dt}(t)\cdot\vec
x+\frac{d\vec z}{dt}(t)\right)\times \left(A(t)\cdot\vec D\right),
\end{cases}
\end{equation}
by the second equality in
\er{yuythfgfyftydtydtydtyddyyyhhddhhhredPPN111}, the first equality
in \er{MaxVacFull1bjkgjhjhgjgjgkjfhjfdghghligioiuittrPPN} and
\er{MaxVacFull1bjkgjhjhgjgjgkjfhjfdghghligioiuittrPPN22} we deduce
\begin{equation}\label{yuythfgfyftydtydtydtyddyyyhhddhhhredPPN111222}
\vec A'=A(t)\cdot \vec A,
\end{equation}
i.e. if $\vec A$ satisfies calibration
\er{MaxVacFull1bjkgjhjhgjgjgkjfhjfdghghligioiuittrPPN22} then it is
a proper vector field. On the other hand, by \er{vhfffngghPPN} we
have
\begin{equation}\label{vhfffngghPPNjuiuio}
\nabla_{\vec x}\Psi_0= -\vec D-\frac{1}{c}\left(\frac{\partial\vec
A}{\partial t}-\vec v\times curl_{\vec x}\vec A+\nabla_{\vec
x}\left(\vec A\cdot\vec v\right)\right).
%=-\nabla_{\vec x}\Psi-\frac{1}{c}\frac{\partial\vec A}{\partial t}+\frac{1}{c}\vec v\times curl_{\vec x}\vec A
\end{equation}
Thus by \er{yuythfgfyftydtydtydtyddyyyhhddhhhredPPN111222} and
\er{yuythfgfyftydtydtydtyddyyyhhddhhhredPPN111}, using Proposition
\ref{yghgjtgyrtrt}
%from the Appendix
we deduce that $\nabla_{\vec
x}\Psi_0$ is a proper vector field, i.e.
\begin{equation}\label{vhfffngghhjghhgPPNghghghuigigg}
\nabla_{\vec x'}\Psi'_0=A(t)\cdot\nabla_{\vec x}\Psi_0.
\end{equation}
So
\begin{equation}\label{vhfffngghhjghhgPPNghghghutghffohjh}
\Psi'_0=\Psi_0,
\end{equation}
i.e. $\Psi_0$ is a is proper scalar field, invariant under the
change of non-inertial cartesian coordinate systems. This explains
why we called $\Psi_0$ the proper scalar electromagnetic potential.
%
%
%
\begin{comment}
On the other hand, by \er{apfrm7} and \er{vhfffngghPPN} using
\er{apfrm6} we deduce
\begin{multline}\label{vhfffngghhjghhgPPN}
\nabla_{\vec x}\left(\frac{1}{c}\vec A\cdot\vec v-\Psi\right)=
\frac{1}{c}\nabla_{\vec x}\left(\vec A\cdot\vec v\right)+\vec
D+\frac{1}{c}\frac{\partial\vec A}{\partial t}-\frac{1}{c}\vec
v\times curl_{\vec x}\vec A\\=\frac{1}{c}\nabla_{\vec x}\left(\vec
A\cdot\vec v\right)+\vec D+\frac{1}{c}\frac{\partial\vec A}{\partial
t}+\frac{1}{c}d_{\vec x}\vec A\cdot\vec v-\frac{1}{c}\left\{d_{\vec
x}\vec A\right\}^T\cdot\vec v\\=\vec D+\frac{1}{c}\frac{\partial\vec
A}{\partial t}+\frac{1}{c}d_{\vec x}\vec A\cdot\vec v+
\frac{1}{c}\left\{d_{\vec x}\vec v\right\}^T\cdot\vec A=\vec
D+\frac{1}{c}\frac{\partial\vec A}{\partial t}+\frac{1}{c}d_{\vec
x}\vec A\cdot\vec v-\frac{1}{c}d_{\vec x}\vec v\cdot\vec A+
\frac{1}{c}\left(d_{\vec x}\vec v+\left\{d_{\vec x}\vec
v\right\}^T\right)\cdot\vec A\\= \vec
D+\frac{1}{c}\left(\frac{\partial\vec A}{\partial t}-curl_{\vec
x}\left(\vec v\times\vec A\right)+\left(div_{\vec x}\vec
A\right)\vec v\right)-\frac{1}{c}\left(div_{\vec x}\vec v\right)\vec
A+ \frac{1}{c}\left(d_{\vec x}\vec v+\left\{d_{\vec x}\vec
v\right\}^T\right)\cdot\vec A.
\end{multline}
Thus by \er{vhfffngghhjghhgPPN} and
\er{yuythfgfyftydtydtydtyddyyyhhddhhhredPPN111222}, using
Proposition \ref{yghgjtgyrtrt}
%from the Appendix
we deduce
\begin{equation}\label{vhfffngghhjghhgPPNghghgh}
\nabla_{\vec x'}\left(\frac{1}{c}\vec A'\cdot\vec
v'-\Psi'\right)=A(t)\cdot\nabla_{\vec x}\left(\frac{1}{c}\vec
A\cdot\vec v-\Psi\right)
\end{equation}
So
\begin{equation}\label{vhfffngghhjghhgPPNghghghutghff}
\left(\frac{1}{c}\vec A'\cdot\vec
v'-\Psi'\right)=\left(\frac{1}{c}\vec A\cdot\vec v-\Psi\right),
\end{equation}
i.e. the scalar field $\zeta:=\left(\frac{1}{c}\vec A'\cdot\vec
v'-\Psi'\right)$ is proper scalar field, invariant under the change
of non-inertial cartesian coordinate systems.
\end{comment}
%
%
%
Then by \er{vhfffngghhjghhgPPNghghghutghffohjh} and
\er{vhfffngghhjghhgPPNghghghutghffugghjhjkjjkl} we deduce
\begin{equation}\label{vhfffngghhjghhgPPNghghghutghffghhg}
\left(\frac{1}{c}\vec A'\cdot\vec
v'-\Psi'\right)=\left(\frac{1}{c}\vec A\cdot\vec v-\Psi\right).
\end{equation}
Therefore, by \er{vhfffngghhjghhgPPNghghghutghffghhg},
\er{yuythfgfyftydtydtydtyddyyyhhddhhhredPPN111222}  and the fact
that
\begin{equation}
\label{NoIn5redPPNghjg}\vec v'=A(t)\cdot \vec
v+\frac{dA}{dt}(t)\cdot\vec x+\frac{d\vec z}{dt}(t),
\end{equation}
we deduce
\begin{equation}\label{vhfffngghhjghhgPPNghghghutghfflklhjkj}
\frac{1}{c}\vec A\cdot\left(\vec
v+A^T(t)\cdot\frac{dA}{dt}(t)\cdot\vec x+A^T(t)\cdot\frac{d\vec
z}{dt}(t)\right)-\Psi'=\frac{1}{c}\vec A\cdot\vec v-\Psi.
\end{equation}
So
\begin{equation}\label{vhfffngghhjghhgPPNghghghutghfflklhjkjhjg}
\Psi'=\Psi+\frac{1}{c}\vec
A\cdot\left(A^T(t)\cdot\frac{dA}{dt}(t)\cdot\vec
x+A^T(t)\cdot\frac{d\vec
z}{dt}(t)\right)=\Psi+\frac{1}{c}\left(A(t)\cdot\vec
A\right)\cdot\left(\frac{dA}{dt}(t)\cdot\vec x+\frac{d\vec
z}{dt}(t)\right).
\end{equation}
Therefore, under the change of some non-inertial cartesian
coordinate system $(*)$ to another cartesian coordinate system
$(**)$, given by \er{noninchredPPN111}, the electromagnetic
potentials transform as:
\begin{equation}\label{vhfffngghhjghhgPPNghghghutghfflklhjkjhjhjjgjkghhj}
\begin{cases}
\Psi'=
%\Psi+\frac{1}{c}\vec A\cdot\left(A^T(t)\cdot\frac{dA}{dt}(t)\cdot\vec x+A^T(t)\cdot\frac{d\vec z}{dt}(t)\right)
\Psi+\frac{1}{c}\left(\frac{dA}{dt}(t)\cdot\vec x+\frac{d\vec
z}{dt}(t)\right)\cdot\left(A(t)\cdot\vec A\right)
\\
\vec A'=A(t)\cdot \vec A\\
\Psi'_0:=\left(\Psi'-\frac{1}{c}\vec A'\cdot\vec
v'\right)=\Psi_0:=\left(\Psi-\frac{1}{c}\vec A\cdot\vec v\right).
\end{cases}
\end{equation}
In particular, under the Galilean transformations
\er{noninchgravortbstrjgghguittu1} the electromagnetic potentials
transform as:
%
%
%
\begin{comment}
\begin{equation}\label{noninchgravortbstrjgghguittu1GG}
\begin{cases}
\vec x'=\vec x+\vec wt,\\
t'=t,
\end{cases}
\end{equation}
\end{comment}
%
%
%
\begin{equation}\label{vhfffngghhjghhgPPNghghghutghfflklhjkjhjhjjgjkghhjhhhjhgjgu}
\begin{cases}
\Psi'=
%\Psi+\frac{1}{c}\vec A\cdot\left(A^T(t)\cdot\frac{dA}{dt}(t)\cdot\vec x+A^T(t)\cdot\frac{d\vec z}{dt}(t)\right)
\Psi+\frac{1}{c}\vec w\cdot\vec A
\\
\vec A'=\vec A\\
\Psi'_0=\Psi_0.
%\\ \left(\frac{1}{c}\vec A'\cdot\vec v'-\Psi'\right)=\left(\frac{1}{c}\vec A\cdot\vec v-\Psi\right).
\end{cases}
\end{equation}
In the proof of
\er{vhfffngghhjghhgPPNghghghutghfflklhjkjhjhjjgjkghhj} we used
equality \er{MaxVacFull1bjkgjhjhgjgjgkjfhjfdghghligioiuittrPPN22}
only for proof of equality
\er{yuythfgfyftydtydtydtyddyyyhhddhhhredPPN111222}. Thus relations
\er{vhfffngghhjghhgPPNghghghutghfflklhjkjhjhjjgjkghhj} are still
valid for every choice of calibration of $(\Psi,\Psi_0,\vec A)$,
which implies \er{yuythfgfyftydtydtydtyddyyyhhddhhhredPPN111222}. In
particular if $w$ is a proper scalar field i.e. $w'=w$ and if
$(\tilde\Psi,\tilde\Psi_0,\tilde{\vec A})$ is another choice of
electromagnetic potentials defined by
\begin{equation}\label{MaxVacFull1bjkgjhjhgjgjgkjfhjfdghghligioiuittrPPNhjkjhkjgghhjjhjkjggh}
\begin{cases}
\tilde\Psi=\Psi+\frac{1}{c}\frac{\partial w}{\partial t}\\
\tilde{\vec A}=\vec A-\nabla_{\vec x}w\\
\tilde\Psi_0=\Psi_0+\frac{1}{c}\left(\frac{\partial w}{\partial
t}+\vec v\cdot\nabla_{\vec x}w\right),
\end{cases}
\end{equation}
then, by Proposition \ref{yghgjtgyrtrt}
%from the Appendix
we have
\begin{equation}\label{vhfffngghhjghhgPPNghghghutghfflklhjkjhjhjjgjkghhjhjkhjg}
\begin{cases}
\tilde\Psi'= \tilde\Psi+\frac{1}{c}\left(\frac{dA}{dt}(t)\cdot\vec
x+\frac{d\vec z}{dt}(t)\right)\cdot\left(A(t)\cdot\tilde{\vec
A}\right)
\\
%\tilde\Psi'=\tilde\Psi+\frac{1}{c}\left(\frac{dA}{dt}(t)\cdot\vec x+\frac{d\vec z}{dt}(t)\right)\cdot\left(A(t)\cdot\vec A\right)\\
\tilde{\vec A}'=A(t)\cdot \tilde{\vec A}\\
\tilde\Psi'_0=\tilde\Psi_0.
\end{cases}
\end{equation}
On the other hand, we always can fine a proper scalar field $w$ for
calibration to illuminate $\tilde\Psi_0$ in
\er{MaxVacFull1bjkgjhjhgjgjgkjfhjfdghghligioiuittrPPNhjkjhkjgghhjjhjkjggh}.
%
%
%
%
\begin{comment}
or to make it to be equal to the Coulomb's Newtonian potential, i.e.
to satisfy
%$\psi_0(\vec x,t)$, which satisfies
\begin{equation}\label{columbPPNb}
-\Delta_{\vec x}\tilde\Psi_0\equiv 4\pi\rho.
\end{equation}
For the first choice of calibration

For the second case we have
\begin{equation}\label{vhfffngghPPNhjkhjjklk}
\begin{cases}
-\Delta_{\vec x}\tilde\Psi_0\equiv 4\pi\rho
\\
-\frac{1}{c}div_{\vec x}\left\{\frac{\partial\tilde{\vec
A}}{\partial t}-\vec v\times curl_{\vec x}\tilde{\vec
A}+\nabla_{\vec x}\left(\tilde{\vec A}\cdot\vec v\right)\right\}=0
\\
\vec B= curl_{\vec x} \tilde{\vec A}\\
\vec E=-\nabla_{\vec
x}\tilde\Psi_0-\frac{1}{c}\frac{\partial\tilde{\vec A}}{\partial
t}-\frac{1}{c}\nabla_{\vec x}\left(\tilde{\vec A}\cdot\vec v\right)
%=-\nabla_{\vec x}\Psi-\frac{1}{c}\frac{\partial\vec A}{\partial t}
\\
 \vec D=-\nabla_{\vec
x}\tilde\Psi_0-\frac{1}{c}\left(\frac{\partial\tilde{\vec
A}}{\partial t}-\vec v\times curl_{\vec x}\tilde{\vec
A}+\nabla_{\vec x}\left(\tilde{\vec A}\cdot\vec v\right)\right)
%=-\nabla_{\vec x}\Psi-\frac{1}{c}\frac{\partial\vec A}{\partial t}+\frac{1}{c}\vec v\times curl_{\vec x}\vec A
\\
\vec H\equiv curl_{\vec x} \tilde{\vec A}-\vec v\times
%\left(-\nabla_{\vec x}\Psi-\frac{1}{c}\frac{\partial\vec A}{\partial t}+\frac{1}{c}\vec v\times curl_{\vec x}\vec A\right).
\left(\nabla_{\vec
x}\tilde\Psi_0+\frac{1}{c}\left(\frac{\partial\tilde{\vec
A}}{\partial t}-\vec v\times curl_{\vec x}\tilde{\vec
A}+\nabla_{\vec x}\left(\tilde{\vec A}\cdot\vec
v\right)\right)\right).
%\\ div_{\vec x}\vec A\equiv 0.
\end{cases}
\end{equation}
%where $\tilde\Psi_0$ satisfies \er{columbPPNb}.
In both cases
\end{comment}
%
%
%
%
Then we have $\tilde\Psi_0\equiv 0$ and the electromagnetic fields
are fully represented by the vectorial electromagnetic potential
$\tilde{\vec A}$ analogously as the vectorial gravitational
potential represents the gravitational field. For this case, we
rewrite \er{vhfffngghPPN} as
\begin{equation}\label{vhfffngghPPNhjkhj}
\begin{cases}
\tilde\Psi_0=0\\
-\frac{1}{c}div_{\vec x}\left\{\frac{\partial\tilde{\vec
A}}{\partial t}-\vec v\times curl_{\vec x}\tilde{\vec
A}+\nabla_{\vec x}\left(\tilde{\vec A}\cdot\vec
v\right)\right\}=4\pi \rho
\\
\vec B= curl_{\vec x} \tilde{\vec A}\\
\vec E=-\frac{1}{c}\frac{\partial\tilde{\vec A}}{\partial
t}-\frac{1}{c}\nabla_{\vec x}\left(\tilde{\vec A}\cdot\vec v\right)
%=-\nabla_{\vec x}\Psi-\frac{1}{c}\frac{\partial\vec A}{\partial t}
\\
 \vec D=-\frac{1}{c}\left(\frac{\partial\tilde{\vec A}}{\partial
t}-\vec v\times curl_{\vec x}\tilde{\vec A}+\nabla_{\vec
x}\left(\tilde{\vec A}\cdot\vec v\right)\right)
%=-\nabla_{\vec x}\Psi-\frac{1}{c}\frac{\partial\vec A}{\partial t}+\frac{1}{c}\vec v\times curl_{\vec x}\vec A
\\
\vec H\equiv curl_{\vec x} \tilde{\vec A}-\frac{1}{c}\,\vec v\times
%\left(-\nabla_{\vec x}\Psi-\frac{1}{c}\frac{\partial\vec A}{\partial t}+\frac{1}{c}\vec v\times curl_{\vec x}\vec A\right).
\left(\frac{1}{c}\left(\frac{\partial\tilde{\vec A}}{\partial
t}-\vec v\times curl_{\vec x}\tilde{\vec A}+\nabla_{\vec
x}\left(\tilde{\vec A}\cdot\vec v\right)\right)\right).
%\\ div_{\vec x}\vec A\equiv 0.
\end{cases}
\end{equation}
Moreover, in this case
\er{vhfffngghhjghhgPPNghghghutghfflklhjkjhjhjjgjkghhjhjkhjg} is
satisfied.
%
%
%
\begin{comment}
Next by the second equation in \er{vhfffngghPPNhjkhj} and by
\er{vhfffngghhjghhgjlkhjhkPPP} in Proposition \ref{yghgjtgyrtrt} we
deduce
\begin{multline}\label{hbuy8uiyhiyh}
-4\pi c\rho=div_{\vec x}\left\{\frac{\partial\tilde{\vec
A}}{\partial t}-\vec v\times curl_{\vec x}\tilde{\vec
A}+\nabla_{\vec x}\left(\tilde{\vec A}\cdot\vec
v\right)\right\}=\\div_{\vec x}\left\{\frac{\partial\vec A}{\partial
t}+\left(div_{\vec x}\vec A\right)\vec v-\left(div_{\vec x}\vec
v\right)\vec A+ \left(d_{\vec x}\vec v+\left\{d_{\vec x}\vec
v\right\}^T\right)\cdot\vec A\right\} =\\
\frac{\partial}{\partial t}\left(div_{\vec x}\vec A\right)+div_{\vec
x}\left\{\left(div_{\vec x}\vec A\right)\vec v-\left(div_{\vec
x}\vec v\right)\vec A+ \left(d_{\vec x}\vec v+\left\{d_{\vec x}\vec
v\right\}^T\right)\cdot\vec A\right\}
\end{multline}
\end{comment}
%
%
%
%
%
%
%SSD2







\section{Lagrangian of the Electromagnetic field}\label{bhjghjfghfg} We would like to
present a Lagrangian and a variational principle for the
electromagnetic field and to obtain the Maxwell equations in the
form \er{MaxVacFull1bjkgjhjhgjaaaPPN} from this principle. Given
known the charge distribution $\rho:=\rho(\vec x,t)$, the current
distribution $\vec j:=\vec j(\vec x,t)$ and the vectorial
gravitational potential $\vec v:=\vec v(\vec x,t)$, consider a
Lagrangian density $L_1$ defined by
\begin{multline}\label{vhfffngghkjgghPPN}
L_1\left(\vec A,\Psi,\vec
x,t\right):=\frac{1}{8\pi}\left|-\nabla_{\vec
x}\Psi-\frac{1}{c}\frac{\partial\vec A}{\partial t}+\frac{1}{c}\vec
v\times curl_{\vec x}\vec A\right|^2-\frac{1}{8\pi}\left|curl_{\vec
x}\vec A\right|^2-\left(\rho\Psi-\frac{1}{c}\vec A\cdot\vec
j\right).
\end{multline}
We investigate stationary points of the functional
\begin{equation}\label{btfffygtgyggyPPN}
J=\int_0^T\int_{\mathbb{R}^3}L_1\left(\vec A,\Psi,\vec
x,t\right)d\vec x dt.
\end{equation}
We denote
\begin{equation}\label{guigjgjffghPPN}
\begin{cases}
\vec D=-\nabla_{\vec x}\Psi-\frac{1}{c}\frac{\partial\vec
A}{\partial t}+\frac{1}{c}\vec
v\times curl_{\vec x}\vec A\\
\vec B=curl_{\vec x}\vec A
\\
\vec E=-\nabla_{\vec x}\Psi-\frac{1}{c}\frac{\partial\vec A}{\partial t}=\vec D-\frac{1}{c}\vec v\times\vec B\\
\vec H=curl_{\vec x}\vec A+\frac{1}{c}\vec
v\times\left(-\nabla_{\vec x}\Psi-\frac{1}{c}\frac{\partial\vec
A}{\partial t}+\frac{1}{c}\vec v\times curl_{\vec x}\vec
A\right)=\vec B+\frac{1}{c}\vec
v\times\vec D\\
\Psi_0:=\Psi-\frac{1}{c}\vec A\cdot\vec v.
\end{cases}
\end{equation}
So we can write:
\begin{multline}\label{vhfffngghkjgghPPNggjgjjkgj}
L_1\left(\vec A,\Psi,\vec x,t\right):=\frac{1}{8\pi}\left|\vec
D\right|^2-\frac{1}{8\pi}\left|\vec
B\right|^2-\left(\rho\Psi-\frac{1}{c}\vec A\cdot\vec
j\right)\\=\frac{1}{8\pi}\left|\vec
D\right|^2-\frac{1}{8\pi}\left|\vec
B\right|^2-\rho\Psi_0+\frac{1}{c}\vec A\cdot(\vec j-\rho\vec v),
\end{multline}
and by \er{guigjgjffghPPN} we have:
\begin{equation}\label{guigjgjffghjhkkgPPN}
\begin{cases}
curl_{\vec x}\vec E+\frac{1}{c}\frac{\partial\vec B}{\partial t}=0\\
div_{\vec x}\vec B=0.
\end{cases}
\end{equation}
Moreover by \er{vhfffngghkjgghPPN} and \er{apfrm3} we have
\begin{equation}\label{vhfffngghkjgghggtghjgfhjhjkghghPPN}
0=\frac{\delta L_1}{\delta \Psi}=\frac{1}{4\pi}div_{\vec x}\vec
D-\rho,
\end{equation}
and
\begin{equation}\label{vhfffngghkjgghggtghjgfhjhjkghghyuiuuPPN}
0=\frac{\delta L_1}{\delta \vec A}=\frac{1}{c}\vec j+\frac{1}{4\pi
c}\frac{\partial\vec D}{\partial t}-\frac{1}{4\pi}curl_{\vec x}\vec
B-\frac{1}{4\pi c}curl_{\vec x}\left(\vec v\times \vec
D\right)=\frac{1}{c}\vec j+\frac{1}{4\pi c}\frac{\partial\vec
D}{\partial t}-\frac{1}{4\pi}curl_{\vec x}\vec H.
\end{equation}
So by \er{vhfffngghkjgghggtghjgfhjhjkghghPPN},
\er{vhfffngghkjgghggtghjgfhjhjkghghyuiuuPPN}, \er{guigjgjffghPPN}
and \er{guigjgjffghjhkkgPPN} we obtain the Maxwell equations in the
form:
\begin{equation}\label{guigjgjffghguygjyfPPN}
\begin{cases}
curl_{\vec x}\vec H=\frac{4\pi}{c}\vec j+\frac{\partial\vec
D}{\partial
t}\\
div_{\vec x}\vec D=4\pi\rho\\
curl_{\vec x}\vec E+\frac{1}{c}\frac{\partial\vec B}{\partial t}=0\\
div_{\vec x}\vec B=0\\
\vec E=\vec D-\frac{1}{c}\vec v\times\vec B\\
\vec H=\vec B+\frac{1}{c}\vec v\times\vec D.
\end{cases}
\end{equation}
Note also that, using \er{vhfffngghkjgghPPNggjgjjkgj}, by
\er{vhfffngghhjghhgPPNghghghutghfflklhjkjhjhjjgjkghhj} and
\er{yuythfgfyftydtydtydtyddyyyhhddhhhredPPN111} the Lagrangian $L_1$
is invariant, under the change of inertial or non-inertial
coordinate system, given by \er{noninchredPPN111}, i.e. for this
change we have
\begin{equation}\label{vhfffngghkjgghPPNggjgjjkgjjhhjkghg}
L'_1\left(\vec A',\Psi',\vec x',t'\right)=L_1\left(\vec A,\Psi,\vec
x,t\right).
\end{equation}


































\section{Local gravitational time and Maxwell equations in a
non-rotating coordinate system}\label{GIGIGU}
Throughout this
section consider an inertial or more generally a non-rotating
cartesian coordinate system $\bf{(*)}$. Then, as before, in this
system we have
\begin{equation}\label{jhhjgjhuiiu}
\vec v(\vec x,t)=\nabla_{\vec x}Z(\vec x,t),
\end{equation}
where $\vec v$ is the vectorial gravitational potential and $Z$ is a
scalar field. Then define a scalar field $\tau:=\tau(\vec x,t)$ by
the following:
\begin{equation}\label{jhhjgjhuiiuiy}
\tau(\vec x,t)=t+\frac{1}{c^2}Z(\vec x,t).
\end{equation}
We call the quantity $\tau(\vec x,t)$ by the name local
gravitational time. The name "local" and "gravitational" is quite
clear, since $\tau$ depend on the space and time variables and
derived by characteristic function $Z$ of the gravitational field.
The name "time" will be clarified bellow. Note also that, using
\er{noninchgravortbstrjgghguittu1intmmjhhj} in remark \ref{ugyugg},
one can easily deduce that under the change of inertial coordinate
system $\bf{(*)}$ to $\bf{(**)}$ given by the Galilean
Transformation
\begin{equation}\label{noninchgravortbstrjgghguittu1intmmkkkk}
\begin{cases}
\vec x'=\vec x+\vec wt,\\
t'=t,
\end{cases}
\end{equation}
the local gravitational time $\tau$ transforms as:
\begin{multline}\label{noninchgravortbstrjgghguittu1intmmjhhjjj}
\tau'(\vec x',t'):= t'+\frac{1}{c^2}Z'(\vec
x',t')=\left(1+\frac{|\vec w|^2}{2c^2}\right)t+\frac{1}{c^2}Z(\vec
x,t)+ \frac{1}{c^2}\vec w\cdot\vec x\\=\tau(\vec
x,t)+\frac{1}{c^2}\vec w\cdot\vec x+\frac{|\vec w|^2}{2c^2}t \approx
\tau(\vec x,t)+\frac{1}{c^2}\vec w\cdot\vec x,
\end{multline}
where the last equality in
\er{noninchgravortbstrjgghguittu1intmmjhhjjj} is valid if
$\frac{|\vec w|^2}{c^2}\ll 1$. So, under
\er{noninchgravortbstrjgghguittu1intmmkkkk} we have:
\begin{equation}\label{noninchgravortbstrjgghguittu1intmmjhhjhjhj}
\tau'\,=\,\tau+\frac{1}{c^2}\vec w\cdot\vec x+\frac{|\vec
w|^2}{2c^2}t \,\approx\, \tau+\frac{1}{c^2}\vec w\cdot\vec x,
\end{equation}
Next consider the Maxwell equations in the vacuum of the form:
\begin{equation}\label{MaxMedFullGGffgguiuiouioKK}
\begin{cases}
curl_{\vec x} \vec H= \frac{4\pi}{c}\vec j+
\frac{1}{c}\frac{\partial \vec D}{\partial t},\\
div_{\vec x}\vec D= 4\pi\rho,\\
curl_{\vec x} \vec E+\frac{1}{c}\frac{\partial \vec B}{\partial t}=0,\\
div_{\vec x} \vec B=0,\\
\vec E=
%\gamma_0
\vec D-\frac{1}{c}\,\vec v\times \vec B,\\
%\quad\quad\text{for}\;\;(\vec x,t)\in\R^3\times[0,+\infty)\\
\vec H=
%\kappa_0
\vec B+\frac{1}{c}\,\vec v\times \vec D.
%D_0:=E+\frac{1}{c}\,v\times
%B\quad\quad\text{for}\;\;(\vec x,t)\in\R^3\times[0,+\infty)\\
%H:=B+\frac{1}{c}\,v\times
%D_0\quad\quad\text{for}\;\;(\vec x,t)\in\R^3\times[0,+\infty).
\end{cases}
\end{equation}
where $\vec E$ is the electric field, $\vec B$ is the magnetic
field, $\vec D$ is the electric displacement field and $\vec H$ is
the $\vec H$-magnetic field, $\vec v:=\vec v(\vec x,t)=\nabla_{\vec
x}Z(\vec x,t)$ is the vectorial gravitational potential, $\rho$ is
the charge density and $\vec j$ is the current density. Next
consider a curvilinear change of variables given by:
\begin{equation}\label{giuuihjghgghjgj78zzrrZZffhhhggygghghjhvbKK}
\begin{cases}
t'=\tau(\vec x,t):=t+\frac{Z(\vec x,t)}{c^2}\\
\vec x'=\vec x.
\end{cases}
\end{equation}
Then by the chain rule, for every vector field $\vec F$ we have:
\begin{equation}\label{giuuihjghgghjgj78zzrrZZffhhhggygghghjhvbKKkk}
\begin{cases}
\frac{\partial \vec F}{\partial t}=\frac{\partial \vec F}{\partial
t'}\left(1+\frac{1}{c^2}\frac{\partial Z}{\partial t}\right),\\
d_{\vec x}\vec F=d_{\vec x'}\vec F+\frac{1}{c^2}\frac{\partial \vec
F}{\partial
t'}\otimes\nabla_{\vec x}Z,\\
div_{\vec x}\vec F=div_{\vec x'}\vec F+\frac{1}{c^2}\frac{\partial
\vec
F}{\partial t'}\cdot\nabla_{\vec x}Z,\\
curl_{\vec x}\vec F=curl_{\vec x'}\vec F+\frac{1}{c^2}\nabla_{\vec
x}Z\times\frac{\partial \vec F}{\partial t'}.
\end{cases}
\end{equation}
Thus inserting \er{giuuihjghgghjgj78zzrrZZffhhhggygghghjhvbKKkk}
into \er{MaxMedFullGGffgguiuiouioKK}, since $\vec v=\nabla_{\vec
x}Z$, we deduce
\begin{equation}\label{MaxMedFullGGffgguiuiouiogghghKK}
\begin{cases}
curl_{\vec x'}\vec H+\frac{1}{c^2}\vec v\times\frac{\partial \vec
H}{\partial t'}= \frac{4\pi}{c}\vec j+ \frac{1}{c}\frac{\partial
\vec D}{\partial
t'}\left(1+\frac{1}{c^2}\frac{\partial Z}{\partial t}\right),\\
div_{\vec x'}\vec D+\frac{1}{c^2}\frac{\partial \vec
D}{\partial t'}\cdot\vec v= 4\pi\rho,\\
curl_{\vec x'}\vec E+\frac{1}{c^2}\vec v\times\frac{\partial \vec
E}{\partial t'}+\frac{1}{c}\frac{\partial \vec B}{\partial
t'}\left(1+\frac{1}{c^2}\frac{\partial Z}{\partial t}\right)=0,\\
div_{\vec x'}\vec B+\frac{1}{c^2}\frac{\partial \vec
B}{\partial t'}\cdot\vec v=0,\\
\vec E=\vec D-\frac{1}{c}\,\vec v\times \vec B,\\
%\quad\quad\text{for}\;\;(\vec x,t)\in\R^3\times[0,+\infty)\\
\vec H=\vec B+\frac{1}{c}\,\vec v\times \vec D.
%D_0:=E+\frac{1}{c}\,v\times
%B\quad\quad\text{for}\;\;(\vec x,t)\in\R^3\times[0,+\infty)\\
%H:=B+\frac{1}{c}\,v\times
%D_0\quad\quad\text{for}\;\;(\vec x,t)\in\R^3\times[0,+\infty).
\end{cases}
\end{equation}
In particular if $Z$ is independent of $t$ or quasistatic then we
rewrite \er{MaxMedFullGGffgguiuiouiogghghKK} as:
\begin{equation}\label{MaxMedFullGGffgguiuiouiogghghhgghhjhjKK}
\begin{cases}
curl_{\vec x'}\vec H= \frac{4\pi}{c}\vec j+
\frac{1}{c}\frac{\partial }{\partial
t'}\left(\vec D-\frac{1}{c}\,\vec v\times \vec H\right),\\
div_{\vec x'}\vec D+\frac{1}{c}\vec v\cdot curl_{\vec x'}\vec H= 4\pi\left(\rho+\frac{1}{c^2}\,\vec v\cdot\vec j\right),\\
curl_{\vec x'}\vec E+\frac{1}{c}\frac{\partial}{\partial
t'}\left(\vec B+\frac{1}{c}\,\vec v\times \vec E\right)=0,\\
div_{\vec x'}\vec B-\frac{1}{c}\vec v\cdot curl_{\vec x'}\vec E=0,\\
\vec E=\vec D-\frac{1}{c}\,\vec v\times \vec B,\\
%\quad\quad\text{for}\;\;(\vec x,t)\in\R^3\times[0,+\infty)\\
\vec H=\vec B+\frac{1}{c}\,\vec v\times \vec D.
%D_0:=E+\frac{1}{c}\,v\times
%B\quad\quad\text{for}\;\;(\vec x,t)\in\R^3\times[0,+\infty)\\
%H:=B+\frac{1}{c}\,v\times
%D_0\quad\quad\text{for}\;\;(\vec x,t)\in\R^3\times[0,+\infty).
\end{cases}
\end{equation}
I.e.
\begin{equation}\label{MaxMedFullGGffgguiuiouiogghghhgghhjhjhjjgKK}
\begin{cases}
curl_{\vec x'}\vec H= \frac{4\pi}{c}\vec j+
\frac{1}{c}\frac{\partial }{\partial
t'}\left(\vec D-\frac{1}{c}\,\vec v\times \vec H\right),\\
div_{\vec x'}\left(\vec D-\frac{1}{c}\,\vec v\times \vec H\right)= 4\pi\left(\rho+\frac{1}{c^2}\,\vec v\cdot\vec j\right),\\
curl_{\vec x'}\vec E+\frac{1}{c}\frac{\partial}{\partial
t'}\left(\vec B+\frac{1}{c}\,\vec v\times \vec E\right)=0,\\
div_{\vec x'}\left(\vec B+\frac{1}{c}\,\vec v\times \vec E\right)=0,\\
\vec E=\vec D-\frac{1}{c}\,\vec v\times \left(\vec H-\frac{1}{c}\,\vec v\times \vec D\right),\\
%\quad\quad\text{for}\;\;(\vec x,t)\in\R^3\times[0,+\infty)\\
\vec H=\vec B+\frac{1}{c}\,\vec v\times \left(\vec
E+\frac{1}{c}\,\vec v\times \vec B\right)
\\
\vec E=\vec D-\frac{1}{c}\,\vec v\times \vec B,\\
%\quad\quad\text{for}\;\;(\vec x,t)\in\R^3\times[0,+\infty)\\
\vec H=\vec B+\frac{1}{c}\,\vec v\times \vec D.
%D_0:=E+\frac{1}{c}\,v\times
%B\quad\quad\text{for}\;\;(\vec x,t)\in\R^3\times[0,+\infty)\\
%H:=B+\frac{1}{c}\,v\times
%D_0\quad\quad\text{for}\;\;(\vec x,t)\in\R^3\times[0,+\infty).
\end{cases}
\end{equation}
In particular, denoting
\begin{equation}\label{giuuihjghgghjgj78zzrrZZffhhhggygghghjhvbgghhjyuuyKK}
\begin{cases}
\vec E^*:=\vec D-\frac{1}{c}\,\vec v\times \vec H=\vec
E-\frac{1}{c^2}\vec v\times\left(\vec v\times\vec D\right)
\\
\vec H^*:=\vec B+\frac{1}{c}\,\vec v\times \vec E=\vec
H-\frac{1}{c^2}\vec v\times\left(\vec v\times\vec B\right),
\end{cases}
\end{equation}
by \er{MaxMedFullGGffgguiuiouiogghghhgghhjhjhjjgKK} we rewrite the
Maxwell equations in the new curvilinear coordinates in the case of
time independent $\vec v$ as:
\begin{equation}\label{MaxMedFullGGffgguiuiouiogghghhgghhjhjhjjgghhgKK}
\begin{cases}
curl_{\vec x'}\vec H= \frac{4\pi}{c}\vec j+
\frac{1}{c}\frac{\partial \vec E^*}{\partial
t'},\\
div_{\vec x'}\vec E^*= 4\pi\left(\rho+\frac{1}{c^2}\,\vec v\cdot\vec j\right),\\
curl_{\vec x'}\vec E+\frac{1}{c}\frac{\partial\vec H^*}{\partial
t'}=0,\\
div_{\vec x'}\vec H^*=0,\\
\vec E^*=\vec E-\frac{1}{c^2}\vec v\times\left(\vec v\times\vec
D\right)
\\
\vec H^*=\vec H-\frac{1}{c^2}\vec v\times\left(\vec v\times\vec
B\right)\\
\vec E=\vec D-\frac{1}{c}\,\vec v\times \vec B,\\
%\quad\quad\text{for}\;\;(\vec x,t)\in\R^3\times[0,+\infty)\\
\vec H=\vec B+\frac{1}{c}\,\vec v\times \vec D.
%D_0:=E+\frac{1}{c}\,v\times
%B\quad\quad\text{for}\;\;(\vec x,t)\in\R^3\times[0,+\infty)\\
%H:=B+\frac{1}{c}\,v\times
%D_0\quad\quad\text{for}\;\;(\vec x,t)\in\R^3\times[0,+\infty).
\end{cases}
\end{equation}
In particular, in the approximation, up to the order
$\left(\frac{|\vec v|}{c}\right)^2\ll 1$ we have $\vec
E^*\approx\vec E$ and $\vec H^*\approx\vec H$ and then the
approximate Maxwell equations have the form:
\begin{equation}\label{MaxMedFullGGffgguiuiouiogghghhgghhjhjhjjgghhgjhgghhhjkiljklKK}
\begin{cases}
curl_{\vec x'}\vec H= \frac{4\pi}{c}\vec j+
\frac{1}{c}\frac{\partial \vec E}{\partial
t'},\\
div_{\vec x'}\vec E= 4\pi\left(\rho+\frac{1}{c^2}\,\vec v\cdot\vec j\right),\\
curl_{\vec x'}\vec E+\frac{1}{c}\frac{\partial\vec H}{\partial
t'}=0,\\
div_{\vec x'}\vec H=0,\\
\vec E=\vec D-\frac{1}{c}\,\vec v\times \vec B,\\
%\quad\quad\text{for}\;\;(\vec x,t)\in\R^3\times[0,+\infty)\\
\vec H=\vec B+\frac{1}{c}\,\vec v\times \vec D.
%D_0:=E+\frac{1}{c}\,v\times
%B\quad\quad\text{for}\;\;(\vec x,t)\in\R^3\times[0,+\infty)\\
%H:=B+\frac{1}{c}\,v\times
%D_0\quad\quad\text{for}\;\;(\vec x,t)\in\R^3\times[0,+\infty).
\end{cases}
\end{equation}
The first four equations in
\er{MaxMedFullGGffgguiuiouiogghghhgghhjhjhjjgghhgjhgghhhjkiljklKK}
form a following system of equation:
\begin{equation}\label{MaxMedFullGGffgguiuiouiogghghhgghhjhjhjjgghhgjhgghhhjkiljklKKHH}
\begin{cases}
curl_{\vec x'}\vec H= \frac{4\pi}{c}\vec j^*+
\frac{1}{c}\frac{\partial \vec E}{\partial
t'},\\
div_{\vec x'}\vec E= 4\pi\rho^*,\\
curl_{\vec x'}\vec E+\frac{1}{c}\frac{\partial\vec H}{\partial
t'}=0,\\
div_{\vec x'}\vec H=0,
\end{cases}
\end{equation}
where
\begin{equation}\label{fgjyhyfgfjjjk}
\vec j^*:=\vec
j\quad\text{and}\quad\rho^*:=\left(\rho+\frac{1}{c^2}\,\vec
v\cdot\vec j\right)
\end{equation}
The system
\er{MaxMedFullGGffgguiuiouiogghghhgghhjhjhjjgghhgjhgghhhjkiljklKKHH}
coincides with the classical Maxwell equations of the usual
Electrodynamics and is similar to \er{MaxMedFullGGffgguiuiouioKK}
for the case $\vec v\equiv 0$. Therefore, given known $\vec v$,
$\rho$ and $\vec j$,
\er{MaxMedFullGGffgguiuiouiogghghhgghhjhjhjjgghhgjhgghhhjkiljklKKHH}
could be solved as easy as the usual wave equation, for example by
the method of retarded potentials. Then backward to
\er{giuuihjghgghjgj78zzrrZZffhhhggygghghjhvbKK} change of variables
could be made in order to deduce the electromagnetic fields in
coordinates $(\vec x,t)$. Next note that, since we defined $t'=\tau$
all the above clarifies the name "time" of the quantity $\tau$.
Finally, we would like to note that if we have a motion of some
material body with the place $\vec r(t)$ and the velocity $\vec
u(t):=\frac{d\vec r}{dt}(t)$ and we associate the local
gravitational time $\tau$ with this body then clearly
\begin{equation}\label{fgjyhyfgfjjjkjkkj}
d\tau\,=\, \left(1+\frac{1}{c^2}\vec u(t)\cdot \vec v\left(\vec
r(t),t\right)\right)\,dt\,\approx\, dt,
\end{equation}
where the last equality in \er{fgjyhyfgfjjjkjkkj} is valid if we
have
\begin{equation}\label{fgjyhyfgfjjjkkkk}
\left(\frac{|\vec v|}{c}\right)^2\ll
1\quad\text{and}\quad\left(\frac{|\vec u(t)|}{c}\right)^2\ll 1.
\end{equation}
So we can use the local gravitational time $\tau$ in the approximate
calculations instead of the true time $t$.




































\section{Motion of particles in external gravitational-electromagnetic field}
\subsection{Lagrangian of the motion of a finite system of classical
particles in an outer gravitational-electromagnetic
field}\label{hggyugyuy} Given a system of $n$ particles with
inertial masses $m_1,\ldots,m_n$, charges
$\sigma_1,\ldots,\sigma_n$, places $\vec r_1(t),\ldots,\vec r_n(t)$
and velocities $\vec r'_1(t),\ldots, \vec r'_n(t)$ in the outer
gravitational field with vectorial potential $\vec v(\vec x,t)$, the
outer electromagnetical fields with potentials $\vec A(\vec x,t)$
and $\vec \Psi(\vec x,t)$ and additional conservative field with the
classical scalar potential $V(\vec y_1,\ldots,\vec y_n,t)$, consider
a Lagrangian:
\begin{multline}\label{vhfffngghkjgghfjjSYSPN}
L_0\left(\frac{d\vec r_1}{dt},\ldots,\frac{d\vec r_n}{dt},,\vec
r_1,\ldots,\vec r_n,t\right):=\\
\sum_{j=1}^{n}\left\{\frac{m_j}{2}\left|\frac{d\vec r_j}{dt}-\vec
v(\vec r_j,t)\right|^2-\sigma_j\left(\Psi(\vec
r_j,t)-\frac{1}{c}\vec A(\vec r_j,t)\cdot\frac{d\vec
r_j}{dt}\right)\right\}+V\left(\vec r_1,\ldots,\vec r_n,t\right).
\end{multline}
This Lagrangian is invariant under the change of inertial and
non-inertial cartesian coordinate systems. We investigate stationary
points of the functional
\begin{equation}\label{btfffygtgyggyijhhkkSYSPN}
J_0=\int_0^T L_0\left(\frac{d\vec r_1}{dt},\ldots,\frac{d\vec
r_n}{dt},,\vec r_1,\ldots,\vec r_n,t\right)dt.
\end{equation}
Then for every $j=1,\ldots,n$ we have
\begin{multline}\label{vhfffngghkjgghggtghjgfhjoyuiyuyhiyyukukySYSPN}
\frac{\delta L_0}{\delta \vec r_j}=-m_j\frac{d}{dt}\left(\frac{d\vec
r_j}{dt}-\vec v(\vec
r_j,t)\right)-\frac{\sigma_j}{c}\frac{d}{dt}\left(\vec A(\vec
r_j,t)\right)-m_j\left\{\nabla_{\vec x}\vec v(\vec
r_j,t)\right\}^T\cdot\left(\frac{d\vec r_j}{dt}-\vec v(\vec
r_j,t)\right)\\-\sigma_j\left(\nabla_{\vec x}\Psi(\vec
r_j,t)-\frac{1}{c}\left\{d_{\vec x}\vec A(\vec
r_j,t)\right\}^T\cdot\frac{d\vec r_j}{dt}\right)+\nabla_{\vec
y_j}V\left(\vec r_1,\ldots,\vec r_n,t\right)=\\-m_j\frac{d^2\vec
r_j}{dt^2}+m_j\left(\frac{\partial}{\partial t}\vec v(\vec
r_j,t)+\nabla_{\vec x}\left(\frac{1}{2}\left|\vec v(\vec
r_j,t)\right|^2\right)-\frac{1}{c}\frac{d\vec r_j}{dt}\times
curl_{\vec x}\vec v(\vec r_j,t)\right)\\+\sigma_j\left(-\nabla_{\vec
x}\Psi(\vec r_j,t)-\frac{1}{c}\frac{\partial}{\partial t}\left(\vec
A(\vec r_j,t)\right)+\frac{1}{c}\frac{d\vec r_j}{dt}\times
curl_{\vec x}\vec A(\vec r_j,t)\right)+\nabla_{\vec y_j}V\left(\vec
r_1,\ldots,\vec r_n,t\right)=0,
\end{multline}
So denoting
\begin{equation}\label{guigjgjffghguygjyfghgghjfgfSYSPN}
\begin{cases}
\vec E=-\nabla_{\vec x}\Psi-\frac{1}{c}\frac{\partial\vec
A}{\partial t}\\
\vec B=curl_{\vec x}\vec A
\end{cases}
\end{equation}
we rewrite \er{vhfffngghkjgghggtghjgfhjoyuiyuyhiyyukukySYSPN} as
\begin{multline}\label{vhfffngghkjgghggtghjgfhjoyuiyuyhiyyukukyihyuSYSPN}
m_j\frac{d^2\vec r_j}{dt^2}=m_j\left(\frac{\partial}{\partial t}\vec
v(\vec r_j,t)+\nabla_{\vec x}\left(\frac{1}{2}\left|\vec v(\vec
r_j,t)\right|^2\right)-\frac{d\vec r_j}{dt}\times curl_{\vec x}\vec
v(\vec r_j,t)\right)+\sigma_j\vec E(\vec
r_j,t)+\frac{\sigma_j}{c}\frac{d\vec r_j}{dt}\times \vec B(\vec
r_j,t)\\+\nabla_{\vec y_j}V\left(\vec r_1,\ldots,\vec
r_n,t\right)=\sigma_j\vec E(\vec
r_j,t)+\frac{\sigma_j}{c}\frac{d\vec r_j}{dt}\times \vec B(\vec
r_j,t)+\nabla_{\vec y_j}V\left(\vec r_1,\ldots,\vec r_n,t\right)+\\
m_j\left(\frac{\partial}{\partial t}\vec v(\vec r_j,t)+d_{\vec
x}\vec v(\vec r_j,t)\cdot\vec v(\vec r_j,t)-\left(\frac{d\vec
r_j}{dt}-\vec v(\vec r_j,t)\right)\times curl_{\vec x}\vec v(\vec
r_j,t)\right).
\end{multline}
So for each particle we get the second law of Newton, consistent
with
\er{noninchgravortbstrjgghguittu2gjgghhjhghjhjgghgghghghtytythvfghfgghjgg},
including the gravitational and the Lorentz force.




Next for every $j=1,\ldots,n$ define the generalized moment of the
particle $m_j$ by
\begin{equation}\label{guytyurtydftyiujhSYSPN}
\vec P_j:=\nabla_{\vec r'_j}L_0\left(\vec r'_1,\ldots,\vec r'_n,\vec
r_1,\ldots,\vec r_n,t\right)=m_j \frac{d\vec r_j}{dt}-m_j\vec v(\vec
r_j,t)+\frac{\sigma_j}{c}\vec A(\vec r_j,t).
\end{equation}
Then
\begin{equation}\label{guytyurtydftyiujhioyhuyhSYSPN}
\frac{d\vec r_j}{dt}=\frac{1}{m_j}\vec P_j+\vec v(\vec
r_j,t)-\frac{\sigma_j}{m_jc}\vec A(\vec r_j,t).
\end{equation}
Thus if we consider a Hamiltonian
\begin{equation}\label{vhfffngghkjgghfjjhyjjfgSYSPN}
H_0\left(\vec P_1,\ldots,\vec P_n,\vec r_1,\ldots,\vec
r_n,t\right):=\sum_{j=1}^{n}\vec P_j\cdot\frac{d\vec
r_j}{dt}-L_0\left(\frac{d\vec r_1}{dt},\ldots,\frac{d\vec
r_n}{dt},\vec r_1,\ldots,\vec r_n,t\right)
\end{equation}
then by \er{vhfffngghkjgghfjjSYSPN},
\er{vhfffngghkjgghfjjhyjjfgSYSPN} and
\er{guytyurtydftyiujhioyhuyhSYSPN} we have:
\begin{multline}\label{vhfffngghkjgghfjjghghghSYSPN}
H_0\left(\vec P_1,\ldots,\vec P_n,\vec r_1,\ldots,\vec
r_n,t\right)=-V\left(\vec r_1,\ldots,\vec
r_n,t\right)+\sum_{j=1}^{n}\vec P_j\cdot\frac{d\vec
r_j}{dt}\\
-\sum_{j=1}^{n}\left(\frac{m_j}{2}\left|\frac{d\vec r_j}{dt}-\vec
v(\vec r_j,t)\right|^2-\sigma_j\left(\Psi(\vec
r_j,t)-\frac{1}{c}\vec A(\vec
r_j,t)\cdot\frac{d\vec r_j}{dt}\right)\right)=\\
\sum_{j=1}^{n}\vec P_j\cdot\left(\frac{1}{m_j}\vec P_j+\vec v(\vec
r_j,t)-\frac{\sigma_j}{m_jc}\vec A(\vec
r_j,t)\right)-\sum_{j=1}^{n}\frac{m_j}{2}\left|\frac{1}{m_j}\vec
P_j-\frac{\sigma_j}{m_jc}\vec A(\vec
r_j,t)\right|^2\\+\sum_{j=1}^{n}\sigma_j\left(\Psi(\vec
r_j,t)-\frac{1}{c}\vec A(\vec r_j,t)\cdot\left(\frac{1}{m_j}\vec
P_j+\vec v(\vec r_j,t)-\frac{\sigma_j}{m_jc}\vec A(\vec
r_j,t)\right)\right)-V\left(\vec r_1,\ldots,\vec r_n,t\right) =\\
%\sum_{j=1}^{n}\vec P_j\cdot\vec v(\vec r_j,t)+ \sum_{j=1}^{n}\vec P_j\cdot\left(\frac{1}{m_j}\vec P_j-\frac{\sigma_j}{m_jc}\vec A(\vec r_j,t)\right)-\sum_{j=1}^{n}\frac{m_j}{2}\left|\frac{1}{m_j}\vec
%P_j-\frac{\sigma_j}{m_jc}\vec A(\vec r_j,t)\right|^2\\-\sum_{j=1}^{n}\frac{\sigma_j}{c}\vec A(\vec r_j,t)\cdot\left(\frac{1}{m_j}\vec P_j-\frac{\sigma_j}{m_jc}\vec A(\vec r_j,t)\right)-\sum_{j=1}^{n}\sigma_j\left(\Psi(\vec
%r_j,t)-\frac{1}{c}\vec A(\vec r_j,t)\cdot\vec v(\vec r_j,t)\right)-V\left(\vec r_1,\ldots,\vec r_n,t\right)=\\
\sum_{j=1}^{n}\vec P_j\cdot\vec v(\vec r_j,t)+
\sum_{j=1}^{n}\frac{1}{2m_j}\left|\vec P_j-\frac{\sigma_j}{c}\vec
A(\vec r_j,t)\right|^2+\sum_{j=1}^{n}\sigma_j\left(\Psi(\vec
r_j,t)-\frac{1}{c}\vec A(\vec r_j,t)\cdot\vec v(\vec
r_j,t)\right)-V\left(\vec r_1,\ldots,\vec r_n,t\right).
\end{multline}






















\subsection{Shr\"{o}dinger equation for a finite system of quantum
particles}\label{hggyugyuy1} Consider the motion of a system of $n$
quantum micro-particle with inertial masses $m_1,\ldots,m_n$ and the
charges $\sigma_1,\ldots,\sigma_n$ in the outer gravitational and
electromagnetical field with characteristics $\vec v(\vec x,t)$,
$\vec A(\vec x,t)$ and $\Psi(\vec x,t)$ and additional conservative
field with potential $V(\vec y_1,\ldots,\vec y_n,t)$, not taking
into account the spin interaction. The Shr\"{o}dinger equation for
this system of particles is
\begin{equation}\label{vhfffngghkjgghfjjghghghhjghjgghkghggkghghjghSYSPN}
i\hbar\frac{\partial\psi}{\partial t}=\hat H_0\cdot\psi,
\end{equation}
where $\psi(\vec x_1,\ldots,\vec x_n,t)\in\mathbb{C}$ is a wave
function and $\hat H_0$ is the Hamiltonian operator. Since by
\er{vhfffngghkjgghfjjghghghSYSPN} the Hamiltonian for a
macro-particles has the form
\begin{multline}\label{vhfffngghkjgghfjjghghghhjghjgghkghggSYSPN}
H_{\text{macro}}\left(\vec P_1,\ldots,\vec P_n,\vec r_1,\ldots,\vec
r_n,t\right)=-V\left(\vec r_1,\ldots,\vec r_n,t\right)+
\sum_{j=1}^{n}\frac{1}{2}\vec P_j\cdot\vec v(\vec
r_j,t)+\sum_{j=1}^{n}\frac{1}{2}\vec v(\vec r_j,t)\cdot\vec P_j\\+
\sum_{j=1}^{n}\frac{1}{2m_j}\left|\vec P_j-\frac{\sigma_j}{c}\vec
A(\vec r_j,t)\right|^2+\sum_{j=1}^{n}\sigma_j\left(\Psi(\vec
r_j,t)-\frac{1}{c}\vec A(\vec r_j,t)\cdot\vec v(\vec r_j,t)\right),
\end{multline}
we built the Hamiltonian operator as
\begin{multline}\label{vhfffngghkjgghfjjghghghhjghjgghkghgghjhggjjkgSYSPN}
\hat H_0\cdot\psi=-\sum_{j=1}^{n}\frac{i\hbar}{2}div_{\vec
x_j}\left\{\psi\vec v(\vec
x_j,t)\right\}-\sum_{j=1}^{n}\frac{i\hbar}{2}\vec v(\vec
x_j,t)\cdot\nabla_{\vec x_j}\psi\\+
\sum_{j=1}^{n}\left\{\frac{1}{2m_j}\left(-i\hbar\nabla_{\vec
x_j}-\frac{\sigma_j}{c}\vec A(\vec
x_j,t)\right)\circ\left(-i\hbar\nabla_{\vec
x_j}-\frac{\sigma_j}{c}\vec A(\vec
x_j,t)\right)\right\}\cdot\psi\\+\sum_{j=1}^{n}\sigma_j\left(\Psi(\vec
x_j,t)-\frac{1}{c}\vec A(\vec x_j,t)\cdot\vec v(\vec
x_j,t)\right)\cdot\psi-V\left(\vec x_1,\ldots,\vec
x_n,t\right)\cdot\psi=
-\sum_{j=1}^{n}\frac{\hbar^2}{2m_j}\Delta_{\vec
x_j}\psi-\sum_{j=1}^{n}\frac{i\hbar}{2}div_{\vec x_j}\left\{\psi\vec
v(\vec x_j,t)\right\}\\-\sum_{j=1}^{n}\frac{i\hbar}{2}\vec v(\vec
x_j,t)\cdot\nabla_{\vec x_j}\psi+\sum_{j=1}^{n}\frac{
i\hbar\sigma_j}{2m_jc}div_{\vec x_j}\left\{\psi\vec A(\vec
x_j,t)\right\}+\sum_{j=1}^{n}\frac{ i\hbar\sigma_j}{2m_jc}\vec
A(\vec x_j,t)\cdot\nabla_{\vec
x_j}\psi\\+\sum_{j=1}^{n}\left(\sigma_j\Psi(\vec
x_j,t)-\frac{\sigma_j}{c}\vec A(\vec x_j,t)\cdot\vec v(\vec
x_j,t)+\frac{\sigma^2_j}{2m_jc^2}\left|\vec A(\vec
x_j,t)\right|^2\right)\psi-V\left(\vec x_1,\ldots,\vec
x_n,t\right)\psi,
\end{multline}
Thus the corresponding Shr\"{o}dinger equation will be
\begin{multline}\label{vhfffngghkjgghfjjghghghhjghjgghkghgghjhggjjkgfgdSYSPN}
i\hbar\frac{\partial\psi}{\partial t}=\hat H_0\cdot\psi=
-\sum_{j=1}^{n}\frac{\hbar^2}{2m_j}\Delta_{\vec
x_j}\psi-\sum_{j=1}^{n}\frac{i\hbar}{2}div_{\vec x_j}\left\{\psi\vec
v(\vec x_j,t)\right\}-\sum_{j=1}^{n}\frac{i\hbar}{2}\vec v(\vec
x_j,t)\cdot\nabla_{\vec x_j}\psi\\+\sum_{j=1}^{n}\frac{
i\hbar\sigma_j}{2m_jc}div_{\vec x_j}\left\{\psi\vec A(\vec
x_j,t)\right\}+\sum_{j=1}^{n}\frac{ i\hbar\sigma_j}{2m_jc}\vec
A(\vec x_j,t)\cdot\nabla_{\vec
x_j}\psi\\+\sum_{j=1}^{n}\left(\sigma_j\Psi(\vec
x_j,t)-\frac{\sigma_j}{c}\vec A(\vec x_j,t)\cdot\vec v(\vec
x_j,t)+\frac{\sigma^2_j}{2m_jc^2}\left|\vec A(\vec
x_j,t)\right|^2\right)\psi-V\left(\vec x_1,\ldots,\vec
x_n,t\right)\psi.
\end{multline}
So
%
%
%
\begin{comment}
\begin{multline}\label{vhfffngghkjgghfjjghghghhjghjgghkghgghjhggjjkgfgdiyfgfSYS}
i\hbar\left(\frac{\partial\psi}{\partial t}+\vec v(\vec
x,t)\cdot\nabla_{\vec x}\psi\right)+\frac{i\hbar}{2}\left(div_{\vec
x}\vec v(\vec x,t)\right)\psi= -\frac{\hbar^2}{2m}\Delta_{\vec
x}\psi+\frac{ i\hbar\sigma}{2mc}div_{\vec x}\left\{\psi\vec A(\vec
x,t)\right\}+\frac{ i\hbar\sigma}{2mc}\vec A(\vec
x,t)\cdot\nabla_{\vec x}\psi\\+\left(\sigma\Psi(\vec
x,t)-\frac{\sigma}{c}\vec A(\vec x,t)\cdot\vec v(\vec
x,t)+\frac{\sigma^2}{2mc^2}\left|\vec A(\vec
x,t)\right|^2-V\left(\vec x,t\right)\right)\psi,
\end{multline}
\end{comment}
%
%
%
%i.e.
\begin{multline}\label{vhfffngghkjgghfjjghghghhjghjgghkghgghjhggjjkgfgdiyfgfjkjgjgSYSPN}
i\hbar\left(\frac{\partial\psi}{\partial t}+\sum_{j=1}^{n}\vec
v(\vec x_j,t)\cdot\nabla_{\vec
x_j}\psi\right)+\sum_{j=1}^{n}\frac{i\hbar}{2}\left(div_{\vec
x_j}\vec v(\vec x_j,t)\right)\psi=
-\sum_{j=1}^{n}\frac{\hbar^2}{2m_j}\Delta_{\vec x_j}\psi-V\left(\vec
x_1,\ldots,\vec x_n,t\right)\psi\\+\sum_{j=1}^{n}\frac{
i\hbar\sigma_j}{2m_jc}div_{\vec x_j}\left\{\psi\vec A(\vec
x_j,t)\right\}+\sum_{j=1}^{n}\frac{ i\hbar\sigma_j}{2m_jc}\vec
A(\vec x_j,t)\cdot\nabla_{\vec
x_j}\psi\\+\sum_{j=1}^{n}\left(\sigma_j\Psi(\vec
x_j,t)-\frac{\sigma_j}{c}\vec A(\vec x_j,t)\cdot\vec v(\vec
x_j,t)+\frac{\sigma^2_j}{2m_jc^2}\left|\vec A(\vec
x_j,t)\right|^2\right)\psi.
\end{multline}
Next consider a change of some non-inertial cartesian coordinate
system $(*)$ to another cartesian coordinate system $(**)$ of the
form:
\begin{equation}\label{noninchgravortbstrghgggSYSPN}
\begin{cases}
\vec x'=A(t)\cdot\vec x+\vec z(t),\\
t'=t,
\end{cases}
\end{equation}
where $A(t)\in SO(3)$ is a rotation, i.e. $A(t)\in \R^{3\times 3}$,
$det\, A(t)>0$ and $A(t)\cdot A^T(t)=I$. Then, since
\begin{equation}\label{vyfgjhgjhvhgghSYSPN}
\begin{cases}
\psi'=\psi
\\
V'=V
\\
\vec A'=A(t)\cdot\vec A
\\
\Psi'-\frac{1}{c}\vec A'\cdot\vec v'=\Psi-\frac{1}{c}\vec A\cdot\vec
v,
\end{cases}
\end{equation}
we deduce that  the Shr\"{o}dinger equation of the form
\er{vhfffngghkjgghfjjghghghhjghjgghkghgghjhggjjkgfgdiyfgfjkjgjgSYSPN}
is invariant under the change of non-inertial cartesian coordinate
system. So the quantum mechanical laws are also invariant in every
non-inertial cartesian coordinate system.











Next, assume that in inertial coordinate system $(*)$ we have:
\begin{equation}
\label{MaxVacFull1ninshtrgravortghhghgjkgghklhjgkghghjjkjhjkkggjkhjkhjjhhfhjhklkhkhjjklzzzyyyhjggjhgghhjhNWNWBWHWNWSYSPN}
\begin{cases}
curl_{\vec x}\vec v= 0,\\
\frac{\partial\vec v}{\partial t}+d_\vec x\vec v\cdot\vec v=
-\nabla_{\vec x}\Phi,
%\frac{d\vec u}{dt}=-curl_{\vec x}\vec v\times(\vec u-\vec v)+\partial_t\vec v+(\nabla_{\vec x}\vec v)\cdot\vec v+\nabla_{\vec x}\psi_0+\frac{1}{m}\vec F.
\end{cases}
\end{equation}
%
%
%
\begin{comment}
\begin{equation}
\label{MaxVacFull1ninshtrgravortghhghgjkgghklhjgkghghjjkjhjkkggjkhjkhjjhhfhjhklkhkhjjklzzzyyyhjggjhgghhjhNWBWHWNWSYSPN}
\begin{cases}
curl_{\vec x}\vec v= 0,\\
div_{\vec x}\left\{\frac{\partial\vec v}{\partial t}+d_\vec x\vec
v\cdot\vec v\right\}= -\Delta_{\vec x}\Phi,
%\frac{d\vec u}{dt}=-curl_{\vec x}\vec v\times(\vec u-\vec v)+\partial_t\vec v+(\nabla_{\vec x}\vec v)\cdot\vec v+\nabla_{\vec x}\psi_0+\frac{1}{m}\vec F.
\end{cases}
\end{equation}
We can rewrite
\er{MaxVacFull1ninshtrgravortghhghgjkgghklhjgkghghjjkjhjkkggjkhjkhjjhhfhjhklkhkhjjklzzzyyyhjggjhgghhjhNWBWHWNWSYSPN}
as
\end{comment}
%
%
%
where $\Phi$ is the scalar gravitational potential. Since in the
system $(*)$ we have $curl_{\vec x}\vec v=0$ we can rewrite
\er{MaxVacFull1ninshtrgravortghhghgjkgghklhjgkghghjjkjhjkkggjkhjkhjjhhfhjhklkhkhjjklzzzyyyhjggjhgghhjhNWNWBWHWNWSYSPN}
as
\begin{equation}
\label{MaxVacFull1ninshtrgravortghhghgjkgghklhjgkghghjjkjhjkkggjkhjkhjjhhfhjhklkhkhjjklzzzyyyhjggjhgghhjhNWNWNWNWNWBWHWNWSYSPN}
\begin{cases}
\vec v=\nabla_{\vec x}Z,\\
\frac{\partial Z}{\partial t}+\frac{1}{2}\left|\nabla_{\vec
x}Z\right|^2=-\Phi.
%\frac{d\vec u}{dt}=-curl_{\vec x}\vec v\times(\vec u-\vec v)+\partial_t\vec v+(\nabla_{\vec x}\vec v)\cdot\vec v+\nabla_{\vec x}\psi_0+\frac{1}{m}\vec F.
\end{cases}
\end{equation}
Thus by
\er{MaxVacFull1ninshtrgravortghhghgjkgghklhjgkghghjjkjhjkkggjkhjkhjjhhfhjhklkhkhjjklzzzyyyhjggjhgghhjhNWNWNWNWNWBWHWNWSYSPN}
%, using the fact that $div_{\vec x}\vec A=0$
we rewrite
\er{vhfffngghkjgghfjjghghghhjghjgghkghgghjhggjjkgfgdiyfgfjkjgjgSYSPN}
as
\begin{multline}\label{vhfffngghkjgghfjjghghghhjghjgghkghgghjhggjjkgfgdiyfgfjkjgjggjjgSYSPN}
i\hbar\frac{\partial\psi}{\partial
t}+\sum_{j=1}^{n}i\hbar\nabla_{\vec x_j}Z(\vec
x_j,t)\cdot\nabla_{\vec
x_j}\psi+\sum_{j=1}^{n}\frac{i\hbar}{2}\left(\Delta_{\vec x_j}Z(\vec
x_j,t)\right)\psi +\sum_{j=1}^{n}\frac{\hbar^2}{2m_j}\Delta_{\vec
x_j}\psi=\\
\sum_{j=1}^{n}\frac{ i\hbar\sigma_j}{2m_jc}\left(div_{\vec x_j}\vec
A(\vec x_j,t)\right)\psi
%+\frac{i\hbar\sigma}{2mc}div_{\vec x}\left\{\psi\vec A\right\}
+\sum_{j=1}^{n}\frac{ i\hbar\sigma_j}{m_jc}\vec A(\vec
x_j,t)\cdot\nabla_{\vec
x_j}\psi-\sum_{j=1}^{n}\frac{\sigma_j}{c}\left(\vec A(\vec
x_j,t)\cdot\nabla_{\vec x_j}Z(\vec
x_j,t)\right)\psi\\+\sum_{j=1}^{n}\left(\sigma_j\Psi(\vec
x_j,t)+\frac{\sigma^2_j}{2m_jc^2}\left|\vec A(\vec
x_j,t)\right|^2\right)\psi-V\psi.
\end{multline}
Then multiplying
\er{vhfffngghkjgghfjjghghghhjghjgghkghgghjhggjjkgfgdiyfgfjkjgjggjjgSYSPN}
by factor $e^{\sum_{j=1}^{n}\frac{im_j}{\hbar}Z(\vec x_j,t)}$ gives:
%\er{vhfffngghkjgghfjjghghghhjghjgghkghgghjhggjjkgfgdiyfgfjkjgjgSYSPN} as
\begin{multline}\label{vhfffngghkjgghfjjghghghhjghjgghkghgghjhggjjkgfgdiyfgfjkjgjggjjgugyyjSYSPN}
i\hbar\frac{\partial\psi}{\partial
t}e^{\sum_{j=1}^{n}\frac{im_j}{\hbar}Z(\vec
x_j,t)}+\sum_{k=1}^{n}i\hbar\left(\nabla_{\vec x_k}Z(\vec
x_k,t)\cdot\nabla_{\vec
x_k}\psi\right)e^{\sum_{j=1}^{n}\frac{im_j}{\hbar}Z(\vec
x_j,t)}\\+\sum_{k=1}^{n}\frac{i\hbar}{2}\left(\Delta_{\vec
x_k}Z(\vec x_k,t)\right)e^{\sum_{j=1}^{n}\frac{im_j}{\hbar}Z(\vec
x_j,t)}\psi
 +\sum_{k=1}^{n}\frac{\hbar^2}{2m_k}\left(\Delta_{\vec x_k}\psi\right)e^{\sum_{j=1}^{n}\frac{im_j}{\hbar}Z(\vec x_j,t)}=\\ \sum_{k=1}^{n}\frac{ i\hbar\sigma_k}{2m_kc}\left(div_{\vec x_k}\vec
A(\vec x_k,t)\right)e^{\sum_{j=1}^{n}\frac{im_j}{\hbar}Z(\vec
x_j,t)}\psi +
%+\frac{i\hbar\sigma}{2mc}div_{\vec x}\left\{\psi\vec A\right\}
\sum_{k=1}^{n}\frac{ i\hbar\sigma_k}{m_kc}\left(\vec A(\vec
x_k,t)\cdot\nabla_{\vec
x_k}\psi\right)e^{\sum_{j=1}^{n}\frac{im_j}{\hbar}Z(\vec
x_j,t)}\\-\sum_{k=1}^{n}\frac{\sigma_k}{c}\left(\vec A(\vec
x_k,t)\cdot\nabla_{\vec x_k}Z(\vec
x_k,t)\right)e^{\sum_{j=1}^{n}\frac{im_j}{\hbar}Z(\vec
x_j,t)}\psi-V\left(e^{\sum_{j=1}^{n}\frac{im_j}{\hbar}Z(\vec
x_j,t)}\psi\right)\\+\sum_{k=1}^{n}\left(\sigma_k\Psi(\vec
x_k,t)+\frac{\sigma^2_k}{2m_kc^2}\left|\vec A(\vec
x_k,t)\right|^2\right)\left(e^{\sum_{j=1}^{n}\frac{im_j}{\hbar}Z(\vec
x_j,t)}\psi\right).
\end{multline}
We rewrite
\er{vhfffngghkjgghfjjghghghhjghjgghkghgghjhggjjkgfgdiyfgfjkjgjggjjgugyyjSYSPN}
as
\begin{multline}\label{vhfffngghkjgghfjjghghghhjghjgghkghgghjhggjjkgfgdiyfgfjkjgjggjjgugyyjjkgghgjSYSPN}
i\hbar\frac{\partial}{\partial
t}\left(e^{\sum_{j=1}^{n}\frac{im_j}{\hbar}Z(\vec
x_j,t)}\psi\right)+\sum_{k=1}^{n}\frac{\hbar^2}{2m_k}\Delta_{\vec
x_k}\left(e^{\sum_{j=1}^{n}\frac{im_j}{\hbar}Z(\vec
x_j,t)}\psi\right)=\\ \sum_{k=1}^{n}\frac{
i\hbar\sigma_k}{2m_kc}\left(div_{\vec x_k}\vec A(\vec
x_k,t)\right)e^{\sum_{j=1}^{n}\frac{im_j}{\hbar}Z(\vec x_j,t)}\psi+
%+\frac{i\hbar\sigma}{2mc}div_{\vec x}\left\{\psi\vec A\right\}
\sum_{k=1}^{n}\frac{ i\hbar\sigma_k}{m_kc}\vec A(\vec
x_k,t)\cdot\nabla_{\vec
x_k}\left(e^{\sum_{j=1}^{n}\frac{im_j}{\hbar}Z(\vec
x_j,t)}\psi\right)
%-\frac{\sigma}{c}\left(\vec A\cdot\nabla_{\vec x}Z\right)e^{\frac{im}{\hbar}Z}\psi
\\+\sum_{k=1}^{n}\left(\sigma_k\Psi(\vec
x_k,t)+\frac{\sigma^2_k}{2m_kc^2}\left|\vec A(\vec
x_k,t)\right|^2\right)\left(e^{\sum_{j=1}^{n}\frac{im_j}{\hbar}Z(\vec
x_j,t)}\psi\right)-V\left(e^{\sum_{j=1}^{n}\frac{im_j}{\hbar}Z(\vec
x_j,t)}\psi\right)\\-\sum_{k=1}^{n}m_k\left(\frac{\partial
Z}{\partial t}(\vec x_k,t)+\frac{1}{2}\left|\nabla_{\vec x_k}Z(\vec
x_k,t)\right|^2\right)\left(e^{\sum_{j=1}^{n}\frac{im_j}{\hbar}Z(\vec
x_j,t)}\psi\right).
\end{multline}
Therefore, inserting
\er{MaxVacFull1ninshtrgravortghhghgjkgghklhjgkghghjjkjhjkkggjkhjkhjjhhfhjhklkhkhjjklzzzyyyhjggjhgghhjhNWNWNWNWNWBWHWNWSYSPN}
into
\er{vhfffngghkjgghfjjghghghhjghjgghkghgghjhggjjkgfgdiyfgfjkjgjggjjgugyyjjkgghgjSYSPN}
gives
\begin{multline}\label{vhfffngghkjgghfjjghghghhjghjgghkghgghjhggjjkgfgdiyfgfjkjgjggjjgugyyjjkgghgjjhhjkkSYSPN}
i\hbar\frac{\partial}{\partial
t}\left(e^{\sum_{j=1}^{n}\frac{im_j}{\hbar}Z(\vec
x_j,t)}\psi\right)=-\sum_{k=1}^{n}\frac{\hbar^2}{2m_k}\Delta_{\vec
x}\left(e^{\sum_{j=1}^{n}\frac{im_j}{\hbar}Z(\vec
x_j,t)}\psi\right)-V\left(e^{\sum_{j=1}^{n}\frac{im_j}{\hbar}Z(\vec
x_j,t)}\psi\right)
%+\frac{i\hbar\sigma}{2mc}div_{\vec x}\left\{\psi\vec A\right\}
\\+\sum_{k=1}^{n}\frac{ i\hbar\sigma_k}{2m_kc}\left(div_{\vec x_k}\vec
A(\vec x_k,t)\right)e^{\sum_{j=1}^{n}\frac{im_j}{\hbar}Z(\vec
x_j,t)}\psi +\sum_{k=1}^{n}\frac{ i\hbar\sigma_k}{m_kc}\vec A(\vec
x_k,t)\cdot\nabla_{\vec
x_k}\left(e^{\sum_{j=1}^{n}\frac{im_j}{\hbar}Z(\vec
x_j,t)}\psi\right)\\
%-\frac{\sigma}{c}\left(\vec A\cdot\nabla_{\vec x}Z\right)e^{\frac{im}{\hbar}Z}\psi
+\sum_{k=1}^{n}\left(\sigma_k\Psi(\vec
x_k,t)+\frac{\sigma^2_k}{2m_kc^2}\left|\vec A(\vec
x_k,t)\right|^2+m_k\Phi(\vec
x_k,t)\right)\left(e^{\sum_{j=1}^{n}\frac{im_j}{\hbar}Z(\vec
x_j,t)}\psi\right).
\end{multline}
Then denoting
\begin{equation}\label{jhgghfhj}
\psi_1:=e^{\sum_{j=1}^{n}\frac{im_j}{\hbar}Z(\vec x_j,t)}\psi,
\end{equation}
%and using the fact that $div_{\vec x}\vec A=0$
we obtain in the
coordinate system $(*)$ the Shr\"{o}dinger equation in the form
\begin{multline}\label{vhfffngghkjgghfjjghghghhjghjgghkghgghjhggjjkgfgdiyfgfjkjgjggjjgugyyjjkgghgjjhhjkkhjhjkSYSPN}
i\hbar\frac{\partial\psi_1}{\partial
t}=-\sum_{j=1}^{n}\frac{\hbar^2}{2m_j}\Delta_{\vec
x_j}\psi_1+\sum_{j=1}^{n}\frac{ i\hbar\sigma_j}{2m_jc}div_{\vec
x_j}\left\{\psi_1\vec A(\vec x_j,t)\right\}
%+\frac{i\hbar\sigma}{2mc}div_{\vec x}\left\{\psi\vec A\right\}
+\sum_{j=1}^{n}\frac{ i\hbar\sigma_j}{2m_jc}\vec A(\vec
x_j,t)\cdot\nabla_{\vec x_j}\psi_1
%-\frac{\sigma}{c}\left(\vec A\cdot\nabla_{\vec x}Z\right)e^{\frac{im}{\hbar}Z}\psi
\\+\sum_{j=1}^{n}\left(\sigma_j\Psi(\vec
x_j,t)+\frac{\sigma^2_j}{2m_jc^2}\left|\vec A(\vec
x_j,t)\right|^2+m_j\Phi(\vec x_j,t)\right)\psi_1-V\psi_1,
\end{multline}
which coincides with the classical Shr\"{o}dinger equation for this
case.
\begin{remark}\label{gyytghffg}
Note that by \er{noninchgravortbstrjgghguittu1intmmjhhj} in Remark
\ref{ugyugg}, equality \er{jhgghfhj} implies that under the change
of coordinate system given by the Galilean Transformation
\begin{equation}\label{noninchgravortbstrjgghguittu1intmmkuk}
\begin{cases}
\vec x'=\vec x+\vec wt,\\
t'=t,
\end{cases}
\end{equation}
the quantity $\psi_1$ transforms as:
\begin{equation}\label{jhgghfhjhjhjhyuyuy}
\psi'_1:=e^{\sum_{j=1}^{n}\frac{im_j}{\hbar}(\vec w\cdot\vec
x_j+\frac{1}{2}|\vec w|^2t)}\psi_1,
\end{equation}
provided that $\psi'=\psi$. Moreover, \er{jhgghfhjhjhjhyuyuy}
coincides with the classical law of transformation of the wave
function, under the Galilean Transformation (see section 17 in
\cite{LL}, the end of the section).
\end{remark}
%\begin{equation}\label{noninchgravortbstrjgghguittu1intmmjhhj}
%Z'(\vec x',t')=Z(\vec x,t)+\vec w\cdot\vec x+\frac{1}{2}|\vec w|^2t,
%\end{equation}

















Next, again consider the motion and interaction of system of $n$
quantum micro-particles having inertial masses $m_1,\ldots, m_n$ and
the charges $\sigma_1,\ldots,\sigma_n$ with the known gravitational
and electromagnetical field with potentials $\vec v(\vec x,t)$,
$\vec A(\vec x,t)$ and $\Psi(\vec x,t)$ and additional conservative
field with potential $V(\vec y_1,\ldots,\vec y_n,t)$. Then consider
a Lagrangian density $L$ defined by
\begin{multline}\label{vhfffngghkjgghDDmmkkksmfggffgZZ}
L_2\left(\psi,\vec A,\Psi,\vec v,\vec x_1,\ldots,\vec x_n,t\right):=\\
\frac{i\hbar}{2}\left(\left(\frac{\partial\psi}{\partial
t}+\sum\limits_{k=1}^{n}\vec v(\vec x_k,t)\cdot\nabla_{\vec
x_k}\psi\right)\cdot\bar\psi-\psi\cdot\left(\frac{\partial\bar\psi}{\partial
t}+\sum\limits_{k=1}^{n}\vec v(\vec x_k,t)\cdot\nabla_{\vec
x_k}\bar\psi\right)\right)-\sum\limits_{k=1}^{n}\frac{\hbar^2}{2m}\nabla_{\vec
x_k}\psi\cdot\nabla_{\vec x_k}\bar\psi\\+V\left(\vec x_1,\ldots,\vec
x_n,t\right)\psi\cdot\bar\psi
-\sum\limits_{k=1}^{n}\frac{\hbar\sigma i}{2mc}\left(\nabla_{\vec
x_k}\psi\cdot\bar\psi-\psi\cdot\nabla_{\vec
x_k}\bar\psi\right)\cdot\vec A(\vec
x_k,t)-\sum\limits_{k=1}^{n}\frac{\sigma^2}{2mc^2}\left|\vec
A(\vec x_k,t)\right|^2\psi\cdot\bar\psi\\
-\sum\limits_{k=1}^{n}\sigma\left(\Psi(\vec x_k,t)-\frac{1}{c}\vec
v(\vec x_k,t)\cdot\vec A(\vec x_k,t)\right)\psi\cdot\bar\psi,
\end{multline}
where $\psi(\vec x_1,\ldots,\vec x_n,t)\in\mathbb{C}$ is a wave
function of the system. Then, as before, it can be proven that $L$
is invariant under the change of inertial or non-inertial cartesian
coordinate systems of the form
\begin{equation*}
\begin{cases}
t'=t\\
%\vec x'=A(t)\cdot\vec x+\vec z(t)\\
\vec x'_k=A(t)\cdot\vec x_k+\vec
z(t)\quad\quad\forall k=1,\ldots,n,\end{cases}
\end{equation*}
provided that $\psi'=\psi$. We investigate stationary points of the
functional
\begin{equation}\label{btfffygtgyggyDDmmkkksmZZ}
J=\int_0^T\int_{\left(\mathbb{R}^3\right)^n}L\left(\psi,\vec
A,\Psi,\vec v,\vec x_1,\ldots,\vec x_n,t\right)d\vec x_1\ldots,d\vec
x_n dt.
\end{equation}
Then,
\begin{multline}\label{vhfffngghkjgghDDmmkkkghgsmZZ}
0=\frac{\delta L_2}{\delta (\bar\psi)}=
i\hbar\left(\frac{\partial\psi}{\partial
t}+\sum\limits_{k=1}^{n}\frac{1}{2}\vec v(\vec
x_k,t)\cdot\nabla_{\vec
x_k}\psi+\sum\limits_{k=1}^{n}\frac{1}{2}div_{\vec
x_k}\left\{\psi\vec v(\vec
x_k,t)\right\}\right)+\sum\limits_{k=1}^{n}\frac{\hbar^2}{2m}\Delta_{\vec
x_k}\psi\\+V\left(\vec x_1,\ldots,\vec x_n,t\right)\psi
-\frac{\hbar\sigma i}{2mc}\left(\sum\limits_{k=1}^{n}\vec A(\vec
x_k,t)\cdot\nabla_{\vec x_k}\psi+\sum\limits_{k=1}^{n}div_{\vec
x_k}\left\{\psi\vec A(\vec
x_k,t)\right\}\right)-\sum\limits_{k=1}^{n}\frac{\sigma^2}{2mc^2}\left|\vec
A(\vec x_k,t)\right|^2\psi
\\-\sum\limits_{k=1}^{n}\sigma\left(\Psi(\vec x_k,t)-\frac{1}{c}\vec
v(\vec x_k,t)\cdot\vec A(\vec x_k,t)\right)\psi,
\end{multline}
and
\begin{multline}\label{vhfffngghkjgghDDmmkkkghguhyuysmZZ}
0=\frac{\delta L_2}{\delta (\psi)}= \bar
{(i)}\hbar\left(\frac{\partial\bar\psi}{\partial
t}+\sum\limits_{k=1}^{n}\frac{1}{2}\vec v(\vec
x_k,t)\cdot\nabla_{\vec
x_k}\bar\psi+\sum\limits_{k=1}^{n}\frac{1}{2}div_{\vec
x_k}\left\{\bar\psi\vec v(\vec
x_k,t)\right\}\right)+\sum\limits_{k=1}^{n}\frac{\hbar^2}{2m}\Delta_{\vec
x_k}\bar\psi \\+V\left(\vec x_1,\ldots,\vec x_n,t\right)\bar\psi
-\sum\limits_{k=1}^{n}\frac{\hbar\sigma \bar {(i)}}{2mc}\left(\vec
A(\vec x_k,t)\cdot\nabla_{\vec x_k}\bar\psi+div_{\vec
x_k}\left\{\bar\psi\vec A(\vec
x_k,t)\right\}\right)-\sum\limits_{k=1}^{n}\frac{\sigma^2}{2mc^2}\left|\vec
A(\vec x_k,t)\right|^2\bar\psi
\\-\sum\limits_{k=1}^{n}\sigma\left(\Psi(\vec x_k,t)-\frac{1}{c}\vec v(\vec x_k,t)\cdot\vec
A(\vec x_k,t)\right)\bar\psi,
\end{multline}
where the last equality is just the complex conjugate of
\er{vhfffngghkjgghDDmmkkkghgsmZZ}. So we get that the Euler-Lagrange
equation for \er{btfffygtgyggyDDmmkkksmZZ} coincides with the
Schr\"{o}dinger equation of the form
\er{vhfffngghkjgghfjjghghghhjghjgghkghgghjhggjjkgfgdiyfgfjkjgjgSYSPN}.





















\subsection{Shr\"{o}dinger-Pauli equation for a spin-half quantum
particle}\label{hjjghjghggh} Consider the motion of a spin-half
quantum micro-particle with inertial mass $m$ and the charge
$\sigma$ in the outer gravitational and electromagnetical field with
characteristics $\vec v(\vec x,t)$, $\vec A(\vec x,t)$ and
$\Psi(\vec x,t)$ and additional conservative field with potential
$V(\vec y,t)$. Since the Hamiltonian for a macro-particle has the
form
\begin{multline}\label{vhfffngghkjgghfjjghghghSYShmyuuiiuuhmhmiopoopnnkkllkkkZZ}
H_{\text{macro}}\left(\vec P,\vec r,t\right)=\\
%
%
%
\frac{m}{2}\left|\vec P-\frac{\sigma}{c}\vec A(\vec r,t)\right|^2
%+\sum_{j=1}^{n}m_j\left(1+\frac{1}{c^2}\left|\frac{1}{m_j}\vec P_j-\frac{\sigma_j}{m_jc}\vec A(\vec r_j,t)\right|^2\right)^{-\frac{1}{2}}\left|\frac{1}{m_j}\vec P_j-\frac{\sigma_j}{m_jc}\vec A(\vec r_j,t)\right|^2
+\sigma\left(\Psi(\vec r,t)-\frac{1}{c}\vec v(\vec r,t)\cdot\vec
A(\vec r,t)\right)-V\left(\vec r,t\right)+\frac{1}{2}\vec v(\vec
r,t)\cdot\vec P+\frac{1}{2}\vec P\cdot\vec v(\vec r,t),
\end{multline}
we built the Hamiltonian operator, taking into account the spin
interaction as
\begin{multline}\label{vhfffngghkjgghfjjghghghSYShmyuuiiuuhmhmiopoopnniukjhjkkkllkkkZZ}
\hat H_0\cdot\psi= -\frac{\hbar^2}{2m}\Delta_{\vec
x}\psi+\frac{i\hbar\sigma}{2mc}div_{\vec x}\left\{\psi_1\vec A(\vec
x,t)\right\}+\frac{i\hbar\sigma}{2mc}\nabla_{\vec x}\psi\cdot\vec
A(\vec x,t)+\frac{\sigma^2}{2mc^2}\left|\vec A(\vec
x,t)\right|^2\psi\\+\sigma\left(\Psi(\vec x,t)-\frac{1}{c}\vec
v(\vec x,t)\cdot\vec A(\vec x,t)\right)\psi-V\left(\vec
x,t\right)\psi-\frac{i\hbar}{2}div_{\vec x}\left\{\psi\vec v(\vec
x,t)\right\}-\frac{i\hbar}{2}\nabla_{\vec x}\psi\cdot\vec v(\vec
x,t)
\\-\frac{g\sigma\hbar}{2mc}\vec
S\cdot\left(curl_{\vec x}\vec A(\vec
x,t)\,\psi\right)+\frac{\hbar}{4}\vec S\cdot\left(curl_{\vec x}\vec
v(\vec x,t)\,\psi\right),
\end{multline}
where $\psi(\vec x,t)=\left(\psi_1(\vec x,t),\psi_2(\vec
x,t)\right)\in\mathbb{C}^2$ is a two-component wave function, $\hat
H_0$ is the Hamiltonian operator, $\vec S:=(S_1,S_2,S_3)$,
$$S_1=\left(\begin{matrix}0&1\\1&0\end{matrix}\right),\quad
S_2=\left(\begin{matrix}0&-i\\i&0\end{matrix}\right)\quad
S_3=\left(\begin{matrix}1&0\\0&-1\end{matrix}\right)$$ are Pauli
matrices and $g$ is a constant that depends on the type of the
particle (for electron we have $g=1$). Note that, in addition to the
classical term of the spin-magnetic interaction, we added another
term to the Hamiltonian, namely $\frac{\hbar}{4}\vec
S\cdot\left(curl_{\vec x}\vec v(\vec x,t)\,\psi\right)$. This term
vanishes in every non-rotating and, in particular, in every inertial
coordinate system, however it provides the invariance of the
Shr\"{o}dinger-Pauli equation, under the change of non-inertial
cartesian coordinate system, as can be seen in the following Theorem
\ref{gjghghgghgintintrrZZ}. The Shr\"{o}dinger-Pauli equation for
this particle is
\begin{equation}\label{vhfffngghkjgghfjjghghghhjghjgghkghggkghghjghSYSPNnnkkllkkkZZ}
i\hbar\frac{\partial\psi}{\partial t}=\hat H_0\cdot\psi.
\end{equation}
 Thus,
\begin{multline}\label{vhfffngghkjgghfjjghghghSYShmyuuiiuuhmhmiopoopnniukjhjkk;l;lkhjjkihjjhkkkkkZZ}
i\hbar\frac{\partial\psi}{\partial
t}=-\frac{\hbar^2}{2m}\Delta_{\vec
x}\psi+\frac{i\hbar\sigma}{2mc}div_{\vec x}\left\{\psi\vec A(\vec
x,t)\right\}+\frac{i\hbar\sigma}{2mc}\nabla_{\vec x}\psi\cdot\vec
A(\vec x,t)+\frac{\sigma^2}{2mc^2}\left|\vec A(\vec
x,t)\right|^2\psi\\
%-\frac{\sigma\hbar}{2mc}\vec S\cdot\left(curl_{\vec x}\vec A(\vec x,t)\,\psi\right)
%
%
%
%+\sum_{j=1}^{n}m_j\left(1+\frac{1}{c^2}\left|\frac{1}{m_j}\vec P_j-\frac{\sigma_j}{m_jc}\vec A(\vec r_j,t)\right|^2\right)^{-\frac{1}{2}}\left|\frac{1}{m_j}\vec P_j-\frac{\sigma_j}{m_jc}\vec A(\vec r_j,t)\right|^2
+\sigma\left(\Psi(\vec x,t)-\frac{1}{c}\vec v(\vec x,t)\cdot\vec
A(\vec x,t)\right)\psi-V\left(\vec
x,t\right)\psi-\frac{i\hbar}{2}div_{\vec x}\left\{\psi\vec v(\vec
x,t)\right\}-\frac{i\hbar}{2}\nabla_{\vec x}\psi\cdot\vec v(\vec
x,t)\\+\frac{\hbar}{2}\vec S\cdot\left(\left(\frac{1}{2}curl_{\vec
x}\vec v(\vec x,t)-\frac{g\sigma}{mc}curl_{\vec x}\vec A(\vec
x,t)\right)\,\psi\right).
\end{multline}
I.e.
\begin{multline}\label{vhfffngghkjgghfjjghghghSYShmyuuiiuuhmhmiopoopnniukjhjkk;l;lkhjjkihjjhkkkkkjjjZZ}
i\hbar\left(\frac{\partial\psi}{\partial t}+\frac{1}{2}div_{\vec
x}\left\{\psi\vec v(\vec x,t)\right\}+\frac{1}{2}\nabla_{\vec
x}\psi\cdot\vec v(\vec x,t)\right)\\=-\frac{\hbar^2}{2m}\Delta_{\vec
x}\psi+\frac{i\hbar\sigma}{2mc}div_{\vec x}\left\{\psi\vec A(\vec
x,t)\right\}+\frac{i\hbar\sigma}{2mc}\nabla_{\vec x}\psi\cdot\vec
A(\vec x,t)+\frac{\sigma^2}{2mc^2}\left|\vec A(\vec
x,t)\right|^2\psi\\
%-\frac{\sigma\hbar}{2mc}\vec S\cdot\left(curl_{\vec x}\vec A(\vec x,t)\,\psi\right)
%
%
%
%+\sum_{j=1}^{n}m_j\left(1+\frac{1}{c^2}\left|\frac{1}{m_j}\vec P_j-\frac{\sigma_j}{m_jc}\vec A(\vec r_j,t)\right|^2\right)^{-\frac{1}{2}}\left|\frac{1}{m_j}\vec P_j-\frac{\sigma_j}{m_jc}\vec A(\vec r_j,t)\right|^2
+\sigma\left(\Psi(\vec x,t)-\frac{1}{c}\vec v(\vec x,t)\cdot\vec
A(\vec x,t)\right)\psi-V\left(\vec
x,t\right)\psi+\frac{\hbar}{2}\vec
S\cdot\left(\left(\frac{1}{2}curl_{\vec x}\vec v(\vec
x,t)-\frac{g\sigma}{mc}curl_{\vec x}\vec A(\vec
x,t)\right)\,\psi\right).
\end{multline}
%where $\vec S:=(S_1,S_2,S_3)$ and
%$$S_1=\left(\begin{matrix}0&1\\1&0\end{matrix}\right),\quad
%S_2=\left(\begin{matrix}0&-i\\i&0\end{matrix}\right)\quad
%S_3=\left(\begin{matrix}1&0\\0&-1\end{matrix}\right)$$ are Pauli matrices.
\begin{theorem}\label{gjghghgghgintintrrZZ}
Consider that the change of some cartesian coordinate system $(*)$
to another cartesian coordinate system $(**)$ is given by
\begin{equation}\label{noninchgravortbstrjgghguittu2intrrrZZ}
\begin{cases}
\vec x'=A(t)\cdot\vec x+\vec z(t),\\
t'=t,
\end{cases}
\end{equation}
where $A(t)\in SO(3)$ is a rotation. Next, assume that in the
coordinate system $(**)$ we observe a validity of the
Shr\"{o}dinger-Pauli equation of the form:
\begin{multline}\label{MaxVacFull1ninshtrredPPNintrrZZ}
i\hbar\left(\frac{\partial\psi'}{\partial t'}+\frac{1}{2}div_{\vec
x'}\left\{\psi'\vec v'\right\}+\frac{1}{2}\nabla_{\vec
x'}\psi'\cdot\vec v'\right)=-\frac{\hbar^2}{2m'}\Delta_{\vec
x'}\psi'+\frac{i\hbar\sigma'}{2m'c}div_{\vec x'}\left\{\psi'\vec
A'\right\}+\frac{i\hbar\sigma'}{2m'c}\nabla_{\vec x'}\psi'\cdot\vec
A'\\+\frac{(\sigma')^2}{2m'c^2}\left|\vec A'\right|^2\psi'
%-\frac{\sigma\hbar}{2mc}\vec S\cdot\left(curl_{\vec x}\vec A(\vec x,t)\,\psi\right)
%
%
%
%+\sum_{j=1}^{n}m_j\left(1+\frac{1}{c^2}\left|\frac{1}{m_j}\vec P_j-\frac{\sigma_j}{m_jc}\vec A(\vec r_j,t)\right|^2\right)^{-\frac{1}{2}}\left|\frac{1}{m_j}\vec P_j-\frac{\sigma_j}{m_jc}\vec A(\vec r_j,t)\right|^2
+\sigma'\left(\Psi'-\frac{1}{c}\vec v'\cdot\vec
A'\right)\psi'-V'\psi'+\frac{\hbar}{2}\vec
S\cdot\left(\left(\frac{1}{2}curl_{\vec x'}\vec
v'-\frac{g'\sigma'}{m'c}curl_{\vec x'}\vec A'\right)\,\psi'\right),
\end{multline}
where $\psi\in\mathbb{C}^2$. Then in the coordinate system $(*)$ we
have the validity of Shr\"{o}dinger-Pauli equation of the same as
\er{MaxVacFull1ninshtrredPPNintrrZZ} form:
\begin{multline}\label{MaxVacFull1ninshtrhjkkredPPNintrrZZ}
i\hbar\left(\frac{\partial\psi}{\partial t}+\frac{1}{2}div_{\vec
x}\left\{\psi\vec v\right\}+\frac{1}{2}\nabla_{\vec x}\psi\cdot\vec
v\right)=-\frac{\hbar^2}{2m}\Delta_{\vec
x}\psi+\frac{i\hbar\sigma}{2mc}div_{\vec x}\left\{\psi\vec
A\right\}+\frac{i\hbar\sigma}{2mc}\nabla_{\vec x}\psi\cdot\vec
A\\+\frac{\sigma^2}{2mc^2}\left|\vec A\right|^2\psi
%-\frac{\sigma\hbar}{2mc}\vec S\cdot\left(curl_{\vec x}\vec A(\vec x,t)\,\psi\right)
%
%
%
%+\sum_{j=1}^{n}m_j\left(1+\frac{1}{c^2}\left|\frac{1}{m_j}\vec P_j-\frac{\sigma_j}{m_jc}\vec A(\vec r_j,t)\right|^2\right)^{-\frac{1}{2}}\left|\frac{1}{m_j}\vec P_j-\frac{\sigma_j}{m_jc}\vec A(\vec r_j,t)\right|^2
+\sigma\left(\Psi-\frac{1}{c}\vec v\cdot\vec
A\right)\psi-V\psi+\frac{\hbar}{2}\vec
S\cdot\left(\left(\frac{1}{2}curl_{\vec x}\vec
v-\frac{g\sigma}{mc}curl_{\vec x}\vec A\right)\,\psi\right).
\end{multline}
provided that
\begin{equation}\label{yuythfgfyftydtydtydtyddyyyhhddhhhredPPN111hgghjgintintrrZZ}
\begin{cases}
g'=g\\
V'=V,\\
\sigma'=\sigma,\\
m'=m,\\
\vec v'=A(t)\cdot \vec v+\frac{dA}{dt}(t)\cdot\vec x+\frac{d\vec z}{dt}(t),\\
\vec A'=A(t)\cdot \vec A,\\
\Psi'-\vec v'\cdot\vec A'=\Psi-\vec v\cdot\vec A,\\
\psi'=U(t)\cdot\psi,
\end{cases}
\end{equation}
where $U(t)\in SU(2)$ is some special unitary $2\times 2$ matrix
i.e. $U(t)\in\mathbb{C}^{2\times 2}$, $det\,U(t)=1$, $U(t)\cdot
U^*(t)=I$ where $U^*(t)$ is the Hermitian adjoint to $U(t)$ matrix:
$U^*(t):=\bar U(t)^T$ and $I$ is the identity $2\times 2$ matrix.
Moreover, $U(t)$ is characterized by:
\begin{equation}\label{gyfyfgfgfgghZZ}
U^*(t)\cdot\vec S\cdot U(t)=A(t)\cdot\vec S,
\end{equation}
that means
\begin{multline*}
\left(U^*(t)\cdot S_1\cdot U(t),U^*(t)\cdot S_2\cdot
U(t),U^*(t)\cdot S_3\cdot U(t)\right)=\\
\left(a_{11}(t)S_1+a_{12}(t)S_2+a_{13}(t)S_3\,,\,a_{21}(t)S_1+a_{22}(t)S_2+a_{23}(t)S_3\,,\,a_{31}(t)S_1+a_{32}(t)S_2+a_{33}(t)S_3\right),
\end{multline*}
where $A(t)=\left\{a_{mk}(t)\right\}_{\{1\leq m,k\leq 3\}}$.
\end{theorem}
\begin{proof}
It is well known that we have
\begin{equation}\label{gyjyfghfZZ}
\begin{cases}
S^2_1=S^2_2=S^2_3=I,\\
S_1\cdot S_2=-S_2\cdot S_1=iS_3,\quad S_2\cdot S_3=-S_3\cdot
S_2=iS_1,\quad S_3\cdot S_1=-S_1\cdot S_3=iS_2.
\end{cases}
\end{equation}
Next, it is well known that $SO(3)$ is smoothly double covered by
$SU(2)$ and the cover mapping is regular and locally one to one. In
particular, for every $t$ there exists $U(t)\in SU(2)$ such that
\er{gyfyfgfgfgghZZ} is satisfied
%, i.e.
%\begin{equation}\label{gyfyfgfgfgghjhjh}
%U^*(t)\cdot\vec S\cdot U(t)=A(t)\cdot\vec S,
%\end{equation}
(note that the seconde choice is $(-U(t))$). Moreover, by Implicit
Function Theorem we deduce that if $A(t)$ is differentiable by $t$
then $U(t)$ is also differentiable by $t$. Thus if
$\psi'=U(t)\cdot\psi$ in \er{MaxVacFull1ninshtrredPPNintrrZZ}, then
by \er{MaxVacFull1ninshtrredPPNintrrZZ},
\er{yuythfgfyftydtydtydtyddyyyhhddhhhredPPN111hgghjgintintrrZZ} and
proposition \ref{yghgjtgyrtrt}
%from Appendix
we have:
\begin{multline}\label{MaxVacFull1ninshtrredPPNintrrghghZZ}
i\hbar
U(t)\cdot\left(U^{-1}(t)\cdot\frac{dU}{dt}(t)\cdot\psi+\frac{\partial\psi}{\partial
t}+\frac{1}{2}div_{\vec x}\left\{\psi\vec
v\right\}+\frac{1}{2}\nabla_{\vec x}\psi\cdot\vec v\right)
=\\U(t)\cdot\left(-\frac{\hbar^2}{2m}\Delta_{\vec
x}\psi+\frac{i\hbar\sigma}{2mc}div_{\vec x}\left\{\psi\vec
A\right\}+\frac{i\hbar\sigma}{2mc}\nabla_{\vec x}\psi\cdot\vec
A+\frac{\sigma^2}{2mc^2}\left|\vec A\right|^2\psi
%-\frac{\sigma\hbar}{2mc}\vec S\cdot\left(curl_{\vec x}\vec A(\vec x,t)\,\psi\right)
%
%
%
%+\sum_{j=1}^{n}m_j\left(1+\frac{1}{c^2}\left|\frac{1}{m_j}\vec P_j-\frac{\sigma_j}{m_jc}\vec A(\vec r_j,t)\right|^2\right)^{-\frac{1}{2}}\left|\frac{1}{m_j}\vec P_j-\frac{\sigma_j}{m_jc}\vec A(\vec r_j,t)\right|^2
+\sigma\left(\Psi-\frac{1}{c}\vec v\cdot\vec
A\right)\psi-V\psi\right)\\+\frac{\hbar}{2}\vec
S\cdot\left(A(t)\cdot\left(\frac{1}{2}curl_{\vec
x}\left(A^{-1}(t)\cdot\frac{dA}{dt}(t)\cdot \vec
x\right)+\frac{1}{2}curl_{\vec x}\vec v-\frac{g\sigma}{mc}curl_{\vec
x}\vec A\right)\,U(t)\cdot\psi\right).
\end{multline}
Thus, since $U(t)$ is unitary and then $U^{-1}(t)=U^*(t)$ and
$A^{-1}(t)=A^T(t)$, by \er{MaxVacFull1ninshtrredPPNintrrghghZZ} we
have
\begin{multline}\label{MaxVacFull1ninshtrredPPNintrrghghghghZZ}
\frac{\hbar}{4}\left(4iU^*(t)\cdot\frac{dU}{dt}(t)\cdot\psi-
U^*(t)\cdot\vec S\cdot U(t)\cdot\left(A(t)\cdot\left(curl_{\vec
x}\left(A^T(t)\cdot\frac{dA}{dt}(t)\cdot \vec
x\right)\right)\psi\right)\right)
\\+i\hbar
\left(\frac{\partial\psi}{\partial t}+\frac{1}{2}div_{\vec
x}\left\{\psi\vec v\right\}+\frac{1}{2}\nabla_{\vec x}\psi\cdot\vec
v\right) =\\-\frac{\hbar^2}{2m}\Delta_{\vec
x}\psi+\frac{i\hbar\sigma}{2mc}div_{\vec x}\left\{\psi\vec
A\right\}+\frac{i\hbar\sigma}{2mc}\nabla_{\vec x}\psi\cdot\vec
A+\frac{\sigma^2}{2mc^2}\left|\vec A\right|^2\psi
%-\frac{\sigma\hbar}{2mc}\vec S\cdot\left(curl_{\vec x}\vec A(\vec x,t)\,\psi\right)
%
%
%
%+\sum_{j=1}^{n}m_j\left(1+\frac{1}{c^2}\left|\frac{1}{m_j}\vec P_j-\frac{\sigma_j}{m_jc}\vec A(\vec r_j,t)\right|^2\right)^{-\frac{1}{2}}\left|\frac{1}{m_j}\vec P_j-\frac{\sigma_j}{m_jc}\vec A(\vec r_j,t)\right|^2
+\sigma\left(\Psi-\frac{1}{c}\vec v\cdot\vec
A\right)\psi-V\psi\\+\frac{\hbar}{2}U^*(t)\cdot\vec S\cdot
U(t)\cdot\left(A(t)\cdot\left(\frac{1}{2}curl_{\vec x}\vec
v-\frac{g\sigma}{mc}curl_{\vec x}\vec A\right)\,\psi\right).
\end{multline}
Thus by \er{MaxVacFull1ninshtrredPPNintrrghghghghZZ} and
\er{gyfyfgfgfgghZZ} we deduce:
\begin{multline}\label{MaxVacFull1ninshtrredPPNintrrghghghghuhjihZZ}
\frac{\hbar}{4}\left(4iU^*(t)\cdot\frac{dU}{dt}(t)\cdot\psi- \vec
S\cdot \left(curl_{\vec x}\left(A^T(t)\cdot\frac{dA}{dt}(t)\cdot
\vec x\right)\right)\psi\right)
\\+i\hbar
\left(\frac{\partial\psi}{\partial t}+\frac{1}{2}div_{\vec
x}\left\{\psi\vec v\right\}+\frac{1}{2}\nabla_{\vec x}\psi\cdot\vec
v\right) =\\-\frac{\hbar^2}{2m}\Delta_{\vec
x}\psi+\frac{i\hbar\sigma}{2mc}div_{\vec x}\left\{\psi\vec
A\right\}+\frac{i\hbar\sigma}{2mc}\nabla_{\vec x}\psi\cdot\vec
A+\frac{\sigma^2}{2mc^2}\left|\vec A\right|^2\psi
%-\frac{\sigma\hbar}{2mc}\vec S\cdot\left(curl_{\vec x}\vec A(\vec x,t)\,\psi\right)
%
%
%
%+\sum_{j=1}^{n}m_j\left(1+\frac{1}{c^2}\left|\frac{1}{m_j}\vec P_j-\frac{\sigma_j}{m_jc}\vec A(\vec r_j,t)\right|^2\right)^{-\frac{1}{2}}\left|\frac{1}{m_j}\vec P_j-\frac{\sigma_j}{m_jc}\vec A(\vec r_j,t)\right|^2
+\sigma\left(\Psi-\frac{1}{c}\vec v\cdot\vec
A\right)\psi-V\psi\\+\frac{\hbar}{2}\vec S\cdot
\left(\left(\frac{1}{2}curl_{\vec x}\vec
v-\frac{g\sigma}{mc}curl_{\vec x}\vec A\right)\,\psi\right).
\end{multline}
On the other hand differentiating the identities $A^T(t)\cdot
A(t)=I$ and $U^*(t)\cdot U(t)=I$ by $t$ we deduce that the real
$3\times 3$ matrix $A^T(t)\cdot\frac{dA}{dt}(t)$ is antisymmetric
and the complex $2\times 2$ matrix $iU^*(t)\cdot\frac{dU}{dt}(t)$ is
Hermitian self-adjoint. Moreover, differentiating the identity
$det\,U(t)=1$ and using that $U^*(t)=U^{-1}(t)$ we deduce that the
matrix $iU^*(t)\cdot\frac{dU}{dt}(t)$ is traceless. In particular,
since $A^T(t)\cdot\frac{dA}{dt}(t)$ is antisymmetric, there exists
$\vec w(t)=(w_1(t),w_2(t),w_3(t))\in\mathbb{R}^3$ such that
\begin{equation}\label{gyfyfgfgfgghuyyuyyuZZ}
A^T(t)\cdot\frac{dA}{dt}(t)\cdot \vec x=\vec w(t)\times\vec
x\quad\quad\forall\vec x\in\mathbb{R}^3,
\end{equation}
and then,
\begin{equation}\label{gyfyfgfgfgghuyyuyyuuyhjjhZZ}
curl_{\vec x}\left(A^T(t)\cdot\frac{dA}{dt}(t)\cdot \vec
x\right)=2\vec w(t).
\end{equation}
On the other hand, since the matrix $iU^*(t)\cdot\frac{dU}{dt}(t)$
is Hermitian self-adjoint and traceless, clearly there exist $\vec
q(t)=(q_1(t),q_2(t),q_3(t))\in\mathbb{R}^3$ such that
\begin{equation}\label{gyfyfgfgfgghuyyuyyuuyhjjhgfhhjghZZ}
iU^*(t)\cdot\frac{dU}{dt}(t)=\vec q(t)\cdot\vec
S:=q_1(t)S_1+q_2(t)S_2+q_3(t)S_3.
\end{equation}
Finally differentiating the identity \er{gyfyfgfgfgghZZ} by $t$ we
deduce
\begin{equation}\label{gyfyfgfgfgghyuyyugghhgZZ}
\frac{dU^*}{dt}(t)\cdot\vec S\cdot U(t)+U^*(t)\cdot\vec S\cdot
\frac{dU}{dt}(t)=\frac{dA}{dt}(t)\cdot\vec S,
\end{equation}
and then again inserting \er{gyfyfgfgfgghZZ} into
\er{gyfyfgfgfgghyuyyugghhgZZ} and using the antisymmetry of
$A^T(t)\cdot\frac{dA}{dt}(t)$ and the Hermitian property of
$iU^*(t)\cdot\frac{dU}{dt}(t)$ we obtain
\begin{equation}\label{gyfyfgfgfgghyuyyugghhgghghhhjZZ}
-i\left(\left(A(t)\cdot\vec S\right)\cdot \left(iU^*(t)\cdot
\frac{dU}{dt}(t)\right)-\left(i\cdot
U^*(t)\cdot\frac{dU}{dt}(t)\right)\cdot \left(A(t)\cdot\vec
S\right)\right)=A(t)\cdot\left(A^T(t)\cdot\frac{dA}{dt}(t)\right)\cdot\vec
S,
\end{equation}
that implies:
\begin{equation}\label{gyfyfgfgfgghyuyyugghhgghghhhjuytytytyZZ}
-i\left(\vec S\cdot \left(iU^*(t)\cdot
\frac{dU}{dt}(t)\right)-\left(i\cdot
U^*(t)\cdot\frac{dU}{dt}(t)\right)\cdot \vec
S\right)=\left(A^T(t)\cdot\frac{dA}{dt}(t)\right)\cdot\vec S,
\end{equation}
Then inserting \er{gyfyfgfgfgghuyyuyyuZZ} and
\er{gyfyfgfgfgghuyyuyyuuyhjjhgfhhjghZZ} into
\er{gyfyfgfgfgghyuyyugghhgghghhhjuytytytyZZ} we deduce
\begin{equation}\label{gyfyfgfgfgghyuyyugghhgghghhhjuytytytytytyyuuyZZ}
-i\left(\vec S\cdot
\left(q_1(t)S_1+q_2(t)S_2+q_3(t)S_3\right)-\left(q_1(t)S_1+q_2(t)S_2+q_3(t)S_3\right)\cdot
\vec S\right)=\vec w(t)\times\vec S.
\end{equation}
Thus by \er{gyjyfghfZZ} and
\er{gyfyfgfgfgghyuyyugghhgghghhhjuytytytytytyyuuyZZ} we get
\begin{equation}\label{gyfyfgfgfgghyuyyugghhgghghhhjuytytytytytyyuuyghhgZZ}
2\vec q(t)=\vec w(t).
\end{equation}
Therefore, \er{gyfyfgfgfgghyuyyugghhgghghhhjuytytytytytyyuuyghhgZZ},
\er{gyfyfgfgfgghuyyuyyuuyhjjhgfhhjghZZ} and
\er{gyfyfgfgfgghuyyuyyuuyhjjhZZ} together imply:
\begin{equation}\label{gyfyfgfgfgghyuyyugghhgghghhhjuytytytytytyyuuyghhgjhjhZZ}
4iU^*(t)\cdot\frac{dU}{dt}(t)=\vec S\cdot \left(curl_{\vec
x}\left(A^T(t)\cdot\frac{dA}{dt}(t)\cdot \vec x\right)\right),
\end{equation}
and inserting
\er{gyfyfgfgfgghyuyyugghhgghghhhjuytytytytytyyuuyghhgjhjhZZ} into
\er{MaxVacFull1ninshtrredPPNintrrghghghghuhjihZZ} we finally
conclude \er{MaxVacFull1ninshtrhjkkredPPNintrrZZ}.
\end{proof}











Next, again consider the motion of a quantum micro-particle with
spin-half, inertial mass $m$ and the charges $\sigma$ with the known
gravitational and electromagnetical field with characteristics $\vec
v(\vec x,t)$, $\vec A(\vec x,t)$ and $\Psi(\vec x,t)$ and additional
conservative field with potential $V(\vec y,t)$, taking into the
account spin interaction. Then consider a Lagrangian density $L$
defined by
\begin{multline}\label{vhfffngghkjgghDDmmkkkZZ}
L\left(\psi,\vec x,t\right):=
\frac{i\hbar}{2}\left(\left(\frac{\partial\psi}{\partial t}+\vec
v\cdot\nabla_{\vec
x}\psi\right)\cdot\bar\psi-\psi\cdot\left(\frac{\partial\bar\psi}{\partial
t}+\vec v\cdot\nabla_{\vec
x}\bar\psi\right)\right)-\frac{\hbar^2}{2m}\nabla_{\vec
x}\psi\cdot\nabla_{\vec x}\bar\psi+V\left(\vec
x,t\right)\psi\cdot\bar\psi\\
-\frac{\hbar\sigma i}{2mc}\left(\nabla_{\vec
x}\psi\cdot\bar\psi-\psi\cdot\nabla_{\vec x}\bar\psi\right)\cdot\vec
A-\frac{\sigma^2}{2mc^2}\left|\vec A\right|^2\psi\cdot\bar\psi
-\sigma\left(\Psi-\frac{1}{c}\vec v\cdot\vec
A\right)\psi\cdot\bar\psi\\ -\frac{\hbar}{2}\left(\left(\vec
S\cdot\left(\frac{1}{2}curl_{\vec x}\vec
v-\frac{g\sigma}{mc}curl_{\vec x}\vec
A\right)\right)\cdot\psi\right)\cdot\bar\psi,
\end{multline}
where $\psi\in \mathbb{C}^2$ is a two-component wave function. Then
similarly to the proof of Theorem \ref{gjghghgghgintintrrZZ} we can
prove that $L$ is invariant under the change of inertial or
non-inertial cartesian coordinate system, given by
\er{noninchgravortbstrjgghguittu2intrrrZZ}, provided that we take
into account
\er{yuythfgfyftydtydtydtyddyyyhhddhhhredPPN111hgghjgintintrrZZ}. We
investigate stationary points of the functional
\begin{equation}\label{btfffygtgyggyDDmmkkkZZ}
J=\int_0^T\int_{\mathbb{R}^3}L\left(\psi,\vec x,t\right)d\vec x dt.
\end{equation}
Then, by \er{vhfffngghkjgghDDmmkkkZZ} we have
\begin{multline}\label{vhfffngghkjgghDDmmkkkghgZZ}
0=\frac{\delta L}{\delta (\bar\psi)}=
i\hbar\left(\frac{\partial\psi}{\partial t}+\frac{1}{2}\vec
v\cdot\nabla_{\vec x}\psi+\frac{1}{2}div_{\vec x}\left\{\psi\vec
v\right\}\right)+\frac{\hbar^2}{2m}\Delta_{\vec x}\psi+V\left(\vec
x,t\right)\psi\\
-\frac{\hbar\sigma i}{2mc}\left(\vec A\cdot\nabla_{\vec
x}\psi+div_{\vec x}\left\{\psi\vec
A\right\}\right)-\frac{\sigma^2}{2mc^2}\left|\vec A\right|^2\psi
-\sigma\left(\Psi-\frac{1}{c}\vec v\cdot\vec A\right)\psi
\\-\frac{\hbar}{2}\left(\vec S\cdot\left(\frac{1}{2}curl_{\vec x}\vec
v-\frac{g\sigma}{mc}curl_{\vec x}\vec A\right)\right)\cdot\psi,
\end{multline}
and
\begin{multline}\label{vhfffngghkjgghDDmmkkkghguhyuyZZ}
0=\frac{\delta L}{\delta (\psi)}= \bar
{(i)}\hbar\left(\frac{\partial\bar\psi}{\partial t}+\frac{1}{2}\vec
v\cdot\nabla_{\vec x}\bar\psi+\frac{1}{2}div_{\vec
x}\left\{\bar\psi\vec
v\right\}\right)+\frac{\hbar^2}{2m}\Delta_{\vec x}\bar\psi
+V\left(\vec
x,t\right)\bar\psi\\
-\frac{\hbar\sigma \bar {(i)}}{2mc}\left(\vec A\cdot\nabla_{\vec
x}\bar\psi+div_{\vec x}\left\{\bar\psi\vec
A\right\}\right)-\frac{\sigma^2}{2mc^2}\left|\vec A\right|^2\bar\psi
-\sigma\left(\Psi-\frac{1}{c}\vec v\cdot\vec A\right)\bar\psi
\\-\frac{\hbar}{2}\left(\vec S\cdot\left(\frac{1}{2}curl_{\vec x}\vec
v-\frac{g\sigma}{mc}curl_{\vec x}\vec A\right)\right)\cdot\bar\psi.
\end{multline}
Note that the last equality is just the complex conjugate of
equality \er{vhfffngghkjgghDDmmkkkghgZZ}. So we get that the
Euler-Lagrange equation for \er{vhfffngghkjgghDDmmkkkZZ} coincides
with the Shr\"{o}dinger-Pauli equation in the form of
\er{vhfffngghkjgghfjjghghghSYShmyuuiiuuhmhmiopoopnniukjhjkk;l;lkhjjkihjjhkkkkkjjjZZ}.




















%SSD4
\section{Relation between the gravitational and inertial masses and
conservation laws}\label{GravElectro}
\subsection{Basic assumptions and their consequences}
We assumed before that the electromagnetic field is influenced by
the gravitational field. We also can assume that the gravitational
field is influenced by the electromagnetic field. We remind that we
assume the first approximation of the law of gravitation in the form
of
\er{MaxVacFull1ninshtrgravortghhghgjkgghklhjgkghghjjkjhjkkggjkhjkhjjhhfhjhklkhkhjjklzzzyyyNWBWHWPPN222jkgghgg}.
I.e.
\begin{equation}
\label{MaxVacFull1ninshtrgravortghhghgjkgghklhjgkghghjjkjhjkkggjkhjkhjjhhfhjhklkhkhjjklzzzyyyNWBWHWPPN222jkgghggyuhg}
\begin{cases}
curl_{\vec x}\left(curl_{\vec x}\vec v\right)= 0,\\
\frac{\partial}{\partial t}\left(div_{\vec x}\vec v\right)+div_{\vec
x}\left\{\left(div_{\vec x}\vec v\right)\vec
v\right\}+\frac{1}{4}\left|d_{\vec x}\vec v+\{d_{\vec x}\vec
v\}^T\right|^2-\left(div_{\vec x}\vec v\right)^2= -4\pi GM,
%\frac{d\vec u}{dt}=-curl_{\vec x}\vec v\times(\vec u-\vec v)+\partial_t\vec v+(\nabla_{\vec x}\vec v)\cdot\vec v+\nabla_{\vec x}\psi_0+\frac{1}{m}\vec F.
\end{cases}
\end{equation}
%ZZTOP
%where $\vec v$ satisfies:
%
%
%
\begin{comment}
\begin{equation}
\label{MaxVacFull1ninshtrgravortghhghgjkgghklhjgkghghjjkjhjkkggjkhjkhjjhhfhjhklkhkhjjklzzzyyyNWBWHWPPN222jkgghgghojyuoki}
\begin{cases}
curl_{\vec x}\left(curl_{\vec x}\vec v\right)= 0,\\
\frac{\partial}{\partial t}\left(div_{\vec x}\vec v\right)+\vec
v\cdot\nabla_{\vec x}\left(div_{\vec x}\vec
v\right)+\frac{1}{4}\left|d_{\vec x}\vec v+\{d_{\vec x}\vec
v\}^T\right|^2= -4\pi GM,
%\frac{d\vec u}{dt}=-curl_{\vec x}\vec v\times(\vec u-\vec v)+\partial_t\vec v+(\nabla_{\vec x}\vec v)\cdot\vec v+\nabla_{\vec x}\psi_0+\frac{1}{m}\vec F.
\end{cases}
\end{equation}
\end{comment}
%
%
%
where $M$ is the density of gravitational masses. However, till now
we said nothing about the relation between the density of inertial
and gravitational masses. If $\mu$ is the density of inertial
masses, then consistently with the classical Newtonian theory of
gravitation we assume that in the absence of essential
electromagnetic fields we should have
\begin{equation}\label{gghjgghfghd}
M=\mu.
\end{equation}
In order to satisfy the laws of conservation of the linear and
angular momentums and energy, consider the following conserved
proper scalar field $Q$, that we call
"electromagnetical-gravitational" mass density, which is negligible
in the absence of electromagnetic fields and satisfies the identity
\begin{equation}
\label{MaxVacFull1ninshtrgravortghhghgjkgghklhjgkghghjjkjhjkkggjkhjkhjjhhfhjhklkhkhjjklzzzyyyNWNWBWHWPPN}
\frac{\partial Q}{\partial t}+div_{\vec x}\left\{Q\vec v\right\}=-
div_\vec x\left\{\frac{1}{4\pi c}\vec D\times \vec B\right\}
%\frac{d\vec u}{dt}=-curl_{\vec x}\vec v\times(\vec u-\vec v)+\partial_t\vec v+(\nabla_{\vec x}\vec v)\cdot\vec v+\nabla_{\vec x}\psi_0+\frac{1}{m}\vec F.
\end{equation}
in the general case. Then, instead of \er{gghjgghfghd}, for the
general case of gravitational-electromagnetic fields we consider the
following relation between the gravitational and inertial mass
densities
\begin{equation}\label{gghjgghfghdkjgjj}
M=\mu+Q.
\end{equation}
Then by
\er{MaxVacFull1ninshtrgravortghhghgjkgghklhjgkghghjjkjhjkkggjkhjkhjjhhfhjhklkhkhjjklzzzyyyNWBWHWPPN222jkgghggyuhg}
and \er{gghjgghfghdkjgjj} we have the following law of gravitation:
\begin{equation}
\label{MaxVacFull1ninshtrgravortghhghgjkgghklhjgkghghjjkjhjkkggjkhjkhjjhhfhjhklkhkhjjklzzzyyyNWBWHWPPN}
\begin{cases}
curl_{\vec x}\left(curl_{\vec x}\vec v\right)= 0,\\
\frac{\partial}{\partial t}\left(div_{\vec x}\vec v\right)+div_{\vec
x}\left\{\left(div_{\vec x}\vec v\right)\vec
v\right\}+\frac{1}{4}\left|d_{\vec x}\vec v+\{d_{\vec x}\vec
v\}^T\right|^2-\left(div_{\vec x}\vec v\right)^2=  -4\pi G(\mu+Q).
%\frac{d\vec u}{dt}=-curl_{\vec x}\vec v\times(\vec u-\vec v)+\partial_t\vec v+(\nabla_{\vec x}\vec v)\cdot\vec v+\nabla_{\vec x}\psi_0+\frac{1}{m}\vec F.
\end{cases}
\end{equation}
%
%
%
\begin{comment}
\begin{equation}
\label{MaxVacFull1ninshtrgravortghhghgjkgghklhjgkghghjjkjhjkkggjkhjkhjjhhfhjhklkhkhjjklzzzyyyNWBWHWPPN222jkgghggyuhgghjhg}
\begin{cases}
curl_{\vec x}\left(curl_{\vec x}\vec v\right)= 0,\\
\frac{\partial}{\partial t}\left(div_{\vec x}\vec v\right)+\vec
v\cdot\nabla_{\vec x}\left(div_{\vec x}\vec
v\right)+\frac{1}{4}\left|d_{\vec x}\vec v+\{d_{\vec x}\vec
v\}^T\right|^2= -4\pi G(\mu+Q).
%\frac{d\vec u}{dt}=-curl_{\vec x}\vec v\times(\vec u-\vec v)+\partial_t\vec v+(\nabla_{\vec x}\vec v)\cdot\vec v+\nabla_{\vec x}\psi_0+\frac{1}{m}\vec F.
\end{cases}
\end{equation}
\end{comment}
%
%
%
%%%where $G>0$ is a gravitational constant.
% and $k\in\mathbb{R}$ is a some constant parameter.
Then as before, we deduce that the laws
\er{MaxVacFull1ninshtrgravortghhghgjkgghklhjgkghghjjkjhjkkggjkhjkhjjhhfhjhklkhkhjjklzzzyyyNWNWBWHWPPN}
and
\er{MaxVacFull1ninshtrgravortghhghgjkgghklhjgkghghjjkjhjkkggjkhjkhjjhhfhjhklkhkhjjklzzzyyyNWBWHWPPN}
are invariant under the change of non-inertial cartesian coordinate
system, provided that $Q'=Q$. We can rewrite
\er{MaxVacFull1ninshtrgravortghhghgjkgghklhjgkghghjjkjhjkkggjkhjkhjjhhfhjhklkhkhjjklzzzyyyNWBWHWPPN}
as
\begin{equation}
\label{MaxVacFull1ninshtrgravortghhghgjkgghklhjgkghghjjkjhjkkggjkhjkhjjhhfhjhklkhkhjjklzzzyyyhjfgfkjgjNWBWHWPPN}
\begin{cases}
curl_{\vec x}\left(curl_{\vec x}\vec v\right)= 0,\\
div_{\vec x}\left\{\frac{\partial\vec v}{\partial t}+d_\vec x\vec
v\cdot\vec v+\frac{1}{2}\vec v\times curl_{\vec x}\vec v\right\}
%+\frac{1}{4}\left|curl_{\vec x}\vec v\right|^2
= -4\pi G(\mu+Q).
\end{cases}
\end{equation}
In particular in the inertial coordinate system $(*)$ we have:
\begin{equation}
\label{MaxVacFull1ninshtrgravortghhghgjkgghklhjgkghghjjkjhjkkggjkhjkhjjhhfhjhklkhkhjjklzzzyyyhjggjhgghhjhNWBWHWPPN}
\begin{cases}
curl_{\vec x}\vec v= 0,\\
div_{\vec x}\left\{\frac{\partial\vec v}{\partial t}+d_\vec x\vec
v\cdot\vec v\right\}= -4\pi G(\mu+Q),
%\frac{d\vec u}{dt}=-curl_{\vec x}\vec v\times(\vec u-\vec v)+\partial_t\vec v+(\nabla_{\vec x}\vec v)\cdot\vec v+\nabla_{\vec x}\psi_0+\frac{1}{m}\vec F.
\end{cases}
\end{equation}
that we can rewrite as
\begin{equation}
\label{MaxVacFull1ninshtrgravortghhghgjkgghklhjgkghghjjkjhjkkggjkhjkhjjhhfhjhklkhkhjjklzzzyyyhjggjhgghhjhNWNWBWHWPPN}
\begin{cases}
curl_{\vec x}\vec v= 0,\\
\frac{\partial\vec v}{\partial t}+d_\vec x\vec v\cdot\vec v=
-\nabla_{\vec x}\Phi,
%\frac{d\vec u}{dt}=-curl_{\vec x}\vec v\times(\vec u-\vec v)+\partial_t\vec v+(\nabla_{\vec x}\vec v)\cdot\vec v+\nabla_{\vec x}\psi_0+\frac{1}{m}\vec F.
\end{cases}
\end{equation}
where $\Phi$ is the scalar gravitational potential: a proper scalar
field which satisfies in every coordinate system:
\begin{equation}
\label{MaxVacFull1ninshtrgravortghhghgjkgghklhjgkghghjjkjhjkkggjkhjkhjjhhfhjhklkhkhjjklzzzyyyhjggjhgghhjhNWNWNWBWHWPPN}
\Delta_{\vec x}\Phi=4\pi G(\mu+Q).
\end{equation}
%Here $G$ is the gravitational constant.
Since in the system $(*)$ we have $curl_{\vec x}\vec v=0$ we can
write
\begin{equation}
\label{MaxVacFull1ninshtrgravortghhghgjkgghklhjgkghghjjkjhjkkggjkhjkhjjhhfhjhklkhkhjjklzzzyyyhjggjhgghhjhNWNWNWNWNWBWHWPPN}
\begin{cases}
\vec v=\nabla_{\vec x}Z,\\
\frac{\partial Z}{\partial t}+\frac{1}{2}\left|\nabla_{\vec
x}Z\right|^2=-\Phi.
%\frac{d\vec u}{dt}=-curl_{\vec x}\vec v\times(\vec u-\vec v)+\partial_t\vec v+(\nabla_{\vec x}\vec v)\cdot\vec v+\nabla_{\vec x}\psi_0+\frac{1}{m}\vec F.
\end{cases}
\end{equation}
\begin{remark}\label{ghghvghhggghKKK}
Lemma \ref{fgbfghfh} from Appendix gives some insight that the
"electromagnetical-gravitational" mass density $Q$ in
\er{MaxVacFull1ninshtrgravortghhghgjkgghklhjgkghghjjkjhjkkggjkhjkhjjhhfhjhklkhkhjjklzzzyyyNWNWBWHWPPN}
should have the values of the same order as the quantity
$\frac{1}{c^2}\left(|\vec D|^2+|\vec B|^2\right)$ and therefore, in
the usual circumstances is negligible with respect to the inertial
mass density $\mu$. Thus we can write $Q\approx 0$ in
\er{MaxVacFull1ninshtrgravortghhghgjkgghklhjgkghghjjkjhjkkggjkhjkhjjhhfhjhklkhkhjjklzzzyyyNWBWHWPPN},
i.e. the force of gravity in an inertial coordinate system
approximately equals to the classical Newtonian force of gravity.
\end{remark}





\subsection{Conservation of the momentum, angular momentum and
energy} Consider the Maxwell equation in the vacuum in some
cartesian coordinate system $(*)$:
\begin{equation}\label{MaxVacFull1bjkgjhjhgjaaahkjhhjzzzyyykkknnnNWBWHWPPN}
\begin{cases}
curl_{\vec x} \vec H= \frac{4\pi}{c}\vec j+\frac{1}{c}\frac{\partial
\vec D}{\partial t},\\
div_{\vec x} \vec D= 4\pi\rho,\\
curl_{\vec x} \vec E+\frac{1}{c}\frac{\partial \vec B}{\partial t}=0,\\
div_{\vec x} \vec B=0,\\
\vec E=\vec D-\frac{1}{c}\,\vec v\times \vec B,\\
\vec H=\vec B+\frac{1}{c}\,\vec v\times \vec D,
\end{cases}
\end{equation}
and consistently with
\er{noninchgravortbstrjgghguittu2gjgghhjhghjhjgghgghghghtytythvfghfgghjgg},
consider in the system $(*)$ the second Law of Newton for the moving
continuum with the inertial mass density $\mu$ and the field of
velocities $\vec u$:
\begin{equation}\label{MaxVacFull1ninshtrgravortghhghgjkgghklhjgkghghjjkjhjkkggjkhjkhjjhhfhjhkzzzyyykkknnnNWBWHWPPN}
\frac{\partial\vec u}{\partial t}+d_{\vec x}\vec u\cdot \vec
u=-(\vec u-\vec v)\times curl_{\vec x}\vec v +\partial_t\vec
v+d_{\vec x}\vec v\cdot \vec v+\frac{1}{\mu}\tilde{\vec F},
%\frac{d\vec u}{dt}=-curl_{\vec x}\vec v\times(\vec u-\vec v)+\partial_t\vec v+(\nabla_{\vec x}\vec v)\cdot\vec v+\nabla_{\vec x}\psi_0+\frac{1}{m}\vec F.
\end{equation}
where $\tilde{\vec F}$ is the total volume density of all
non-gravitational forces acting on the continuum with mass density
$\mu$. Thus, again by \er{apfrm9}, we rewrite
\er{MaxVacFull1ninshtrgravortghhghgjkgghklhjgkghghjjkjhjkkggjkhjkhjjhhfhjhkzzzyyykkknnnNWBWHWPPN}
as:
\begin{multline}\label{MaxVacFull1ninshtrgravortghhghgjkgghklhjgkghghjjkjhjkkggjkhjkhjjhhfhjhkjkhbbgjhzzzyyykkknnnNWBWHWPPN}
\mu\left(\frac{\partial\vec u}{\partial t}+d_{\vec x}\vec u\cdot
\vec u\right)=\frac{\partial(\mu\vec u)}{\partial t}+div_{\vec
x}\left\{\mu\vec u\otimes\vec u\right\}=\\-\mu\vec u\times
curl_{\vec x}\vec v+\mu\partial_{t}\vec
v+\mu\nabla_{\vec x}\left(\frac{1}{2}|\vec v|^2\right)+\rho\vec E+\frac{1}{c}\vec j\times\vec B+\vec F=\\
=-\mu(\vec u-\vec v)\times curl_{\vec x}\vec v +\mu\partial_t\vec v+
d_{\vec x}\vec v\cdot (\mu\vec v)+\rho\vec E+\frac{1}{c}\vec
j\times\vec B+\vec F.
%\frac{d\vec u}{dt}=-curl_{\vec x}\vec v\times(\vec u-\vec v)+\partial_t\vec v+(\nabla_{\vec x}\vec v)\cdot\vec v+\nabla_{\vec x}\psi_0+\frac{1}{m}\vec F.
\end{multline}
where $\rho\vec E+\frac{1}{c}\vec j\times\vec B$ is the volume
density of the Lorentz force and $\vec F$ is the total volume
density of all non-gravitational and non-electromagnetic forces
acting on the continuum with mass density $\mu$, which satisfies the
continuum equation:
\begin{equation}\label{MaxVacFull1ninshtrgravortghhghgjkgghklhjgkghghjjkjhjkkggjkhjkhjjhhfhjhkhjjkhjgzzzyyykkknnnNWBWHWPPN}
\frac{\partial\mu}{\partial t}+div_{\vec x}\left(\mu\vec u\right)=0.
\end{equation}
Then, again by \er{apfrm9}, we rewrite
\er{MaxVacFull1ninshtrgravortghhghgjkgghklhjgkghghjjkjhjkkggjkhjkhjjhhfhjhkjkhbbgjhzzzyyykkknnnNWBWHWPPN}
as
\begin{equation}\label{vhfffngghkjgghggtghjgfhjoyuiyuyhiyyukukyihyuSYSPNNWhgjgghyuyy8yuyughghhCCmmGGKK}
\mu\frac{\partial}{\partial t}\left\{\left(\vec u-\vec v\right)
\right\} + \mu \,d_{\vec x}\left\{\left(\vec u-\vec
v\right)\right\}\cdot\vec u +\mu \left\{d_{\vec x}\vec
v\right\}^T\cdot\left(\vec u-\vec v\right)=\rho\vec
E+\frac{1}{c}\vec j\times \vec B+\vec F.
\end{equation}
Thus by
\er{vhfffngghkjgghggtghjgfhjoyuiyuyhiyyukukyihyuSYSPNNWhgjgghyuyy8yuyughghhCCmmGGKK}
and
\er{MaxVacFull1ninshtrgravortghhghgjkgghklhjgkghghjjkjhjkkggjkhjkhjjhhfhjhkhjjkhjgzzzyyykkknnnNWBWHWPPN}
we obtain
\begin{equation}\label{vhfffngghkjgghggtghjgfhjoyuiyuyhiyyukukyihyuSYSPNNWhgjgghyuyy8yuyughghhghghhjjhCCmmGGKK}
\frac{\partial}{\partial t}\left\{ \mu \left(\vec u-\vec v\right)
\right\} + div_{\vec x}\left\{\mu \left(\vec u-\vec
v\right)\otimes\vec u\right\} +\mu \left\{d_{\vec x}\vec
v\right\}^T\cdot\left(\vec u-\vec v\right)=\rho\vec
E+\frac{1}{c}\vec j\times \vec B+\vec F.
\end{equation}
Moreover, multiplying
\er{vhfffngghkjgghggtghjgfhjoyuiyuyhiyyukukyihyuSYSPNNWhgjgghyuyy8yuyughghhCCmmGGKK}
by $(\vec u-\vec v)$ and using
\er{MaxVacFull1ninshtrgravortghhghgjkgghklhjgkghghjjkjhjkkggjkhjkhjjhhfhjhkhjjkhjgzzzyyykkknnnNWBWHWPPN}
we have:
\begin{multline}\label{vhfffngghkjgghggtghjgfhjoyuiyuyhiyyukukyihyuSYSPNNWhgjgghyuyy8yuyughghhyttyytCCmmGGKK}
\frac{\partial}{\partial t}\left\{\frac{\mu}{2}\left|\vec u-\vec
v\right|^2\right\}+div_{\vec x}\left\{\frac{\mu}{2}\left|\vec u-\vec
v\right|^2\vec u\right\}=
\\ -\frac{1}{2}\left(d_{\vec x}\vec v+\left\{d_{\vec x}\vec v\right\}^T\right)^T:\left\{\mu\left(\vec u-\vec v\right)\otimes \left(\vec
u-\vec v\right)\right\}+\vec j\cdot\vec E -\vec v\cdot\left(\rho\vec
E+\frac{1}{c}\vec j\times \vec B\right)+(\vec u-\vec v)\cdot\vec F.
\end{multline}
On the other hand, by Lemma \ref{guigiukhn} and Lemma \ref{fgbfghfh}
in the Appendix we have:
\begin{multline}\label{hvkgkjgkjbjkjjkgjglhhkhjyuyghjhhjhjzzzyyykkknnnNWBWHWPPNNewZZZghghghCCmmGGKK}
\frac{\partial}{\partial t}\left(\frac{1}{4\pi c}\,\vec D\times \vec
B\right)+div_\vec x\left\{\left(\frac{1}{4\pi c}\vec D\times \vec
B\right)\otimes \vec v\right\}=-(d_\vec x \vec
v)^T\cdot\left(\frac{1}{4\pi c}\vec D\times \vec
B\right)\\+\frac{1}{4\pi}div_\vec x\left\{\vec D\otimes \vec D+\vec
B\otimes \vec B-\frac{1}{2}\left(|\vec D|^2+|\vec
B|^2\right)I\right\}-\left(\rho \vec E+\frac{1}{c}\vec j\times \vec
B\right),
%=\\ \frac{1}{c}\left\{d_\vec x(D\times B)\right\}^T\cdot v+div_{\vec x}\left\{D\otimes D+B\otimes
%B-\frac{1}{2}\left(|D|^2+|B|^2+\frac{2}{c}v\cdot(D\times
%B)\right)I\right\}-4\pi\rho E-\frac{4\pi}{c}\vec j\times B.
\end{multline}
and
\begin{multline}\label{hvkgkjgkjbjbbklnknhihiokhhfjffghvjmbjhjkhlkzzzyyykkknnnNWBWHWPPNNewZZZgghghCCmmGGKK}
\frac{\partial}{\partial t}\left(\frac{|\vec D|^2+|\vec
B|^2}{8\pi}\right)+div_\vec x\left\{\left(\frac{|\vec D|^2+|\vec
B|^2}{8\pi}\right)\vec v\right\}=\\
\frac{1}{4\pi}div_\vec x\left\{(\vec D\otimes \vec D+ \vec B\otimes
\vec B)\cdot \vec v-\frac{1}{2}\left(|\vec D|^2+|\vec
B|^2\right)\vec v-c \vec D\times \vec B\right\}
\\-\left\{\frac{1}{4\pi}\left(div_\vec x\left\{\vec D\otimes \vec D+\vec B\otimes
\vec B-\frac{1}{2}\left(|\vec D|^2+|\vec
B|^2\right)I\right\}\right)-\left(\rho \vec E+\frac{1}{c}\,\vec
j\times \vec B\right)\right\}\cdot \vec v-\vec j\cdot \vec E =\\
-\frac{c}{4\pi}div_\vec x\left\{ \vec D\times \vec
B\right\}+\left(\rho \vec E+\frac{1}{c}\,\vec j\times \vec
B\right)\cdot \vec v-\vec j\cdot \vec E
\\+\frac{1}{8\pi}\left(d_{\vec x}\vec v+\left\{d_{\vec x}\vec
v\right\}^T\right):\left\{\vec D\otimes \vec D+\vec B\otimes \vec
B-\frac{1}{2}\left(|\vec D|^2+|\vec B|^2\right)I\right\}.
\end{multline}
Thus by
\er{vhfffngghkjgghggtghjgfhjoyuiyuyhiyyukukyihyuSYSPNNWhgjgghyuyy8yuyughghhghghhjjhCCmmGGKK}
and
\er{hvkgkjgkjbjkjjkgjglhhkhjyuyghjhhjhjzzzyyykkknnnNWBWHWPPNNewZZZghghghCCmmGGKK}
we have
\begin{multline}\label{vhfffngghkjgghggtghjgfhjoyuiyuyhiyyukukyihyuSYSPNNWhgjgghyuyy8yuyughghhghghhjjhhjhjkhjCCmmGGKK}
\frac{\partial}{\partial t}\left\{ \mu \left(\vec u-\vec v\right)
+\frac{1}{4\pi c}\,\vec D\times \vec B\right\}+ div_{\vec
x}\left\{\left(\mu \left(\vec u-\vec v\right)+\frac{1}{4\pi c}\vec
D\times \vec B\right)\otimes \vec v\right\}
\\+\left\{d_{\vec x}\vec v\right\}^T\cdot\left\{\mu\left(\vec u-\vec v\right)
+\frac{1}{4\pi c}\vec D\times \vec B\right\}=\\
%- div_{\vec x}\left\{\left(1-\frac{1}{c^2}\left|\vec u-\vec v\right|^2\right)^{-\frac{1}{2}}\left(\vec u-\vec v\right)\otimes(\vec u-\vec v)\right\}
\frac{1}{4\pi}div_\vec x\left\{\vec D\otimes \vec D+\vec B\otimes
\vec B-\frac{1}{2}\left(|\vec D|^2+|\vec B|^2\right)I-4\pi\mu
\left(\vec u-\vec v\right)\otimes(\vec u-\vec v)\right\}+\vec F,
\end{multline}
and by
\er{vhfffngghkjgghggtghjgfhjoyuiyuyhiyyukukyihyuSYSPNNWhgjgghyuyy8yuyughghhyttyytCCmmGGKK}
and
\er{hvkgkjgkjbjbbklnknhihiokhhfjffghvjmbjhjkhlkzzzyyykkknnnNWBWHWPPNNewZZZgghghCCmmGGKK}
we have
\begin{multline}\label{vhfffngghkjgghggtghjgfhjoyuiyuyhiyyukukyihyuSYSPNNWhgjgghyuyy8yuyughghhyttyytgghhgCCmmGGKK}
\frac{\partial}{\partial t}\left\{\frac{\mu}{2}\left|\vec u-\vec
v\right|^2+\frac{|\vec D|^2+|\vec B|^2}{8\pi}\right\}+div_{\vec
x}\left\{\left(\frac{\mu}{2}\left|\vec u-\vec v\right|^2+\frac{|\vec
D|^2+|\vec B|^2}{8\pi}\right)\vec v\right\}\\+div_{\vec
x}\left\{\frac{\mu}{2}\left|\vec u-\vec v\right|^2(\vec u-\vec
v)+\frac{c}{4\pi}\vec D\times \vec B\right\}=
-\frac{1}{2}\left(d_{\vec x}\vec v+\left\{d_{\vec x}\vec
v\right\}^T\right):\left\{\mu\left(\vec u-\vec v\right)\otimes
\left(\vec u-\vec v\right)\right\}\\+\frac{1}{8\pi}\left(d_{\vec
x}\vec v+\left\{d_{\vec x}\vec v\right\}^T\right):\left\{\vec
D\otimes \vec D+\vec B\otimes \vec B-\frac{1}{2}\left(|\vec
D|^2+|\vec B|^2\right)I
%-4\pi\mu\left(1-\frac{1}{c^2}\left|\vec u-\vec v\right|^2\right)^{-\frac{1}{2}}\left(\vec u-\vec v\right)\otimes \left(\vec u-\vec v\right)
\right\}
%-div_\vec x\left\{ \frac{c}{4\pi}\vec D\times \vec B\right\}
+(\vec u-\vec v)\cdot\vec F.
\end{multline}
In particular by
\er{vhfffngghkjgghggtghjgfhjoyuiyuyhiyyukukyihyuSYSPNNWhgjgghyuyy8yuyughghhyttyytgghhgCCmmGGKK}
and
\er{vhfffngghkjgghggtghjgfhjoyuiyuyhiyyukukyihyuSYSPNNWhgjgghyuyy8yuyughghhghghhjjhhjhjkhjCCmmGGKK}
we have:
\begin{multline}\label{vhfffngghkjgghggtghjgfhjoyuiyuyhiyyukukyihyuSYSPNNWhgjgghyuyy8yuyughghhyttyytgghhghghgghjhkjhCCmmGGKK}
\frac{\partial}{\partial t}\left\{ \frac{\mu}{2}\left|\vec u-\vec
v\right|^2+\frac{|\vec D|^2+|\vec B|^2}{8\pi}\right\}+div_{\vec
x}\left\{\left(\frac{\mu}{2}\left|\vec u-\vec
v\right|^2+\left(\frac{|\vec D|^2+|\vec
B|^2}{8\pi}\right)\right)\vec v\right\}\\+div_{\vec
x}\left\{\frac{\mu}{2}\left|\vec u-\vec v\right|^2(\vec u-\vec
v)+\frac{c}{4\pi}\vec D\times \vec B\right\}= -\vec
v\cdot\frac{\partial}{\partial t}\left\{ \mu
 \left(\vec
u-\vec v\right) +\frac{1}{4\pi c}\,\vec D\times \vec B\right\} \\-
div_{\vec x}\left\{\left(\left(\mu\left(\vec u-\vec
v\right)+\frac{1}{4\pi c}\vec D\times \vec B\right)\cdot\vec
v\right)\vec v\right\}-div_{\vec x}\left\{\mu \left(\left(\vec
u-\vec v\right)\cdot\vec v\right) \left(\vec u-\vec
v\right)\right\}\\+\frac{1}{4\pi}div_{\vec x}\left\{\left(\vec
D\otimes \vec D+\vec B\otimes \vec B-\frac{1}{2}\left(|\vec
D|^2+|\vec B|^2\right)I\right)\cdot\vec v
%-4\pi\mu\left(1-\frac{1}{c^2}\left|\vec u-\vec v\right|^2\right)^{-\frac{1}{2}}\left(\vec u-\vec v\right)\otimes \left(\vec u-\vec v\right)
\right\}
%-div_\vec x\left\{ \frac{c}{4\pi}\vec D\times \vec B\right\}
+\vec u\cdot\vec F,
\end{multline}
and thus
\begin{multline}\label{vhfffngghkjgghggtghjgfhjoyuiyuyhiyyukukyihyuSYSPNNWhgjgghyuyy8yuyughghhyttyytgghhghghgghjhkjhCCmmGGKKuiiuui}
\frac{\partial}{\partial t}\left\{ \frac{\mu}{2}\left|\vec u-\vec
v\right|^2+\frac{|\vec D|^2+|\vec B|^2}{8\pi}+\vec v\cdot\left( \mu
 \left(\vec
u-\vec v\right) +\frac{1}{4\pi c}\,\vec D\times \vec
B\right)\right\}\\+div_{\vec x}\left\{\left(\frac{\mu}{2}\left|\vec
u-\vec v\right|^2+\left(\frac{|\vec D|^2+|\vec
B|^2}{8\pi}\right)\right)\vec v\right\}+div_{\vec
x}\left\{\frac{\mu}{2}\left|\vec u-\vec v\right|^2(\vec u-\vec
v)+\frac{c}{4\pi}\vec D\times \vec B\right\}\\= \frac{\partial\vec
v}{\partial t}\cdot\left( \mu
 \left(\vec
u-\vec v\right) +\frac{1}{4\pi c}\,\vec D\times \vec B\right)-
div_{\vec x}\left\{\left(\left(\mu\left(\vec u-\vec
v\right)+\frac{1}{4\pi c}\vec D\times \vec B\right)\cdot\vec
v\right)\vec v\right\}\\-div_{\vec x}\left\{\mu \left(\left(\vec
u-\vec v\right)\cdot\vec v\right) \left(\vec u-\vec
v\right)\right\}+\frac{1}{4\pi}div_{\vec x}\left\{\left(\vec
D\otimes \vec D+\vec B\otimes \vec B-\frac{1}{2}\left(|\vec
D|^2+|\vec B|^2\right)I\right)\cdot\vec v
%-4\pi\mu\left(1-\frac{1}{c^2}\left|\vec u-\vec v\right|^2\right)^{-\frac{1}{2}}\left(\vec u-\vec v\right)\otimes \left(\vec u-\vec v\right)
\right\}
%-div_\vec x\left\{ \frac{c}{4\pi}\vec D\times \vec B\right\}
+\vec u\cdot\vec F.
\end{multline}
Moreover, by
\er{vhfffngghkjgghggtghjgfhjoyuiyuyhiyyukukyihyuSYSPNNWhgjgghyuyy8yuyughghhghghhjjhhjhjkhjCCmmGGKK}
and \er{apfrm9} we have
\begin{multline}\label{vhfffngghkjgghggtghjgfhjoyuiyuyhiyyukukyihyuSYSPNNWhgjgghyuyy8yuyughghhghghhjjhhjhjkhjCCmmGGKKJJKK}
\frac{\partial}{\partial t}\left\{ \mu \left(\vec u-\vec v\right)
+\frac{1}{4\pi c}\,\vec D\times \vec B\right\}+ div_{\vec
x}\left\{\left(\mu \left(\vec u-\vec v\right)+\frac{1}{4\pi c}\vec
D\times \vec B\right)\otimes \vec v+\vec v\otimes\left(\mu\left(\vec
u-\vec v\right)+\frac{1}{4\pi c}\vec D\times \vec B\right)\right\}
\\+\left(\mu\left(\vec u-\vec v\right)
+\frac{1}{4\pi c}\vec D\times \vec B\right)\times curl_{\vec x}\vec
v-\left(div_{\vec x}\left\{\mu \left(\vec u-\vec v\right)
+\frac{1}{4\pi c}\,\vec D\times \vec B\right\}\right)\vec v=\\
%- div_{\vec x}\left\{\left(1-\frac{1}{c^2}\left|\vec u-\vec v\right|^2\right)^{-\frac{1}{2}}\left(\vec u-\vec v\right)\otimes(\vec u-\vec v)\right\}
\frac{1}{4\pi}div_\vec x\left\{\vec D\otimes \vec D+\vec B\otimes
\vec B-\frac{1}{2}\left(|\vec D|^2+|\vec B|^2\right)I-4\pi\mu
\left(\vec u-\vec v\right)\otimes(\vec u-\vec v)\right\}+\vec F,
\end{multline}
and by
\er{vhfffngghkjgghggtghjgfhjoyuiyuyhiyyukukyihyuSYSPNNWhgjgghyuyy8yuyughghhghghhjjhhjhjkhjCCmmGGKK}
and \er{apfrm6kkk} we have
\begin{multline}\label{vhfffngghkjgghggtghjgfhjoyuiyuyhiyyukukyihyuSYSPNNWhgjgghyuyy8yuyughghhghghhjjhhjhjkhjCCmmGGKKJJKKMod}
\frac{\partial}{\partial t}\left\{ \mu \left(\vec u-\vec v\right)
+\frac{1}{4\pi c}\,\vec D\times \vec B\right\}- curl_{\vec
x}\left\{\vec v\times\left(\mu\left(\vec u-\vec
v\right)+\frac{1}{4\pi c}\vec D\times \vec B\right)\right\}
\\+\left(div_{\vec x}\left\{\mu \left(\vec u-\vec v\right)
+\frac{1}{4\pi c}\,\vec D\times \vec B\right\}\right)\vec
v+\left(d_{\vec x}\vec v+\left\{d_{\vec x}\vec
v\right\}^T\right)\cdot\left\{\mu\left(\vec u-\vec v\right)
+\frac{1}{4\pi c}\vec D\times \vec B\right\}=\\
%- div_{\vec x}\left\{\left(1-\frac{1}{c^2}\left|\vec u-\vec v\right|^2\right)^{-\frac{1}{2}}\left(\vec u-\vec v\right)\otimes(\vec u-\vec v)\right\}
\frac{1}{4\pi}div_\vec x\left\{\vec D\otimes \vec D+\vec B\otimes
\vec B-\frac{1}{2}\left(|\vec D|^2+|\vec B|^2\right)I-4\pi\mu
\left(\vec u-\vec v\right)\otimes(\vec u-\vec v)\right\}+\vec F.
\end{multline}
On the other hand for every vector fields $\Gamma:\R^3\to\R^3$ and
$\Lambda:\R^3\to\R^3$ and every scalar field $P:\R^3\to\R$ we have:
\begin{multline}\label{yguiyiu7yytyutkhjzzzyyykkknnnNWBWHWPPNKK}
\vec x\times div_{\vec
x}\{\Gamma\otimes\Lambda+\Lambda\otimes\Gamma\}=div_{\vec
x}\left\{(\vec x\times\Gamma)\otimes\Lambda+(\vec
x\times\Lambda)\otimes\Gamma\right\},\\
\vec x\times div_{\vec x}\{P\Gamma\otimes\Gamma\}=div_{\vec
x}\left\{P(\vec
x\times\Gamma)\otimes\Gamma\right\}\quad\text{and}\quad\vec
x\times\nabla_{\vec x}P=-curl_{\vec x}\{P\,\vec x\}.
\end{multline}
Thus inserting \er{yguiyiu7yytyutkhjzzzyyykkknnnNWBWHWPPNKK} into
\er{vhfffngghkjgghggtghjgfhjoyuiyuyhiyyukukyihyuSYSPNNWhgjgghyuyy8yuyughghhghghhjjhhjhjkhjCCmmGGKKJJKK}
we obtain:
\begin{multline}\label{hvkgkjgkjbjkjjkgjglhhkhjyuyghjhhjhjfghfdhgdhdfdhzzzyyyjffjjkkhhkgjgkjhfhjhgfffjgjhgjffgjggjgjggjhgkkkggjgjgmomnnnNWKK}
\frac{\partial}{\partial t}\left\{\vec x\times\left(\mu(\vec u-\vec
v)+\frac{1}{4\pi c}\,\vec D\times \vec B\right)\right\}+div_\vec
x\left\{\mu\left(\vec x\times(\vec u-\vec v)\right)\otimes(\vec
u-\vec v)\right\}\\+div_{\vec x}\left\{\left(\vec x \times\vec
v\right)\otimes\left(\mu(\vec u-\vec v)+\frac{1}{4\pi c}\vec D\times
\vec B\right)+\left(\vec x\times\left(\mu(\vec u-\vec
v)+\frac{1}{4\pi c}\vec D\times \vec B\right)\right)\otimes \vec
v\right\}\\-\vec x\times\left\{\left(div_{\vec x}\left\{\mu(\vec
u-\vec v)+\frac{1}{4\pi c}\vec D\times \vec B\right\}\right)\vec
v-\left(\mu(\vec u-\vec v)+\frac{1}{4\pi c}\vec
D\times \vec B\right)\times curl_{\vec x}\vec v\right\}\\
=\frac{1}{4\pi}div_\vec x\left\{(\vec x\times \vec D)\otimes \vec
D+(\vec x\times\vec B)\otimes \vec B\right\}+curl_{\vec
x}\left\{\frac{1}{2}\left(|\vec D|^2+|\vec B|^2\right)\vec
x\right\}+\vec x\times\vec F.
%=\\ \frac{1}{c}\left\{d_\vec x(D\times B)\right\}^T\cdot v+div_{\vec x}\left\{D\otimes D+B\otimes
%B-\frac{1}{2}\left(|D|^2+|B|^2+\frac{2}{c}v\cdot(D\times
%B)\right)I\right\}-4\pi\rho E-\frac{4\pi}{c}\vec j\times B.
\end{multline}

 Next assume that the system $(*)$ is inertial. Then, since by \er{MaxVacFull1ninshtrgravortghhghgjkgghklhjgkghghjjkjhjkkggjkhjkhjjhhfhjhklkhkhjjklzzzyyyNWNWBWHWPPN}
and
\er{MaxVacFull1ninshtrgravortghhghgjkgghklhjgkghghjjkjhjkkggjkhjkhjjhhfhjhkhjjkhjgzzzyyykkknnnNWBWHWPPN}
we have:
\begin{equation}
\label{MaxVacFull1ninshtrgravortghhghgjkgghklhjgkghghjjkjhjkkggjkhjkhjjhhfhjhklkhkhjjklzzzyyyNWNWBWHWPPNKKKK}
\frac{\partial}{\partial t}\left(\mu+Q\right)+div_{\vec
x}\left\{\left(\mu+Q\right)\vec v\right\}=- div_\vec
x\left\{\mu(\vec u-\vec v)+\frac{1}{4\pi c}\vec D\times \vec
B\right\},
%\frac{d\vec u}{dt}=-curl_{\vec x}\vec v\times(\vec u-\vec v)+\partial_t\vec v+(\nabla_{\vec x}\vec v)\cdot\vec v+\nabla_{\vec x}\psi_0+\frac{1}{m}\vec F.
\end{equation}
by
\er{MaxVacFull1ninshtrgravortghhghgjkgghklhjgkghghjjkjhjkkggjkhjkhjjhhfhjhklkhkhjjklzzzyyyNWNWBWHWPPNKKKK},
\er{MaxVacFull1ninshtrgravortghhghgjkgghklhjgkghghjjkjhjkkggjkhjkhjjhhfhjhklkhkhjjklzzzyyyhjggjhgghhjhNWNWBWHWPPN}
and
\er{MaxVacFull1ninshtrgravortghhghgjkgghklhjgkghghjjkjhjkkggjkhjkhjjhhfhjhklkhkhjjklzzzyyyhjggjhgghhjhNWNWNWBWHWPPN}
we have
\begin{multline}\label{vhfffngghkjgghggtghjgfhjoyuiyuyhiyyukukyihyuSYSPNNWhgjgghyuyy8yuyughghhghghhjjhhjhjkhjCCmmGGKKJJKKjkhjh}
-\left(div_{\vec x}\left\{\mu \left(\vec u-\vec v\right)
+\frac{1}{4\pi c}\,\vec D\times \vec B\right\}\right)\vec
v=\left(\frac{\partial}{\partial t}\left(\mu+Q\right)\right)\vec
v+\left(div_{\vec x}\left\{\left(\mu+Q\right)\vec
v\right\}\right)\vec v=\\
\frac{\partial}{\partial t}\left(\left(\mu+Q\right)\vec
v\right)+div_{\vec x}\left\{\left(\mu+Q\right)\vec v\otimes\vec
v\right\}-\left(\mu+Q\right)\left(\frac{\partial\vec v}{\partial
t}+d_{\vec x}\vec v\cdot\vec v\right)=\\ \frac{\partial}{\partial
t}\left(\left(\mu+Q\right)\vec v\right)+div_{\vec
x}\left\{\left(\mu+Q\right)\vec v\otimes\vec
v\right\}+\left(\mu+Q\right)\nabla_{\vec x}\Phi=\\
\frac{\partial}{\partial t}\left(\left(\mu+Q\right)\vec
v\right)+div_{\vec x}\left\{\left(\mu+Q\right)\vec v\otimes\vec
v\right\}+\frac{1}{4\pi G}\left(\Delta_{\vec
x}\Phi\right)\nabla_{\vec x}\Phi=\\
\frac{\partial}{\partial t}\left(\left(\mu+Q\right)\vec
v\right)+div_{\vec x}\left\{\left(\mu+Q\right)\vec v\otimes\vec
v\right\}+\frac{1}{4\pi G}div_{\vec x}\left\{\nabla_{\vec
x}\Phi\otimes \nabla_{\vec x}\Phi-\frac{1}{2}\left|\nabla_{\vec
x}\Phi\right|^2I\right\}.
\end{multline}
Moreover, by \er{yguiyiu7yytyutkhjzzzyyykkknnnNWBWHWPPNKK} and
\er{vhfffngghkjgghggtghjgfhjoyuiyuyhiyyukukyihyuSYSPNNWhgjgghyuyy8yuyughghhghghhjjhhjhjkhjCCmmGGKKJJKKjkhjh}
we have
\begin{multline}\label{vhfffngghkjgghggtghjgfhjoyuiyuyhiyyukukyihyuSYSPNNWhgjgghyuyy8yuyughghhghghhjjhhjhjkhjCCmmGGKKJJKKjkhjhyyyuu}
-\left(div_{\vec x}\left\{\mu \left(\vec u-\vec v\right)
+\frac{1}{4\pi c}\,\vec D\times \vec B\right\}\right)\vec
x\times\vec v= \frac{\partial}{\partial
t}\left(\left(\mu+Q\right)\vec x\times\vec v\right)+div_{\vec
x}\left\{\left(\mu+Q\right)\left(\vec x\times\vec
v\right)\otimes\vec v\right\}\\+\frac{1}{4\pi G}div_{\vec
x}\left\{\left(\vec x\times\nabla_{\vec x}\Phi\right)\otimes
\nabla_{\vec x}\Phi\right\}+\frac{1}{8\pi G}curl_{\vec
x}\left\{\left|\nabla_{\vec x}\Phi\right|^2\vec x\right\}.
\end{multline}
Therefore, by inserting
\er{vhfffngghkjgghggtghjgfhjoyuiyuyhiyyukukyihyuSYSPNNWhgjgghyuyy8yuyughghhghghhjjhhjhjkhjCCmmGGKKJJKKjkhjh}
into
\er{vhfffngghkjgghggtghjgfhjoyuiyuyhiyyukukyihyuSYSPNNWhgjgghyuyy8yuyughghhghghhjjhhjhjkhjCCmmGGKKJJKK}
and using
\er{MaxVacFull1ninshtrgravortghhghgjkgghklhjgkghghjjkjhjkkggjkhjkhjjhhfhjhklkhkhjjklzzzyyyhjggjhgghhjhNWNWBWHWPPN}
we deduce the following conservation of the momentum:
\begin{multline}\label{vhfffngghkjgghggtghjgfhjoyuiyuyhiyyukukyihyuSYSPNNWhgjgghyuyy8yuyughghhghghhjjhhjhjkhjCCmmGGKKJJKKhghgghghgh}
\frac{\partial}{\partial t}\left\{ \mu\vec u+Q\vec v +\frac{1}{4\pi
c}\,\vec D\times \vec B\right\}+ div_{\vec x}\left\{\mu\vec
u\otimes\vec u+Q\vec v\otimes\vec v+\left(\frac{1}{4\pi c}\vec
D\times \vec B\right)\otimes \vec v+\vec v\otimes\left(\frac{1}{4\pi
c}\vec D\times \vec B\right)\right\}
\\+\frac{1}{4\pi G}div_{\vec x}\left\{\nabla_{\vec
x}\Phi\otimes \nabla_{\vec x}\Phi-\frac{1}{2}\left|\nabla_{\vec
x}\Phi\right|^2I\right\}-div_\vec x\left\{\mu
\left(\vec u-\vec v\right)\otimes(\vec u-\vec v)\right\}=\\
\frac{\partial}{\partial t}\left\{ \mu \left(\vec u-\vec v\right)
+\frac{1}{4\pi c}\,\vec D\times \vec B\right\}+ div_{\vec
x}\left\{\left(\mu \left(\vec u-\vec v\right)+\frac{1}{4\pi c}\vec
D\times \vec B\right)\otimes \vec v+\vec v\otimes\left(\mu\left(\vec
u-\vec v\right)+\frac{1}{4\pi c}\vec D\times \vec B\right)\right\}
\\+\frac{\partial}{\partial t}\left(\left(\mu+Q\right)\vec
v\right)+div_{\vec x}\left\{\left(\mu+Q\right)\vec v\otimes\vec
v\right\}+\frac{1}{4\pi G}div_{\vec x}\left\{\nabla_{\vec
x}\Phi\otimes \nabla_{\vec x}\Phi-\frac{1}{2}\left|\nabla_{\vec
x}\Phi\right|^2I\right\}=\\
%- div_{\vec x}\left\{\left(1-\frac{1}{c^2}\left|\vec u-\vec v\right|^2\right)^{-\frac{1}{2}}\left(\vec u-\vec v\right)\otimes(\vec u-\vec v)\right\}
\frac{1}{4\pi}div_\vec x\left\{\vec D\otimes \vec D+\vec B\otimes
\vec B-\frac{1}{2}\left(|\vec D|^2+|\vec B|^2\right)I-4\pi\mu
\left(\vec u-\vec v\right)\otimes(\vec u-\vec v)\right\}+\vec F,
\end{multline}
and by inserting
\er{vhfffngghkjgghggtghjgfhjoyuiyuyhiyyukukyihyuSYSPNNWhgjgghyuyy8yuyughghhghghhjjhhjhjkhjCCmmGGKKJJKKjkhjhyyyuu}
into
\er{hvkgkjgkjbjkjjkgjglhhkhjyuyghjhhjhjfghfdhgdhdfdhzzzyyyjffjjkkhhkgjgkjhfhjhgfffjgjhgjffgjggjgjggjhgkkkggjgjgmomnnnNWKK}
and using
\er{MaxVacFull1ninshtrgravortghhghgjkgghklhjgkghghjjkjhjkkggjkhjkhjjhhfhjhklkhkhjjklzzzyyyhjggjhgghhjhNWNWBWHWPPN}
we deduce the following conservation of the angular momentum:
\begin{multline}\label{hvkgkjgkjbjkjjkgjglhhkhjyuyghjhhjhjfghfdhgdhdfdhzzzyyyjffjjkkhhkgjgkjhfhjhgfffjgjhgjffgjggjgjggjhgkkkggjgjgmomnnnNWKKhgjggh}
\frac{\partial}{\partial t}\left\{\vec x\times\left(\mu\vec u+Q\vec
v+\frac{1}{4\pi c}\,\vec D\times \vec B\right)\right\}+\frac{1}{4\pi
G}div_{\vec x}\left\{\left(\vec x\times\nabla_{\vec
x}\Phi\right)\otimes \nabla_{\vec x}\Phi\right\}+\frac{1}{8\pi
G}curl_{\vec x}\left\{\left|\nabla_{\vec x}\Phi\right|^2\vec
x\right\}\\+div_{\vec x}\left\{\mu\left(\vec x\times\vec
u\right)\otimes\vec u+Q\left(\vec x\times\vec v\right)\otimes\vec
v+\left(\vec x \times\vec v\right)\otimes\left(\frac{1}{4\pi c}\vec
D\times \vec B\right)+\left(\vec x\times\left(\frac{1}{4\pi c}\vec
D\times \vec
B\right)\right)\otimes \vec v\right\}=\\
\frac{\partial}{\partial t}\left\{\vec x\times\left(\mu(\vec u-\vec
v)+\frac{1}{4\pi c}\,\vec D\times \vec B\right)\right\}+div_\vec
x\left\{\mu\left(\vec x\times(\vec u-\vec v)\right)\otimes(\vec
u-\vec v)\right\}\\+div_{\vec x}\left\{\left(\vec x \times\vec
v\right)\otimes\left(\mu(\vec u-\vec v)+\frac{1}{4\pi c}\vec D\times
\vec B\right)+\left(\vec x\times\left(\mu(\vec u-\vec
v)+\frac{1}{4\pi c}\vec D\times \vec B\right)\right)\otimes \vec
v\right\}\\+\frac{\partial}{\partial t}\left(\left(\mu+Q\right)\vec
x\times\vec v\right)+div_{\vec x}\left\{\left(\mu+Q\right)\left(\vec
x\times\vec v\right)\otimes\vec v\right\}\\+\frac{1}{4\pi
G}div_{\vec x}\left\{\left(\vec x\times\nabla_{\vec
x}\Phi\right)\otimes \nabla_{\vec x}\Phi\right\}+\frac{1}{8\pi
G}curl_{\vec
x}\left\{\left|\nabla_{\vec x}\Phi\right|^2\vec x\right\}\\
=\frac{1}{4\pi}div_\vec x\left\{(\vec x\times \vec D)\otimes \vec
D+(\vec x\times\vec B)\otimes \vec B\right\}+curl_{\vec
x}\left\{\frac{1}{2}\left(|\vec D|^2+|\vec B|^2\right)\vec
x\right\}+\vec x\times\vec F.
%=\\ \frac{1}{c}\left\{d_\vec x(D\times B)\right\}^T\cdot v+div_{\vec x}\left\{D\otimes D+B\otimes
%B-\frac{1}{2}\left(|D|^2+|B|^2+\frac{2}{c}v\cdot(D\times
%B)\right)I\right\}-4\pi\rho E-\frac{4\pi}{c}\vec j\times B.
\end{multline}
Finally, by
\er{MaxVacFull1ninshtrgravortghhghgjkgghklhjgkghghjjkjhjkkggjkhjkhjjhhfhjhklkhkhjjklzzzyyyNWNWBWHWPPNKKKK},
\er{MaxVacFull1ninshtrgravortghhghgjkgghklhjgkghghjjkjhjkkggjkhjkhjjhhfhjhklkhkhjjklzzzyyyhjggjhgghhjhNWNWBWHWPPN}
and
\er{MaxVacFull1ninshtrgravortghhghgjkgghklhjgkghghjjkjhjkkggjkhjkhjjhhfhjhklkhkhjjklzzzyyyhjggjhgghhjhNWNWNWBWHWPPN}
we have
\begin{multline}\label{vhfffngghkjgghggtghjgfhjoyuiyuyhiyyukukyihyuSYSPNNWhgjgghyuyy8yuyughghhyttyytgghhghghgghjhkjhCCmmGGKKuiiuuihjgghghhhjhjhjhj}
\frac{\partial\vec v}{\partial t}\cdot\left( \mu
 \left(\vec
u-\vec v\right) +\frac{1}{4\pi c}\,\vec D\times \vec
B\right)=-\nabla_{\vec x}\left(\Phi+\frac{1}{2}\left|\vec
v\right|^2\right)\cdot \left( \mu
 \left(\vec
u-\vec v\right) +\frac{1}{4\pi c}\,\vec D\times \vec B\right)=\\
\left(\Phi+\frac{1}{2}\left|\vec v\right|^2\right) \left(div_{\vec
x}\left\{ \mu
 \left(\vec
u-\vec v\right) +\frac{1}{4\pi c}\,\vec D\times \vec
B\right\}\right)-div_{\vec x}\left\{\left(\Phi+\frac{1}{2}\left|\vec
v\right|^2\right)\left( \mu
 \left(\vec
u-\vec v\right) +\frac{1}{4\pi c}\,\vec D\times \vec
B\right)\right\}=\\
-\left(\Phi+\frac{1}{2}\left|\vec v\right|^2\right)
\left(\frac{\partial}{\partial t}\left(\mu+Q\right)+div_{\vec
x}\left\{\left(\mu+Q\right)\vec v\right\}\right)-div_{\vec
x}\left\{\left(\Phi+\frac{1}{2}\left|\vec v\right|^2\right)\left(
\mu
 \left(\vec
u-\vec v\right) +\frac{1}{4\pi c}\,\vec D\times \vec
B\right)\right\}\\= -\left(\Phi+\frac{1}{2}\left|\vec
v\right|^2\right) \left(\frac{\partial}{\partial
t}\left(\mu+Q\right)\right)+\left(\mu+Q\right)\nabla_{\vec
x}\left(\Phi+\frac{1}{2}\left|\vec v\right|^2\right)\cdot\vec v-
div_{\vec x}\left\{\left(\Phi+\frac{1}{2}\left|\vec
v\right|^2\right)\left(\mu+Q\right)\vec v\right\}\\-div_{\vec
x}\left\{\left(\Phi+\frac{1}{2}\left|\vec v\right|^2\right)\left(
\mu
 \left(\vec
u-\vec v\right) +\frac{1}{4\pi c}\,\vec D\times \vec
B\right)\right\}=-\frac{1}{4\pi G}\Phi
\left(\frac{\partial}{\partial
t}\left(\Delta_{\vec x}\Phi\right)\right)\\
-\left(\frac{1}{2}\left|\vec v\right|^2\right)
\left(\frac{\partial}{\partial
t}\left(\mu+Q\right)\right)-\left(\mu+Q\right)\frac{\partial\vec
v}{\partial t}\cdot\vec v- div_{\vec
x}\left\{\left(\Phi+\frac{1}{2}\left|\vec
v\right|^2\right)\left(\mu+Q\right)\vec v\right\}\\-div_{\vec
x}\left\{\left(\Phi+\frac{1}{2}\left|\vec v\right|^2\right)\left(
\mu
 \left(\vec
u-\vec v\right) +\frac{1}{4\pi c}\,\vec D\times \vec
B\right)\right\}=\frac{1}{8\pi G} \frac{\partial}{\partial
t}\left(\left|\nabla_{\vec x}\Phi\right|^2\right) -
\frac{\partial}{\partial
t}\left(\frac{1}{2}\left(\mu+Q\right)\left|\vec
v\right|^2\right)\\-\frac{1}{4\pi G}div_{\vec x}\left\{\Phi
\frac{\partial}{\partial t}\left(\nabla_{\vec
x}\Phi\right)\right\}-div_{\vec
x}\left\{\left(\Phi+\frac{1}{2}\left|\vec v\right|^2\right)\left(
\mu \vec u+Q\vec v+\frac{1}{4\pi c}\,\vec D\times \vec
B\right)\right\}
\end{multline}
Then by inserting
\er{vhfffngghkjgghggtghjgfhjoyuiyuyhiyyukukyihyuSYSPNNWhgjgghyuyy8yuyughghhyttyytgghhghghgghjhkjhCCmmGGKKuiiuuihjgghghhhjhjhjhj}
into
\er{vhfffngghkjgghggtghjgfhjoyuiyuyhiyyukukyihyuSYSPNNWhgjgghyuyy8yuyughghhyttyytgghhghghgghjhkjhCCmmGGKKuiiuui}
we deuce:
\begin{multline}\label{vhfffngghkjgghggtghjgfhjoyuiyuyhiyyukukyihyuSYSPNNWhgjgghyuyy8yuyughghhyttyytgghhghghgghjhkjhCCmmGGKKuiiuuihjgghgh}
\frac{\partial}{\partial t}\left\{ \frac{\mu}{2}\left|\vec u-\vec
v\right|^2+\frac{|\vec D|^2+|\vec B|^2}{8\pi}+\vec v\cdot\left( \mu
 \left(\vec
u-\vec v\right) +\frac{1}{4\pi c}\,\vec D\times \vec
B\right)\right\}\\+div_{\vec x}\left\{\left(\frac{\mu}{2}\left|\vec
u-\vec v\right|^2+\left(\frac{|\vec D|^2+|\vec
B|^2}{8\pi}\right)\right)\vec v\right\}+div_{\vec
x}\left\{\frac{\mu}{2}\left|\vec u-\vec v\right|^2(\vec u-\vec
v)+\frac{c}{4\pi}\vec D\times \vec B\right\}\\= \frac{1}{8\pi G}
\frac{\partial}{\partial t}\left(\left|\nabla_{\vec
x}\Phi\right|^2\right) - \frac{\partial}{\partial
t}\left(\frac{1}{2}\left(\mu+Q\right)\left|\vec
v\right|^2\right)\\-\frac{1}{4\pi G}div_{\vec x}\left\{\Phi
\frac{\partial}{\partial t}\left(\nabla_{\vec
x}\Phi\right)\right\}-div_{\vec
x}\left\{\left(\Phi+\frac{1}{2}\left|\vec v\right|^2\right)\left(
\mu \vec u+Q\vec v+\frac{1}{4\pi c}\,\vec D\times \vec
B\right)\right\}\\- div_{\vec x}\left\{\left(\left(\mu\left(\vec
u-\vec v\right)+\frac{1}{4\pi c}\vec D\times \vec B\right)\cdot\vec
v\right)\vec v\right\}\\-div_{\vec x}\left\{\mu \left(\left(\vec
u-\vec v\right)\cdot\vec v\right) \left(\vec u-\vec
v\right)\right\}+\frac{1}{4\pi}div_{\vec x}\left\{\left(\vec
D\otimes \vec D+\vec B\otimes \vec B-\frac{1}{2}\left(|\vec
D|^2+|\vec B|^2\right)I\right)\cdot\vec v
%-4\pi\mu\left(1-\frac{1}{c^2}\left|\vec u-\vec v\right|^2\right)^{-\frac{1}{2}}\left(\vec u-\vec v\right)\otimes \left(\vec u-\vec v\right)
\right\}
%-div_\vec x\left\{ \frac{c}{4\pi}\vec D\times \vec B\right\}
+\vec u\cdot\vec F.
\end{multline}
Then, using \er{apfrm1} and the last two equalities in
\er{MaxVacFull1bjkgjhjhgjaaahkjhhjzzzyyykkknnnNWBWHWPPN},  we
rewrite
\er{vhfffngghkjgghggtghjgfhjoyuiyuyhiyyukukyihyuSYSPNNWhgjgghyuyy8yuyughghhyttyytgghhghghgghjhkjhCCmmGGKKuiiuuihjgghgh}
in the form of the following conservation of the energy:
\begin{multline}\label{vhfffngghkjgghggtghjgfhjoyuiyuyhiyyukukyihyuSYSPNNWhgjgghyuyy8yuyughghhyttyytgghhghghgghjhkjhCCmmGGKKuiiuuihjgghghiuiuiui}
\frac{\partial}{\partial t}\left\{ \frac{\mu}{2}\left|\vec
u\right|^2+\frac{Q}{2}\left|\vec v\right|^2+\frac{\vec D\cdot\vec
E+\vec B\cdot\vec H}{8\pi}-\frac{1}{8\pi G} \left|\nabla_{\vec
x}\Phi\right|^2\right\}\\+div_{\vec x}\left\{\frac{\mu}{2}\left|\vec
u\right|^2\vec u+\frac{Q}{2}\left|\vec v\right|^2\vec
v+\left(\frac{\vec D\cdot\vec E+\vec B\cdot\vec H}{8\pi}\right)\vec
v+\frac{1}{8\pi c}\left|\vec v\right|^2\left(\vec D\times \vec
B\right)\right\}+div_{\vec x}\left\{\frac{c}{4\pi}\vec D\times \vec
B\right\}\\=-\frac{1}{4\pi G}div_{\vec x}\left\{\Phi
\frac{\partial}{\partial t}\left(\nabla_{\vec
x}\Phi\right)\right\}-div_{\vec x}\left\{\Phi\left( \mu \vec u+Q\vec
v+\frac{1}{4\pi c}\,\vec D\times \vec
B\right)\right\}\\+\frac{1}{4\pi}div_{\vec x}\left\{\left(\vec
D\otimes \vec D+\vec B\otimes \vec B-\frac{1}{2}\left(|\vec
D|^2+|\vec B|^2\right)I\right)\cdot\vec v
%-4\pi\mu\left(1-\frac{1}{c^2}\left|\vec u-\vec v\right|^2\right)^{-\frac{1}{2}}\left(\vec u-\vec v\right)\otimes \left(\vec u-\vec v\right)
\right\}
%-div_\vec x\left\{ \frac{c}{4\pi}\vec D\times \vec B\right\}
+\vec u\cdot\vec F.
\end{multline}
%
%
%
%
%
%
\begin{comment}
\begin{equation}
\label{MaxVacFull1ninshtrgravortghhghgjkgghklhjgkghghjjkjhjkkggjkhjkhjjhhfhjhklkhkhjjklzzzyyyhjggjhgghhjhNWNWBWHWPPN}
\begin{cases}
curl_{\vec x}\vec v= 0,\\
\frac{\partial\vec v}{\partial t}+d_\vec x\vec v\cdot\vec v=
-\nabla_{\vec x}\Phi,
%\frac{d\vec u}{dt}=-curl_{\vec x}\vec v\times(\vec u-\vec v)+\partial_t\vec v+(\nabla_{\vec x}\vec v)\cdot\vec v+\nabla_{\vec x}\psi_0+\frac{1}{m}\vec F.
\end{cases}
\end{equation}
where $\Phi$ is the scalar gravitational potential: a proper scalar
field which satisfies in every coordinate system:
\begin{equation}
\label{MaxVacFull1ninshtrgravortghhghgjkgghklhjgkghghjjkjhjkkggjkhjkhjjhhfhjhklkhkhjjklzzzyyyhjggjhgghhjhNWNWNWBWHWPPN}
\Delta_{\vec x}\Phi=4\pi G(\mu+Q).
\end{equation}
\begin{equation}
\label{MaxVacFull1ninshtrgravortghhghgjkgghklhjgkghghjjkjhjkkggjkhjkhjjhhfhjhklkhkhjjklzzzyyyNWNWBWHWPPN}
\frac{\partial Q}{\partial t}+div_{\vec x}\left\{Q\vec v\right\}=-
div_\vec x\left\{\frac{1}{4\pi c}\vec D\times \vec B\right\}
%\frac{d\vec u}{dt}=-curl_{\vec x}\vec v\times(\vec u-\vec v)+\partial_t\vec v+(\nabla_{\vec x}\vec v)\cdot\vec v+\nabla_{\vec x}\psi_0+\frac{1}{m}\vec F.
\end{equation}
in the general case. Then, instead of \er{gghjgghfghd}, for the
general case of gravitational-electromagnetic fields we consider the
following relation between the gravitational and inertial mass
densities
\begin{equation}\label{gghjgghfghdkjgjj}
M=\mu+Q.
\end{equation}
\begin{equation}\label{MaxVacFull1ninshtrgravortghhghgjkgghklhjgkghghjjkjhjkkggjkhjkhjjhhfhjhkhjjkhjgzzzyyykkknnnNWBWHWPPN}
\frac{\partial\mu}{\partial t}+div_{\vec x}\left(\mu\vec u\right)=0.
\end{equation}



%
Then by
\er{MaxVacFull1ninshtrgravortghhghgjkgghklhjgkghghjjkjhjkkggjkhjkhjjhhfhjhklkhkhjjklzzzyyyhjggjhgghhjhNWNWBWHWPPN}
and
\er{MaxVacFull1ninshtrgravortghhghgjkgghklhjgkghghjjkjhjkkggjkhjkhjjhhfhjhkhjjkhjgzzzyyykkknnnNWBWHWPPN}
we can rewrite
\er{MaxVacFull1ninshtrgravortghhghgjkgghklhjgkghghjjkjhjkkggjkhjkhjjhhfhjhkjkhbbgjhzzzyyykkknnnNWBWHWPPN}
in the system $(*)$ as
\begin{equation}\label{MaxVacFull1ninshtrgravortghhghgjkgghklhjgkghghjjkjhjkkggjkhjkhjjhhfhjhkjkhbbgjhzzzyyykkknnnNWBWHWHJKJJGJGHJJGGPPN}
\mu\bigg(\frac{\partial\vec u}{\partial t}+d_{\vec x}\vec u\cdot\vec
u\bigg)=\frac{\partial(\mu\vec u)}{\partial t}+div_{\vec
x}\left\{\mu\vec u\otimes\vec u\right\} =-\mu\nabla_{\vec
x}\Phi+\rho\vec E+\frac{1}{c}\vec j\times\vec B+\vec F.
%\frac{d\vec u}{dt}=-curl_{\vec x}\vec v\times(\vec u-\vec v)+\partial_t\vec v+(\nabla_{\vec x}\vec v)\cdot\vec v+\nabla_{\vec x}\psi_0+\frac{1}{m}\vec F.
\end{equation}
Moreover, by
\er{MaxVacFull1ninshtrgravortghhghgjkgghklhjgkghghjjkjhjkkggjkhjkhjjhhfhjhkjkhbbgjhzzzyyykkknnnNWBWHWPPN}
and
\er{MaxVacFull1ninshtrgravortghhghgjkgghklhjgkghghjjkjhjkkggjkhjkhjjhhfhjhkhjjkhjgzzzyyykkknnnNWBWHWPPN},
using
\er{MaxVacFull1ninshtrgravortghhghgjkgghklhjgkghghjjkjhjkkggjkhjkhjjhhfhjhklkhkhjjklzzzyyyhjggjhgghhjhNWNWBWHWPPN}
we have
\begin{multline}\label{MaxVacFull1ninshtrgravortghhghgjkgghklhjgkghghjjkjhjkkggjkhjkhjjhhfhjhkjkhbbgjhhkjhhklhzzzyyykkknnnNWBWHWPPN}
\frac{\partial}{\partial t}\left(\frac{\mu|\vec
u|^2}{2}\right)+div_{\vec x}\left\{\left(\frac{\mu|\vec
u|^2}{2}\right)\vec u\right\}=\mu\vec u\cdot\left(\partial_{t}\vec
v+d_{\vec x}\vec v\cdot\vec v\right)+\vec u\cdot\left(\rho\vec
E+\frac{1}{c}\vec j\times\vec B\right)+\vec F\cdot\vec u= -\mu\vec
u\cdot\nabla_{\vec x}\Phi\\+\vec j\cdot\vec E+\vec F\cdot\vec
u=-div_{\vec x}\left(\Phi\mu\vec u\right)+\Phi\,div_{\vec
x}\left(\mu\vec u\right)+\vec j\cdot\vec E+\vec F\cdot\vec
u=-div_{\vec x}\left(\Phi\mu\vec u\right)-\Phi\,\frac{\partial\vec
\mu}{\partial t}+\vec j\cdot\vec E+\vec F\cdot\vec u.
%\frac{d\vec u}{dt}=-curl_{\vec x}\vec v\times(\vec u-\vec v)+\partial_t\vec v+(\nabla_{\vec x}\vec v)\cdot\vec v+\nabla_{\vec x}\psi_0+\frac{1}{m}\vec F.
\end{multline}
On the other hand, in the Appendix we proved:
\begin{multline}\label{hvkgkjgkjbjkjjkgjglhhkhjyuyghjhhjhjzzzyyykkknnnNWBWHWPPN}
\frac{\partial}{\partial t}\left(\frac{1}{4\pi c}\,\vec D\times \vec
B\right)+div_\vec x\left\{\left(\frac{1}{4\pi c}\vec D\times \vec
B\right)\otimes \vec v\right\}=-(d_\vec x \vec
v)^T\cdot\left(\frac{1}{4\pi c}\vec D\times \vec
B\right)\\+\frac{1}{4\pi}div_\vec x\left\{\vec D\otimes \vec D+\vec
B\otimes \vec B-\frac{1}{2}\left(|\vec D|^2+|\vec
B|^2\right)I\right\}-\left(\rho \vec E+\frac{1}{c}\vec j\times \vec
B\right),
%=\\ \frac{1}{c}\left\{d_\vec x(D\times B)\right\}^T\cdot v+div_{\vec x}\left\{D\otimes D+B\otimes
%B-\frac{1}{2}\left(|D|^2+|B|^2+\frac{2}{c}v\cdot(D\times
%B)\right)I\right\}-4\pi\rho E-\frac{4\pi}{c}\vec j\times B.
\end{multline}
and
\begin{multline}\label{hvkgkjgkjbjbbklnknhihiokhhfjffghvjmbjhjkhlkzzzyyykkknnnNWBWHWPPN}
\frac{\partial}{\partial t}\left(\frac{|\vec D|^2+|\vec
B|^2}{8\pi}\right)+div_\vec x\left\{\left(\frac{|\vec D|^2+|\vec
B|^2}{8\pi}\right)\vec v\right\}=\\
\frac{1}{4\pi}div_\vec x\left\{(\vec D\otimes \vec D+ \vec B\otimes
\vec B)\cdot \vec v-\frac{1}{2}\left(|\vec D|^2+|\vec
B|^2\right)\vec v-c \vec D\times \vec B\right\}
\\-\left\{\frac{1}{4\pi}\left(div_\vec x\left\{\vec D\otimes \vec D+\vec B\otimes
\vec B-\frac{1}{2}\left(|\vec D|^2+|\vec
B|^2\right)I\right\}\right)-\left(\rho \vec E+\frac{1}{c}\,\vec
j\times \vec B\right)\right\}\cdot \vec v-\vec j\cdot \vec E.
\end{multline}
In particular, by
\er{hvkgkjgkjbjkjjkgjglhhkhjyuyghjhhjhjzzzyyykkknnnNWBWHWPPN} we
have
\begin{multline}\label{hvkgkjgkjbjkjjkgjglhhkhjyuyghjhhjhjkhkzzzyyykkknnnNWBWHWPPN}
\vec v\cdot\frac{\partial}{\partial t}\left(\frac{1}{4\pi c}\,\vec
D\times \vec B\right)+div_\vec x\left\{\left(\left(\frac{1}{4\pi
c}\vec D\times \vec B\right)\cdot\vec v\right)\vec
v\right\}=\\+\left\{\frac{1}{4\pi}div_\vec x\left\{\vec D\otimes
\vec D+\vec B\otimes \vec B-\frac{1}{2}\left(|\vec D|^2+|\vec
B|^2\right)I\right\}-\left(\rho \vec E+\frac{1}{c}\vec j\times \vec
B\right)\right\}\cdot\vec v.
%=\\ \frac{1}{c}\left\{d_\vec x(D\times B)\right\}^T\cdot v+div_{\vec x}\left\{D\otimes D+B\otimes
%B-\frac{1}{2}\left(|D|^2+|B|^2+\frac{2}{c}v\cdot(D\times
%B)\right)I\right\}-4\pi\rho E-\frac{4\pi}{c}\vec j\times B.
\end{multline}
Inserting
\er{hvkgkjgkjbjkjjkgjglhhkhjyuyghjhhjhjkhkzzzyyykkknnnNWBWHWPPN}
into
\er{hvkgkjgkjbjbbklnknhihiokhhfjffghvjmbjhjkhlkzzzyyykkknnnNWBWHWPPN}
gives:
\begin{multline}\label{hvkgkjgkjbjbbklnknhihiokhhfjffghvjmbjhjkhlkzzzyyykkkfjfffffhfnnnNWBWHWPPN}
\frac{\partial}{\partial t}\left(\frac{|\vec D|^2+|\vec
B|^2}{8\pi}+\frac{1}{4\pi c}\left(\vec D\times \vec
B\right)\cdot\vec v\right)+div_\vec x\left\{\left(\frac{|\vec
D|^2+|\vec B|^2}{8\pi}\right)\vec v+\left(\left(\frac{1}{4\pi c}\vec
D\times \vec B\right)\cdot\vec v\right)\vec v\right\}=\\
\left(\frac{1}{4\pi c}\,\vec D\times \vec
B\right)\cdot\frac{\partial\vec v}{\partial t}+
\frac{1}{4\pi}div_\vec x\left\{(\vec D\otimes \vec D+ \vec B\otimes
\vec B)\cdot \vec v-\frac{1}{2}\left(|\vec D|^2+|\vec
B|^2\right)\vec v-c \vec D\times \vec B\right\}-\vec j\cdot \vec E.
\end{multline}
Then inserting
\er{hvkgkjgkjbjkjjkgjglhhkhjyuyghjhhjhjzzzyyykkknnnNWBWHWPPN} into
\er{MaxVacFull1ninshtrgravortghhghgjkgghklhjgkghghjjkjhjkkggjkhjkhjjhhfhjhkjkhbbgjhzzzyyykkknnnNWBWHWHJKJJGJGHJJGGPPN}
we obtain
\begin{multline}\label{hvkgkjgkjbjkjjkgjglhhkhjyuyghjhhjhjzzzyyykkknnnNWBWHWhjklhkjPPN}
\frac{\partial}{\partial t}\left(\mu\vec u+\frac{1}{4\pi c}\,\vec
D\times \vec B\right)+div_\vec x\left\{\mu\vec u\otimes\vec
u+\left(\frac{1}{4\pi c}\vec D\times \vec B\right)\otimes \vec
v\right\}=\\\frac{1}{4\pi}div_\vec x\left\{\vec D\otimes \vec D+\vec
B\otimes \vec B-\frac{1}{2}\left(|\vec D|^2+|\vec
B|^2\right)I\right\}-(d_\vec x \vec v)^T\cdot\left(\frac{1}{4\pi
c}\vec D\times \vec B\right)-\mu\nabla_{\vec x}\Phi+\vec F,
%=\\ \frac{1}{c}\left\{d_\vec x(D\times B)\right\}^T\cdot v+div_{\vec x}\left\{D\otimes D+B\otimes
%B-\frac{1}{2}\left(|D|^2+|B|^2+\frac{2}{c}v\cdot(D\times
%B)\right)I\right\}-4\pi\rho E-\frac{4\pi}{c}\vec j\times B.
\end{multline}
and inserting
\er{hvkgkjgkjbjbbklnknhihiokhhfjffghvjmbjhjkhlkzzzyyykkkfjfffffhfnnnNWBWHWPPN}
into
\er{MaxVacFull1ninshtrgravortghhghgjkgghklhjgkghghjjkjhjkkggjkhjkhjjhhfhjhkjkhbbgjhhkjhhklhzzzyyykkknnnNWBWHWPPN}
we obtain
\begin{multline}\label{hvkgkjgkjbjbbklnknhihiokhhfjffghvjmbjhjkhlkzzzyyykkkfjfffffhfnnnNWBWHWghjgjhPPN}
\frac{\partial}{\partial t}\left(\frac{\mu|\vec u|^2}{2}+\frac{|\vec
D|^2+|\vec B|^2}{8\pi}+\frac{1}{4\pi c}\left(\vec D\times \vec
B\right)\cdot\vec v\right)\\+div_\vec x\left\{\left(\frac{\mu|\vec
u|^2}{2}\right)\vec u+\left(\frac{|\vec D|^2+|\vec
B|^2}{8\pi}\right)\vec v+\left(\left(\frac{1}{4\pi c}\vec
D\times \vec B\right)\cdot\vec v\right)\vec v\right\}=\\
\frac{1}{4\pi}div_\vec x\left\{(\vec D\otimes \vec D+ \vec B\otimes
\vec B)\cdot \vec v-\frac{1}{2}\left(|\vec D|^2+|\vec
B|^2\right)\vec v-c \vec D\times \vec B-4\pi\Phi\mu\vec u\right\}\\+
\left(\frac{1}{4\pi c}\,\vec D\times \vec
B\right)\cdot\frac{\partial\vec v}{\partial
t}-\Phi\,\frac{\partial\vec \mu}{\partial t}+\vec F\cdot\vec u.
\end{multline}
Then, by
\er{MaxVacFull1ninshtrgravortghhghgjkgghklhjgkghghjjkjhjkkggjkhjkhjjhhfhjhklkhkhjjklzzzyyyhjggjhgghhjhNWNWBWHWPPN},
\er{MaxVacFull1ninshtrgravortghhghgjkgghklhjgkghghjjkjhjkkggjkhjkhjjhhfhjhklkhkhjjklzzzyyyNWNWBWHWPPN}
and
\er{MaxVacFull1ninshtrgravortghhghgjkgghklhjgkghghjjkjhjkkggjkhjkhjjhhfhjhkhjjkhjgzzzyyykkknnnNWBWHWPPN},
using
\er{hvkgkjgkjbjkjjkgjglhhkhjyuyghjhhjhjzzzyyykkknnnNWBWHWhjklhkjPPN}
 we obtain that in the system $(*)$ we have:
\begin{multline}\label{hvkgkjgkjbjkjjkgjglhhkhjyuyghjhhjhjfghfdhgdhdfdhzzzyyyjffjjkkhhkgjgkjhfhjhgfffjgjhgjffgjggjgjggjhgkkkggjgjgnnnmmmNWNWBWHWPPN}
\frac{\partial}{\partial t}\left(\mu\vec u+\frac{1}{4\pi c}\,\vec
D\times \vec B\right)+div_\vec x\left\{\mu\vec u\otimes\vec
u+\left(\frac{1}{4\pi c}\vec D\times \vec B\right)\otimes \vec
v\right\}=\\
\frac{1}{4\pi}div_\vec x\left\{\vec D\otimes \vec D+\vec B\otimes
\vec B-\frac{1}{2}\left(|\vec D|^2+|\vec B|^2\right)I\right\}-d_\vec
x \vec v\cdot\left(\frac{1}{4\pi c}\vec D\times \vec
B\right)-\mu\nabla_{\vec x}\Phi+\vec F
=-\mu\nabla_{\vec x}\Phi+\\
div_\vec x\left\{\frac{1}{4\pi}\left(\vec D\otimes \vec D+\vec
B\otimes \vec B-\frac{1}{2}\left(|\vec D|^2+|\vec
B|^2\right)I\right)-\vec v\otimes\left(\frac{1}{4\pi c}\vec D\times
\vec B\right)\right\}+\left(div_{\vec x}\left\{\frac{1}{4\pi c}\vec
D\times \vec B\right\}\right)\vec v\\+\vec F= -\mu\nabla_{\vec
x}\Phi+\vec F + div_\vec x\left\{\frac{1}{4\pi}\left(\vec D\otimes
\vec D+\vec B\otimes \vec B-\frac{1}{2}\left(|\vec D|^2+|\vec
B|^2\right)I\right)-\vec v\otimes\left(\frac{1}{4\pi c}\vec D\times
\vec B\right)\right\}\\-\left(\frac{\partial Q}{\partial
t}+div_{\vec x}\left\{Q\vec v\right\}\right)\vec v= div_\vec
x\left\{\frac{1}{4\pi}\left(\vec D\otimes \vec D+\vec B\otimes \vec
B-\frac{1}{2}\left(|\vec D|^2+|\vec B|^2\right)I\right)-\vec
v\otimes\left(\frac{1}{4\pi c}\vec D\times \vec
B\right)\right\}\\+\vec F-\mu\nabla_{\vec x}\Phi
-\frac{\partial}{\partial t}\left(Q\vec v\right)-div_{\vec
x}\left\{Q\vec v\otimes\vec v\right\}+Q\left(\frac{\partial\vec
v}{\partial t}+d_{\vec x}\vec v\cdot\vec v\right)=\vec
F-(\mu+Q)\nabla_{\vec x}\Phi
-\frac{\partial}{\partial t}\left(Q\vec v\right)\\
+div_\vec x\left\{\frac{1}{4\pi}\left(\vec D\otimes \vec D+\vec
B\otimes \vec B-\frac{1}{2}\left(|\vec D|^2+|\vec
B|^2\right)I\right)-\vec v\otimes\left(\frac{1}{4\pi c}\vec D\times
\vec B\right)-Q\vec v\otimes\vec v\right\}.
\end{multline}
%
%
Thus by
\er{hvkgkjgkjbjkjjkgjglhhkhjyuyghjhhjhjfghfdhgdhdfdhzzzyyyjffjjkkhhkgjgkjhfhjhgfffjgjhgjffgjggjgjggjhgkkkggjgjgnnnmmmNWNWBWHWPPN}
and
\er{MaxVacFull1ninshtrgravortghhghgjkgghklhjgkghghjjkjhjkkggjkhjkhjjhhfhjhklkhkhjjklzzzyyyhjggjhgghhjhNWNWNWBWHWPPN}
we obtain
\begin{multline}\label{hvkgkjgkjbjkjjkgjglhhkhjyuyghjhhjhjfghfdhgdhdfdhzzzyyyjffjjkkhhkgjgkjhfhjhgfffjgjhgjffgjggjgjggjhgkkkggjgjgnnnmmmNWNWNWBWHWPPN}
\frac{\partial}{\partial t}\left(\mu\vec u+Q\vec v+\frac{1}{4\pi
c}\,\vec D\times \vec B\right)+div_\vec x\left\{\mu\vec u\otimes\vec
u+\left(\frac{1}{4\pi c}\vec D\times \vec B\right)\otimes \vec
v+\vec v\otimes\left(\frac{1}{4\pi c}\vec D\times \vec
B\right)+Q\vec v\otimes\vec v\right\}\\=\vec F-\frac{1}{4\pi
G}\left(\Delta_{\vec x}\Phi\right)\nabla_{\vec x}\Phi
+\frac{1}{4\pi}div_\vec x\left\{\vec D\otimes \vec D+\vec B\otimes
\vec B-\frac{1}{2}\left(|\vec D|^2+|\vec B|^2\right)I\right\}\\
=\frac{1}{4\pi}div_\vec x\left\{\vec D\otimes \vec D+\vec B\otimes
\vec B-\frac{1}{2}\left(|\vec D|^2+|\vec
B|^2\right)I-\frac{1}{G}\nabla_{\vec x}\Phi\otimes\nabla_{\vec
x}\Phi+\frac{1}{2G}\left|\nabla_{\vec x}\Phi\right|^2 I\right\}+\vec
F.
\end{multline}
%
%
%
%
%
So, in the system $(*)$ we have:
\begin{multline}\label{hvkgkjgkjbjkjjkgjglhhkhjyuyghjhhjhjfghfdhgdhdfdhzzzyyyjffjjkkhhkgjgkjhfhjhgfffjgjhgjffgjggjgjggjhgkkkggjgjgnnnmmmNWNWNWNWBWHWPPN}
\frac{\partial}{\partial t}\left(\mu\vec u+Q\vec v+\frac{1}{4\pi
c}\,\vec D\times \vec B\right)+div_\vec x\left\{\mu\vec u\otimes\vec
u+Q\vec v\otimes\vec v+\left(\frac{1}{4\pi c}\vec D\times \vec
B\right)\otimes \vec v+\vec v\otimes\left(\frac{1}{4\pi c}\vec
D\times \vec
B\right)\right\}\\
=\frac{1}{4\pi}div_\vec x\left\{\vec D\otimes \vec D+\vec B\otimes
\vec B-\frac{1}{2}\left(|\vec D|^2+|\vec
B|^2\right)I-\frac{1}{G}\nabla_{\vec x}\Phi\otimes\nabla_{\vec
x}\Phi+\frac{1}{2G}\left|\nabla_{\vec x}\Phi\right|^2 I\right\}+\vec
F.
\end{multline}
On the other hand for every vector fields $\Gamma:\R^3\to\R^3$ and
$\Lambda:\R^3\to\R^3$ and every scalar field $P:\R^3\to\R$ we have:
\begin{multline}\label{yguiyiu7yytyutkhjzzzyyykkknnnNWBWHWPPN}
\vec x\times div_{\vec
x}\{\Gamma\otimes\Lambda+\Lambda\otimes\Gamma\}=div_{\vec
x}\left\{(\vec x\times\Gamma)\otimes\Lambda+(\vec
x\times\Lambda)\otimes\Gamma\right\},\\
\vec x\times div_{\vec x}\{P\Gamma\otimes\Gamma\}=div_{\vec
x}\left\{P(\vec
x\times\Gamma)\otimes\Gamma\right\}\quad\text{and}\quad\vec
x\times\nabla_{\vec x}P=-curl_{\vec x}\{P\,\vec x\}.
\end{multline}
Thus inserting \er{yguiyiu7yytyutkhjzzzyyykkknnnNWBWHWPPN} into
\er{hvkgkjgkjbjkjjkgjglhhkhjyuyghjhhjhjfghfdhgdhdfdhzzzyyyjffjjkkhhkgjgkjhfhjhgfffjgjhgjffgjggjgjggjhgkkkggjgjgnnnmmmNWNWNWNWBWHWPPN}
we infer that in the system $(*)$ we have:
\begin{multline}\label{hvkgkjgkjbjkjjkgjglhhkhjyuyghjhhjhjfghfdhgdhdfdhzzzyyyjffjjkkhhkgjgkjhfhjhgfffjgjhgjffgjggjgjggjhgkkkggjgjgmomnnnmmmNWNWBWHWPPN}
\frac{\partial}{\partial t}\left(\vec x\times(\mu\vec u)+\vec
x\times(Q\vec v)+\vec x\times\left(\frac{1}{4\pi c}\,\vec D\times
\vec B\right)\right)\\+div_\vec x\left\{\mu(\vec x\times\vec
u)\otimes\vec u+Q(\vec x\times\vec v)\otimes\vec v+\left(\vec
x\times\left(\frac{1}{4\pi c}\vec D\times \vec
B\right)\right)\otimes \vec v+(\vec x\times\vec
v)\otimes\left(\frac{1}{4\pi c}\vec D\times \vec
B\right)\right\}\\
=\frac{1}{4\pi}div_\vec x\left\{(\vec x\times\vec D)\otimes \vec
D+(\vec x\times\vec B)\otimes \vec B-\frac{1}{G}(\vec
x\times\nabla_{\vec x}\Phi)\otimes\nabla_{\vec
x}\Phi\right\}\\+\frac{1}{8\pi}curl_{\vec x}\left\{\left(|\vec
D|^2+|\vec B|^2-\frac{1}{G}\left|\nabla_{\vec
x}\Phi\right|^2\right)\vec x\right\}+ \vec x\times \vec F.
\end{multline}
%
%
%
%
%
Furthermore, by
\er{MaxVacFull1ninshtrgravortghhghgjkgghklhjgkghghjjkjhjkkggjkhjkhjjhhfhjhklkhkhjjklzzzyyyhjggjhgghhjhNWNWBWHWPPN},
\er{MaxVacFull1ninshtrgravortghhghgjkgghklhjgkghghjjkjhjkkggjkhjkhjjhhfhjhklkhkhjjklzzzyyyNWNWBWHWPPN},
\er{MaxVacFull1ninshtrgravortghhghgjkgghklhjgkghghjjkjhjkkggjkhjkhjjhhfhjhkhjjkhjgzzzyyykkknnnNWBWHWPPN}
and
\er{MaxVacFull1ninshtrgravortghhghgjkgghklhjgkghghjjkjhjkkggjkhjkhjjhhfhjhklkhkhjjklzzzyyyhjggjhgghhjhNWNWNWBWHWPPN}
%\er{MaxVacFull1ninshtrgravortghhghgjkgghklhjgkghghjjkjhjkkggjkhjkhjjhhfhjhklkhkhjjklzzzyyyhjggjhgghhjhNWNWNWNWBWZZZ}
using
\er{hvkgkjgkjbjbbklnknhihiokhhfjffghvjmbjhjkhlkzzzyyykkkfjfffffhfnnnNWBWHWghjgjhPPN}
we deduce that in the system $(*)$ we have:
\begin{multline}\label{zzzzzzzzzl}
\frac{\partial}{\partial t}\left(\frac{\mu|\vec u|^2}{2}+\frac{|\vec
D|^2+|\vec B|^2}{8\pi}+\frac{1}{4\pi c}\left(\vec D\times \vec
B\right)\cdot\vec v\right)\\+div_\vec x\left\{\left(\frac{\mu|\vec
u|^2}{2}\right)\vec u+\left(\frac{|\vec D|^2+|\vec
B|^2}{8\pi}\right)\vec v+\left(\left(\frac{1}{4\pi c}\vec
D\times \vec B\right)\cdot\vec v\right)\vec v\right\}=\\
\frac{1}{4\pi}div_\vec x\left\{(\vec D\otimes \vec D+ \vec B\otimes
\vec B)\cdot \vec v-\frac{1}{2}\left(|\vec D|^2+|\vec
B|^2\right)\vec v-c \vec D\times \vec B-4\pi\Phi\mu\vec u\right\}\\-
\left(\frac{1}{4\pi c}\,\vec D\times \vec B\right)\cdot\nabla_{\vec
x}\left(\Phi+\frac{1}{2}\left|\vec
v\right|^2\right)-\Phi\,\frac{\partial\vec \mu}{\partial t}+\vec
F\cdot\vec u=\\
\frac{1}{4\pi}div_\vec x\left\{(\vec D\otimes \vec D+ \vec B\otimes
\vec B)\cdot \vec v-\frac{1}{2}\left(|\vec D|^2+|\vec
B|^2\right)\vec v-c \vec D\times \vec B-4\pi\Phi\mu\vec
u\right\}\\-div_{\vec x}\left\{\left(\Phi+\frac{1}{2}\left|\vec
v\right|^2\right)\left(\frac{1}{4\pi c}\,\vec D\times \vec
B\right)\right\}+div_{\vec x}\left\{\frac{1}{4\pi c}\,\vec D\times
\vec B\right\}\cdot\left(\Phi+\frac{1}{2}\left|\vec
v\right|^2\right)-\Phi\,\frac{\partial\vec \mu}{\partial t}+\vec
F\cdot\vec u=\\
\frac{1}{4\pi}div_\vec x\left\{(\vec D\otimes \vec D+ \vec B\otimes
\vec B)\cdot \vec v-\frac{1}{2}\left(|\vec D|^2+|\vec
B|^2\right)\vec v-c \vec D\times \vec B-4\pi\Phi\mu\vec
u\right\}\\-div_{\vec x}\left\{\left(\Phi+\frac{1}{2}\left|\vec
v\right|^2\right)\left(\frac{1}{4\pi c}\,\vec D\times \vec
B\right)\right\}-\Phi\,\frac{\partial\vec \mu}{\partial t}+\vec
F\cdot\vec u\\
- \Phi\left(\frac{\partial Q}{\partial t}+div_{\vec x}\left\{Q\vec
v\right\}\right)- \frac{1}{2}\left|\vec
v\right|^2\left(\frac{\partial Q}{\partial t}+div_{\vec
x}\left\{Q\vec v\right\}\right).
\end{multline}
Thus by \er{zzzzzzzzzl} we have
\begin{multline}\label{MaxVacFull1ninshtrgravortghhghgjkgghklhjgkghghjjkjhjkkggjkhjkhjjhhfhjhkjkhbbgjhhkjhhklhzzzyyykkkgkhjjhgfhjjffghuikkgkjghhjkjknnnmmmNWNWBWHWPPN}
\frac{\partial}{\partial t}\left(\frac{\mu|\vec u|^2}{2}+\frac{|\vec
D|^2+|\vec B|^2}{8\pi}+\frac{1}{4\pi c}\left(\vec D\times \vec
B\right)\cdot\vec v\right)\\+div_\vec x\left\{\left(\frac{\mu|\vec
u|^2}{2}\right)\vec u+\left(\frac{|\vec D|^2+|\vec
B|^2}{8\pi}\right)\vec v+\left(\left(\frac{1}{4\pi c}\vec
D\times \vec B\right)\cdot\vec v\right)\vec v\right\}=\\
\frac{1}{4\pi}div_\vec x\left\{(\vec D\otimes \vec D+ \vec B\otimes
\vec B)\cdot \vec v-\frac{1}{2}\left(|\vec D|^2+|\vec
B|^2\right)\vec v-c \vec D\times \vec B-4\pi\Phi\mu\vec
u\right\}\\-div_{\vec x}\left\{\left(\Phi+\frac{1}{2}\left|\vec
v\right|^2\right)\left(\frac{1}{4\pi c}\,\vec D\times \vec
B\right)\right\}+\vec F\cdot\vec u -\frac{\Phi}{4\pi
G}\frac{\partial}{\partial t}(\Delta_{\vec x}\Phi)-div_{\vec
x}\left\{\Phi Q\vec v\right\}+Q\vec v\cdot\nabla_{\vec x}\Phi\\-
\frac{\partial}{\partial t}\left(\frac{Q}{2}\left|\vec
v\right|^2\right)+ Q\vec v\cdot\frac{\partial\vec v}{\partial
t}+Q\vec v\cdot\left(d_{\vec x}\vec v\cdot\vec v\right)- div_{\vec
x}\left\{\frac{Q}{2}\left|\vec v\right|^2\vec v\right\}=
\\
\frac{1}{4\pi}div_\vec x\left\{(\vec D\otimes \vec D+ \vec B\otimes
\vec B)\cdot \vec v-\frac{1}{2}\left(|\vec D|^2+|\vec
B|^2\right)\vec v-c \vec D\times \vec B-4\pi\Phi\mu\vec
u\right\}\\-div_{\vec x}\left\{\left(\Phi+\frac{1}{2}\left|\vec
v\right|^2\right)\left(\frac{1}{4\pi c}\,\vec D\times \vec
B\right)\right\}-div_{\vec x}\left\{\Phi Q\vec
v\right\}-\frac{1}{4\pi G}div_{\vec
x}\left\{\Phi\frac{\partial}{\partial t}(\nabla_{\vec
x}\Phi)\right\}\\- div_{\vec x}\left\{\frac{Q}{2}\left|\vec
v\right|^2\vec v\right\}- \frac{\partial}{\partial
t}\left(\frac{Q}{2}\left|\vec v\right|^2-\frac{1}{8\pi
G}\big|\nabla_{\vec x}\Phi\big|^2\right)+\vec F\cdot\vec u.
\end{multline}
So:
\begin{multline}\label{MaxVacFull1ninshtrgravortghhghgjkgghklhjgkghghjjkjhjkkggjkhjkhjjhhfhjhkjkhbbgjhhkjhhklhzzzyyykkkgkhjjhgfhjjffghuikkgkjghhjkjknnnmmmNWNWBWHWhhjhhPPN}
\frac{\partial}{\partial t}\left(\frac{\mu|\vec
u|^2}{2}+\frac{Q}{2}\left|\vec v\right|^2+\frac{|\vec D|^2+|\vec
B|^2}{8\pi}+\frac{1}{4\pi c}\left(\vec D\times \vec
B\right)\cdot\vec v-\frac{1}{8\pi G}\big|\nabla_{\vec
x}\Phi\big|^2\right)\\+div_\vec x\left\{\left(\frac{\mu|\vec
u|^2}{2}\right)\vec u+\left(\frac{Q|\vec v|^2}{2}\right)\vec
v+\left(\frac{|\vec D|^2+|\vec B|^2}{8\pi}\right)\vec
v+\left(\left(\frac{1}{4\pi c}\vec D\times \vec B\right)\cdot\vec
v\right)\vec v\right\}=
\\
\frac{1}{4\pi}div_\vec x\left\{(\vec D\otimes \vec D+ \vec B\otimes
\vec B)\cdot \vec v-\frac{1}{2}\left(|\vec D|^2+|\vec
B|^2\right)\vec v-c \vec D\times \vec B-4\pi\Phi(\mu\vec u+Q\vec
v)\right\}\\-div_{\vec x}\left\{\left(\Phi+\frac{1}{2}\left|\vec
v\right|^2\right)\left(\frac{1}{4\pi c}\,\vec D\times \vec
B\right)\right\}-\frac{1}{4\pi G}div_{\vec
x}\left\{\Phi\frac{\partial}{\partial t}(\nabla_{\vec
x}\Phi)\right\}+\vec F\cdot\vec u.
\end{multline}
Using \er{MaxVacFull1bjkgjhjhgjaaahkjhhjzzzyyykkknnnNWBWHWPPN} we
can rewrite
\er{MaxVacFull1ninshtrgravortghhghgjkgghklhjgkghghjjkjhjkkggjkhjkhjjhhfhjhkjkhbbgjhhkjhhklhzzzyyykkkgkhjjhgfhjjffghuikkgkjghhjkjknnnmmmNWNWBWHWhhjhhPPN}
as:
\begin{multline}\label{MaxVacFull1ninshtrgravortghhghgjkgghklhjgkghghjjkjhjkkggjkhjkhjjhhfhjhkjkhbbgjhhkjhhklhzzzyyykkkgkhjjhgfhjjffghuikkgkjghhjkjknnnmmmNWNWBWHWhhjhhENPPNyuuyttjk}
\frac{\partial}{\partial t}\left(\frac{\mu|\vec
u|^2}{2}+\frac{Q}{2}\left|\vec v\right|^2+\frac{\vec D\cdot\vec
E+\vec B\cdot\vec H}{8\pi}-\frac{1}{8\pi G}\big|\nabla_{\vec
x}\Phi\big|^2\right)=\\
\frac{\partial}{\partial t}\left(\frac{\mu|\vec
u|^2}{2}+\frac{Q}{2}\left|\vec v\right|^2+\frac{|\vec D|^2+|\vec
B|^2}{8\pi}+\frac{1}{4\pi c}\left(\vec D\times \vec
B\right)\cdot\vec v-\frac{1}{8\pi G}\big|\nabla_{\vec
x}\Phi\big|^2\right)=\\-div_\vec x\left\{\left(\frac{\mu|\vec
u|^2}{2}\right)\vec u+\left(\frac{Q|\vec v|^2}{2}\right)\vec
v+\left(\frac{|\vec D|^2+|\vec B|^2}{8\pi}\right)\vec
v+\left(\left(\frac{1}{4\pi c}\vec D\times \vec B\right)\cdot\vec
v\right)\vec v\right\}
\\
+\frac{1}{4\pi}div_\vec x\left\{(\vec D\otimes \vec D+ \vec B\otimes
\vec B)\cdot \vec v-\frac{1}{2}\left(|\vec D|^2+|\vec
B|^2\right)\vec v-c \vec D\times \vec B-4\pi\Phi(\mu\vec u+Q\vec
v)\right\}\\-div_{\vec x}\left\{\left(\Phi+\frac{1}{2}\left|\vec
v\right|^2\right)\left(\frac{1}{4\pi c}\,\vec D\times \vec
B\right)\right\}-\frac{1}{4\pi G}div_{\vec
x}\left\{\Phi\frac{\partial}{\partial t}(\nabla_{\vec
x}\Phi)\right\}+\vec F\cdot\vec u =\\-div_\vec
x\left\{\left(\frac{\mu|\vec u|^2}{2}\right)\vec
u+\left(\frac{Q|\vec v|^2}{2}\right)\vec v+\frac{1}{2}\left|\vec
v\right|^2\left(\frac{1}{4\pi c}\,\vec D\times \vec
B\right)+\left(\frac{\vec D\cdot\vec E+\vec B\cdot\vec
H}{8\pi}\right)\vec v\right\}
\\
+\frac{1}{4\pi}div_\vec x\left\{(\vec D\otimes \vec D+ \vec B\otimes
\vec B)\cdot \vec v-\frac{1}{2}\left(|\vec D|^2+|\vec
B|^2\right)\vec v-c \vec D\times \vec B\right\}\\-div_\vec
x\left\{\Phi\left(\mu\vec u+Q\vec v+\frac{1}{4\pi c}\,\vec D\times
\vec B\right)\right\}
%-div_{\vec x}\left\{\Phi\left(\frac{1}{4\pi c}\,\vec D\times \vec B\right)\right\}
-\frac{1}{4\pi G}div_{\vec
x}\left\{\Phi\frac{\partial}{\partial t}(\nabla_{\vec
x}\Phi)\right\}+\vec F\cdot\vec u.
\end{multline}
\end{comment}
%
%
%
%
As a consequence of
\er{vhfffngghkjgghggtghjgfhjoyuiyuyhiyyukukyihyuSYSPNNWhgjgghyuyy8yuyughghhghghhjjhhjhjkhjCCmmGGKKJJKKhghgghghgh},
\er{hvkgkjgkjbjkjjkgjglhhkhjyuyghjhhjhjfghfdhgdhdfdhzzzyyyjffjjkkhhkgjgkjhfhjhgfffjgjhgjffgjggjgjggjhgkkkggjgjgmomnnnNWKKhgjggh}
and
\er{vhfffngghkjgghggtghjgfhjoyuiyuyhiyyukukyihyuSYSPNNWhgjgghyuyy8yuyughghhyttyytgghhghghgghjhkjhCCmmGGKKuiiuuihjgghghiuiuiui}
we infer that we have the following proposition:
\begin{proposition}
Consider the Maxwell equation for the vacuum in the form
\er{MaxVacFull1bjkgjhjhgjaaahkjhhjzzzyyykkknnnNWBWHWPPN} and the
second Law of Newton for the moving continuum in the form
\er{MaxVacFull1ninshtrgravortghhghgjkgghklhjgkghghjjkjhjkkggjkhjkhjjhhfhjhkjkhbbgjhzzzyyykkknnnNWBWHWPPN}.
Next, assume that in some cartesian coordinate system $(*)$ we
observe the gravitational law in the form of
\er{MaxVacFull1ninshtrgravortghhghgjkgghklhjgkghghjjkjhjkkggjkhjkhjjhhfhjhklkhkhjjklzzzyyyhjggjhgghhjhNWNWBWHWPPN},
\er{MaxVacFull1ninshtrgravortghhghgjkgghklhjgkghghjjkjhjkkggjkhjkhjjhhfhjhklkhkhjjklzzzyyyhjggjhgghhjhNWNWNWBWHWPPN}
and
\er{MaxVacFull1ninshtrgravortghhghgjkgghklhjgkghghjjkjhjkkggjkhjkhjjhhfhjhklkhkhjjklzzzyyyNWNWBWHWPPN}.
Then in the system $(*)$ we have the following conservation laws of
the linear and angular momentums and energy:
\begin{multline}\label{hvkgkjgkjbjkjjkgjglhhkhjyuyghjhhjhjfghfdhgdhdfdhzzzyyyjffjjkkhhkgjgkjhfhjhgfffjgjhgjffgjggjgjggjhgkkkggjgjgnnnmmmNWNWNWNWBWHWMPPN}
\frac{\partial}{\partial t}\left(\mu\vec u+Q\vec v+\frac{1}{4\pi
c}\,\vec D\times \vec B\right)=\\-div_\vec x\left\{\mu\vec
u\otimes\vec u+Q\vec v\otimes\vec v+\left(\frac{1}{4\pi c}\vec
D\times \vec B\right)\otimes \vec v+\vec v\otimes\left(\frac{1}{4\pi
c}\vec D\times \vec
B\right)\right\}\\
+\frac{1}{4\pi}div_\vec x\left\{\vec D\otimes \vec D+\vec B\otimes
\vec B-\frac{1}{2}\left(|\vec D|^2+|\vec
B|^2\right)I-\frac{1}{G}\nabla_{\vec x}\Phi\otimes\nabla_{\vec
x}\Phi+\frac{1}{2G}\left|\nabla_{\vec x}\Phi\right|^2 I\right\}+\vec
F,
\end{multline}
\begin{multline}\label{hvkgkjgkjbjkjjkgjglhhkhjyuyghjhhjhjfghfdhgdhdfdhzzzyyyjffjjkkhhkgjgkjhfhjhgfffjgjhgjffgjggjgjggjhgkkkggjgjgmomnnnmmmNWNWBWHWMMPPN}
\frac{\partial}{\partial t}\left(\vec x\times(\mu\vec u)+\vec
x\times(Q\vec v)+\vec x\times\left(\frac{1}{4\pi c}\,\vec D\times
\vec B\right)\right)=\\-div_\vec x\left\{\mu(\vec x\times\vec
u)\otimes\vec u+Q(\vec x\times\vec v)\otimes\vec v+\left(\vec
x\times\left(\frac{1}{4\pi c}\vec D\times \vec
B\right)\right)\otimes \vec v+(\vec x\times\vec
v)\otimes\left(\frac{1}{4\pi c}\vec D\times \vec
B\right)\right\}\\
+\frac{1}{4\pi}div_\vec x\left\{(\vec x\times\vec D)\otimes \vec
D+(\vec x\times\vec B)\otimes \vec B-\frac{1}{G}(\vec
x\times\nabla_{\vec x}\Phi)\otimes\nabla_{\vec
x}\Phi\right\}\\+\frac{1}{8\pi}curl_{\vec x}\left\{\left(|\vec
D|^2+|\vec B|^2-\frac{1}{G}\left|\nabla_{\vec
x}\Phi\right|^2\right)\vec x\right\}+ \vec x\times \vec F,
\end{multline}
and
\begin{multline}\label{MaxVacFull1ninshtrgravortghhghgjkgghklhjgkghghjjkjhjkkggjkhjkhjjhhfhjhkjkhbbgjhhkjhhklhzzzyyykkkgkhjjhgfhjjffghuikkgkjghhjkjknnnmmmNWNWBWHWhhjhhENPPN}
\frac{\partial}{\partial t}\left(\frac{\mu|\vec
u|^2}{2}+\frac{Q}{2}\left|\vec v\right|^2+\frac{\vec D\cdot\vec
E+\vec B\cdot\vec H}{8\pi}-\frac{1}{8\pi G}\big|\nabla_{\vec
x}\Phi\big|^2\right)=\\
=-div_\vec x\left\{\left(\frac{\mu|\vec u|^2}{2}\right)\vec
u+\left(\frac{Q|\vec v|^2}{2}\right)\vec v+\frac{1}{2}\left|\vec
v\right|^2\left(\frac{1}{4\pi c}\,\vec D\times \vec
B\right)+\left(\frac{\vec D\cdot\vec E+\vec B\cdot\vec
H}{8\pi}\right)\vec v\right\}
\\
+\frac{1}{4\pi}div_\vec x\left\{(\vec D\otimes \vec D+ \vec B\otimes
\vec B)\cdot \vec v-\frac{1}{2}\left(|\vec D|^2+|\vec
B|^2\right)\vec v-c \vec D\times \vec B\right\}\\-div_\vec
x\left\{\Phi\left(\mu\vec u+Q\vec v+\frac{1}{4\pi c}\,\vec D\times
\vec B\right)\right\}
%-div_{\vec x}\left\{\Phi\left(\frac{1}{4\pi c}\,\vec D\times \vec B\right)\right\}
-\frac{1}{4\pi G}div_{\vec x}\left\{\Phi\frac{\partial}{\partial
t}(\nabla_{\vec x}\Phi)\right\}+\vec F\cdot\vec u.
\end{multline}
%
%
%
\begin{comment}
\begin{multline}
\frac{\partial}{\partial t}\left(\frac{\mu|\vec
u|^2}{2}+\frac{Q}{2}\left|\vec v\right|^2+\frac{|\vec D|^2+|\vec
B|^2}{8\pi}+\frac{1}{4\pi c}\left(\vec D\times \vec
B\right)\cdot\vec v-\frac{1}{8\pi G}\big|\nabla_{\vec
x}\Phi\big|^2\right)=\\-div_\vec x\left\{\left(\frac{\mu|\vec
u|^2}{2}\right)\vec u+\left(\frac{Q|\vec v|^2}{2}\right)\vec
v+\left(\frac{|\vec D|^2+|\vec B|^2}{8\pi}\right)\vec
v+\left(\left(\frac{1}{4\pi c}\vec D\times \vec B\right)\cdot\vec
v\right)\vec v\right\}
\\
+\frac{1}{4\pi}div_\vec x\left\{(\vec D\otimes \vec D+ \vec B\otimes
\vec B)\cdot \vec v-\frac{1}{2}\left(|\vec D|^2+|\vec
B|^2\right)\vec v-c \vec D\times \vec B-\Phi(\mu\vec u+Q\vec
v)\right\}\\-div_{\vec x}\left\{\left(\Phi+\frac{1}{2}\left|\vec
v\right|^2\right)\left(\frac{1}{4\pi c}\,\vec D\times \vec
B\right)\right\}-\frac{1}{4\pi G}div_{\vec
x}\left\{\Phi\frac{\partial}{\partial t}(\nabla_{\vec
x}\Phi)\right\}+\vec F\cdot\vec u.
\end{multline}
\end{comment}
%
%
%
\end{proposition}




























































\section{Lagrangian of the unified Gravitational-Electromagnetic field}\label{ghjfhgfdhd} Given known the distribution of inertial mass
density of some continuum medium $\mu:=\mu(\vec x,t)$, the field of
velocities of this medium $\vec u:=\vec u(\vec x,t)$, the charge
density $\rho:=\rho(\vec x,t)$ and the current density $\vec j:=\vec
j(\vec x,t)$ consider a Lagrangian density $L$ defined by
\begin{multline}\label{vhfffngghkjgghDD}
L\left(\vec A,\Psi,\vec v,\Phi,\vec p,\vec
x,t\right):=\frac{1}{8\pi}\left|-\nabla_{\vec
x}\Psi-\frac{1}{c}\frac{\partial\vec A}{\partial t}+\frac{1}{c}\vec
v\times curl_{\vec x}\vec A\right|^2-\frac{1}{8\pi}\left|curl_{\vec
x}\vec A\right|^2-\left(\rho\Psi-\frac{1}{c}\vec A\cdot\vec
j\right)\\+\frac{\mu}{2}\left|\vec u-\vec v\right|^2+
%\frac{k}{2}
\frac{1}{2}\left(d_{\vec x}\vec v+\left\{d_{\vec x}\vec
v\right\}^T\right)\cdot\left(d_{\vec x}\vec p+\left\{d_{\vec x}\vec
p\right\}^T\right)-2\left(div_{\vec x}\vec v\right)\left(div_{\vec
x}\vec p\right)\\+\frac{1}{4\pi G}\left(div_{\vec x}\vec
v\right)\left(\frac{\partial \Phi}{\partial t}+\vec
v\cdot\nabla_{\vec x} \Phi\right)+\frac{1}{4\pi
G}\Phi\left(div_{\vec x}\vec v\right)^2-\frac{\Phi}{16\pi
G}\left|d_{\vec x}\vec v+\left\{d_{\vec x}\vec
v\right\}^T\right|^2+\frac{1}{8\pi G}\left|\nabla_{\vec
x}\Phi\right|^2,
\end{multline}
where $\vec p$ is some proper vector field. Then $L$ is invariant
under the change of inertial or non-inertial cartesian coordinate
system. We investigate stationary points of the functional
\begin{equation}\label{btfffygtgyggyDD}
J=\int_0^T\int_{\mathbb{R}^3}L\left(\vec A,\Psi,\vec v,\Phi,\vec
p,\vec x,t\right)d\vec x dt.
\end{equation}
We denote
\begin{equation}\label{guigjgjffghDD}
\begin{cases}
\vec D=-\nabla_{\vec x}\Psi-\frac{1}{c}\frac{\partial\vec
A}{\partial t}+\frac{1}{c}\vec
v\times curl_{\vec x}\vec A\\
\vec B=curl_{\vec x}\vec A
\\
\vec E=-\nabla_{\vec x}\Psi-\frac{1}{c}\frac{\partial\vec A}{\partial t}=\vec D-\frac{1}{c}\vec v\times\vec B\\
\vec H=curl_{\vec x}\vec A+\frac{1}{c}\vec
v\times\left(-\nabla_{\vec x}\Psi-\frac{1}{c}\frac{\partial\vec
A}{\partial t}+\frac{1}{c}\vec v\times curl_{\vec x}\vec
A\right)=\vec B+\frac{1}{c}\vec v\times\vec D.
\end{cases}
\end{equation}
Then by \er{guigjgjffghDD} we have:
\begin{equation}\label{guigjgjffghjhkkgDD}
\begin{cases}
curl_{\vec x}\vec E+\frac{1}{c}\frac{\partial\vec B}{\partial t}=0\\
div_{\vec x}\vec B=0.
\end{cases}
\end{equation}
Moreover by \er{vhfffngghkjgghDD},  \er{apfrm3} and \er{apfrm6} we
have
\begin{equation}\label{vhfffngghkjgghggtghjgfhjyuiutDD}
\frac{\delta L}{\delta \vec p}=-div_{\vec x}\left(d_{\vec x}\vec
v+\left\{d_{\vec x}\vec v\right\}^T\right)+2\nabla_{\vec
x}\left(div_{\vec x}\vec v\right)=curl_{\vec x}\left(curl_{\vec
x}\vec v\right)=0,
\end{equation}
\begin{equation}\label{vhfffngghkjgghggtghjgfhjDD}
\frac{\delta L}{\delta \Phi}=-\frac{1}{4\pi
G}\left(\frac{\partial}{\partial t}\left\{div_{\vec x}\vec
v\right\}+\vec v\cdot\nabla_{\vec x}\left(div_{\vec x}\vec
v\right)+\frac{1}{4}\left|d_{\vec x}\vec v+\left\{d_{\vec x}\vec
v\right\}^T\right|^2\right)-\frac{1}{4\pi G}\Delta_{\vec x}\Phi=0,
\end{equation}
\begin{multline}\label{vhfffngghkjgghggtghjgfhjoyuiyuDD}
\frac{\delta L}{\delta \vec v}=-\left(\mu\vec u-\mu\vec
v+\frac{1}{4\pi c}\vec D\times\vec B\right)-div_{\vec
x}\left(d_{\vec x}\vec p+\left\{d_{\vec x}\vec
p\right\}^T\right)+2\nabla_{\vec x}\left(div_{\vec x}\vec
p\right)\\+\frac{1}{4\pi G}div_{\vec x}\left\{\left(d_{\vec x}\vec
v+\left\{d_{\vec x}\vec v\right\}^T\right)\Phi\right\}-\frac{1}{2\pi
G}\nabla_{\vec x}\left(\Phi\left(div_{\vec x}\vec
v\right)\right)-\frac{1}{4\pi G}\nabla_{\vec x}\left(\frac{\partial
\Phi}{\partial t}+\vec v\cdot\nabla_{\vec x}
\Phi\right)\\+\frac{1}{4\pi G}\left(div_{\vec x}\vec
v\right)\nabla_{\vec x}\Phi=-\left(\mu\vec u-\mu\vec v+\frac{1}{4\pi
c}\vec D\times\vec B\right)+curl_{\vec x}\left(curl_{\vec x}\vec
p\right)-\frac{1}{4\pi G}\Phi\,curl_{\vec x}\left(curl_{\vec x}\vec
v\right)\\-\frac{1}{4\pi G}\left(\frac{\partial}{\partial
t}\left(\nabla_{\vec x}\Phi\right)-curl_{\vec x}\left(\vec
v\times\nabla_{\vec x}\Phi\right)+\left(\Delta_{\vec
x}\Phi\right)\vec v\right)=0,
\end{multline}
\begin{equation}\label{vhfffngghkjgghggtghjgfhjhjkghghDD}
\frac{\delta L}{\delta \Psi}=\frac{1}{4\pi}div_{\vec x}\vec
D-\rho=0,
\end{equation}
and
\begin{equation}\label{vhfffngghkjgghggtghjgfhjhjkghghyuiuuDD}
\frac{\delta L}{\delta \vec A}=\frac{1}{c}\vec j+\frac{1}{4\pi
c}\frac{\partial\vec D}{\partial t}-\frac{1}{4\pi}curl_{\vec x}\vec
B-\frac{1}{4\pi c}curl_{\vec x}\left(\vec v\times \vec
D\right)=\frac{1}{c}\vec j+\frac{1}{4\pi c}\frac{\partial\vec
D}{\partial t}-\frac{1}{4\pi}curl_{\vec x}\vec H=0.
\end{equation}
So
\begin{equation}\label{guigjgjffghguygjyfDD}
\begin{cases}
curl_{\vec x}\vec H=\frac{4\pi}{c}\vec j+\frac{\partial\vec
D}{\partial
t}\\
div_{\vec x}\vec D=4\pi\rho\\
curl_{\vec x}\vec E+\frac{1}{c}\frac{\partial\vec B}{\partial t}=0\\
div_{\vec x}\vec B=0\\
\vec E=\vec D-\frac{1}{c}\vec v\times\vec B\\
\vec H=\vec B+\frac{1}{c}\vec v\times\vec D\\
curl_{\vec x}\left(curl_{\vec x}\vec v\right)=0
\\
\frac{\partial}{\partial t}\left\{div_{\vec x}\vec v\right\}+\vec
v\cdot\nabla_{\vec x}\left(div_{\vec x}\vec
v\right)+\frac{1}{4}\left|d_{\vec x}\vec v+\left\{d_{\vec x}\vec
v\right\}^T\right|^2=-\Delta_{\vec x}\Phi\\
\left(\mu\vec u-\mu\vec v+\frac{1}{4\pi c}\vec D\times\vec
B\right)=curl_{\vec x}\left(curl_{\vec x}\vec p\right)-\frac{1}{4\pi
G}\left(\frac{\partial}{\partial t}\left(\nabla_{\vec
x}\Phi\right)-curl_{\vec x}\left(\vec v\times\nabla_{\vec
x}\Phi\right)+\left(\Delta_{\vec x}\Phi\right)\vec v\right).
\end{cases}
\end{equation}
In particular, using continuum equation $\partial_t\mu+div_{\vec
x}\left(\mu\vec u\right)=0$ from the last equality in
\er{guigjgjffghguygjyfDD} we deduce
\begin{equation*}
%\label{vhfffngghkjgghggtghjgfhjhjkghghyuiuukjhjhhg}
\frac{\partial}{\partial t}\left(\frac{1}{4\pi G}\Delta_{\vec
x}\Phi-\mu\right)+div_{\vec x}\left\{\left(\frac{1}{4\pi
G}\Delta_{\vec x}\Phi-\mu\right)\vec v\right\}=-div_{\vec
x}\left\{\frac{1}{4\pi c}\vec D\times\vec B\right\}.
\end{equation*}
Thus denoting $Q=\Delta_{\vec x}\Phi/4\pi G-\mu$ we deduce
\begin{equation}\label{guigjgjffghguygjyfghggDD}
\begin{cases}
curl_{\vec x}\vec H=\frac{4\pi}{c}\vec j+\frac{\partial\vec
D}{\partial
t}\\
div_{\vec x}\vec D=4\pi\rho\\
curl_{\vec x}\vec E+\frac{1}{c}\frac{\partial\vec B}{\partial t}=0\\
div_{\vec x}\vec B=0\\
\vec E=\vec D-\frac{1}{c}\vec v\times\vec B\\
\vec H=\vec B+\frac{1}{c}\vec v\times\vec D\\
curl_{\vec x}\left(curl_{\vec x}\vec v\right)=0
\\
\frac{\partial}{\partial t}\left\{div_{\vec x}\vec v\right\}+\vec
v\cdot\nabla_{\vec x}\left(div_{\vec x}\vec
v\right)+\frac{1}{4}\left|d_{\vec x}\vec v+\left\{d_{\vec x}\vec
v\right\}^T\right|^2=-4\pi G(\mu+Q)\\
\frac{\partial Q}{\partial t}+div_{\vec x}\left(Q\vec
v\right)=-div_{\vec x}\left\{\frac{1}{4\pi c}\vec D\times\vec
B\right\}.
\end{cases}
\end{equation}
%SSD5























































































































































\section{Covariant formulation of the physical laws in the
four-dimensional non-relativistic space-time}\label{CVFRM}
\subsection{Four-vectors, four-covectors and tensors in the
four-dimensional non-relativistic space-time} First of all we would
like to remind the definitions of the vectors, covectors and
covariant and contravariant tensors of second order in
$\mathbb{R}^4$.
\begin{definition}
Given $\mathcal{S}$, that is a certain subgroup of the group of all
smooth non-degenerate invertible transformations from $\mathbb{R}^4$
onto $\mathbb{R}^4$ having the form
\begin{equation}\label{fgjfjhgghyuyyu}
\begin{cases}
x'^0=f^{(0)}(x^0,x^1,x^2,x^3),\\
x'^1=f^{(1)}(x^0,x^1,x^2,x^3),\\
x'^2=f^{(2)}(x^0,x^1,x^2,x^3),\\
x'^3=f^{(3)}(x^0,x^1,x^2,x^3),
\end{cases}
\end{equation}
we say that a one-component field $a:=a(x^0,x^1,x^2,x^3)$ is a
scalar field on the group $\mathcal{S}$, if under the coordinate
transformation in the group $\mathcal{S}$ of the form
\er{fgjfjhgghyuyyu} this field transforms as:
\begin{equation}\label{fgjfjhgghhgjgihhi}
a'=a.
\end{equation}
Next we say that a four-component field $(a^0,a^1,a^2,a^3)$ is a
four-vector field on the group $\mathcal{S}$, if under the
coordinate transformation in the group $\mathcal{S}$ of the form
\er{fgjfjhgghyuyyu} every of four components of this field
transforms as:
\begin{equation}\label{fgjfjhgghhgjg}
a'^j=\sum_{k=0}^{3}\frac{\partial f^{(j)}}{\partial
x^k}a^k\quad\quad\forall j=0,1,2,3.
\end{equation}
Next we say that a four-component field $(a_0,a_1,a_2,a_3)$ is a
four-covector field on the group $\mathcal{S}$, if under the
coordinate transformation in the group $\mathcal{S}$ of the form
\er{fgjfjhgghyuyyu} every of four components of this field
transforms as:
\begin{equation}\label{fgjfjhgghhgjghjhj}
a_j=\sum_{k=0}^{3}\frac{\partial f^{(k)}}{\partial
x^j}a'_k\quad\quad\forall j=0,1,2,3.
\end{equation}
Furthermore, we say that a $16$-component field
$\{a_{mn}\}_{m,n=0,1,2,3}$ is a two times covariant tensor field on
the group $\mathcal{S}$, if under the coordinate transformation in
the group $\mathcal{S}$ of the form \er{fgjfjhgghyuyyu} every of
$16$ components of this field transforms as:
\begin{equation}\label{fgjfjhgghhgjghjhjkkkkjjk}
a_{mn}=\sum_{j=0}^{3}\sum_{k=0}^{3}\frac{\partial f^{(k)}}{\partial
x^m}\frac{\partial f^{(j)}}{\partial x^n}a'_{kj}\quad\quad\forall\,
m,n=0,1,2,3.
\end{equation}
Next we say that a $16$-component field $\{a^{mn}\}_{m,n=0,1,2,3}$
is a two times contravariant tensor field on the group
$\mathcal{S}$, if under the coordinate transformation in the group
$\mathcal{S}$ of the form \er{fgjfjhgghyuyyu} every of $16$
components of this field transforms as:
\begin{equation}\label{fgjfjhgghhgjghjhjkkkkgggh}
a'^{mn}=\sum_{j=0}^{3}\sum_{k=0}^{3}\frac{\partial f^{(m)}}{\partial
x^k}\frac{\partial f^{(n)}}{\partial x^j}a^{kj}\quad\quad\forall\,
m,n=0,1,2,3.
\end{equation}
Then it is well known that for every two four-vectors
$(a^0,a^1,a^2,a^3)$ and $(b^0,b^1,b^2,b^3)$ on $\mathcal{S}$, the
$16$-component field $\{c^{mn}\}_{m,n=0,1,2,3}$, defined in every
coordinate system by
\begin{equation}\label{fgjfjhgghhgjghjhjkkkkgjghghjljl}
c^{mn}:=a^mb^n\quad\quad\forall\, m,n=0,1,2,3,
\end{equation}
is a two times contravariant tensor on $\mathcal{S}$. Moreover, for
every two four-covectors $(a_0,a_1,a_2,a_3)$ and $(b_0,b_1,b_2,b_3)$
on $\mathcal{S}$, the $16$-component field
$\{c_{mn}\}_{m,n=0,1,2,3}$, defined in every coordinate system by
\begin{equation}\label{fgjfjhgghhgjghjhjkkkkgjghghkkkj}
c_{mn}:=a_mb_n\quad\quad\forall\, m,n=0,1,2,3,
\end{equation}
is a two times covariant tensor on $\mathcal{S}$. It is also well
known that if $\{a^{mn}\}_{m,n=0,1,2,3}$ is a two times
contravariant tensor field on the group $\mathcal{S}$ and if a
$16$-component field $\{b_{mn}\}_{m,n=0,1,2,3}$ satisfies
\begin{equation}\label{fgjfjhgghhgjghjhjkkkkgjghghuiiiu}
\sum_{k=0}^{3}a^{mk}b_{kn}=\begin{cases}
1\quad\text{if}\quad m=n\\
0\quad\text{if}\quad m\neq n
\end{cases}\quad\quad\forall\, m,n=0,1,2,3,
\end{equation}
then $\{b_{mn}\}_{m,n=0,1,2,3}$ is a two times covariant tensor on
$\mathcal{S}$. Next it is well known that, given a four-covector
$(a_0,a_1,a_2,a_3)$ a four-vector $(b^0,b^1,b^2,b^3)$, a two times
covariant tensor $\{c_{mn}\}_{m,n=0,1,2,3}$ and a two times
contravariant tensor $\{d^{mn}\}_{m,n=0,1,2,3}$ on the group
$\mathcal{S}$, the quantities
\begin{equation}\label{fgjfjhgghhgjghjhjkkkkgjghghuiiiukljk}
\sum_{k=0}^{3}a_kb^k\quad\text{and}\quad
\sum_{m=0}^{3}\sum_{n=0}^{3}c_{mn}d^{mn}
\end{equation}
are scalars on $\mathcal{S}$, the four-component fields defined by
\begin{equation}\label{fgjfjhgghhgjghjhjkkkkgjghghuiiiulkkj}
\Big\{\sum_{k=0}^{3}d^{mk}a_{k}\Big\}_{m=0,1,2,3}\quad\text{and}\quad
\Big\{\sum_{k=0}^{3}c_{mk}b^{k}\Big\}_{m=0,1,2,3}
\end{equation}
are four-vector and four-covector on $\mathcal{S}$ and moreover,
$16$-component fields $\{\hat c^{mn}\}_{m,n=0,1,2,3}$ and $\{\hat
d_{mn}\}_{m,n=0,1,2,3}$ defined by
\begin{equation}\label{fgjfjhgghhgjghjhjkkkkgjghghuiiiulkkjlkkl}
\hat
c^{mn}:=\sum_{k=0}^{3}\sum_{j=0}^{3}d^{mj}d^{nk}c_{jk}\quad\text{and}\quad
%\Big\{\sum_{k=0}^{3}c_{mk}b^{k}\Big\}_{m=0,1,2,3}
\hat
d_{mn}:=\sum_{j=0}^{3}\sum_{k=0}^{3}c_{mj}c_{nk}d^{jk}\quad\quad\forall\,
m,n=0,1,2,3,
\end{equation}
are two times contravariant and two times covariant tensors on
$\mathcal{S}$. Next, it is also well known that given a two times
covariant tensor $\{c_{mn}\}_{m,n=0,1,2,3}$ and a two times
contravariant tensor $\{d^{mn}\}_{m,n=0,1,2,3}$ on the group
$\mathcal{S}$ the $16$-component fields $\{c_{nm}\}_{m,n=0,1,2,3}$
and $\{d^{nm}\}_{m,n=0,1,2,3}$ are also two times covariant and two
times contravariant tensors on $\mathcal{S}$. Finally, it is well
known that, if $a:=a(x^0,x^1,x^2,x^3)$ is a scalar field on the
group $\mathcal{S}$, then the four-component field
$(w_0,w_1,w_2,w_3)$ defined by:
\begin{equation}\label{fgjfjhgghhgjghjhjkkkkgjghghuiiiulkkjlkklplikkl}
w_j:=\frac{\partial a}{\partial x^j}\quad\quad\forall\,j=0,1,2,3,
\end{equation}
is a \underline{four-covector} field on the group $\mathcal{S}$.
\end{definition}
Next consider the four-dimensional space-time $\mathbb{R}^4$, such
that for every point in space $\vec x=(x_1,x_2,x_3)\in\mathbb{R}^3$
and every instant of time $t$ we correspond the point
$(x^0,x^1,x^2,x^3)\in\mathbb{R}^4$ that has the form:
\begin{equation}\label{fgjfjhggh}
(x^0,x^1,x^2,x^3):=(ct,x_1,x_2,x_3)=\left(ct,\vec x\right),
\end{equation}
where $c$ is the universal constant in Maxwell equations for vacuum.
In this space we denote by $\mathcal{S}_0$, the subgroup of the
group of smooth non-degenerate invertible mappings, containing
transformations of the form
\begin{equation}\label{noninchgravortbstrjgghguittu2intrrrZZygjyghhj}
\begin{cases}
x'^0=x^0
\\
x'^j=\sum\limits_{k=1}^{3}A_{jk}\left(\frac{x^0}{c}\right)\,x_k+z_j\left(\frac{x^0}{c}\right)\quad\forall
j=1,2,3,
\end{cases}
\end{equation}
where $$\left\{A_{jk}(t)\right\}_{j,k=1,2,3}=A(t):\mathbb{R}\to
SO(3)$$ is a rotation, smoothly dependent on $t$ and
$$\left(z_1(t),z_2(t),z_3(t)\right)=\vec
z(t):\mathbb{R}\to\mathbb{R}^3$$ also smoothly dependent on $t$.
Then in the terms of time $t$ and three-dimensional space we rewrite
\er{noninchgravortbstrjgghguittu2intrrrZZygjyghhj} as:
\begin{equation}\label{noninchgravortbstrjgghguittu2intrrrZZygjyg}
\begin{cases}
\vec x'=A(t)\cdot\vec x+\vec z(t),\\
t'=t,
\end{cases}
\end{equation}
where $A(t)\in SO(3)$ is a rotation. I.e. the group $\mathcal{S}_0$
represents all transformations of cartesian non-inertial coordinate
systems in the non-relativistic space-time. It can be easily checked
by trivial calculations that $\mathcal{S}_0$ is indeed a group, i.e.
for every two transformations $f,g\in\mathcal{S}_0$ the composition
$g\circ f$ and the inverse transformation $f^{(-1)}$ are also
contained in $\mathcal{S}_0$, thats mean that they also have a form
of \er{noninchgravortbstrjgghguittu2intrrrZZygjyghhj}. Next assume
that a four-covector $(a_0,a_1,a_2,a_3)$ and a four-vector
$(b^0,b^1,b^2,b^3)$ on the group $\mathcal{S}_0$ are given. Then by
inserting \er{noninchgravortbstrjgghguittu2intrrrZZygjyghhj} into
\er{fgjfjhgghhgjg} and \er{fgjfjhgghhgjghjhj} we obtain the
following laws of transformations under the acting in the group
$\mathcal{S}_0$:
\begin{equation}\label{fgjfjhgghhgjghjhjijhoj}
\begin{cases}
a_0=a'_0+\sum_{k=1}^{3}\frac{1}{c}\left(\sum\limits_{j=1}^{3}\frac{dA_{kj}}{dt}\left(\frac{x^0}{c}\right)\,x_j+\frac{d
z_k}{dt}\left(\frac{x^0}{c}\right)\right)\,a'_k
\\
a_j=\sum_{k=1}^{3}A_{kj}\left(\frac{x^0}{c}\right)\,a'_k\quad\quad\forall
j=1,2,3,
\end{cases}
\end{equation}
and
\begin{equation}\label{fgjfjhgghhgjgiuouoiuu}
\begin{cases}
b'^0=b^0
\\
b'^j=\frac{1}{c}\left(\sum\limits_{k=1}^{3}\frac{dA_{jk}}{dt}\left(\frac{x^0}{c}\right)\,x_k+\frac{d
z_j}{dt}\left(\frac{x^0}{c}\right)\right)\,b^0+\sum_{k=1}^{3}A_{jk}\left(\frac{x^0}{c}\right)\,b^k\quad\quad\forall
j=1,2,3.
\end{cases}
\end{equation}
In particular, since $A(t)\in SO(3)$ and thus
\begin{equation}\label{fgjfjhgghhgjgiuouoiuu1}
\sum_{j=1}^{3}A_{mj}(t)A_{nj}(t)=\begin{cases}1\quad\text{if}\quad m=n\\
0\quad\text{if}\quad m\neq n
\end{cases}
\quad\quad\quad\forall m,n=1,2,3,
\end{equation}
by \er{fgjfjhgghhgjghjhjijhoj} we deduce:
\begin{equation}\label{fgjfjhgghhgjghjhjijhojpiii}
\begin{cases}
a'_0=a_0-\sum_{k=1}^{3}\frac{1}{c}\left(\sum\limits_{j=1}^{3}\frac{dA_{kj}}{dt}\left(\frac{x^0}{c}\right)\,x_j+\frac{d
z_k}{dt}\left(\frac{x^0}{c}\right)\right)\left(\sum_{j=1}^{3}A_{kj}\left(\frac{x^0}{c}\right)\,a_j\right)
\\
a'_k=\sum_{j=1}^{3}A_{kj}\left(\frac{x^0}{c}\right)\,a_j\quad\quad\forall
k=1,2,3.
\end{cases}
\end{equation}
So, by \er{fgjfjhgghhgjghjhjijhojpiii} and
\er{fgjfjhgghhgjgiuouoiuu} we obtained the following laws of
transformation of four-covectors and four-vectors in the group
$\mathcal{S}_0$, i.e. under the change of non-inertial cartesian
coordinate systems:
\begin{equation}\label{fgjfjhgghhgjghjhjijhojpiiihjhj}
\begin{cases}
a'_0=a_0-\sum_{k=1}^{3}\frac{1}{c}\left(\sum\limits_{j=1}^{3}\frac{dA_{kj}}{dt}\left(\frac{x^0}{c}\right)\,x_j+\frac{d
z_k}{dt}\left(\frac{x^0}{c}\right)\right)\left(\sum_{j=1}^{3}A_{kj}\left(\frac{x^0}{c}\right)\,a_j\right)
\\
a'_k=\sum_{j=1}^{3}A_{kj}\left(\frac{x^0}{c}\right)\,a_j\quad\quad\forall
k=1,2,3,
\end{cases}
\end{equation}
and
\begin{equation}\label{fgjfjhgghhgjgiuouoiuujkjkjk}
\begin{cases}
b'^0=b^0
\\
b'^j=\frac{1}{c}\left(\sum\limits_{k=1}^{3}\frac{dA_{jk}}{dt}\left(\frac{x^0}{c}\right)\,x_k+\frac{d
z_j}{dt}\left(\frac{x^0}{c}\right)\right)\,b^0+\sum_{k=1}^{3}A_{jk}\left(\frac{x^0}{c}\right)\,b^k\quad\quad\forall
j=1,2,3.
\end{cases}
\end{equation}
%
%
%
\begin{comment}
\begin{equation}\label{fgjfjhgghhgjghjhjijhojihjhj}
\begin{cases}
a'_0=\frac{\partial f^{(0)}}{\partial
x^0}a_0+\sum_{k=1}^{3}\frac{\partial f^{(k)}}{\partial x^0}a_k
\\
a'_j=\frac{\partial f^{(0)}}{\partial
x^j}a_0+\sum_{k=1}^{3}\frac{\partial f^{(k)}}{\partial
x^j}a_k\quad\quad\forall j=1,2,3,
\end{cases}
\end{equation}
and
\begin{equation}\label{fgjfjhgghhgjgiuouoiuuhjhj}
\begin{cases}
b'^0=\frac{\partial f^{(0)}}{\partial
x^0}b^0+\sum_{k=1}^{3}\frac{\partial f^{(0)}}{\partial x^k}b^k
\\
b'^j=\frac{\partial f^{(j)}}{\partial
x^0}b^0+\sum_{k=1}^{3}\frac{\partial f^{(j)}}{\partial
x^k}b^k\quad\quad\forall j=1,2,3.
\end{cases}
\end{equation}
\end{comment}
%
%
%
Therefore, if we denote the four-vector $(b^0,b^1,b^2,b^3)$ and the
four-covector $(a_0,a_1,a_2,a_3)$ on the group $\mathcal{S}_0$ as:
\begin{equation}\label{fgjfjhgghhgjghjhjijhojihjhjjijhjjj}
\begin{cases}
(b^0,b^1,b^2,b^3)=\left(\sigma,\frac{1}{c}\vec
b\right)\quad\text{where}\quad
\sigma:=b^0\;\;\text{and}\;\;\vec b:=c(b^1,b^2,b^3)\in\mathbb{R}^3,\\
(a_0,a_1,a_2,a_3)=(\psi,-\vec a)\quad\text{where}\quad
\psi:=a_0\;\;\text{and}\;\;\vec a:=-(a_1,a_2,a_3)\in\mathbb{R}^3,
\end{cases}
\end{equation}
then by \er{fgjfjhgghhgjghjhjijhojpiiihjhj} and
\er{fgjfjhgghhgjgiuouoiuujkjkjk} in the terms of time $t$ and three
dimensional space $\vec x$, we obtain the following laws of
transformations of $\sigma$, $\vec b$, $\psi$ and $\vec a$ under the
change of non-inertial cartesian coordinate system:
\begin{equation}\label{fgjfjhgghhgjgiuouoiuujkjkjkojkoiu}
\begin{cases}
\sigma'=\sigma
\\
\vec b'=A\left(t\right)\cdot\vec
b+\left(\frac{dA}{dt}\left(t\right)\cdot\vec x+\frac{d \vec
z}{dt}\left(t\right)\right)\sigma,
\end{cases}
\end{equation}
and
\begin{equation}\label{fgjfjhgghhgjghjhjijhojpiiihjhjuiouj}
\begin{cases}
\psi'=\psi+\frac{1}{c}\left(\frac{dA}{dt}\left(t\right)\cdot\vec
x+\frac{d \vec z}{dt}\left(t\right)\right)\cdot\left(A(t)\cdot\vec
a\right)
\\
\vec a'=A(t)\cdot\vec a.
\end{cases}
\end{equation}
In particular, if $\sigma:=b^0$ is the first coordinate of an
arbitrary four-vector $(b^0,b^1,b^2,b^3)$ on the group
$\mathcal{S}_0$, then $\sigma$ is a proper scalar field in the
frames of Definition \ref{bggghghgj}. Moreover, if $\vec
a:=-(a_1,a_2,a_3)$, where $a_1,a_2,a_3$ are the last three
coordinates of an arbitrary four-covector $(a_0,a_1,a_2,a_3)$ on the
group $\mathcal{S}_0$, then $\vec a$ is a proper vector field in the
frames of Definition \ref{bggghghgj}.

Next, since by Definition \ref{bggghghgj} every three-dimensional
speed-like vector field $\vec u$ transforms under the change of
non-inertial cartesian coordinate system as:
\begin{equation}
\label{NoIn3redPPN'}\vec u'=A(t)\cdot \vec
u+\frac{dA}{dt}\left(t\right)\cdot\vec x+\frac{d \vec
z}{dt}\left(t\right),
\end{equation}
by comparing \er{NoIn3redPPN'} with
\er{fgjfjhgghhgjgiuouoiuujkjkjkojkoiu} we deduce that for every
speed-like vector field $\vec u$ the four-component field
$(u^0,u^1,u^2,u^3)$ defined by
\begin{equation}\label{fgjfjhgghhgjghjhjijhojihjhjjijhjjjjjuii}
(u^0,u^1,u^2,u^3):=\left(1,\frac{1}{c}\vec
u\right)\quad\text{where}\quad
u^0=1\;\;\text{and}\;\;(u^1,u^2,u^3)=\frac{1}{c}\vec
u\in\mathbb{R}^3,
\end{equation}
is a four-vector field on the group $\mathcal{S}_0$. We call such
four-vectors by the name vectors of type $1$. In particular, if
$\vec u$ is the velocity field, then the quantity defined by
\er{fgjfjhgghhgjghjhjijhojihjhjjijhjjjjjuii} is a a four-vector
field on the group $\mathcal{S}_0$ that we call the four-dimensional
speed. Regarding the field of velocity $\vec u$ we also can give a
different argumentation that the four-component field
$(u^0,u^1,u^2,u^3)$ defined by
\er{fgjfjhgghhgjghjhjijhojihjhjjijhjjjjjuii} is a four-vector field
on the group $\mathcal{S}_0$: indeed it is well known from Tensor
Analysys that if $\left(x^0(s),x^1(s),x^2(s),x^3(s)\right)$ is a
curve in $\mathbb{R}^4$, parameterized by some scalar parameter $s$,
then the four-component field
$\left(\frac{dx^0}{ds}(s),\frac{dx^1}{ds}(s),\frac{dx^2}{ds}(s),\frac{dx^3}{ds}(s)\right)$
is a four-vector field on an arbitrary group $\mathcal{S}$ and, in
particular, on the group $\mathcal{S}_0$. Thus, if $\vec
r(t)=\left(r_1(t),r_2(t),r_3(t)\right)$ is a three-dimensional
trajectory of the motion of some particle, parameterized by the
global time $t$, then if we consider a curve
$\frac{1}{c}\left(ct,r_1(t),r_2(t),r_3(t)\right)$ in $\mathbb{R}^4$,
parameterized by the global time $t$, then the four-component field:
\begin{equation}\label{fgjfjhgghhgjghjhjijhojihjhjjijhjjjjjuiijhjhh}
\left(1,\frac{1}{c}\frac{d\vec r}{dt}(t)\right):=
\left(1,\frac{1}{c}\frac{dr_1}{dt}(t),\frac{1}{c}\frac{dr_2}{dt}(t),\frac{1}{c}\frac{dr_3}{dt}(t)\right)
\end{equation}
is a four-vector field on the group $\mathcal{S}_0$.


 Similarly, if $\vec v$ is the vectorial gravitational
potential, then since $\vec v$ is a speed-like vector field, the
four-component field $(v^0,v^1,v^2,v^3)$ defined by
\begin{equation}\label{fgjfjhgghhgjghjhjijhojihjhjjijhjjjjjuiijjjk}
(v^0,v^1,v^2,v^3):=\left(1,\frac{1}{c}\vec
v\right)\quad\text{where}\quad
v^0=1\;\;\text{and}\;\;(v^1,v^2,v^3)=\frac{1}{c}\vec v,
\end{equation}
is also a four-vector field on the group $\mathcal{S}_0$ that we
call the four-dimensional gravitational potential.

Moreover, by \er{fgjfjhgghhgjgiuouoiuujkjkjkojkoiu}, for every speed
like vector field $\vec u$ and every proper scalar field $\sigma$
the four-component field $(b^0,b^1,b^2,b^3)$ defined by
\begin{equation}\label{fgjfjhgghhgjghjhjijhojihjhjjijhjjjjjuiijjjkl}
(b^0,b^1,b^2,b^3):=\left(\sigma,\frac{\sigma}{c}\vec
u\right)\quad\text{where}\quad
b^0=\sigma\;\;\text{and}\;\;(b^1,b^2,b^3)=\frac{\sigma}{c}\vec u,
\end{equation}
is also a four-vector field on the group $\mathcal{S}_0$. In
particular, if we consider the field of four-dimensional moment of a
particle $(p^0,p^1,p^2,p^3)$ defined by
\begin{equation}\label{fgjfjhgghhgjghjhjijhojihjhjjijhjjjjjuiijjjklihh}
(p^0,p^1,p^2,p^3):=\left(m,\frac{1}{c}(m\vec
u)\right)\quad\text{where}\quad
p^0=m\;\;\text{and}\;\;(p^1,p^2,p^3)=\frac{1}{c}(m\vec u),
\end{equation}
where $m$ is the mass of the particle and $\vec u$ is the velocity
of the particle, then $(p^0,p^1,p^2,p^3)$ is also a four-vector on
the group $\mathcal{S}_0$. Moreover, by comparing \er{NoIn4redPPN}
and \er{NoIn6redPPN} with \er{fgjfjhgghhgjgiuouoiuujkjkjkojkoiu} we
deduce that if we consider the field of four-dimensional electric
current $(j^0,j^1,j^2,j^3)$ defined by
\begin{equation}\label{fgjfjhgghhgjghjhjijhojihjhjjijhjjjjjuiijjjklihhojjjo}
(j^0,j^1,j^2,j^3):=\left(\rho,\frac{1}{c}\,\vec
j\right)\quad\text{where}\quad
j^0=\rho\;\;\text{and}\;\;(j^1,j^2,j^3)=\frac{1}{c}\,\vec j,
\end{equation}
where $\rho$ is the electric charge density and $\vec j$ is the
electric current density, then $(j^0,j^1,j^2,j^3)$ is also a
four-vector on the group $\mathcal{S}_0$.

On the other hand, for every proper three-dimensional vector field
$\vec G$ that satisfies due to Definition \ref{bggghghgj}:
\begin{equation}
\label{NoIn1redPPN'}\vec G'=A(t)\cdot\vec G,
\end{equation}
by comparing \er{NoIn1redPPN'} with
\er{fgjfjhgghhgjgiuouoiuujkjkjkojkoiu} we deduce that the
four-component field $(G^0,G^1,G^2,G^3)$ defined by
\begin{equation}\label{fgjfjhgghhgjghjhjijhojihjhjjijhjjjjjuiiklkllo;}
(G^0,G^1,G^2,G^3):=\left(0,\vec G\right)\quad\text{where}\quad
G^0=0\;\;\text{and}\;\;(G^1,G^2,G^3)=\vec G,
\end{equation}
is also a four-vector field on the group $\mathcal{S}_0$.  We call
such four-vectors by the name vectors of type $0$.

Next, since by
\er{vhfffngghhjghhgPPNghghghutghfflklhjkjhjhjjgjkghhj} the scalar
electromagnetic potential $\Psi$ and the vector electromagnetic
potential $\vec A$, under the change of non-inertial cartesian
coordinate system transform as:
\begin{equation}\label{vhfffngghhjghhgPPNghghghutghfflklhjkjhjhjjgjkghhjhhjhjkhjghlkk}
\begin{cases}
\Psi'=
%\Psi+\frac{1}{c}\vec A\cdot\left(A^T(t)\cdot\frac{dA}{dt}(t)\cdot\vec x+A^T(t)\cdot\frac{d\vec z}{dt}(t)\right)
\Psi+\frac{1}{c}\left(\frac{dA}{dt}(t)\cdot\vec x+\frac{d\vec
z}{dt}(t)\right)\cdot\left(A(t)\cdot\vec A\right)
\\
\vec A'=A(t)\cdot \vec A,
\end{cases}
\end{equation}
by comparing
\er{vhfffngghhjghhgPPNghghghutghfflklhjkjhjhjjgjkghhjhhjhjkhjghlkk}
with \er{fgjfjhgghhgjghjhjijhojpiiihjhjuiouj} we deduce that the
four-component field $(A_0,A_1,A_2,A_3)$ defined as
\begin{equation}\label{fgjfjhgghhgjghjhjijhojihjhjjijhjjjljljpk}
(A_0,A_1,A_2,A_3)=(\Psi,-\vec A)\quad\text{where}\quad
A_0=\Psi\;\;\text{and}\;\;(A_1,A_2,A_3)=-\vec A,
\end{equation}
is a four-\underline{covector} field on the group $\mathcal{S}_0$.
We call this  four-covector field by the name four dimensional
electromagnetic potential. Next, since $(A_0,A_1,A_2,A_3)$ is a
four-covector field on the group $\mathcal{S}_0$, then it is well
known from the tensor analysis that the $16$-component field
$\{F_{ij}\}_{0\leq i,j\leq 3}$ defined in every non-inertial
cartesian coordinate system by
\begin{equation}\label{huohuioy89gjjhjffffff3478zzrrZZZhjhhjhhjjhhffGGhjjh}
F_{ij}:=\frac{\partial A_j}{\partial x^i}-\frac{\partial
A_i}{\partial x^j}\quad\quad\forall\, i,j=0,1,2,3\,,
\end{equation}
is an antisymmetric two times covariant tensor field on the group
$\mathcal{S}_0$, which we call the covariant tensor of the
electromagnetic field. In particular, by inserting
\er{fgjfjhgghhgjghjhjijhojihjhjjijhjjjljljpk} and \er{fgjfjhggh}
into \er{huohuioy89gjjhjffffff3478zzrrZZZhjhhjhhjjhhffGGhjjh} we
deduce:
\begin{equation}\label{huohuioy89gjjhjffffff3478zzrrZZZhjhhjhhjjhhffGGhjjhiuui}
\begin{cases}
F_{00}=0\\ F_{0j}=-F_{j0}=-\frac{1}{c}\frac{\partial(-A_j)}{\partial
t}-\frac{\partial \Psi}{\partial
x^j}\quad\quad\forall\, j=1,2,3\\
F_{jj}=0\quad\quad\forall\, j=1,2,3
\\
F_{ij}=-F_{ji}=\frac{\partial (-A_i)}{\partial x^j}-\frac{\partial
(-A_j)}{\partial x^i}\quad\quad\forall\, i\neq j=1,2,3\,,
\end{cases}
\end{equation}
Thus if as in \er{MaxVacFull1bjkgjhjhgjgjgkjfhjfdghghligioiuittrPPN}
we denote:
\begin{equation}\label{MaxVacFull1bjkgjhjhgjgjgkjfhjfdghghligioiuittrPPNkkk}
\begin{cases}
\vec B:= curl_{\vec x} \vec A,\\
\vec E:=-\nabla_{\vec x}\Psi-\frac{1}{c}\frac{\partial\vec
A}{\partial t},
%,\\ div_{\vec x}\vec A\equiv 0,
\end{cases}
\end{equation}
then denoting  $\vec E:=(E_1,E_2,E_3)$ and $\vec B:=(B_1,B_2,B_3)$,
by \er{MaxVacFull1bjkgjhjhgjgjgkjfhjfdghghligioiuittrPPNkkk}
% and \er{fgjfjhgghhgjghjhjijhojihjhjjijhjjjljljpk}
we rewrite
\er{huohuioy89gjjhjffffff3478zzrrZZZhjhhjhhjjhhffGGhjjhiuui} as:
\begin{equation}\label{huohuioy89gjjhjffffff3478zzrrZZZhjhhjhhjjhhffGGhjjhiuuijkjjk}
\begin{cases}
F_{00}=0\\ F_{0j}=-F_{j0}=E_j\quad\quad\forall\, j=1,2,3\\
F_{jj}=0\quad\quad\forall\, j=1,2,3
\\
F_{12}=-F_{21}=-B_3
%\frac{\partial (-A_1)}{\partial x^2}-\frac{\partial(-A_2)}{\partial x^1}
\\
F_{13}=-F_{31}=B_2
%\frac{\partial (-A_1)}{\partial x^3}-\frac{\partial(-A_3)}{\partial x^1}
\\
F_{23}=-F_{32}=-B_1\,.
%\frac{\partial (-A_2)}{\partial x^3}-\frac{\partial (-A_3)}{\partial x^2}
\end{cases}
\end{equation}


Next assume that $T:=\left\{T_{ij}\right\}_{i,j=1,2,3}\in
\R^{3\times 3}$ is a $9$-component proper matrix valued field,
which, being a proper matrix field, by Definition \ref{bggghghgj}
satisfies:
\begin{equation}\label{uguyytfddddgghjjghjjjihohjjk}
T'=A(t)\cdot T\cdot A^T(t)=A(t)\cdot T\cdot
\left\{A(t)\right\}^{-1}.
\end{equation}
Next consider a $16$-component field $\{\mathcal{T}^{ij}\}_{0\leq
i,j\leq 3}$ defined in every non-inertial cartesian coordinate
system by
\begin{equation}\label{huohuioy89gjjhjffffff3478zzrrZZZhjhhjhhjjhhffGGhjjhuiiuihhj}
\begin{cases}
\mathcal{T}^{00}=0
\\
\mathcal{T}^{0j}=\mathcal{T}^{j0}=0\quad\quad\forall\, j=1,2,3
\\
\mathcal{T}^{ij}:=T_{ij}\quad\quad\forall\, i,j=1,2,3\,,
\end{cases}
\end{equation}
Then by inserting \er{noninchgravortbstrjgghguittu2intrrrZZygjyghhj}
and \er{uguyytfddddgghjjghjjjihohjjk} into
\er{fgjfjhgghhgjghjhjkkkkgggh}, we can prove that the field
$\{\mathcal{T}^{ij}\}_{0\leq i,j\leq 3}$ defined by
\er{huohuioy89gjjhjffffff3478zzrrZZZhjhhjhhjjhhffGGhjjhuiiuihhj} is
a two times \underline{contravariant} tensor field on the group
$\mathcal{S}_0$. Indeed, by \er{fgjfjhgghhgjghjhjkkkkgggh} for every
two times contravariant tensor field $\{a^{ij}\}_{0\leq i,j\leq 3}$
we have
\begin{multline}\label{fgjfjhgghhgjghjhjkkkkggghkjjk}
a'^{mn}=\frac{\partial f^{(m)}}{\partial x^0}\frac{\partial
f^{(n)}}{\partial x^0}a^{00}+\sum_{k=1}^{3}\frac{\partial
f^{(m)}}{\partial x^k}\frac{\partial f^{(n)}}{\partial
x^0}a^{k0}+\sum_{j=1}^{3}\frac{\partial f^{(m)}}{\partial
x^0}\frac{\partial f^{(n)}}{\partial
x^j}a^{0j}\\+\sum_{j=1}^{3}\sum_{k=1}^{3}\frac{\partial
f^{(m)}}{\partial x^k}\frac{\partial f^{(n)}}{\partial
x^j}a^{kj}\quad\quad\forall\, m,n=0,1,2,3.
\end{multline}
Then, since by \er{noninchgravortbstrjgghguittu2intrrrZZygjyghhj} we
have $\frac{\partial f^{(0)}}{\partial x^0}=1$, $\frac{\partial
f^{(0)}}{\partial x^k}=0$ $\forall k=1,2,3$ and $\frac{\partial
f^{(m)}}{\partial x^k}=A_{mk}\left(\frac{x^0}{c}\right)$ $\forall
k,m=1,2,3$, in the case where $a^{00}=0$ and $a^{0j}=a^{j0}=0$
$\forall j=1,2,3$  we rewrite \er{fgjfjhgghhgjghjhjkkkkggghkjjk} as:
\begin{equation}\label{fgjfjhgghhgjghjhjkkkkggghjjkjhgj}
\begin{cases}
a'^{00}=0\\
a'^{j0}=a'^{0j}=0\quad\quad\forall\, j=1,2,3,
\\
a'^{mn}=\sum_{j=1}^{3}\sum_{k=1}^{3}A_{mk}\left(\frac{x^0}{c}\right)A_{nj}\left(\frac{x^0}{c}\right)a^{kj}\quad\quad\forall\,
m,n=1,2,3.
\end{cases}
\end{equation}
that is compatible with
\er{huohuioy89gjjhjffffff3478zzrrZZZhjhhjhhjjhhffGGhjjhuiiuihhj} and
\er{uguyytfddddgghjjghjjjihohjjk}.

In particular, if we consider the $9$-component matrix field $I$
that defined in every cartesian coordinate system as
$I:=\left\{\delta_{ij}\right\}_{1,j=1,2,3}\in \R^{3\times 3}$, where
\begin{equation}\label{huohuioy89gjjhjffffff3478zzrrZZZhjhhjhhjjhhffGGhjjhuiiuihhjkklpkl}
\delta_{ij}:=
\begin{cases}
1\quad\text{if}\quad i=j
\\
0\quad\text{if}\quad i\neq j,
\end{cases}
\end{equation}
which is a proper matrix field, since
\begin{equation}\label{uguyytfddddgghjjghjjjihohjjkkkook}
I=A(t)\cdot I\cdot \left\{A(t)\right\}^{-1},
\end{equation}
then the $16$-component field $\{{\Theta}^{ij}\}_{0\leq i,j\leq 3}$
defined in every non-inertial cartesian coordinate system by
\begin{equation}\label{huohuioy89gjjhjffffff3478zzrrZZZhjhhjhhjjhhffGGhjjhuiiuihhjkkoioi}
\begin{cases}
{\Theta}^{00}=0
\\
{\Theta}^{0j}={\Theta}^{j0}=0\quad\quad\forall\, j=1,2,3
\\
{\Theta}^{ij}:=\delta_{ij}\quad\quad\forall\, i,j=1,2,3
\end{cases}
\end{equation}
is a two times contravariant tensor field on the group
$\mathcal{S}_0$ and moreover, this tensor is symmetric. We call
$\{{\Theta}^{ij}\}_{0\leq i,j\leq 3}$ the contravariant tensor of
the three-dimensional geometry.

Finally the scalar field $\tau:=\tau(x^0,x^1,x^2,x^3)$, defined in
every cartesian coordinate system as
\begin{equation}\label{uguyytfddddgghkjhhjuuiui}
\tau:=\frac{x^0}{c}=t,
\end{equation}
is a scalar on the group $\mathcal{S}_0$. Here $t$ is the global
non-relativistic time. Moreover, by
\er{fgjfjhgghhgjghjhjkkkkgjghghuiiiulkkjlkklplikkl} and
\er{uguyytfddddgghkjhhjuuiui}, the four-component field
$(v_0,v_1,v_2,v_3)$ defined by:
\begin{equation}\label{fgjfjhgghhgjghjhjkkkkgjghghuiiiulkkjlkklplikklkjljghhgghk}
v_0:=c\,\frac{\partial \tau}{\partial
x^0}(x^0,x^1,x^2,x^3)=1\quad\text{and}\quad v_j:=c\,\frac{\partial
\tau}{\partial x^j}(x^0,x^1,x^2,x^3)=0\quad\quad\forall\,j=1,2,3,
\end{equation}
is a \underline{four-covector} field on the group $\mathcal{S}_0$.
\subsection{Pseudo-metric tensors of the four-dimensional space-time}
Consider $\{g^{ij}\}_{0\leq i,j\leq 3}$ to be a two times
contravariant tensor field on the group $\mathcal{S}_0$, defined by
\begin{equation}\label{hoyuiouigyfghgjh3478zzrrZZffGGjkkjojj}
g^{ij}:=v^iv^j-{\Theta}^{ij}\quad\quad\forall\,i,j=0,1,2,3,
\end{equation}
where $\{{\Theta}^{ij}\}_{0\leq i,j\leq 3}$ is the contravariant
tensor of the three-dimensional geometry, defined by
\er{huohuioy89gjjhjffffff3478zzrrZZZhjhhjhhjjhhffGGhjjhuiiuihhjkkoioi}
and being a two times contravariant tensor, and $(v^0,v^1,v^2,v^3)$
is the four-dimensional gravitational potential, defined by
\er{fgjfjhgghhgjghjhjijhojihjhjjijhjjjjjuiijjjk} and being a
four-vector. Then by \er{fgjfjhgghhgjghjhjkkkkgjghghjljl} we obtain
that $\{g^{ij}\}_{0\leq i,j\leq 3}$ is indeed a two times
contravariant tensor field on the group $\mathcal{S}_0$ and
moreover, this tensor is symmetric. Moreover, by
\er{huohuioy89gjjhjffffff3478zzrrZZZhjhhjhhjjhhffGGhjjhuiiuihhjkkoioi}
and \er{fgjfjhgghhgjghjhjijhojihjhjjijhjjjjjuiijjjk} we have:
\begin{equation}\label{hoyuiouigyfghgjh3478zzrrZZffGGjkkj}
\begin{cases}
g^{00}=1\\
g^{ij}=-\delta_{ij}+\frac{v_iv_j}{c^2}\quad\forall 1\leq i,j\leq 3\\
g^{0j}=g^{j0}=\frac{v_j}{c}\quad\forall 1\leq j\leq 3.
\end{cases}
\end{equation}
We call the tensor $\{g^{ij}\}_{0\leq i,j\leq 3}$ the contravariant
pseudo-metric tensor of the four-dimensional space-time. Next
consider a $16$-component field $\{g_{ij}\}_{0\leq i,j\leq 3}$
defined by
\begin{equation}\label{hoyuiouigyfg3478zzrrZZffGGhhjhj}
\begin{cases}
g_{00}=1-\frac{|\vec v|^2}{c^2}\\
g_{ij}=-\delta_{ij}\quad\forall 1\leq i,j\leq 3\\
g_{0j}=g_{j0}=\frac{v_j}{c}\quad\forall 1\leq j\leq 3.
\end{cases}
\end{equation}
where $\vec v=(v_1,v_2,v_3)$ is the three-dimensional vectorial
gravitational potential. Then
\begin{equation*}
\sum_{k=0}^{3}g_{0k}g^{k0}=g_{00}g^{00}+\sum_{k=1}^{3}g_{0k}g^{k0}=
1-\frac{|\vec v|^2}{c^2}+\frac{|\vec v|^2}{c^2}=1,
\end{equation*}
\begin{equation*}
\sum_{k=0}^{3}g_{ik}g^{kj}=g_{i0}g^{0j}+\sum_{k=1}^{3}g_{ik}g^{kj}=
\frac{v_iv_j}{c^2}+\delta_{ij}-\frac{v_iv_j}{c^2}=\delta_{ij}\quad\forall
1\leq i,j\leq 3,
\end{equation*}
and
\begin{equation*}
\sum_{k=0}^{3}g_{ik}g^{k0}=g_{i0}g^{00}+\sum_{k=1}^{3}g_{ik}g^{k0}=\frac{v_i}{c}-\frac{v_i}{c}=0
\quad\forall 1\leq i\leq 3,
\end{equation*}
\begin{multline*}
\sum_{k=0}^{3}g_{0k}g^{kj}=g_{00}g^{0j}+\sum_{k=1}^{3}g_{0k}g^{kj}=
\left(1-\frac{|\vec v|^2}{c^2}\right)\frac{v_j}{c}
-\sum_{k=1}^{3}\frac{v_k}{c}\left(\delta_{kj}-\frac{v_kv_j}{c^2}\right)=0
\quad\forall 1\leq j\leq 3,
\end{multline*}
where $\{g^{ij}\}_{0\leq i,j\leq 3}$ is the contravariant
pseudo-metric tensor of the four-dimensional space-time, defined by
\er{hoyuiouigyfghgjh3478zzrrZZffGGjkkj}. So,
\begin{equation}\label{fgjfjhgghhgjghjhjkkkkgjghghuiiiuujhjh}
\sum_{k=0}^{3}g^{ik}g_{kj}=\begin{cases}
1\quad\text{if}\quad i=j\\
0\quad\text{if}\quad i\neq j
\end{cases}\quad\quad\forall\, i,j=0,1,2,3.
\end{equation}
Therefore, by comparing \er{fgjfjhgghhgjghjhjkkkkgjghghuiiiuujhjh}
and \er{fgjfjhgghhgjghjhjkkkkgjghghuiiiu} we deduce that
$\{g_{ij}\}_{i,j=0,1,2,3}$ is a two times covariant tensor on the
group $\mathcal{S}_0$, and moreover, this tensor is symmetric. We
call the tensor $\{g_{ij}\}_{0\leq i,j\leq 3}$ covariant
pseudo-metric tensor of the four-dimensional space-time. Using
\er{fgjfjhgghhgjghjhjkkkkgjghghuiiiuujhjh} we also obtain that the
pseudo-metric tensors $\{g_{ij}\}_{i,j=0,1,2,3}$ and
$\{g^{ij}\}_{0\leq i,j\leq 3}$ are non-degenerate and they are
inverse of each other. Moreover, it can be easily calculated that if
we consider the $4\times 4$-matrix:
\begin{equation}\label{fgjfjhgghhgjghjhjkkkkgjghghuiiiuujhjhjklj}
G=\{g_{ij}\}_{0\leq i,j\leq 3},
\end{equation}
then
\begin{equation}\label{fgjfjhgghhgjghjhjkkkkgjghghuiiiuujhjh1}
\text{det}\,G=-1.
\end{equation}

Thus, with the covariant and contravariant pseudo-metric tensors we
can lower and lift indexes of arbitrary tensors. In particular given
a four-covector $(a_0,a_1,a_2,a_3)$ and a four-vector
$(b^0,b^1,b^2,b^3)$ on the group $\mathcal{S}_0$ we can define the
corresponding lifted four-vector $(a^0,a^1,a^2,a^3)$ and the
corresponded lowered four-covector $(b_0,b_1,b_2,b_3)$ by
\begin{equation}\label{fgjfjhgghhgjghjhjkkkkgjghghuiiiulkkjKK}
(a^0,a^1,a^2,a^3):=\Big\{\sum_{k=0}^{3}g^{mk}a_{k}\Big\}_{m=0,1,2,3}\quad\text{and}\quad
(b_0,b_1,b_2,b_3):=\Big\{\sum_{k=0}^{3}g_{mk}b^{k}\Big\}_{m=0,1,2,3}
\end{equation}
Then by \er{hoyuiouigyfghgjh3478zzrrZZffGGjkkj},
\er{hoyuiouigyfg3478zzrrZZffGGhhjhj} and
\er{fgjfjhgghhgjghjhjkkkkgjghghuiiiulkkjKK} we have:
\begin{equation}\label{fgjfjhgghhgjghjhjkkkkgjghghuiiiulkkjKKyuyyu}
a^0=a_{0}+\sum_{k=1}^{3}\frac{1}{c}v_k a_{k}\quad\text{and}\quad
a^m=-a_{m}+\frac{1}{c}\bigg(a_{0}+\sum_{k=1}^{3}\frac{1}{c}v_k
a_{k}\bigg)v_m\quad\quad\forall m=1,2,3,
\end{equation}
and
\begin{multline}\label{fgjfjhgghhgjghjhjkkkkgjghghuiiiulkkjKKyuyyuuuui}
b_0=\left(1-\frac{|\vec
v|^2}{c^2}\right)b^{0}+\sum_{k=1}^{3}\frac{1}{c}v_k
b^{k}=b^0-\sum_{k=1}^{3}\frac{1}{c}v_k\left(-
b^{k}+\frac{1}{c}b^0v_k\right)\\ \quad\text{and}\quad
b_m=-b^{m}+\frac{1}{c} b^{0}v_m\quad\quad\forall m=1,2,3.
\end{multline}
We also can rewrite \er{fgjfjhgghhgjghjhjkkkkgjghghuiiiulkkjKKyuyyu}
and \er{fgjfjhgghhgjghjhjkkkkgjghghuiiiulkkjKKyuyyuuuui} as:
\begin{equation}\label{fgjfjhgghhgjghjhjkkkkgjghghuiiiulkkjKKyuyyuioui}
a^0=a_{0}+\sum_{k=1}^{3}\frac{1}{c}v_k a_{k}\quad\text{and}\quad
a^m=-a_{m}+\frac{1}{c}a^0v_m\quad\quad\forall m=1,2,3,
\end{equation}
and
\begin{equation}\label{fgjfjhgghhgjghjhjkkkkgjghghuiiiulkkjKKyuyyuuuuiiuii}
b_0=b^0-\sum_{k=1}^{3}\frac{1}{c}v_kb_k \quad\text{and}\quad
b_m=-b^{m}+\frac{1}{c} b^{0}v_m\quad\quad\forall m=1,2,3.
\end{equation}
In particular, if we consider the scalar field $\Lambda$ on the
group $\mathcal{S}_0$ defined by:
\begin{equation}\label{fgjfjhgghhgjghjhjkkkkgjghghuiiiulkkjKKyuyyu0io}
\Lambda:=b^0a_0+\sum_{k=1}^{3}b^ka_k
\end{equation}
then by inserting
\er{fgjfjhgghhgjghjhjkkkkgjghghuiiiulkkjKKyuyyuioui} and
\er{fgjfjhgghhgjghjhjkkkkgjghghuiiiulkkjKKyuyyuuuuiiuii} into
\er{fgjfjhgghhgjghjhjkkkkgjghghuiiiulkkjKKyuyyu0io} we deduce:
\begin{equation}\label{fgjfjhgghhgjghjhjkkkkgjghghuiiiulkkjKKyuyyu0ioioio}
\Lambda=b^0\left(a^0-\sum_{k=1}^{3}\frac{1}{c}v_ka_k\right)+\sum_{k=1}^{3}\left(-b_{k}+\frac{1}{c}b^0v_k\right)a_k=b^0a^0-\sum_{k=1}^{3}b_{k}a_k.
\end{equation}
So,
\begin{equation}\label{fgjfjhgghhgjghjhjkkkkgjghghuiiiulkkjKKyuyyu0ioioiogghghgh}
\Lambda=b^0a_0+\sum_{k=1}^{3}b^ka_k=b^0a^0-\sum_{k=1}^{3}b_{k}a_k.
\end{equation}

Next, if for every speed-like vector field $\vec u$ we consider the
four-vector field $(u^0,u^1,u^2,u^3)$ defined by
\er{fgjfjhgghhgjghjhjijhojihjhjjijhjjjjjuii} as:
\begin{equation}\label{fgjfjhgghhgjghjhjijhojihjhjjijhjjjjjuiikkj}
(u^0,u^1,u^2,u^3):=\left(1,\frac{1}{c}\vec
u\right)\quad\text{where}\quad
u^0=1\;\;\text{and}\;\;(u^1,u^2,u^3)=\frac{1}{c}\vec
u\in\mathbb{R}^3,
\end{equation}
then, by \er{fgjfjhgghhgjghjhjkkkkgjghghuiiiulkkjKKyuyyuuuuiiuii}
the corresponding lowered four-covector field $(u_0,u_1,u_2,u_3)$
satisfies:
\begin{multline}\label{fgjfjhgghhgjghjhjijhojihjhjjijhjjjjjuiikkjjnj}
(u_0,u_1,u_2,u_3):=\left(1+\frac{1}{c^2}\left(\vec u-\vec
v\right)\cdot\vec v,-\frac{1}{c}\left(\vec u-\vec
v\right)\right)\quad\text{where}\quad\\
u_0=1+\frac{1}{c^2}\left(\vec u-\vec v\right)\cdot\vec
v\;\;\text{and}\;\;(u_1,u_2,u_3)=-\frac{1}{c}\left(\vec u-\vec
v\right)\in\mathbb{R}^3.
\end{multline}
Moreover, in the case where $(u^0,u^1,u^2,u^3)$  is a
four-dimensional speed, we call the corresponding lowered
four-covector field $(u_0,u_1,u_2,u_3)$ by the name four-dimensional
cospeed. In particular, if we consider the four-dimensional
gravitational potential $(v^0,v^1,v^2,v^3)$ defined by
\er{fgjfjhgghhgjghjhjijhojihjhjjijhjjjjjuiijjjk}:
\begin{equation}\label{fgjfjhgghhgjghjhjijhojihjhjjijhjjjjjuiijjjkhjhjhj}
(v^0,v^1,v^2,v^3):=\left(1,\frac{1}{c}\vec
v\right)\quad\text{where}\quad
v^0=1\;\;\text{and}\;\;(v^1,v^2,v^3)=\frac{1}{c}\vec v,
\end{equation}
then by \er{fgjfjhgghhgjghjhjijhojihjhjjijhjjjjjuiikkjjnj} we obtain
that the corresponding lowered four-covector field
$(v_0,v_1,v_2,v_3)$, that we call the four-covector of gravitational
potential, satisfies:
\begin{equation}\label{fgjfjhgghhgjghjhjijhojihjhjjijhjjjjjuiikkjjnj1}
(v_0,v_1,v_2,v_3):=\left(1,0\right)\quad\text{where}\quad
v_0=1\;\;\text{and}\;\;(v_1,v_2,v_3)=0:=(0,0,0).
\end{equation}
Note that the four-covector of gravitational potential, defined by
\er{fgjfjhgghhgjghjhjijhojihjhjjijhjjjjjuiikkjjnj1} coincides with
the four-covector defined by
\er{fgjfjhgghhgjghjhjkkkkgjghghuiiiulkkjlkklplikklkjljghhgghk} as
the gradient of the scalar of global time. Moreover, by
\er{fgjfjhgghhgjghjhjijhojihjhjjijhjjjjjuiijjjkhjhjhj} and
\er{fgjfjhgghhgjghjhjijhojihjhjjijhjjjjjuiikkjjnj1} we clearly have:
\begin{equation}\label{fgjfjhgghhgjghjhjijhojihjhjjijhjjjjjuiijjjkhjhjhjuiiuuuyu}
c^2\left(\sum_{j=0}^{n}\sum_{k=0}^{n}\,g^{jk}\frac{\partial\tau}{\partial
x^j}\frac{\partial\tau}{\partial
x^k}\right)=\sum_{j=0}^{n}\sum_{k=0}^{n}g^{jk}v_jv_k=\sum_{j=0}^{n}\sum_{k=0}^{n}g_{jk}v^jv^k=\sum_{j=0}^{n}v^jv_j=1,
\end{equation}
where $\tau$ is the scalar of the global time on the group
$\mathcal{S}_0$, defined by \er{uguyytfddddgghkjhhjuuiui}.

More generally, if for every speed-like vector field $\vec u$ and
every proper scalar field $\sigma$  we consider the four-vector
field $(b^0,b^1,b^2,b^3)$ on the group $\mathcal{S}_0$ defined by
\er{fgjfjhgghhgjghjhjijhojihjhjjijhjjjjjuiijjjkl} as:
\begin{equation}\label{fgjfjhgghhgjghjhjijhojihjhjjijhjjjjjuiijjjklhhh}
(b^0,b^1,b^2,b^3):=\left(\sigma,\frac{\sigma}{c}\vec
u\right)\quad\text{where}\quad
b^0=\sigma\;\;\text{and}\;\;(b^1,b^2,b^3)=\frac{\sigma}{c}\vec u,
\end{equation}
then by \er{fgjfjhgghhgjghjhjkkkkgjghghuiiiulkkjKKyuyyuuuuiiuii} the
corresponding lowered four-covector field $(b_0,b_1,b_2,b_3)$
satisfies:
\begin{multline}\label{fgjfjhgghhgjghjhjijhojihjhjjijhjjjjjuiikkjjnjjbj}
(b_0,b_1,b_2,b_3):=\left(\sigma\left(1+\frac{1}{c^2}\left(\vec
u-\vec v\right)\cdot\vec v\right),-\frac{\sigma}{c}\left(\vec u-\vec
v\right)\right)\quad\text{where}\quad\\
b_0=\sigma\left(1+\frac{1}{c^2}\left(\vec u-\vec v\right)\cdot\vec
v\right)\;\;\text{and}\;\;(b_1,b_2,b_3)=-\frac{\sigma}{c}\left(\vec
u-\vec v\right).
\end{multline}
In particular, if we consider the field of four-vector of the moment
of a particle $(p^0,p^1,p^2,p^3)$ defined by
\er{fgjfjhgghhgjghjhjijhojihjhjjijhjjjjjuiijjjklihh} as
\begin{equation}\label{fgjfjhgghhgjghjhjijhojihjhjjijhjjjjjuiijjjklihhjjj}
(p^0,p^1,p^2,p^3):=\left(m,\frac{1}{c}(m\vec
u)\right)\quad\text{where}\quad
p^0=m\;\;\text{and}\;\;(p^1,p^2,p^3)=\frac{1}{c}(m\vec u),
\end{equation}
where $m$ is the mass of the particle and $\vec u$ is the velocity
of the particle, then the corresponding lowered four-covector field
$(p_0,p_1,p_2,p_3)$, which we call the four-covector of momentum,
satisfies:
\begin{multline}\label{fgjfjhgghhgjghjhjijhojihjhjjijhjjjjjuiikkjjnjjbjhjh}
(p_0,p_1,p_2,p_3):=\left(m\left(1+\frac{1}{c^2}\left(\vec u-\vec
v\right)\cdot\vec v\right),-\frac{m}{c}\left(\vec u-\vec
v\right)\right)\quad\text{where}\quad\\
p_0=m\left(1+\frac{1}{c^2}\left(\vec u-\vec v\right)\cdot\vec
v\right)\;\;\text{and}\;\;(p_1,p_2,p_3)=-\frac{m}{c}\left(\vec
u-\vec v\right).
\end{multline}
In particular, the scalar field $J_0$ defined by
\begin{equation}\label{fgjfjhgghhgjghjhjkkkkgjghghuiiiulkkjKKyuyyu0ioioiogghghghgghghhg}
J_0:=-\frac{c^2}{2m}\left(p^0p_0+\sum_{k=1}^{3}p^kp_k\right),
\end{equation}
by \er{fgjfjhgghhgjghjhjkkkkgjghghuiiiulkkjKKyuyyu0ioioiogghghgh},
\er{fgjfjhgghhgjghjhjijhojihjhjjijhjjjjjuiijjjklihhjjj} and
\er{fgjfjhgghhgjghjhjijhojihjhjjijhjjjjjuiikkjjnjjbjhjh} satisfies:
\begin{equation}\label{fgjfjhgghhgjghjhjkkkkgjghghuiiiulkkjKKyuyyu0ioioiogghghghgghgh}
J_0=\frac{mc^2}{2}\left(\frac{1}{c^2}\left|\vec u-\vec
v\right|^2-1\right)=\frac{m}{2}\left|\vec u-\vec
v\right|^2-\frac{mc^2}{2}.
\end{equation}
Moreover, if we consider the  four-dimensional electric current
$(j^0,j^1,j^2,j^3)$ defined by
\er{fgjfjhgghhgjghjhjijhojihjhjjijhjjjjjuiijjjklihhojjjo} as
\begin{equation}\label{fgjfjhgghhgjghjhjijhojihjhjjijhjjjjjuiijjjklihhojjjokjkj}
(j^0,j^1,j^2,j^3):=\left(\rho,\frac{1}{c}\,\vec
j\right)\quad\text{where}\quad
j^0=\rho\;\;\text{and}\;\;(j^1,j^2,j^3)=\frac{1}{c}\,\vec j,
\end{equation}
where $\rho$ is the electric charge density and $\vec j$ is the
electric current density, then the corresponding lowered
four-covector field $(j_0,j_1,j_2,j_3)$, which we call the
four-covector of current, satisfies:
\begin{multline}\label{fgjfjhgghhgjghjhjijhojihjhjjijhjjjjjuiikkjjnjjbjhjhiyuyug}
(j_0,j_1,j_2,j_3):=\left(\rho+\frac{1}{c^2}\left(\vec j-\rho\vec
v\right)\cdot\vec v,-\frac{1}{c}\left(\vec j-\rho\vec
v\right)\right)\quad\text{where}\quad\\
j_0=\rho+\frac{1}{c^2}\left(\vec j-\rho\vec v\right)\cdot\vec
v\;\;\text{and}\;\;(j_1,j_2,j_3)=-\frac{1}{c}\left(\vec j-\rho\vec
v\right).
\end{multline}
Finally, if $\Psi$ is the scalar electromagnetic potential and $\vec
A$ is the vector electromagnetic potential and we consider the
four-\underline{covector} field of four dimensional electromagnetic
potential $(A_0,A_1,A_2,A_3)$, defined by
\er{fgjfjhgghhgjghjhjijhojihjhjjijhjjjljljpk} as:
\begin{equation}\label{fgjfjhgghhgjghjhjijhojihjhjjijhjjjljljpkikhj}
(A_0,A_1,A_2,A_3)=(\Psi,-\vec A)\quad\text{where}\quad
A_0=\Psi\;\;\text{and}\;\;(A_1,A_2,A_3)=-\vec A,
\end{equation}
then by inserting \er{fgjfjhgghhgjghjhjijhojihjhjjijhjjjljljpkikhj}
into \er{fgjfjhgghhgjghjhjkkkkgjghghuiiiulkkjKKyuyyuioui} we deduce
that the corresponding lifted four-vector field $(A^0,A^1,A^2,A^3)$,
which we call the four-vector of electromagnetic potential,
satisfies:
\begin{equation}\label{fgjfjhgghhgjghjhjkkkkgjghghuiiiulkkjKKyuyyuiouigjguuuiui}
A^0=\Psi-\frac{1}{c}\vec v\cdot\vec
A\quad\text{and}\quad(A^1,A^2,A^3)=\vec
A+\frac{1}{c}\left(\Psi-\frac{1}{c}\vec v\cdot\vec A\right)\vec v.
\end{equation}
On the other hand, the proper scalar electromagnetic potential
$\Psi_0$ was defined by
\er{vhfffngghhjghhgPPNghghghutghffugghjhjkjjkl} as:
\begin{equation}\label{vhfffngghhjghhgPPNghghghutghffugghjhjkjjklhjgh}
\Psi_0:=\Psi-\frac{1}{c}\vec A\cdot\vec v.
\end{equation}
Thus we rewrite
\er{fgjfjhgghhgjghjhjkkkkgjghghuiiiulkkjKKyuyyuiouigjguuuiui} as:
\begin{equation}\label{fgjfjhgghhgjghjhjkkkkgjghghuiiiulkkjKKyuyyuiouigjguuuiui1}
A^0=\Psi_0\quad\text{and}\quad(A^1,A^2,A^3)=\vec
A+\frac{1}{c}\Psi_0\vec v.
\end{equation}




Next given a two times covariant tensor $\{c_{mn}\}_{m,n=0,1,2,3}$
%and a two times contravariant tensor $\{d^{mn}\}_{m,n=0,1,2,3}$
on the group $\mathcal{S}_0$ by
\er{fgjfjhgghhgjghjhjkkkkgjghghuiiiulkkjlkkl} we consider two times
contravariant tensor
%and two times covariant tensors
on $\mathcal{S}_0$:
$\{c^{mn}\}_{m,n=0,1,2,3}$
%and $\{d_{mn}\}_{m,n=0,1,2,3}$
defined
by:
\begin{equation}\label{fgjfjhgghhgjghjhjkkkkgjghghuiiiulkkjlkklKK}
c^{mn}:=\sum_{k=0}^{3}\sum_{j=0}^{3}g^{mj}g^{nk}c_{jk}
%\quad\text{and}\quadd_{mn}:=\sum_{j=0}^{3}\sum_{k=0}^{3}g_{mj}g_{nk}d^{jk}
\quad\quad\forall\,
m,n=0,1,2,3.
\end{equation}
We rewrite \er{fgjfjhgghhgjghjhjkkkkgjghghuiiiulkkjlkklKK} as:
\begin{multline}\label{khjhhkfgjfjhgghhgjghjhjkkkkgjghghuiiiulkkjlkklKKgfg}
c^{mn}=g^{m0}g^{n0}c_{00}+\sum_{k=1}^{3}g^{m0}g^{nk}c_{0k}+\sum_{j=1}^{3}g^{mj}g^{n0}c_{j0}+\sum_{k=1}^{3}\sum_{j=1}^{3}g^{mj}g^{nk}c_{jk}
%\quad\text{and}\quad\quad\\ d_{mn}=g_{m0}g_{n0}d^{00}+\sum_{k=1}^{3}g_{m0}g_{nk}d^{0k}+\sum_{j=1}^{3}g_{mj}g_{n0}d^{j0}+\sum_{k=1}^{3}\sum_{j=1}^{3}g_{mj}g_{nk}d^{jk}
\quad\forall\,
m,n=0,1,2,3.
\end{multline}
In particular, by inserting \er{hoyuiouigyfghgjh3478zzrrZZffGGjkkj}
and \er{hoyuiouigyfg3478zzrrZZffGGhhjhj} into
\er{khjhhkfgjfjhgghhgjghjhjkkkkgjghghuiiiulkkjlkklKKgfg} we deduce:
\begin{equation}\label{khjhhkfgjfjhgghhgjghjhjkkkkgjghghuiiiulkkjlkklKKgfgjhjjghgjh}
\begin{cases}
c^{00}=c_{00}+\sum_{k=1}^{3}\frac{v_k}{c}c_{0k}+\sum_{j=1}^{3}\frac{v_j}{c}c_{j0}+\sum_{k=1}^{3}\sum_{j=1}^{3}\frac{v_j}{c}\frac{v_k}{c}c_{jk}
\\
c^{m0}=
%\frac{v_m}{c}c_{00}+\sum_{k=1}^{3}\frac{v_m}{c}\frac{v_k}{c}c_{0k}+\sum_{j=1}^{3}\frac{v_m}{c}\frac{v_j}{c}c_{j0}+\sum_{k=1}^{3}\sum_{j=1}^{3}\frac{v_k}{c}\frac{v_m}{c}\frac{v_j}{c}c_{jk}
\frac{v_m}{c}c^{00}-c_{m0}-\sum_{k=1}^{3}\frac{v_k}{c}c_{mk}
\quad\forall\, m=1,2,3,
\\
c^{0n}=\frac{v_n}{c}c^{00}
%\frac{v_n}{c}c_{00}+\sum_{k=1}^{3}\frac{v_n}{c}\frac{v_k}{c}c_{0k}+\sum_{j=1}^{3}\frac{v_j}{c}\frac{v_n}{c}c_{j0}+\sum_{k=1}^{3}\sum_{j=1}^{3}\frac{v_j}{c}\frac{v_n}{c}\frac{v_k}{c}c_{jk}
-c_{0n}
-\sum_{j=1}^{3}\frac{v_j}{c}c_{jn} \quad\forall\, n=1,2,3,
\\
c^{mn}=\frac{v_m}{c}\frac{v_n}{c}c^{00}
%\frac{v_m}{c}\frac{v_n}{c}c_{00}+\sum_{k=1}^{3}\frac{v_m}{c}\frac{v_n}{c}\frac{v_k}{c}c_{0k}+\sum_{j=1}^{3}\frac{v_n}{c}\frac{v_m}{c}\frac{v_j}{c}c_{j0}+\sum_{k=1}^{3}\sum_{j=1}^{3}\frac{v_m}{c}\frac{v_n}{c}\frac{v_j}{c}\frac{v_k}{c}c_{jk}
%-\sum_{k=1}^{3}\sum_{j=1}^{3}\frac{v_m}{c}\frac{v_n}{c}\frac{v_j}{c}\frac{v_k}{c}c_{jk}
-\sum_{k=1}^{3}\frac{v_n}{c}\frac{v_k}{c}c_{mk}-\sum_{j=1}^{3}\frac{v_m}{c}\frac{v_j}{c}c_{jn}
-\frac{v_m}{c}c_{0n}-\frac{v_n}{c}c_{m0}+c_{mn}\quad\forall\,
m,n=1,2,3.
\end{cases}
\end{equation}
We rewrite
\er{khjhhkfgjfjhgghhgjghjhjkkkkgjghghuiiiulkkjlkklKKgfgjhjjghgjh}
as:
\begin{equation}\label{khjhhkfgjfjhgghhgjghjhjkkkkgjghghuiiiulkkjlkklKKgfgjhjjghgjh1}
\begin{cases}
c^{00}=c_{00}+\sum_{k=1}^{3}\frac{v_k}{c}c_{0k}+\sum_{j=1}^{3}\frac{v_j}{c}c_{j0}+\sum_{k=1}^{3}\sum_{j=1}^{3}\frac{v_j}{c}\frac{v_k}{c}c_{jk}
\\
c^{m0}=
%\frac{v_m}{c}c_{00}+\sum_{k=1}^{3}\frac{v_m}{c}\frac{v_k}{c}c_{0k}+\sum_{j=1}^{3}\frac{v_m}{c}\frac{v_j}{c}c_{j0}+\sum_{k=1}^{3}\sum_{j=1}^{3}\frac{v_k}{c}\frac{v_m}{c}\frac{v_j}{c}c_{jk}
\frac{v_m}{c}c^{00}-c_{m0}-\sum_{k=1}^{3}\frac{v_k}{c}c_{mk}
\quad\forall\, m=1,2,3,
\\
c^{0n}=\frac{v_n}{c}c^{00}
%\frac{v_n}{c}c_{00}+\sum_{k=1}^{3}\frac{v_n}{c}\frac{v_k}{c}c_{0k}+\sum_{j=1}^{3}\frac{v_j}{c}\frac{v_n}{c}c_{j0}+\sum_{k=1}^{3}\sum_{j=1}^{3}\frac{v_j}{c}\frac{v_n}{c}\frac{v_k}{c}c_{jk}
-c_{0n} -\sum_{j=1}^{3}\frac{v_j}{c}c_{jn} \quad\forall\, n=1,2,3,
\\
c^{mn}=\frac{v_m}{c}c^{0n}+\frac{v_n}{c}c^{m0}-\frac{v_m}{c}\frac{v_n}{c}c^{00}
%\frac{v_m}{c}\frac{v_n}{c}c_{00}+\sum_{k=1}^{3}\frac{v_m}{c}\frac{v_n}{c}\frac{v_k}{c}c_{0k}+\sum_{j=1}^{3}\frac{v_n}{c}\frac{v_m}{c}\frac{v_j}{c}c_{j0}+\sum_{k=1}^{3}\sum_{j=1}^{3}\frac{v_m}{c}\frac{v_n}{c}\frac{v_j}{c}\frac{v_k}{c}c_{jk}
%-\sum_{k=1}^{3}\sum_{j=1}^{3}\frac{v_m}{c}\frac{v_n}{c}\frac{v_j}{c}\frac{v_k}{c}c_{jk}
%%+2\frac{v_m}{c}\frac{v_n}{c}c^{00}-\sum_{k=1}^{3}\frac{v_n}{c}\frac{v_k}{c}c_{mk}-\sum_{j=1}^{3}\frac{v_m}{c}\frac{v_j}{c}c_{jn}-\frac{v_m}{c}c_{0n}-\frac{v_n}{c}c_{m0}
+c_{mn}\quad\forall\,
m,n=1,2,3.
\end{cases}
\end{equation}
In particular if the tensor $\{c_{mn}\}_{m,n=0,1,2,3}$ is
antisymmetric, i.e. $c_{mn}=-c_{nm}\;\forall \,m,n=0,1,2,3$, then we
simplify
\er{khjhhkfgjfjhgghhgjghjhjkkkkgjghghuiiiulkkjlkklKKgfgjhjjghgjh1}
as
\begin{equation}\label{khjhhkfgjfjhgghhgjghjhjkkkkgjghghuiiiulkkjlkklKKgfgjhjjghgjhhjhj}
\begin{cases}
c^{00}=0
\\
c^{mm}=0 \quad\forall\, m=1,2,3,
\\
c^{0m}=-c^{m0}=
%\frac{v_m}{c}c_{00}+\sum_{k=1}^{3}\frac{v_m}{c}\frac{v_k}{c}c_{0k}+\sum_{j=1}^{3}\frac{v_m}{c}\frac{v_j}{c}c_{j0}+\sum_{k=1}^{3}\sum_{j=1}^{3}\frac{v_k}{c}\frac{v_m}{c}\frac{v_j}{c}c_{jk}
-c_{0m}+\sum_{k=1}^{3}\frac{v_k}{c}c_{mk} \quad\forall\, m=1,2,3,
\\
%\frac{v_n}{c}c_{00}+\sum_{k=1}^{3}\frac{v_n}{c}\frac{v_k}{c}c_{0k}+\sum_{j=1}^{3}\frac{v_j}{c}\frac{v_n}{c}c_{j0}+\sum_{k=1}^{3}\sum_{j=1}^{3}\frac{v_j}{c}\frac{v_n}{c}\frac{v_k}{c}c_{jk}
%%c^{0n}=-c_{0n} -\sum_{j=1}^{3}\frac{v_j}{c}c_{jn} \quad\forall\, n=1,2,3,\\
c^{mn}=\frac{v_m}{c}c^{0n}-\frac{v_n}{c}c^{0m}
%\frac{v_m}{c}\frac{v_n}{c}c_{00}+\sum_{k=1}^{3}\frac{v_m}{c}\frac{v_n}{c}\frac{v_k}{c}c_{0k}+\sum_{j=1}^{3}\frac{v_n}{c}\frac{v_m}{c}\frac{v_j}{c}c_{j0}+\sum_{k=1}^{3}\sum_{j=1}^{3}\frac{v_m}{c}\frac{v_n}{c}\frac{v_j}{c}\frac{v_k}{c}c_{jk}
%-\sum_{k=1}^{3}\sum_{j=1}^{3}\frac{v_m}{c}\frac{v_n}{c}\frac{v_j}{c}\frac{v_k}{c}c_{jk}
%%+2\frac{v_m}{c}\frac{v_n}{c}c^{00}-\sum_{k=1}^{3}\frac{v_n}{c}\frac{v_k}{c}c_{mk}-\sum_{j=1}^{3}\frac{v_m}{c}\frac{v_j}{c}c_{jn}-\frac{v_m}{c}c_{0n}-\frac{v_n}{c}c_{m0}
+c_{mn}\quad\forall\, m,n=1,2,3.
\end{cases}
\end{equation}
In particular, if $\{F_{ij}\}_{0\leq i,j\leq 3}$ is the
antisymmetric two times covariant tensor field of the
electromagnetic field on the group $\mathcal{S}_0$, which by
\er{huohuioy89gjjhjffffff3478zzrrZZZhjhhjhhjjhhffGGhjjhiuuijkjjk}
satisfies:
\begin{equation}\label{huohuioy89gjjhjffffff3478zzrrZZZhjhhjhhjjhhffGGhjjhiuuijkjjkihjhjh}
\begin{cases}
F_{00}=0\\ F_{0j}=-F_{j0}=E_j\quad\quad\forall\, j=1,2,3\\
F_{jj}=0\quad\quad\forall\, j=1,2,3
\\
F_{12}=-F_{21}=-B_3
%\frac{\partial (-A_1)}{\partial x^2}-\frac{\partial(-A_2)}{\partial x^1}
\\
F_{13}=-F_{31}=B_2
%\frac{\partial (-A_1)}{\partial x^3}-\frac{\partial(-A_3)}{\partial x^1}
\\
F_{23}=-F_{32}=-B_1\,,
%\frac{\partial (-A_2)}{\partial x^3}-\frac{\partial (-A_3)}{\partial x^2}
\end{cases}
\end{equation}
where  $\vec E:=(E_1,E_2,E_3)$ and $\vec B:=(B_1,B_2,B_3)$, then by
inserting
\er{huohuioy89gjjhjffffff3478zzrrZZZhjhhjhhjjhhffGGhjjhiuuijkjjkihjhjh}
into
\er{khjhhkfgjfjhgghhgjghjhjkkkkgjghghuiiiulkkjlkklKKgfgjhjjghgjhhjhj}
we deduce:
\begin{equation}\label{khjhhkfgjfjhgghhgjghjhjkkkkgjghghuiiiulkkjlkklKKgfgjhjjghgjhhjhjhhhhh}
\begin{cases}
F^{00}=0
\\
F^{jj}=0 \quad\forall\, j=1,2,3,
\\
%\frac{v_m}{c}c_{00}+\sum_{k=1}^{3}\frac{v_m}{c}\frac{v_k}{c}c_{0k}+\sum_{j=1}^{3}\frac{v_m}{c}\frac{v_j}{c}c_{j0}+\sum_{k=1}^{3}\sum_{j=1}^{3}\frac{v_k}{c}\frac{v_m}{c}\frac{v_j}{c}c_{jk}
F^{01}=-F^{10}=-F_{01}+\frac{v_2}{c}F_{12}+\frac{v_3}{c}F_{13}=-\left(E_1+\frac{1}{c}\left(v_2B_3-v_3B_2\right)\right)\\
F^{02}=-F^{20}=-F_{02}+\frac{v_1}{c}F_{21}+\frac{v_3}{c}F_{23}=-\left(E_2+\frac{1}{c}\left(v_3B_1-v_1B_3\right)\right)\\
F^{03}=-F^{30}=-F_{03}+\frac{v_1}{c}F_{31}+\frac{v_2}{c}F_{32}=-\left(E_3+\frac{1}{c}\left(v_1B_2-v_2B_1\right)\right)
\\
%\frac{v_n}{c}c_{00}+\sum_{k=1}^{3}\frac{v_n}{c}\frac{v_k}{c}c_{0k}+\sum_{j=1}^{3}\frac{v_j}{c}\frac{v_n}{c}c_{j0}+\sum_{k=1}^{3}\sum_{j=1}^{3}\frac{v_j}{c}\frac{v_n}{c}\frac{v_k}{c}c_{jk}
%%c^{0n}=-c_{0n} -\sum_{j=1}^{3}\frac{v_j}{c}c_{jn} \quad\forall\, n=1,2,3,\\

%\frac{v_m}{c}\frac{v_n}{c}c_{00}+\sum_{k=1}^{3}\frac{v_m}{c}\frac{v_n}{c}\frac{v_k}{c}c_{0k}+\sum_{j=1}^{3}\frac{v_n}{c}\frac{v_m}{c}\frac{v_j}{c}c_{j0}+\sum_{k=1}^{3}\sum_{j=1}^{3}\frac{v_m}{c}\frac{v_n}{c}\frac{v_j}{c}\frac{v_k}{c}c_{jk}
%-\sum_{k=1}^{3}\sum_{j=1}^{3}\frac{v_m}{c}\frac{v_n}{c}\frac{v_j}{c}\frac{v_k}{c}c_{jk}
%%+2\frac{v_m}{c}\frac{v_n}{c}c^{00}-\sum_{k=1}^{3}\frac{v_n}{c}\frac{v_k}{c}c_{mk}-\sum_{j=1}^{3}\frac{v_m}{c}\frac{v_j}{c}c_{jn}-\frac{v_m}{c}c_{0n}-\frac{v_n}{c}c_{m0}
F^{12}=-F^{21}=\frac{v_1}{c}F^{02}-\frac{v_2}{c}F^{01}+F_{12}=-\left(B_3+\frac{1}{c}\left(v_1F^{20}-v_2F^{10}\right)\right)\\
F^{13}=-F^{31}=\frac{v_1}{c}F^{03}-\frac{v_3}{c}F^{01}+F_{13}=B_2+\frac{1}{c}\left(v_3F^{10}-v_1F^{30}\right)\\
F^{23}=-F^{32}=\frac{v_2}{c}F^{03}-\frac{v_3}{c}F^{02}+F_{23}=-\left(B_1+\frac{1}{c}\left(v_2F^{30}-v_3F^{20}\right)\right).
\end{cases}
\end{equation}
Thus, as before in \er{MaxVacFullPPN}, denoting:
\begin{equation}\label{MaxVacFullPPNhjjgh}
\begin{cases}
\vec D:=\vec E+\frac{1}{c}\,\vec v\times
\vec B\\
\vec H:=\vec B+\frac{1}{c}\,\vec v\times \vec D,
\end{cases}
\end{equation}
and denoting $\vec D:=(D_1,D_2,D_3)$ and $\vec H:=(H_1,H_2,H_3)$ we
rewrite
\er{khjhhkfgjfjhgghhgjghjhjkkkkgjghghuiiiulkkjlkklKKgfgjhjjghgjhhjhjhhhhh}
as:
\begin{equation}\label{khjhhkfgjfjhgghhgjghjhjkkkkgjghghuiiiulkkjlkklKKgfgjhjjghgjhhjhjhhhhhghhg}
\begin{cases}
F^{00}=0
\\
%\frac{v_m}{c}c_{00}+\sum_{k=1}^{3}\frac{v_m}{c}\frac{v_k}{c}c_{0k}+\sum_{j=1}^{3}\frac{v_m}{c}\frac{v_j}{c}c_{j0}+\sum_{k=1}^{3}\sum_{j=1}^{3}\frac{v_k}{c}\frac{v_m}{c}\frac{v_j}{c}c_{jk}
F^{0j}=-F^{j0}=-D_j\quad\forall\, j=1,2,3,\\
F^{jj}=0 \quad\forall\, j=1,2,3,\\
%F^{01}=-F^{10}=-D_1\\
%-\left(E_1+\frac{1}{c}\left(v_2B_3-v_3B_2\right)\right)\\
%F^{02}=-F^{20}=-D_2\\
%-\left(E_2+\frac{1}{c}\left(v_3B_1-v_1B_3\right)\right)\\
%F^{03}=-F^{30}=-D_3
%-\left(E_3+\frac{1}{c}\left(v_1B_2-v_2B_1\right)\right)
%\\
%\frac{v_n}{c}c_{00}+\sum_{k=1}^{3}\frac{v_n}{c}\frac{v_k}{c}c_{0k}+\sum_{j=1}^{3}\frac{v_j}{c}\frac{v_n}{c}c_{j0}+\sum_{k=1}^{3}\sum_{j=1}^{3}\frac{v_j}{c}\frac{v_n}{c}\frac{v_k}{c}c_{jk}
%%c^{0n}=-c_{0n} -\sum_{j=1}^{3}\frac{v_j}{c}c_{jn} \quad\forall\, n=1,2,3,\\
%\frac{v_m}{c}\frac{v_n}{c}c_{00}+\sum_{k=1}^{3}\frac{v_m}{c}\frac{v_n}{c}\frac{v_k}{c}c_{0k}+\sum_{j=1}^{3}\frac{v_n}{c}\frac{v_m}{c}\frac{v_j}{c}c_{j0}+\sum_{k=1}^{3}\sum_{j=1}^{3}\frac{v_m}{c}\frac{v_n}{c}\frac{v_j}{c}\frac{v_k}{c}c_{jk}
%-\sum_{k=1}^{3}\sum_{j=1}^{3}\frac{v_m}{c}\frac{v_n}{c}\frac{v_j}{c}\frac{v_k}{c}c_{jk}
%%+2\frac{v_m}{c}\frac{v_n}{c}c^{00}-\sum_{k=1}^{3}\frac{v_n}{c}\frac{v_k}{c}c_{mk}-\sum_{j=1}^{3}\frac{v_m}{c}\frac{v_j}{c}c_{jn}-\frac{v_m}{c}c_{0n}-\frac{v_n}{c}c_{m0}
F^{12}=-F^{21}=-H_3\\
%-\left(B_3+\frac{1}{c}\left(v_1F^{20}-v_2F^{10}\right)\right)\\
F^{13}=-F^{31}=H_2\\
%B_2+\frac{1}{c}\left(v_3F^{10}-v_1F^{30}\right)\\
F^{23}=-F^{32}=-H_1.
%-\left(B_1+\frac{1}{c}\left(v_2F^{30}-v_3F^{20}\right)\right)
\end{cases}
\end{equation}
In particular, by
\er{huohuioy89gjjhjffffff3478zzrrZZZhjhhjhhjjhhffGGhjjhiuuijkjjkihjhjh}
and
\er{khjhhkfgjfjhgghhgjghjhjkkkkgjghghuiiiulkkjlkklKKgfgjhjjghgjhhjhjhhhhhghhg},
using \er{MaxVacFullPPNhjjgh} we deduce that the scalar field on the
group $\mathcal{S}_0$: $L_e$, defined as:
\begin{equation}\label{MaxVacFullPPNhjjghjjkjhh}
L_e:=\sum_{j=0}^{3}\sum_{k=0}^{3}F^{jk}F_{jk},
\end{equation}
satisfies
\begin{multline}\label{MaxVacFullPPNhjjghjjkjhh1}
L_e=F^{00}F_{00}+\sum_{k=1}^{3}F^{0k}F_{0k}+\sum_{j=1}^{3}F^{j0}F_{j0}+\sum_{j=1}^{3}\sum_{k=1}^{3}F^{jk}F_{jk}=-2\vec
E\cdot\vec D+2\vec B\cdot\vec H=\\-2\left(\left(\vec
D-\frac{1}{c}\,\vec v\times \vec B\right)\cdot\vec D-\vec
B\cdot\left(\vec B+\frac{1}{c}\,\vec v\times \vec
D\right)\right)=-2\left(|\vec D|^2-|\vec B|^2\right).
\end{multline}
\subsection{Maxwell equations in covariant formulation}
It is well known from Tensor Analysys that if $\{S^{ij}\}_{0\leq
i,j\leq 3}$ is the antisymmetric two times contravariant tensor and
if $\{\xi_{ij}\}_{0\leq i,j\leq 3}$ is a symmetric two times
covariant and non-degenerate tensor, both on the certain group
$\mathcal{S}$, then the four-component field $\{\theta_{k}\}_{0\leq
k\leq 3}$ defined by
\begin{equation}\label{MaxVacFullPPNhjjghjjkjhhoujii}
\theta_k:=\sum_{j=0}^{3}\frac{\partial S^{kj}}{\partial
x^j}+\sum_{j=0}^{3}\frac{S^{kj}}{\sqrt{|\text{det}\,\xi|}}\frac{\partial}{\partial
x^j}\left(\sqrt{|\text{det}\,\xi|}\right)\quad\quad\forall\,
k=0,1,2,3,
\end{equation}
is a four-vector on $\mathcal{S}$. Here $\xi$ is a $4\times
4$-matrix defined by:
\begin{equation}\label{fgjfjhgghhgjghjhjkkkkgjghghuiiiuujhjhjkljjhjhjji}
\xi=\{\xi_{ij}\}_{0\leq i,j\leq 3}.
\end{equation}



In particular, if we consider the $4\times 4$-matrix $G$ defined by
\er{fgjfjhgghhgjghjhjkkkkgjghghuiiiuujhjhjklj} as:
\begin{equation}\label{fgjfjhgghhgjghjhjkkkkgjghghuiiiuujhjhjklj1}
G=\{g_{ij}\}_{0\leq i,j\leq 3},
\end{equation}
that satisfies \er{fgjfjhgghhgjghjhjkkkkgjghghuiiiuujhjh1} in every
cartesian coordinate system, i.e.
\begin{equation}\label{fgjfjhgghhgjghjhjkkkkgjghghuiiiuujhjhhjjg}
\text{det}\,G=-1.
\end{equation}
then for the lifted contravariant tensor of the electromagnetic
field $\{F^{ij}\}_{0\leq i,j\leq 3}$ on the group $\mathcal{S}_0$,
considered in
\er{khjhhkfgjfjhgghhgjghjhjkkkkgjghghuiiiulkkjlkklKKgfgjhjjghgjhhjhjhhhhhghhg},
as in \er{MaxVacFullPPNhjjghjjkjhhoujii} we can define the
four-vector field:
\begin{equation}\label{MaxVacFullPPNhjjghjjkjhhoujiiikjjihjhiuiu}
\left\{\sum_{j=0}^{3}\frac{\partial F^{kj}}{\partial
x^j}+\sum_{j=0}^{3}\frac{F^{kj}}{\sqrt{|\text{det}\,G|}}\frac{\partial}{\partial
x^j}\left(\sqrt{|\text{det}\,G|}\right)\right\}_{0\leq k\leq
3}=\left\{\sum_{j=0}^{3}\frac{\partial F^{kj}}{\partial
x^j}\right\}_{0\leq k\leq 3}
\end{equation}
on the group $\mathcal{S}_0$. Then by
\er{khjhhkfgjfjhgghhgjghjhjkkkkgjghghuiiiulkkjlkklKKgfgjhjjghgjhhjhjhhhhhghhg},
denoting
$$(x^0,x^1,x^2,x^3):=(ct,x_1,x_2,x_3)=(ct,\vec x),$$
we deduce:
\begin{equation}\label{khjhhkfgjfjhgghhgjghjhjkkkkgjghghuiiiulkkjlkklKKgfgjhjjghgjhhjhjhhhhhghhgtyytojjj}
\begin{cases}
\sum_{j=0}^{3}\frac{\partial F^{0j}}{\partial x^j}=-div_{\vec x}\vec D\\
\sum_{j=0}^{3}\frac{\partial F^{1j}}{\partial x^j}=\frac{1}{c}\frac{\partial D_1}{\partial t}-\left(\frac{\partial H_3}{\partial x_2}-\frac{\partial H_2}{\partial x_3}\right)\\
\sum_{j=0}^{3}\frac{\partial F^{2j}}{\partial x^j}=\frac{1}{c}\frac{\partial D_2}{\partial t}-\left(\frac{\partial H_1}{\partial x_3}-\frac{\partial H_3}{\partial x_1}\right)\\
\sum_{j=0}^{3}\frac{\partial F^{3j}}{\partial
x^j}=\frac{1}{c}\frac{\partial D_3}{\partial t}-\left(\frac{\partial
H_2}{\partial x_1}-\frac{\partial H_1}{\partial x_2}\right).
\end{cases}
\end{equation}
I.e.:
\begin{equation}\label{khjhhkfgjfjhgghhgjghjhjkkkkgjghghuiiiulkkjlkklKKgfgjhjjghgjhhjhjhhhhhghhgtyytojjjjj}
\left(\sum_{j=0}^{3}\frac{\partial F^{0j}}{\partial
x^j},\sum_{j=0}^{3}\frac{\partial F^{1j}}{\partial
x^j},\sum_{j=0}^{3}\frac{\partial F^{2j}}{\partial
x^j},\sum_{j=0}^{3}\frac{\partial F^{3j}}{\partial
x^j}\right)=\left(-div_{\vec x}\vec D,
\left(\frac{1}{c}\frac{\partial \vec D}{\partial t}-curl_{\vec
x}\vec H\right)\right).
\end{equation}
Therefore, by
\er{khjhhkfgjfjhgghhgjghjhjkkkkgjghghuiiiulkkjlkklKKgfgjhjjghgjhhjhjhhhhhghhgtyytojjjjj},
the first pair of Maxwell Equations in \er{MaxVacFullPPN}:
\begin{equation}\label{MaxVacFullPPNhjjghyghghiyyhhhj}
\begin{cases}
curl_{\vec x} \vec H=\frac{4\pi}{c}\vec j+\frac{1}{c}\frac{\partial \vec D}{\partial t}\\
div_{\vec x} \vec D= 4\pi\rho,
\end{cases}
\end{equation}
is equivalent to the following equations:
\begin{equation}\label{khjhhkfgjfjhgghhgjghjhjkkkkgjghghuiiiulkkjlkklKKgfgjhjjghgjhhjhjhhhhhghhgtyytojjjjjuyiy}
\left(\sum_{j=0}^{3}\frac{\partial F^{0j}}{\partial
x^j},\sum_{j=0}^{3}\frac{\partial F^{1j}}{\partial
x^j},\sum_{j=0}^{3}\frac{\partial F^{2j}}{\partial
x^j},\sum_{j=0}^{3}\frac{\partial F^{3j}}{\partial
x^j}\right)
%=\left(-div_{\vec x}\vec D,\left(\frac{1}{c}\frac{\partial \vec D}{\partial t}-curl_{\vec x}\vec H\right)\right)
=-4\pi(j^0,j^1,j^2,j^3),
\end{equation}
where $(j^0,j^1,j^2,j^3)$ is the four-vector of electric current on
the group $\mathcal{S}_0$ defined by
\er{fgjfjhgghhgjghjhjijhojihjhjjijhjjjjjuiijjjklihhojjjo} as:
\begin{equation}
\label{fgjfjhgghhgjghjhjijhojihjhjjijhjjjjjuiijjjklihhojjjoouuoiuiu}
(j^0,j^1,j^2,j^3):=\left(\rho,\frac{1}{c}\,\vec j\right)
%\quad\text{where}\quad j^0=\rho\;\;\text{and}\;\;(j^1,j^2,j^3)=\frac{1}{c}\,\vec j,
\end{equation}
%where $\rho$ is the electric charge density and $\vec j$ is the electric current density, then $(j^0,j^1,j^2,j^3)$ is also the four-vector on the group $\mathcal{S}_0$.
Note that in both sides of equation
\er{khjhhkfgjfjhgghhgjghjhjkkkkgjghghuiiiulkkjlkklKKgfgjhjjghgjhhjhjhhhhhghhgtyytojjjjjuyiy}
we have four-vectors and thus
\er{khjhhkfgjfjhgghhgjghjhjkkkkgjghghuiiiulkkjlkklKKgfgjhjjghgjhhjhjhhhhhghhgtyytojjjjjuyiy}
is a covariant form of \er{MaxVacFullPPNhjjghyghghiyyhhhj}. On the
other hand, the second pair of Maxwell Equations in
\er{MaxVacFullPPN}:
\begin{equation}\label{MaxVacFullPPNhjjghyghghiyy}
\begin{cases}
curl_{\vec x} \vec E+\frac{1}{c}\frac{\partial \vec B}{\partial t}=0\\
div_{\vec x} \vec B=0,
\end{cases}
\end{equation}
is equivalent to
\er{MaxVacFull1bjkgjhjhgjgjgkjfhjfdghghligioiuittrPPNkkk}, i.e. to
the following:
\begin{equation}\label{MaxVacFull1bjkgjhjhgjgjgkjfhjfdghghligioiuittrPPNkkkouiuiuuiui}
\begin{cases}
\vec B= curl_{\vec x} \vec A,\\
\vec E=-\nabla_{\vec x}\Psi-\frac{1}{c}\frac{\partial\vec
A}{\partial t},
%,\\ div_{\vec x}\vec A\equiv 0,
\end{cases}
\end{equation}
On the other hand, as before, by
\er{huohuioy89gjjhjffffff3478zzrrZZZhjhhjhhjjhhffGGhjjhiuuijkjjk} we
can rewrite
\er{MaxVacFull1bjkgjhjhgjgjgkjfhjfdghghligioiuittrPPNkkkouiuiuuiui}
in the form of
\er{huohuioy89gjjhjffffff3478zzrrZZZhjhhjhhjjhhffGGhjjh}:
\begin{equation}\label{huohuioy89gjjhjffffff3478zzrrZZZhjhhjhhjjhhffGGhjjhuyuyu}
F_{ij}=\frac{\partial A_j}{\partial x^i}-\frac{\partial
A_i}{\partial x^j}\quad\quad\forall\, i,j=0,1,2,3\,,
\end{equation}
where $(A_0,A_1,A_2,A_3)$ is the four-covector of the
electromagnetic potential on the group $\mathcal{S}_0$ defined by
\er{fgjfjhgghhgjghjhjijhojihjhjjijhjjjljljpk} as:
\begin{equation}\label{fgjfjhgghhgjghjhjijhojihjhjjijhjjjljljpkyuuyyu}
(A_0,A_1,A_2,A_3)=(\Psi,-\vec A).
%\quad\text{where}\quad A_0=\Psi\;\;\text{and}\;\;(A_1,A_2,A_3)=-\vec A,
\end{equation}
Note that in both sides of equation
\er{huohuioy89gjjhjffffff3478zzrrZZZhjhhjhhjjhhffGGhjjhuyuyu} we
have two time covariant tensors, and thus
\er{huohuioy89gjjhjffffff3478zzrrZZZhjhhjhhjjhhffGGhjjhuyuyu} is a
covariant form of \er{MaxVacFullPPNhjjghyghghiyy}. Finally, by
\er{khjhhkfgjfjhgghhgjghjhjkkkkgjghghuiiiulkkjlkklKKgfgjhjjghgjhhjhjhhhhh},
the relations between $(\vec E,\vec B)$ and $(\vec D,\vec H)$ in
\er{MaxVacFullPPNhjjgh}:
\begin{equation}\label{MaxVacFullPPNhjjghojjkjkiou}
\begin{cases}
\vec D=\vec E+\frac{1}{c}\,\vec v\times
\vec B\\
\vec H=\vec B+\frac{1}{c}\,\vec v\times \vec D,
\end{cases}
\end{equation}
are equivalent to the following covariant equations:
\begin{equation}\label{fgjfjhgghhgjghjhjkkkkgjghghuiiiulkkjlkklKKkjkj}
F^{mn}:=\sum_{k=0}^{3}\sum_{j=0}^{3}g^{mj}g^{nk}F_{jk}
%\quad\text{and}\quadd_{mn}:=\sum_{j=0}^{3}\sum_{k=0}^{3}g_{mj}g_{nk}d^{jk}
\quad\quad\forall\, m,n=0,1,2,3.
\end{equation}
Thus by
\er{huohuioy89gjjhjffffff3478zzrrZZZhjhhjhhjjhhffGGhjjhuyuyu},
\er{fgjfjhgghhgjghjhjkkkkgjghghuiiiulkkjlkklKKkjkj} and
\er{khjhhkfgjfjhgghhgjghjhjkkkkgjghghuiiiulkkjlkklKKgfgjhjjghgjhhjhjhhhhhghhgtyytojjjjjuyiy}
together, we deduce that the full system of Maxwell Equations in
\er{MaxVacFullPPN}:
\begin{equation}\label{MaxVacFullPPNhjjghyghghiyyhh}
\begin{cases}
curl_{\vec x} \vec H=\frac{4\pi}{c}\vec j+\frac{1}{c}\frac{\partial \vec D}{\partial t}\\
div_{\vec x} \vec D=4\pi\rho\\
curl_{\vec x} \vec E+\frac{1}{c}\frac{\partial \vec B}{\partial t}=0\\
div_{\vec x} \vec B=
0\\
\vec E=\vec D-\frac{1}{c}\,\vec v\times
\vec B\\
\vec H=\vec B+\frac{1}{c}\,\vec v\times \vec D,
\end{cases}
\end{equation}
is equivalent to the following covariant equations:
\begin{equation}\label{khjhhkfgjfjhgghhgjghjhjkkkkgjghghuiiiulkkjlkklKKgfgjhjjghgjhhjhjhhhhhghhgtyytojjjjjuyiyuu}
\sum_{j=0}^{3}\frac{\partial}{\partial
x^j}\left(\sum_{m=0}^{3}\sum_{n=0}^{3}g^{km}g^{jn}\left(\frac{\partial
A_n}{\partial x^m}-\frac{\partial A_m}{\partial
x^n}\right)\right)=-4\pi j^k\quad\quad\forall\, k=0,1,2,3.
\end{equation}
Note that equations
\er{khjhhkfgjfjhgghhgjghjhjkkkkgjghghuiiiulkkjlkklKKgfgjhjjghgjhhjhjhhhhhghhgtyytojjjjjuyiyuu}
are fully analogous to the covariant formulation of Maxwell
equations in Special Relativity and the only difference is the
choice of the pseudo-metric tensor $\{g^{ij}\}_{0\leq i,j\leq 3}$
(Note that for the Special Relativity case we also have
$\text{det}\,G=-1$). As for the cases of the General relativity, the
covariant formulation of Maxwell equations is still similar to
\er{khjhhkfgjfjhgghhgjghjhjkkkkgjghghuiiiulkkjlkklKKgfgjhjjghgjhhjhjhhhhhghhgtyytojjjjjuyiyuu},
however, in addition to the different choice of the pseudo-metric
tensor $\{g^{ij}\}_{0\leq i,j\leq 3}$ we also have
$\text{det}\,G\neq Const.$ and thus for the full analogy equations
\er{khjhhkfgjfjhgghhgjghjhjkkkkgjghghuiiiulkkjlkklKKgfgjhjjghgjhhjhjhhhhhghhgtyytojjjjjuyiyuu}
should be rewritten in the enlarged form, due to
\er{MaxVacFullPPNhjjghjjkjhhoujii},\er{MaxVacFullPPNhjjghjjkjhhoujiiikjjihjhiuiu}:
\begin{multline}\label{khjhhkfgjfjhgghhgjghjhjkkkkgjghghuiiiulkkjlkklKKgfgjhjjghgjhhjhjhhhhhghhgtyytojjjjjuyiyuughgfg}
\sum_{j=0}^{3}\frac{\partial}{\partial
x^j}\left(\sum_{m=0}^{3}\sum_{n=0}^{3}g^{km}g^{jn}\left(\frac{\partial
A_n}{\partial x^m}-\frac{\partial A_m}{\partial
x^n}\right)\right)+\\
\sum_{j=0}^{3}\frac{1}{\sqrt{|\text{det}\,G|}}\frac{\partial}{\partial
x^j}\left(\sqrt{|\text{det}\,G|}\right)\left(\sum_{m=0}^{3}\sum_{n=0}^{3}g^{km}g^{jn}\left(\frac{\partial
A_n}{\partial x^m}-\frac{\partial A_m}{\partial x^n}\right)\right)
=-4\pi j^k\quad\quad\forall\, k=0,1,2,3.
\end{multline}
Note also that we can rewrite
\er{khjhhkfgjfjhgghhgjghjhjkkkkgjghghuiiiulkkjlkklKKgfgjhjjghgjhhjhjhhhhhghhgtyytojjjjjuyiyuughgfg}
as:
\begin{equation}\label{khjhhkfgjfjhgghhgjghjhjkkkkgjghghuiiiulkkjlkklKKgfgjhjjghgjhhjhjhhhhhghhgtyytojjjjjuyiyuughgfghhj}
\sum_{j=0}^{3}\frac{\partial}{\partial
x^j}\left(\sum_{m=0}^{3}\sum_{n=0}^{3}\sqrt{|\text{det}\,G|}\,g^{km}g^{jn}\left(\frac{\partial
A_n}{\partial x^m}-\frac{\partial A_m}{\partial x^n}\right)\right)
=-4\pi \sqrt{|\text{det}\,G|}\, j^k\quad\quad\forall\, k=0,1,2,3.
\end{equation}








Next by \er{MaxVacFullPPNhjjghjjkjhh} and
\er{MaxVacFullPPNhjjghjjkjhh1} we have
\begin{equation}\label{MaxVacFullPPNhjjghjjkjhhiuyy}
\frac{1}{2}|\vec D|^2-\frac{1}{2}|\vec
B|^2=-\sum_{j=0}^{3}\sum_{k=0}^{3}\frac{1}{4}F^{jk}F_{jk}.
\end{equation}
Therefore, by
\er{fgjfjhgghhgjghjhjijhojihjhjjijhjjjjjuiijjjklihhojjjoouuoiuiu},
\er{fgjfjhgghhgjghjhjijhojihjhjjijhjjjljljpkyuuyyu} and
\er{MaxVacFullPPNhjjghjjkjhhiuyy}, we can rewrite the density of the
Lagrangian of the electromagnetic field, defined in
\er{vhfffngghkjgghPPNggjgjjkgj} as
\begin{equation}\label{vhfffngghkjgghPPNggjgjjkgjoiui}
L_1\left(\vec A,\Psi,\vec
x,t\right):=\frac{1}{4\pi}\left(\frac{1}{2}\left|\vec
D\right|^2-\frac{1}{2}\left|\vec
B\right|^2-4\pi\left(\rho\Psi-\frac{1}{c}\vec A\cdot\vec
j\right)\right),
\end{equation}
in the equivalent covariant form:
\begin{multline}\label{vhfffngghkjgghPPNggjgjjkgjoiuioiggh}
L_1=\frac{1}{4\pi}\left(-\sum_{n=0}^{3}\sum_{k=0}^{3}\frac{1}{4}F^{nk}F_{nk}-\sum_{k=0}^{3}4\pi
j^k A_k\right)=\\
\frac{1}{4\pi}\left(-\sum_{n=0}^{3}\sum_{k=0}^{3}\sum_{m=0}^{3}\sum_{p=0}^{3}\frac{1}{4}g^{mn}g^{pk}\left(\frac{\partial
A_p}{\partial x^m}-\frac{\partial A_m}{\partial
x^p}\right)\left(\frac{\partial A_k}{\partial x^n}-\frac{\partial
A_n}{\partial x^k}\right)-\sum_{k=0}^{3}4\pi j^k A_k\right).
\end{multline}
The density of Lagrangian in
\er{vhfffngghkjgghPPNggjgjjkgjoiuioiggh} is also fully analogous to
the covariant formulation of the Lagrangian density of the
electromagnetic field in Special and General Relativity and the only
difference is the choice of the pseudo-metric tensor
$\{g^{ij}\}_{0\leq i,j\leq 3}$.



\subsection{Covariant formulation of Lagrangian of motion of a classical
charged particle in the external gravitational and electromagnetic
fields} Given a classical charged particle with inertial mass $m$,
charge $\sigma$, three-dimensional place $\vec r(t)$ and
three-dimensional velocity $\frac{d\vec r}{dt}$ in the outer
gravitational field with three-dimensional vectorial potential $\vec
v(\vec x,t)$, the outer electromagnetical field with
three-dimensional vectorial potential $\vec A(\vec x,t)$ and scalar
potential $\vec \Psi(\vec x,t)$, consider a usual Lagrangian that is
a particular case of \er{vhfffngghkjgghfjjSYSPN}:
\begin{equation}\label{vhfffngghkjgghfjjSYSPNkoijjhpoi}
L_0\left(\frac{d\vec r}{dt},t\right):=
\left\{\frac{m}{2}\left|\frac{d\vec r}{dt}-\vec v(\vec
r,t)\right|^2-\sigma\left(\Psi(\vec r,t)-\frac{1}{c}\vec A(\vec
r,t)\cdot\frac{d\vec r}{dt}\right)\right\}.
\end{equation}
Then, since we are interesting in critical points of the functional
\begin{equation}\label{btfffygtgyggyijhhkkSYSPNuhuygyygyggy}
J_0=\int_0^T L_0\left(\frac{d\vec r}{dt},\vec r,t\right)dt,
\end{equation}
adding a constant does not changes the physical meaning of the
Lagrangian and we can rewrite \er{vhfffngghkjgghfjjSYSPNkoijjhpoi}
as:
\begin{equation}\label{vhfffngghkjgghfjjSYSPNkoijjhpoiuui}
L'_0\left(\frac{d\vec r}{dt},t\right):=
\left\{\left(\frac{m}{2}\left|\frac{d\vec r}{dt}-\vec v(\vec
r,t)\right|^2-\frac{mc^2}{2}\right)-\sigma\left(\Psi(\vec
r,t)-\frac{1}{c}\vec A(\vec r,t)\cdot\frac{d\vec
r}{dt}\right)\right\}.
\end{equation}
and \er{btfffygtgyggyijhhkkSYSPNuhuygyygyggy} as
\begin{multline}\label{btfffygtgyggyijhhkkSYSPNuhuygyygyggyuyy}
J'_0:=J_0-\frac{Tmc^2}{2}=\int_0^T L'_0\left(\frac{d\vec r}{dt},\vec
r,t\right)dt=\\ \int_0^T\left\{\left(\frac{m}{2}\left|\frac{d\vec
r}{dt}-\vec v(\vec
r,t)\right|^2-\frac{mc^2}{2}\right)-\sigma\left(\Psi(\vec
r,t)-\frac{1}{c}\vec A(\vec r,t)\cdot\frac{d\vec
r}{dt}\right)\right\}dt,
\end{multline}
Next consider the four-vector field of the momentum on the group
$\mathcal{S}_0$: $\left(p^0(t),p^1(t),p^2(t),p^3(t)\right)$, defined
by \er{fgjfjhgghhgjghjhjijhojihjhjjijhjjjjjuiijhjhh} and
\er{fgjfjhgghhgjghjhjijhojihjhjjijhjjjjjuiijjjklihh} as:
\begin{equation}\label{fgjfjhgghhgjghjhjijhojihjhjjijhjjjjjuiijhjhhioi}
\left(
p^0(t),p^1(t),p^2(t),p^3(t)\right):=\left(m,\frac{m}{c}\frac{d\vec
r}{dt}(t)\right)=
\left(m,\frac{m}{c}\frac{dr_1}{dt}(t),\frac{m}{c}\frac{dr_2}{dt}(t),\frac{m}{c}\frac{dr_3}{dt}(t)\right)
\end{equation}
Then by
\er{fgjfjhgghhgjghjhjkkkkgjghghuiiiulkkjKKyuyyu0ioioiogghghghgghghhg}
and
\er{fgjfjhgghhgjghjhjkkkkgjghghuiiiulkkjKKyuyyu0ioioiogghghghgghgh}
we have
\begin{multline}\label{fgjfjhgghhgjghjhjkkkkgjghghuiiiulkkjKKyuyyu0ioioiogghghghgghghhghgv}
\frac{mc^2}{2}\left(\frac{1}{c^2}\left|\frac{d\vec r}{dt}-\vec
v(\vec r,t)\right|^2-1\right)=\frac{m}{2}\left|\frac{d\vec
r}{dt}-\vec v(\vec
r,t)\right|^2-\frac{mc^2}{2}\\=-\frac{c^2}{2m}\left(\sum_{k=0}^{3}p^kp_k\right)=-\frac{mc^2}{2}\left(\sum_{j=0}^{3}\sum_{k=0}^{3}g_{jk}(\vec
r,t)\,\frac{p^j}{m}\,\frac{p^k}{m}\right).
\end{multline}
On the other hand if we consider the four-covector of the
electromagnetic potential on the group $\mathcal{S}_0$:
$(A_0,A_1,A_2,A_3)$, defined by
\er{fgjfjhgghhgjghjhjijhojihjhjjijhjjjljljpk} as:
\begin{equation}\label{fgjfjhgghhgjghjhjijhojihjhjjijhjjjljljpkyuuyyuhhhhj}
(A_0,A_1,A_2,A_3)=(\Psi,-\vec A),
%\quad\text{where}\quad A_0=\Psi\;\;\text{and}\;\;(A_1,A_2,A_3)=-\vec A,
\end{equation}
then we can write,
\begin{equation}\label{btfffygtgyggyijhhkkSYSPNuhuygyygyggyuyyioiooiihhhj}
\sigma\left(\Psi(\vec r,t)-\frac{1}{c}\vec A(\vec
r,t)\cdot\frac{d\vec r}{dt}\right)=\sum_{k=0}^{3}\sigma A_k(\vec
r,t)\,\frac{p^k}{m}.
\end{equation}
Thus by
\er{fgjfjhgghhgjghjhjkkkkgjghghuiiiulkkjKKyuyyu0ioioiogghghghgghghhghgv}
and \er{btfffygtgyggyijhhkkSYSPNuhuygyygyggyuyyioiooiihhhj} we
rewrite \er{btfffygtgyggyijhhkkSYSPNuhuygyygyggyuyy} in a covariant
form:
\begin{multline}\label{btfffygtgyggyijhhkkSYSPNuhuygyygyggyuyyuyuy}
J'_0=\int_0^T L'_0\left(\frac{d\vec r}{dt},\vec r,t\right)dt=
\int_0^T\left\{-\frac{mc^2}{2}\left(\sum_{j=0}^{3}\sum_{k=0}^{3}g_{jk}(\vec
r,t)\,\frac{p^j}{m}\,\frac{p^k}{m}\right)-\sum_{k=0}^{3}\sigma
A_k(\vec r,t)\,\frac{p^k}{m}\right\}dt.
\end{multline}
Thus if we consider the four-dimensional space-time trajectory of
the particle:
\begin{equation}\label{btfffjhgjghgh}
\left(\chi^0(t),\chi^1(t),\chi^2(t),\chi^3(t)\right)=\left(t,\frac{1}{c}r_1(t),\frac{1}{c}r_2(t),\frac{1}{c}r_3(t)\right),
\end{equation}
then we rewrite \er{btfffygtgyggyijhhkkSYSPNuhuygyygyggyuyyuyuy} as:
\begin{equation}\label{btfffygtgyggyijhhkkSYSPNuhuygyygyggyuyyuyuykuhghg}
J'_0=
\int_0^T\left\{-\frac{mc^2}{2}\left(\sum_{j=0}^{3}\sum_{k=0}^{3}g_{jk}\left(\chi(t)\right)\,\frac{d\chi^j}{dt}\,\frac{d\chi^k}{dt}\right)-\sum_{k=0}^{3}\sigma
A_k\left(\chi(t)\right)\,\frac{d\chi^k}{dt}\right\}dt.
\end{equation}
Moreover, $\left(\frac{d\chi^0}{dt}, \frac{d\chi^1}{dt},
\frac{d\chi^2}{dt}, \frac{d\chi^3}{dt}\right)$ is a four-vector on
the group $\mathcal{S}_0$ and the global non-relativistic time $t$
is the scalar on the group $\mathcal{S}_0$.


Next we also can consider a more general Lagrangian than
\er{btfffygtgyggyijhhkkSYSPNuhuygyygyggyuyyuyuykuhghg}: given a
function $\mathcal{G}(\tau):\mathbb{R}\to\mathbb{R}$ define:
\begin{equation}\label{btfffygtgyggyijhhkkSYSPNuhuygyygyggyuyyuyuykuhghgjjoj}
J_{\mathcal{G}}(\chi)=
\int_0^T\left\{-mc^2\;\mathcal{G}\left(\sum_{j=0}^{3}\sum_{k=0}^{3}g_{jk}\left(\chi(t)\right)\,\frac{d\chi^j}{dt}\,\frac{d\chi^k}{dt}\right)-\sum_{k=0}^{3}\sigma
A_k\left(\chi(t)\right)\,\frac{d\chi^k}{dt}\right\}dt.
\end{equation}
Clearly, \er{btfffygtgyggyijhhkkSYSPNuhuygyygyggyuyyuyuykuhghgjjoj}
is written in covariant form, and in particular,
\er{btfffygtgyggyijhhkkSYSPNuhuygyygyggyuyyuyuykuhghgjjoj} is
invariant under the change of non-inertial cartesian coordinate
systems. In particular, for $\mathcal{G}(\tau):=\frac{1}{2}\tau$ we
obtain \er{btfffygtgyggyijhhkkSYSPNuhuygyygyggyuyyuyuykuhghg}.


Another important particular case is the following choice:
$\mathcal{G}(\tau):=\sqrt{\tau}$. Then we deduce:
\begin{equation}\label{btfffygtgyggyijhhkkSYSPNuhuygyygyggyuyyuyuykuhghgjjojiyy}
J_{rl}(\chi)=
\int_0^T\left\{-mc^2\;\sqrt{\left(\sum_{j=0}^{3}\sum_{k=0}^{3}g_{jk}\left(\chi(t)\right)\,\frac{d\chi^j}{dt}\,\frac{d\chi^k}{dt}\right)}-\sum_{k=0}^{3}\sigma
A_k\left(\chi(t)\right)\,\frac{d\chi^k}{dt}\right\}dt,
\end{equation}
that is in somewhat analogous to the relativistic Lagrangian of the
motion of charged particle. Due to \er{btfffjhgjghgh} we rewrite
\er{btfffygtgyggyijhhkkSYSPNuhuygyygyggyuyyuyuykuhghgjjojiyy} in a
three-dimensional form as:
\begin{equation}\label{btfffygtgyggyijhhkkSYSPNuhuygyygyggyuyyuyuykuhghgjjojiyyyuyuuyy}
J_{rl}(\vec r)= \int_0^T\left\{
-mc^2\sqrt{1-\frac{1}{c^2}\left|\frac{d\vec r}{dt}-\vec v(\vec
r,t)\right|^2}-\sigma\left(\Psi(\vec r,t)-\frac{1}{c}\vec A(\vec
r,t)\cdot\frac{d\vec r}{dt}\right)\right\}dt.
\end{equation}
Thus in the case
$$\frac{1}{c^2}\left|\frac{d\vec r}{dt}-\vec v(\vec
r,t)\right|^2\ll 1,$$ up to additive constant,
\er{btfffygtgyggyijhhkkSYSPNuhuygyygyggyuyyuyuykuhghgjjojiyyyuyuuyy}
becomes to be \er{btfffygtgyggyijhhkkSYSPNuhuygyygyggy}, where $L_0$
is given by \er{vhfffngghkjgghfjjSYSPNkoijjhpoi}. Note that the
Lagrangian in
\er{btfffygtgyggyijhhkkSYSPNuhuygyygyggyuyyuyuykuhghgjjojiyy} has
the following advantage with respect to
\er{btfffygtgyggyijhhkkSYSPNuhuygyygyggyuyyuyuykuhghg}: if we
parameterize the curve in \er{btfffjhgjghgh} by some arbitrary
parameter $s$ that is different from the global time $t$, then
changing variables of integration in
\er{btfffygtgyggyijhhkkSYSPNuhuygyygyggyuyyuyuykuhghgjjojiyy} from
$t$ to $s$ gives:
\begin{equation}\label{btfffygtgyggyijhhkkSYSPNuhuygyygyggyuyyuyuykuhghgjjojiyyjljlgh}
J_{rl}(\chi)=
\int_a^b\left\{-mc^2\;\sqrt{\left(\sum_{j=0}^{3}\sum_{k=0}^{3}g_{jk}\left(\chi(s)\right)\,\frac{d\chi^j}{ds}\,\frac{d\chi^k}{ds}\right)}-\sum_{k=0}^{3}\sigma
A_k\left(\chi(s)\right)\,\frac{d\chi^k}{ds}\right\}ds,
\end{equation}
that has exactly the same form as
\er{btfffygtgyggyijhhkkSYSPNuhuygyygyggyuyyuyuykuhghgjjojiyy},
however $s$ in
\er{btfffygtgyggyijhhkkSYSPNuhuygyygyggyuyyuyuykuhghgjjojiyyjljlgh}
can be \underline{arbitrary} parameter of the curve.

Finally, we would like to note that if the motion of some particle
is ruled by the relativistic-like Lagrangian in
\er{btfffygtgyggyijhhkkSYSPNuhuygyygyggyuyyuyuykuhghgjjojiyyyuyuuyy},
then, although the absolute value of the velocity of the particle
$\left|\frac{d\vec r}{dt}\right|$ can be arbitrary large, the
absolute value of the difference between the velocity of the
particle and the local gravitational potential cannot exceed the
value $c$, i.e.:
\begin{equation}\label{btfffygtgyggyijhhkkSYSPNuhuygyygyggyuyyuyuykuhghgjjojiyyyuyuuyyijyyuyu}
\left|\vec u(t)-\vec v(\vec r,t)\right|:=\left|\frac{d\vec
r}{dt}-\vec v(\vec r,t)\right|\,<\,c\quad\quad\quad\quad\forall\, t,
\end{equation}
provided that
\er{btfffygtgyggyijhhkkSYSPNuhuygyygyggyuyyuyuykuhghgjjojiyyyuyuuyyijyyuyu}
is satisfied in some initial instant of time. Note also that the
quantity in the right hand side of
\er{btfffygtgyggyijhhkkSYSPNuhuygyygyggyuyyuyuykuhghgjjojiyyyuyuuyyijyyuyu}
is invariant under the change of inertial or non-inertial cartesian
coordinate system.



\subsection{Physical laws in curvilinear coordinate systems in the
non-relativistic space-time}
%$\mathbb{R}^4$
Let $\mathcal{S}$ be the group of all smooth non-degenerate
invertible transformations from $\mathbb{R}^4$ onto $\mathbb{R}^4$
having the form \er{fgjfjhgghyuyyu}:
\begin{equation}\label{fgjfjhgghyuyyuyughg}
\begin{cases}
x'^0=f^{(0)}(x^0,x^1,x^2,x^3),\\
x'^1=f^{(1)}(x^0,x^1,x^2,x^3),\\
x'^2=f^{(2)}(x^0,x^1,x^2,x^3),\\
x'^3=f^{(3)}(x^0,x^1,x^2,x^3),
\end{cases}
\end{equation}
and let $\mathcal{S}_0$ be a subgroup of transformations of the form
\er{noninchgravortbstrjgghguittu2intrrrZZygjyghhj}. Then, it is
clear, that given any object that is a scalar, four-vector,
four-covector, two-times covariant tensor or two-times contravariant
tensor on the group $\mathcal{S}_0$, defined in every cartesian
non-inertial coordinate system, we can uniquely extend the
definition of this object, in such a way that it will be defined
also in every curvilinear coordinate systems in $\mathbb{R}^4$ and
will be respectively a scalar, four-vector, four-covector, two-times
covariant tensor or two-times contravariant tensor on the wider
group $\mathcal{S}$. Thus all the physical laws that have a
covariant form preserve their form also in transformations of the
form \er{fgjfjhgghyuyyuyughg} i.e. in curvilinear coordinate
systems. In particular, the Maxwell Equations in every curvilinear
coordinate system have the form of
\er{khjhhkfgjfjhgghhgjghjhjkkkkgjghghuiiiulkkjlkklKKgfgjhjjghgjhhjhjhhhhhghhgtyytojjjjjuyiyuughgfg}
or equivalently of
\er{khjhhkfgjfjhgghhgjghjhjkkkkgjghghuiiiulkkjlkklKKgfgjhjjghgjhhjhjhhhhhghhgtyytojjjjjuyiyuughgfghhj}:
\begin{multline}\label{khjhhkfgjfjhgghhgjghjhjkkkkgjghghuiiiulkkjlkklKKgfgjhjjghgjhhjhjhhhhhghhgtyytojjjjjuyiyuughgfghjjhkpk}
\sum_{j=0}^{3}\frac{\partial}{\partial
x^j}\left(\sum_{m=0}^{3}\sum_{n=0}^{3}g^{km}g^{jn}\left(\frac{\partial
A_n}{\partial x^m}-\frac{\partial A_m}{\partial
x^n}\right)\right)+\\
\sum_{j=0}^{3}\frac{1}{\sqrt{|\text{det}\,G|}}\frac{\partial}{\partial
x^j}\left(\sqrt{|\text{det}\,G|}\right)\left(\sum_{m=0}^{3}\sum_{n=0}^{3}g^{km}g^{jn}\left(\frac{\partial
A_n}{\partial x^m}-\frac{\partial A_m}{\partial x^n}\right)\right)
=-4\pi j^k\quad\quad\forall\, k=0,1,2,3,
\end{multline}
or equivalently:
\begin{equation}\label{khjhhkfgjfjhgghhgjghjhjkkkkgjghghuiiiulkkjlkklKKgfgjhjjghgjhhjhjhhhhhghhgtyytojjjjjuyiyuughgfghhjkkhj}
\sum_{j=0}^{3}\frac{\partial}{\partial
x^j}\left(\sum_{m=0}^{3}\sum_{n=0}^{3}\sqrt{|\text{det}\,G|}\,g^{km}g^{jn}\left(\frac{\partial
A_n}{\partial x^m}-\frac{\partial A_m}{\partial x^n}\right)\right)
=-4\pi \sqrt{|\text{det}\,G|}\, j^k\quad\quad\forall\, k=0,1,2,3.
\end{equation}
Here $\{A_k\}_{k=0,1,2,3}$ is the four-covector of the
electromagnetic potential, $\{j^k\}_{k=0,1,2,3}$ is the four-vector
of the current and $G:=\{g_{kj}\}_{k,j=0,1,2,3}$,
$\{g^{kj}\}_{k,j=0,1,2,3}$ are pseudo-metric covariant and
contravariant tensors. Note, that in curvilinear coordinate system
we can have $\text{det}\,G\neq Const$ and thus we need to consider
the enlarged form
\er{khjhhkfgjfjhgghhgjghjhjkkkkgjghghuiiiulkkjlkklKKgfgjhjjghgjhhjhjhhhhhghhgtyytojjjjjuyiyuughgfg}
instead of
\er{khjhhkfgjfjhgghhgjghjhjkkkkgjghghuiiiulkkjlkklKKgfgjhjjghgjhhjhjhhhhhghhgtyytojjjjjuyiyuu}.
Moreover, the density of the Lagrangian of the electromagnetic field
in every curvilinear coordinate system in $\mathbb{R}^4$ also has a
form of \er{vhfffngghkjgghPPNggjgjjkgjoiuioiggh}:
\begin{multline}\label{vhfffngghkjgghPPNggjgjjkgjoiuioigghjhhh}
L_1=\frac{1}{4\pi}\left(-\sum_{n=0}^{3}\sum_{k=0}^{3}\frac{1}{4}F^{nk}F_{nk}-\sum_{k=0}^{3}4\pi
j^k A_k\right)=\\
\frac{1}{4\pi}\left(-\sum_{n=0}^{3}\sum_{k=0}^{3}\sum_{m=0}^{3}\sum_{p=0}^{3}\frac{1}{4}g^{mn}g^{pk}\left(\frac{\partial
A_p}{\partial x^m}-\frac{\partial A_m}{\partial
x^p}\right)\left(\frac{\partial A_k}{\partial x^n}-\frac{\partial
A_n}{\partial x^k}\right)-\sum_{k=0}^{3}4\pi j^k A_k\right),
\end{multline}
where
%\er{huohuioy89gjjhjffffff3478zzrrZZZhjhhjhhjjhhffGGhjjh}
\begin{equation}\label{huohuioy89gjjhjffffff3478zzrrZZZhjhhjhhjjhhffGGhjjhjhhjhj}
F_{ij}:=\frac{\partial A_j}{\partial x^i}-\frac{\partial
A_i}{\partial x^j}\quad\quad\forall\, i,j=0,1,2,3\,.
\end{equation}


Next the general Lagrangian of motion of the charged particle in the
gravitational and electromagnetic field
\er{btfffygtgyggyijhhkkSYSPNuhuygyygyggyuyyuyuykuhghgjjoj} preserve
its form in every curvilinear coordinate system:
\begin{equation}\label{btfffygtgyggyijhhkkSYSPNuhuygyygyggyuyyuyuykuhghgjjojhjfg}
J_{\mathcal{G}}(\chi)=
\int_0^T\left\{-mc^2\;\mathcal{G}\left(\sum_{j=0}^{3}\sum_{k=0}^{3}g_{jk}\left(\chi(t)\right)\,\frac{d\chi^j}{dt}\,\frac{d\chi^k}{dt}\right)-\sum_{k=0}^{3}\sigma
A_k\left(\chi(t)\right)\,\frac{d\chi^k}{dt}\right\}dt.
\end{equation}
where $t$ is the global time, which is a scalar on the group
$\mathcal{S}$,
\begin{equation}\label{btfffjhgjghghijhh}
\left(\chi^0(t),\chi^1(t),\chi^2(t),\chi^3(t)\right):=\left(\frac{1}{c}x^0(t),\frac{1}{c}x_1(t),\frac{1}{c}x_2(t),\frac{1}{c}x_3(t)\right),
\end{equation}
and $\left(x^0(t),x^1(t),x^2(t),x^3(t)\right)\in\mathbb{R}^4$ is a
four-dimensional space-time trajectory of the particle,
parameterized by the global time. Note that if we denote by $t$ the
scalar of global time, then in a general curvilinear coordinate
system the coordinate $x^0$ can differ from $ct$, and the equality
$x^0=ct$ valid, in general, only in cartesian non-inertial
coordinate systems. However, since the equality in
\er{fgjfjhgghhgjghjhjijhojihjhjjijhjjjjjuiijjjkhjhjhjuiiuuuyu} has a
covariant form, the scalar of the global time $t$ satisfies the
following Eikonal-type equation in every curvilinear coordinate
system:
\begin{equation}\label{fgjfjhgghhgjghjhjijhojihjhjjijhjjjjjuiijjjkhjhjhjuiiuuuyuuoiuuiio}
\sum_{j=0}^{n}\sum_{k=0}^{n}\,g^{jk}(x^0,x^1,x^2,x^3)\,\frac{\partial
t}{\partial x^j}(x^0,x^1,x^2,x^3)\,\frac{\partial t}{\partial
x^k}(x^0,x^1,x^2,x^3)\,=\, \frac{1}{c^2}.
\end{equation}



Next, in the particular case of the relativistic-like Lagrangian
where $\mathcal{G}(\tau):=\sqrt{\tau}$, the Lagrangian in
\er{btfffygtgyggyijhhkkSYSPNuhuygyygyggyuyyuyuykuhghgjjojiyyjljlgh}
also preserve their form in every curvilinear coordinate system:
\begin{equation}\label{btfffygtgyggyijhhkkSYSPNuhuygyygyggyuyyuyuykuhghgjjojiyyjljlghhhhjhj}
J_{rl}(\chi)=
\int_a^b\left\{-mc^2\;\sqrt{\left(\sum_{j=0}^{3}\sum_{k=0}^{3}g_{jk}\left(\chi(s)\right)\,\frac{d\chi^j}{ds}\,\frac{d\chi^k}{ds}\right)}-\sum_{k=0}^{3}\sigma
A_k\left(\chi(s)\right)\,\frac{d\chi^k}{ds}\right\}ds,
\end{equation}
where $s$ is the arbitrary parameter of the trajectory:
\begin{equation}\label{btfffjhgjghghijhhhu}
\left(\chi^0(s),\chi^1(s),\chi^2(s),\chi^3(s)\right):=\left(\frac{1}{c}x^0(s),\frac{1}{c}x_1(s),\frac{1}{c}x_2(s),\frac{1}{c}x_3(s)\right).
\end{equation}
In particular we can take $s:=\chi^0$ in
\er{btfffygtgyggyijhhkkSYSPNuhuygyygyggyuyyuyuykuhghgjjojiyyjljlghhhhjhj}.



Finally, we would like to note the following fact: since in the
absence of essential gravitational masses, in every inertial
coordinate system the three-dimensional vectorial gravitational
potential $\vec v$ is a constant,
% and therefore,
there exists a
unique inertial coordinate system where $\vec v=0$ everywhere. In
this particular system by \er{hoyuiouigyfg3478zzrrZZffGGhhjhj} and
the fact that $\vec v=0$ we have:
\begin{equation}\label{hoyuiouigyfg3478zzrrZZffGGhhjhjiui}
\begin{cases}
g_{00}=1\\
g_{ij}=-\delta_{ij}\quad\forall 1\leq i,j\leq 3\\
g_{0j}=g_{j0}=0\quad\forall 1\leq j\leq 3.
\end{cases}
\end{equation}
and thus the Maxwell equations are the same as in the Special
Relativity. Moreover, in this system the Lagrangian of the motion of
the particle of the form
\er{btfffygtgyggyijhhkkSYSPNuhuygyygyggyuyyuyuykuhghgjjojiyyjljlghhhhjhj}
is also the same as in the Special Relativity. Thus, since Maxwell
equations
\er{khjhhkfgjfjhgghhgjghjhjkkkkgjghghuiiiulkkjlkklKKgfgjhjjghgjhhjhjhhhhhghhgtyytojjjjjuyiyuughgfghjjhkpk}
and the Lagrangian of the motion of particles
\er{btfffygtgyggyijhhkkSYSPNuhuygyygyggyuyyuyuykuhghgjjojiyyjljlghhhhjhj}
preserve their form in every curvilinear coordinate system of the
group $\mathcal{S}$, they stay the same as in Special Relativity
also in the case of every curvilinear coordinate system. Thus in the
particular case of $\mathcal{G}(\tau):=\sqrt{\tau}$ in
\er{btfffygtgyggyijhhkkSYSPNuhuygyygyggyuyyuyuykuhghgjjojhjfg} and
in the absence of essential gravitational masses, the unique formal
mathematical difference between our model and the Special Relativity
is that in the frames of our model we consider the Galilean
Transformations as transformations of the change of inertial
coordinate systems and \er{noninchgravortbstr} as transformations of
the change of non-inertial cartesian coordinate system, however the
Lorenz transformations lead to non-inertial \underline{curvilinear}
coordinate system. In contrast, in the Special Relativity the
fundamental role of the Lorenz transformations, i.e. the
transformations that preserve the form
\er{hoyuiouigyfg3478zzrrZZffGGhhjhjiui} of the pseudo-metric tensor,
is postulated as the role of transformations of the change of
inertial coordinate systems, and at the same time the Galilean
Transformations and transformations \er{noninchgravortbstr} lead to
\underline{curvilinear} non-inertial coordinate system.






















\section{Maxwell equations in the presence of Dielectrics and/or Magnetics}\label{DMPGG}
\subsection{General setting} Consider system
\er{MaxVacFull1bjkgjhjhgjaaaPPN}
%\er{MaxVacFull1ninshtrhjkk}
in some inertial or non-inertial cartesian coordinate system inside
a dielectric and/or magnetic medium:
\begin{equation}\label{MaxVacFullnnnnGG}
\begin{cases}
curl_{\vec x} \vec H_0\equiv \frac{4\pi}{c}\left(\vec j+\vec
j_m+\vec j_p\right)+
\frac{1}{c}\frac{\partial \vec D_0}{\partial t}\quad\text{for}\;\;(\vec x,t)\in\R^3\times[0,+\infty),\\
div_{\vec x} \vec D_0\equiv 4\pi\left(\rho+\rho_p\right)\quad\quad\text{for}\;\;(\vec x,t)\in\R^3\times[0,+\infty),\\
curl_{\vec x} \vec E+\frac{1}{c}\frac{\partial \vec B}{\partial t}\equiv 0\quad\quad\text{for}\;\;(\vec x,t)\in\R^3\times[0,+\infty),\\
div_{\vec x} \vec B\equiv 0\quad\quad\text{for}\;\;(\vec
x,t)\in\R^3\times[0,+\infty),
%D_0:=E+\frac{1}{c}\,v\times
%B\quad\quad\text{for}\;\;(\vec x,t)\in\R^3\times[0,+\infty)\\
%H:=B+\frac{1}{c}\,v\times
%D_0\quad\quad\text{for}\;\;(\vec x,t)\in\R^3\times[0,+\infty).
\end{cases}
\end{equation}
where $\vec E$ is the electric field, $\vec B$ is the magnetic
field, $\vec v:=\vec v(\vec x,t)$ is the vectorial gravitational
potential, $\rho$ is the average (macroscopic) charge density,
$\rho_p$ is the density of the charge of polarization, $\vec j$ is
the average (macroscopic) current density, $\vec j_m$ is the density
of the current of magnetization, $\vec j_p$ is the density of the
current of polarization and
\begin{equation}\label{MaxVacFullnnnngkggjkklhGG}
%\begin{cases}
\vec D_0:=\vec E+\frac{1}{c}\,\vec v\times \vec
B\quad\text{and}\quad
%\quad\quad\text{for}\;\;(\vec x,t)\in\R^3\times[0,+\infty)\\
\vec H_0:=\vec B+\frac{1}{c}\,\vec v\times \vec D_0.
%\quad\quad\text{for}\;\;(\vec x,t)\in\R^3\times[0,+\infty).
%\end{cases}
\end{equation}
It is well known from the Lorentz theory that in the case of a
moving dielectric/magnetic medium
\begin{equation}\label{PolarGG}
%\begin{cases}
\rho_p=-div_{\vec x} \vec P\quad\text{and}\quad
%\quad\quad\text{for}\;\;(\vec x,t)\in\R^3\times[0,+\infty)\\
\vec j_p=\frac{\partial \vec P}{\partial t}-curl_{\vec x}\left(
%\frac{1}{c}\,
\vec u\times \vec P\right),
%\quad\quad\text{for}\;\;(\vec x,t)\in\R^3\times[0,+\infty).
%\end{cases}
\end{equation}
where $\vec P:\R^3\times[0,+\infty)\to\R^3$ is the field of
polarization and $\vec u:=\vec u(\vec x,t)$ is the field of
velocities of the dielectric medium (see also \cite{PC}, page 610).
Furthermore,
\begin{equation}\label{MagnetGG}
\vec j_m=c\, curl_{\vec x} \vec M,
\end{equation}
where $\vec M:\R^3\times[0,+\infty)\to\R^3$ is the field of
magnetization. Thus if we consider
\begin{equation}\label{OprdddGG}
\vec D:=\vec D_0+4\pi \vec P=\vec E+\frac{1}{c}\,\vec v\times \vec
B+4\pi \vec P,
\end{equation}
and
\begin{multline}\label{Oprddd1GG}
%\quad\quad\text{for}\;\;(\vec x,t)\in\R^3\times[0,+\infty)\\
\vec H:=\vec H_0-4\pi \vec M+\frac{4\pi}{c}\,\vec u\times \vec
P=\vec B+\frac{1}{c}\,\vec v\times \vec D_0+\frac{4\pi}{c}\,\vec
u\times \vec P-4\pi \vec M
%B+\frac{1}{c}\,v\times D-4\pi M,
\\=\vec B+\frac{4\pi}{c}\,\vec u\times \vec P+\frac{1}{c}\,\vec v\times
\vec E+\frac{1}{c}\,\vec v\times\left(\frac{1}{c}\,\vec v\times \vec
B\right)-4\pi \vec M,
%\quad\quad\text{for}\;\;(\vec x,t)\in\R^3\times[0,+\infty).
\end{multline}
we obtain the usual Maxwell equations of the form:
\begin{equation}\label{MaxMedFullGG}
\begin{cases}
curl_{\vec x} \vec H\equiv \frac{4\pi}{c}\vec j+
\frac{1}{c}\frac{\partial \vec D}{\partial t}\quad\text{for}\;\;(\vec x,t)\in\R^3\times[0,+\infty),\\
div_{\vec x} \vec D\equiv 4\pi\rho\quad\quad\text{for}\;\;(\vec x,t)\in\R^3\times[0,+\infty),\\
curl_{\vec x} \vec E+\frac{1}{c}\frac{\partial \vec B}{\partial t}\equiv 0\quad\quad\text{for}\;\;(\vec x,t)\in\R^3\times[0,+\infty),\\
div_{\vec x} \vec B\equiv 0\quad\quad\text{for}\;\;(\vec
x,t)\in\R^3\times[0,+\infty),
%D_0:=E+\frac{1}{c}\,v\times
%B\quad\quad\text{for}\;\;(\vec x,t)\in\R^3\times[0,+\infty)\\
%H:=B+\frac{1}{c}\,v\times
%D_0\quad\quad\text{for}\;\;(\vec x,t)\in\R^3\times[0,+\infty).
\end{cases}
\end{equation}
We call $\vec D$ by the electric displacement field and $\vec H$ by
the $\vec H$-magnetic field in a medium.

\subsection{Change of Non-inertial coordinate system} Consider the
change of certain non-inertial cartesian coordinate system $(*)$ to
another cartesian coordinate system $(**)$ of the form
\begin{equation*}
%\label{noninchjhhkh}
\begin{cases}
\vec x'=A(t)\cdot \vec x+\vec z(t),\\
t'=t,
\end{cases}
\end{equation*}
where $A(t)\in SO(3)$ is a rotation. Then, as before in
\er{yuythfgfyftydtydtydtyddyyyhhddhhhredPPN111hgghjg}, denoting
$\vec w(t)=\vec z'(t)$, we have the following relations between the
physical quantities in coordinate systems $(*)$ and $(**)$:
\begin{equation}\label{guigikvbvbGG}
\begin{cases}
\vec E'=A(t)\cdot\vec E-\frac{1}{c}\,\left(A'(t)\cdot\vec x+\vec
w(t)\right)\times \left(A(t)\cdot\vec B\right),\\
\vec B'=A(t)\cdot\vec B,\\
\vec D'_0=A(t)\cdot \vec D_0,\\
\vec H'_0=A(t)\cdot\vec H_0+\frac{1}{c}\,\left(A'(t)\cdot\vec x+\vec
w(t)\right)\times \left(A(t)\cdot\vec D_0\right),\\
\vec P'=A(t)\cdot\vec P,\\
\vec M'=A(t)\cdot\vec M,\\
%\vec v'=A(t)\cdot \vec v+A'(t)\cdot\vec x+\vec w(t)\\
\vec u'=A(t)\cdot \vec u+A'(t)\cdot\vec x+\vec w(t).
\end{cases}
\end{equation}
Plugging it into \er{OprdddGG} and \er{Oprddd1GG} we deduce
\begin{equation}\label{OprdddkkkmGG}
\vec D':=\vec D'_0+4\pi \vec P'=A(t)\cdot\left(\vec D_0+4\pi \vec
P\right)=A(t)\cdot\vec D,
\end{equation}
and
\begin{multline}\label{Oprddd1kkkmGG}
%\quad\quad\text{for}\;\;(\vec x,t)\in\R^3\times[0,+\infty)\\
\vec H':=\vec H'_0-4\pi \vec M'+\frac{4\pi}{c}\,\vec u'\times \vec
P'=A(t)\cdot\vec H_0+\frac{1}{c}\,\left(A'(t)\cdot\vec x+\vec
w(t)\right)\times \left(A(t)\cdot\vec D_0\right)\\-4\pi
A(t)\cdot\vec M+\frac{4\pi}{c}\,\left(A(t)\cdot \vec
u+A'(t)\cdot\vec x+\vec
w(t)\right)\times \left(A(t)\cdot\vec P\right)\\
=A(t)\cdot\left(\vec H_0-4\pi \vec M+\frac{4\pi}{c}\,\vec u\times
\vec P\right)+\frac{1}{c}\,\left(A'(t)\cdot\vec x+\vec
w(t)\right)\times \left(A(t)\cdot\left(\vec D_0+4\pi\vec
P\right)\right)\\= A(t)\cdot\vec H+\frac{1}{c}\,\left(A'(t)\cdot\vec
x+\vec w(t)\right)\times \left(A(t)\cdot\vec D\right) ,
%\quad\quad\text{for}\;\;(\vec x,t)\in\R^3\times[0,+\infty).
\end{multline}
So the expressions of transformations under the change of
non-inertial cartesian coordinate system in a dielectric/magnetic
medium exactly the same as in the vacuum, i.e. having the form of
\begin{equation}\label{guigikvbvbggjklhjkkgjgGGGG}
\begin{cases}
\vec D'=A(t)\cdot \vec D\\
\vec B'=A(t)\cdot\vec B\\
\vec E'=A(t)\cdot\vec E-\frac{1}{c}\,\left(A'(t)\cdot\vec x+\vec
w(t)\right)\times \left(A(t)\cdot\vec B\right)\\
\vec H'=A(t)\cdot\vec H+\frac{1}{c}\,\left(A'(t)\cdot\vec x+\vec
w(t)\right)\times \left(A(t)\cdot\vec D\right).
\end{cases}
\end{equation}







\subsection{Case of simplest dielectrics/magnetics} It is well
known that in the case of simplest homogenous isotropic dielectrics
and/or magnetics we have
\begin{equation}\label{EBDHTrans444knbuihuigGG}
\begin{cases}
\vec P=\gamma\left(\vec E+\frac{1}{c}\,\vec u\times \vec B\right),\\
\vec M=\kappa\vec B,
\end{cases}
\end{equation}
where $\gamma$ and $\kappa$ are material coefficients.
%are invariant under the following transformations
%\begin{equation}\label{Gal444}
%\begin{cases}
%x'=x+wt,\\
%t'=t.
%\end{cases}
%\end{equation}
%and
Using \er{guigikvbvbGG}, it can be easily seen that the laws in
\er{EBDHTrans444knbuihuigGG} are invariant under the changes of
inertial or non-inertial cartesian coordinate system. Next, plugging
\er{EBDHTrans444knbuihuigGG} into \er{OprdddGG} and \er{Oprddd1GG}
gives,
\begin{equation}\label{OprdddsimGG}
\vec D=\vec E+\frac{1}{c}\,\vec v\times \vec B+4\pi\gamma\left(\vec
E+\frac{1}{c}\,\vec u\times \vec B\right),
\end{equation}
and
\begin{equation}\label{Oprddd1simGG}
%\quad\quad\text{for}\;\;(\vec x,t)\in\R^3\times[0,+\infty)\\
\vec H=\left(1-4\pi \kappa\right)\vec B+\frac{4\pi\gamma}{c}\,\vec
u\times \left(\vec E+\frac{1}{c}\,\vec u\times \vec
B\right)+\frac{1}{c}\,\vec v\times\left(\vec E+\frac{1}{c}\,\vec
v\times \vec B\right).
%\quad\quad\text{for}\;\;(\vec x,t)\in\R^3\times[0,+\infty).
\end{equation}
We rewrite \er{OprdddsimGG} as:
\begin{equation}\label{OprdddsimGGffyzz}
\vec E=\frac{1}{1+4\pi\gamma}\vec
D-\frac{1}{c}\,\frac{1}{1+4\pi\gamma}\left(\vec v+4\pi\gamma\vec
u\right)\times \vec B,
\end{equation}
and by \er{OprdddsimGG} and \er{OprdddsimGGffyzz} we rewrite
\er{Oprddd1simGG} as:
\begin{equation}\label{Oprddd1simGGffyzz}
%\quad\quad\text{for}\;\;(\vec x,t)\in\R^3\times[0,+\infty)\\
\vec H=\left(1-4\pi \kappa\right)\vec
B+\frac{1}{c}\,\frac{1}{1+4\pi\gamma}\left(\vec v+4\pi\gamma\vec
u\right)\times \vec
D+\frac{4\pi\gamma}{1+4\pi\gamma}\,\frac{1}{c^2}\,(\vec u-\vec
v)\times\left(\left(\vec u-\vec v\right)\times \vec B\right).
%\quad\quad\text{for}\;\;(\vec x,t)\in\R^3\times[0,+\infty).
\end{equation}
Thus denoting $\gamma_0=\frac{1}{1+4\pi\gamma}$ and $\kappa_0=1-4\pi
\kappa$ and defining the speed-like vector field
\begin{equation}\label{OprdddsimGGffyhjyhhzz}
\vec {\tilde u}:=\left(\gamma_0\vec v+(1-\gamma_0)\vec
u\right)=\frac{1}{1+4\pi\gamma}\left(\vec v+4\pi\gamma\vec u\right),
\end{equation}
by \er{OprdddsimGGffyzz} and \er{Oprddd1simGGffyzz}  we deduce
\begin{equation}\label{OprdddsimGGffykkzz}
\vec E=\gamma_0\vec D-\frac{1}{c}\,\vec {\tilde u}\times \vec B,
\end{equation}
and
\begin{equation}\label{Oprddd1simGGffykkzz}
%\quad\quad\text{for}\;\;(\vec x,t)\in\R^3\times[0,+\infty)\\
\vec H=\kappa_0\vec B+\frac{1}{c}\,\vec {\tilde u}\times \vec
D+\frac{(1-\gamma_0)}{c^2}\,(\vec u-\vec v)\times\left(\left(\vec
u-\vec v\right)\times \vec B\right),
%\quad\quad\text{for}\;\;(\vec x,t)\in\R^3\times[0,+\infty).
\end{equation}
where we call $\gamma_0$ and $\kappa_0$ dielectric and magnetic
permeability of the medium. Thus, by \er{MaxMedFullGG},
\er{OprdddsimGGffyhjyhhzz} \er{OprdddsimGGffykkzz} and
\er{Oprddd1simGGffykkzz} we have
\begin{equation}\label{MaxMedFullGGffykkzz}
\begin{cases}
curl_{\vec x} \vec H=\frac{4\pi}{c}\vec j+
\frac{1}{c}\frac{\partial \vec D}{\partial t},\\
div_{\vec x} \vec D=4\pi\rho,\\
curl_{\vec x} \vec E+\frac{1}{c}\frac{\partial \vec B}{\partial t}=0,\\
div_{\vec x} \vec B=0,\\
\vec E=\gamma_0\vec D-\frac{1}{c}\,\vec {\tilde u}\times \vec B,\\
\vec H=\kappa_0\vec B+\frac{1}{c}\,\vec {\tilde u}\times \vec
D+\frac{(1-\gamma_0)}{c^2}\,(\vec u-\vec v)\times\left(\left(\vec
u-\vec v\right)\times \vec B\right),\\
\vec {\tilde u}:=\left(\gamma_0\vec v+(1-\gamma_0)\vec u\right),
%D_0:=E+\frac{1}{c}\,v\times
%B\quad\quad\text{for}\;\;(\vec x,t)\in\R^3\times[0,+\infty)\\
%H:=B+\frac{1}{c}\,v\times
%D_0\quad\quad\text{for}\;\;(\vec x,t)\in\R^3\times[0,+\infty).
\end{cases}
\end{equation}
where $\vec {\tilde u}$ is a speed-like vector field that we call
the optical displacement of the moving medium. Note that for the
case $\gamma_0=1$ and $\kappa_0=1$, the system
\er{MaxMedFullGGffykkzz} is exactly the same as the corresponding
system in the vacuum. The equations in \er{MaxMedFullGGffykkzz} take
much simpler forms in the case where the quantity
\begin{equation}\label{OprdddsimGGffyhjyhhtygrffgfzz}
\frac{|1-\gamma_0|\cdot|\vec u-\vec v|^2}{c^2}\ll 1
\end{equation} is
negligible, that happens if the absolute value of the difference
between the medium velocity and vectorial gravitational potential is
much less then the constant $c$ or/and $\gamma_0$ is close to the
value $1$. Indeed, in this case, instead of \er{OprdddsimGGffykkzz}
and \er{Oprddd1simGGffykkzz} we obtain the following relations:
\begin{align}\label{OprdddsimsimsimGG}
\vec E=\gamma_0\vec D-\frac{1}{c}\,\vec {\tilde u}\times \vec B,\\
\label{Oprddd1simsimsimGG}
%\quad\quad\text{for}\;\;(\vec x,t)\in\R^3\times[0,+\infty)\\
\vec H=\kappa_0\vec B+\frac{1}{c}\,\vec {\tilde u}\times \vec D.
%\quad\quad\text{for}\;\;(\vec x,t)\in\R^3\times[0,+\infty).
\end{align}
As a consequence we obtain the full system of Maxwell equations in
the medium:
\begin{equation}\label{MaxMedFullGGffykkhjhhzz}
\begin{cases}
curl_{\vec x} \vec H=\frac{4\pi}{c}\vec j+
\frac{1}{c}\frac{\partial \vec D}{\partial t},\\
div_{\vec x} \vec D=4\pi\rho,\\
curl_{\vec x} \vec E+\frac{1}{c}\frac{\partial \vec B}{\partial t}=0,\\
div_{\vec x} \vec B=0\\
\vec E=\gamma_0\vec D-\frac{1}{c}\,\vec {\tilde u}\times \vec B\\
\vec H=\kappa_0\vec B+\frac{1}{c}\,\vec {\tilde u}\times \vec D,\\
\vec {\tilde u}=\left(\gamma_0\vec v+(1-\gamma_0)\vec u\right),
%D_0:=E+\frac{1}{c}\,v\times
%B\quad\quad\text{for}\;\;(\vec x,t)\in\R^3\times[0,+\infty)\\
%H:=B+\frac{1}{c}\,v\times
%D_0\quad\quad\text{for}\;\;(\vec x,t)\in\R^3\times[0,+\infty).
\end{cases}
\end{equation}
where $\vec {\tilde u}$ is the speed-like vector field and
$\gamma_0$ and $\kappa_0$ are dielectric and magnetic permeability
of the medium. Note that \er{MaxMedFullGGffykkhjhhzz} is analogous
to the system of Maxwell equations in the vacuum and it is also
invariant under the change of inertial or non-inertial cartesian
coordinate system, provided that under this transformation we have
\er{guigikvbvbggjklhjkkgjgGGGG}.
































\subsection{Ohm's Law in a conducting medium} It is well known that
the Ohm's Law in a conducting medium has the form
\begin{equation}\label{vjhfhjtjhjhuyyiyGG}
\vec j-\rho\,\vec u=\varepsilon\left(\vec E+\frac{1}{c}\,\vec
u\times \vec B\right),
\end{equation}
where $\vec u$ is the velocity of the medium and $\varepsilon$ is a
material coefficient. As before, using
%\er{yuythfgfyftydtydtydtyddyyyhhddhhhredPPN111hgghjg}
\er{guigikvbvbggjklhjkkgjgGGGG}, it can be easily seen that the
Ohm's Law is invariant under the changes of inertial or non-inertial
cartesian coordinate system.
















































































































\section{Some further consequences of Maxwell equations}\label{CM}
\subsection{General case}\label{gcCM}
Again consider the system of Maxwell equations in the vacuum or in a
medium of the form \er{MaxMedFullGGffykkhjhhzz}:
%\er{MaxVacFull1bjkgjhjhgjaaaPPN}:
\begin{equation}\label{MaxVacFullPPNffGG}
\begin{cases}
curl_{\vec x} \vec H=\frac{4\pi}{c}\vec j+
\frac{1}{c}\frac{\partial \vec D}{\partial t},\\
div_{\vec x} \vec D=4\pi\rho,\\
curl_{\vec x} \vec E+\frac{1}{c}\frac{\partial \vec B}{\partial t}=0,\\
div_{\vec x} \vec B=0\\
\vec E=\gamma_0\vec D-\frac{1}{c}\,\vec {\tilde u}\times \vec B\\
\vec H=\kappa_0\vec B+\frac{1}{c}\,\vec {\tilde u}\times \vec D,\\
\vec {\tilde u}=\left(\gamma_0\vec v+(1-\gamma_0)\vec u\right),
%D_0:=E+\frac{1}{c}\,v\times
%B\quad\quad\text{for}\;\;(\vec x,t)\in\R^3\times[0,+\infty)\\
%H:=B+\frac{1}{c}\,v\times
%D_0\quad\quad\text{for}\;\;(\vec x,t)\in\R^3\times[0,+\infty).
\end{cases}
\end{equation}
%
%
%
\begin{comment}
\begin{equation}\label{MaxVacFullPPNffGG}
\begin{cases}
curl_{\vec x} \vec H\equiv \frac{4\pi}{c}\vec
j+\frac{1}{c}\frac{\partial
\vec D}{\partial t},\\
%\quad\text{for}\;\;(\vec x,t)\in\R^3\times[0,+\infty),\\
div_{\vec x} \vec D\equiv 4\pi\rho,\\
%\quad\quad\text{for}\;\;(\vec x,t)\in\R^3\times[0,+\infty),\\
curl_{\vec x} \vec E+\frac{1}{c}\frac{\partial \vec B}{\partial t}\equiv 0,\\
%\quad\quad\text{for}\;\;(\vec x,t)\in\R^3\times[0,+\infty),\\
div_{\vec x} \vec B\equiv 0,\\
%\quad\quad\text{for}\;\;(\vec x,t)\in\R^3\times[0,+\infty),\\
\vec E=\vec D-\frac{1}{c}\,\vec v\times \vec B,\\
%\quad\quad\text{for}\;\;(\vec x,t)\in\R^3\times[0,+\infty)\\
\vec H=\vec B+\frac{1}{c}\,\vec v\times \vec D,
%\quad\quad\text{for}\;\;(\vec x,t)\in\R^3\times[0,+\infty).
\end{cases}
\end{equation}
\end{comment}
%
%
%
where $\gamma_0\neq 0$ and $\kappa_0\neq 0$ are material
coefficients, $\vec v$ is the vectorial gravitational potential $u$
is the medium velocity and $\vec {\tilde u}=\left(\gamma_0\vec
v+(1-\gamma_0)\vec u\right)$ is the speed-like vector field. Remind
that in the case of the vacuum we have $\gamma_0=\kappa_0=1$, $\vec
{\tilde u}=\vec v$ and equations \er{MaxVacFullPPNffGG} are precise
(in the frames of our model). Otherwise, in the case $\gamma_0\neq
1$ equations \er{MaxVacFullPPNffGG} are just an approximation that
is good enough for the case:
\begin{equation}\label{OprdddsimGGffyhjyhhtygrffgfzzjjj}
\frac{|1-\gamma_0|\cdot|\vec u-\vec v|^2}{c^2}\ll 1.
\end{equation}
Throughout this section we study equation \er{MaxVacFullPPNffGG} in
domains where we assume that the coefficients $\gamma_0\neq 0$ and
$\kappa_0\neq 0$ vary sufficiently slow on the place and time and
thus their spatial and temporal derivatives are negligible. Next
again by the third and the fourth equations in
\er{MaxVacFullPPNffGG} we can write
\begin{equation}\label{MaxVacFull1bjkgjhjhgjgjgkjfhjfdghghligioiuittrPPNgg}
\begin{cases}
\vec B\equiv curl_{\vec x} \vec A,\\
\vec E\equiv-\nabla_{\vec x}\Psi-\frac{1}{c}\frac{\partial\vec
A}{\partial t},
%,\\ div_{\vec x}\vec A\equiv 0,
\end{cases}
\end{equation}
where $\Psi$ and $\vec A$ are the usual scalar and the vectorial
electromagnetic potentials. Then by
\er{MaxVacFull1bjkgjhjhgjgjgkjfhjfdghghligioiuittrPPNgg} and
\er{MaxVacFullPPNffGG} we have
\begin{equation}\label{vhfffngghPPN333yuyuGG}
\begin{cases}
\vec B= curl_{\vec x} \vec A\\
\vec E=-\nabla_{\vec x}\Psi-\frac{1}{c}\frac{\partial\vec
A}{\partial t}\\
 \vec D=-\frac{1}{\gamma_0}\nabla_{\vec
x}\Psi-\frac{1}{\gamma_0 c}\frac{\partial\vec A}{\partial t}+\frac{1}{c\gamma_0}\vec {\tilde u}\times curl_{\vec x}\vec A\\
\vec H= \kappa_0 \,curl_{\vec x} \vec A+\frac{1}{c}\,\vec {\tilde
u}\times\left(-\frac{1}{\gamma_0}\nabla_{\vec
x}\Psi-\frac{1}{\gamma_0 c}\frac{\partial\vec A}{\partial
t}+\frac{1}{\gamma_0 c}\vec {\tilde u}\times curl_{\vec x}\vec
A\right).
%\\ div_{\vec x}\vec A\equiv 0.
\end{cases}
\end{equation}
Next we remind the definition of the proper scalar electromagnetic
potential:
\begin{equation}\label{vhfffngghhjghhgPPNghghghutghffugghjhjkjjklggkkk}
\Psi_0:=\Psi-\frac{1}{c}\vec A\cdot\vec v,
\end{equation}
and remind also that $\vec A$ is a proper vector field and $\Psi_0$
is a proper scalar field. Then in the case of the medium we also
define an additional scalar electromagnetic potential:
\begin{equation}\label{vhfffngghhjghhgPPNghghghutghffugghjhjkjjklgg}
\Psi_1:=\Psi-\frac{1}{c}\vec A\cdot\vec {\tilde u}.
\end{equation}
Then, since $\vec A$ is a proper vector field, we deduce that
$\Psi_1$ is also a proper scalar field. Moreover, in the case of the
vacuum or more generally in the case where $\gamma_0\approx 1$ we
have $\Psi_1=\Psi_0$. Thus by
\er{vhfffngghhjghhgPPNghghghutghffugghjhjkjjklgg} we rewrite
\er{vhfffngghPPN333yuyuGG} as:
\begin{equation}\label{vhfffngghPPNffGG}
\begin{cases}
\vec B= curl_{\vec x} \vec A\\
\vec E=-\nabla_{\vec x}\Psi_1-\frac{1}{c}\frac{\partial\vec
A}{\partial t}-\frac{1}{c}\nabla_{\vec x}\left(\vec A\cdot\vec
{\tilde u}\right)
%=-\nabla_{\vec x}\Psi-\frac{1}{c}\frac{\partial\vec A}{\partial t}
\\
 \vec D=-\frac{1}{\gamma_0}\nabla_{\vec
x}\Psi_1-\frac{1}{\gamma_0 c}\left(\frac{\partial\vec A}{\partial
t}-\vec {\tilde u}\times curl_{\vec x}\vec A+\nabla_{\vec
x}\left(\vec A\cdot\vec {\tilde u}\right)\right)
%=-\nabla_{\vec x}\Psi-\frac{1}{c}\frac{\partial\vec A}{\partial t}+\frac{1}{c}\vec v\times curl_{\vec x}\vec A
\\
\vec H= \kappa_0\,curl_{\vec x} \vec A-\frac{1}{c}\,\vec {\tilde
u}\times
%\left(-\nabla_{\vec x}\Psi-\frac{1}{c}\frac{\partial\vec A}{\partial t}+\frac{1}{c}\vec v\times curl_{\vec x}\vec A\right).
\left(\frac{1}{\gamma_0}\nabla_{\vec x}\Psi_1+\frac{1}{\gamma_0
c}\left(\frac{\partial\vec A}{\partial t}-\vec {\tilde u}\times
curl_{\vec x}\vec A+\nabla_{\vec x}\left(\vec A\cdot\vec {\tilde
u}\right)\right)\right).
%\\ div_{\vec x}\vec A\equiv 0.
\end{cases}
\end{equation}
%%%%%
Using Proposition \ref{yghgjtgyrtrt} we rewrite the third equation
in \er{vhfffngghPPNffGG} as
\begin{equation}\label{vhfffngghPPNffGG1}
\vec D=-\frac{1}{\gamma_0}\nabla_{\vec x}\Psi_1-\frac{1}{\gamma_0
c}\left(\frac{\partial\vec A}{\partial t}-curl_{\vec x}\left(\vec
{\tilde u}\times\vec A\right)+\left(div_{\vec x}\vec A\right)\vec
{\tilde u}+\left(d_{\vec x}\vec {\tilde u}+\left\{d_{\vec x}\vec
{\tilde u}\right\}^T\right)\cdot\vec A-\left(div_{\vec x}\vec
{\tilde u}\right)\vec A\right).
\end{equation}
Then by \er{vhfffngghPPNffGG1}, \er{vhfffngghPPNffGG} and
\er{MaxVacFullPPNffGG} we have
\begin{multline}\label{MaxVacFullPPNmmmffGG}
\frac{1}{\gamma_0 c}\left(\frac{\partial}{\partial t}\left(div_{\vec
x}\vec A\right)+div_{\vec x} \left\{\left(div_{\vec x}\vec
A\right)\vec {\tilde u}\right\}\right)\\+\frac{1}{\gamma_0
c}\,div_{\vec x} \left\{\left(d_{\vec x}\vec {\tilde
u}+\left\{d_{\vec x}\vec {\tilde u}\right\}^T\right)\cdot\vec
A-\left(div_{\vec x}\vec {\tilde u}\right)\vec
A\right\}+\frac{1}{\gamma_0}\,\Delta_{\vec x}\Psi_1=-4\pi\rho
\end{multline}
and
\begin{multline}\label{MaxVacFullPPNnnnffGG}
curl_{\vec x} \left\{\kappa_0\,curl_{\vec x} \vec
A-\frac{1}{\gamma_0 c}\,\vec {\tilde u}\times
%\left(-\nabla_{\vec x}\Psi-\frac{1}{c}\frac{\partial\vec A}{\partial t}+\frac{1}{c}\vec v\times curl_{\vec x}\vec A\right).
\left(\nabla_{\vec x}\Psi_1+\frac{1}{c}\left(\frac{\partial\vec
A}{\partial t}-\vec {\tilde u}\times curl_{\vec x}\vec
A+\nabla_{\vec x}\left(\vec A\cdot\vec {\tilde
u}\right)\right)\right)\right\}=\\
\frac{4\pi}{c}\vec j+\frac{1}{\gamma_0 c}\frac{\partial}{\partial
t}\left\{-\nabla_{\vec x}\Psi_1-\frac{1}{c}\left(\frac{\partial\vec
A}{\partial t}-\vec {\tilde u}\times curl_{\vec x}\vec
A+\nabla_{\vec x}\left(\vec A\cdot\vec {\tilde
u}\right)\right)\right\}.
\end{multline}
Then, we rewrite \er{MaxVacFullPPNmmmffGG} as:
\begin{multline}\label{MaxVacFullPPNmmmffffffhhGG}
-\frac{1}{\gamma_0 c}\left(\frac{\partial}{\partial
t}\left(div_{\vec x}\vec A\right)+div_{\vec x}
\left\{\left(div_{\vec x}\vec A\right)\vec {\tilde
u}\right\}\right)-\frac{1}{\gamma_0}\,\Delta_{\vec
x}\Psi_1\\=4\pi\rho+\frac{1}{\gamma_0 c}\,div_{\vec x}
\left\{\left(d_{\vec x}\vec {\tilde u}+\left\{d_{\vec x}\vec {\tilde
u}\right\}^T\right)\cdot\vec A-\left(div_{\vec x}\vec {\tilde
u}\right)\vec A\right\},
\end{multline}
and \er{MaxVacFullPPNnnnffGG} as:
\begin{multline}\label{MaxVacFullPPNnnnffffffhhGG}
-\kappa_0\,\Delta_{\vec x}\vec A-\frac{1}{\gamma_0 c^2}curl_{\vec x}
\left\{\vec {\tilde u}\times
%\left(-\nabla_{\vec x}\Psi-\frac{1}{c}\frac{\partial\vec A}{\partial t}+\frac{1}{c}\vec v\times curl_{\vec x}\vec A\right).
\left(\frac{\partial\vec A}{\partial t}-\vec {\tilde u}\times
curl_{\vec x}\vec A+\nabla_{\vec x}\left(\vec A\cdot\vec {\tilde
u}\right)\right)\right\}= \frac{4\pi}{c}\vec j\\-\frac{1}{\gamma_0
c^2}\frac{\partial}{\partial t}\left(\frac{\partial\vec A}{\partial
t}-\vec {\tilde u}\times curl_{\vec x}\vec A+\nabla_{\vec
x}\left(\vec A\cdot\vec {\tilde u}\right)\right)-\left(\nabla_{\vec
x}\left(\frac{1}{\gamma_0 c}\frac{\partial}{\partial
t}\Psi_1+\kappa_0\,div_{\vec x} \vec A\right)-\frac{1}{\gamma_0
c}curl_{\vec x} \left(\vec {\tilde u}\times
%\left(-\nabla_{\vec x}\Psi-\frac{1}{c}\frac{\partial\vec A}{\partial t}+\frac{1}{c}\vec v\times curl_{\vec x}\vec A\right).
\nabla_{\vec x}\Psi_1\right)\right).
\end{multline}
Then by \er{MaxVacFullPPNnnnffffffhhGG},
\er{MaxVacFullPPNmmmffffffhhGG} and \er{apfrm6} we deduce:
\begin{multline}\label{MaxVacFullPPNnnnffffffyuughhhGG}
-\kappa_0\,\Delta_{\vec x}\vec A-\frac{1}{\gamma_0 c^2}curl_{\vec x}
\left\{\vec {\tilde u}\times
%\left(-\nabla_{\vec x}\Psi-\frac{1}{c}\frac{\partial\vec A}{\partial t}+\frac{1}{c}\vec v\times curl_{\vec x}\vec A\right).
\left(\frac{\partial\vec A}{\partial t}-\vec {\tilde u}\times
curl_{\vec x}\vec A+\nabla_{\vec x}\left(\vec A\cdot\vec {\tilde
u}\right)\right)\right\}=\frac{4\pi}{c}\vec j\\-\nabla_{\vec
x}\left(\frac{1}{\gamma_0 c}\frac{\partial}{\partial
t}\Psi_1+\frac{1}{\gamma_0 c}\vec {\tilde u}\cdot\nabla_{\vec
x}\Psi_1+\kappa_0\,div_{\vec x} \vec A\right)-\frac{1}{\gamma_0
c^2}\frac{\partial}{\partial t}\left(\frac{\partial\vec A}{\partial
t}-\vec {\tilde u}\times curl_{\vec x}\vec A+\nabla_{\vec
x}\left(\vec A\cdot\vec {\tilde u}\right)\right)\\+\frac{1}{\gamma_0
c}\left(\nabla_{\vec x}\left(\vec {\tilde u}\cdot\nabla_{\vec
x}\Psi_1\right)+curl_{\vec x} \left(\vec {\tilde u}\times
%\left(-\nabla_{\vec x}\Psi-\frac{1}{c}\frac{\partial\vec A}{\partial t}+\frac{1}{c}\vec v\times curl_{\vec x}\vec A\right).
\nabla_{\vec x}\Psi_1\right)\right) =\frac{4\pi}{c}\vec
j\\-\nabla_{\vec x}\left(\frac{1}{\gamma_0
c}\frac{\partial}{\partial t}\Psi_1+\frac{1}{\gamma_0 c}\vec {\tilde
u}\cdot\nabla_{\vec x}\Psi_1+\kappa_0\,div_{\vec x} \vec
A\right)-\frac{1}{\gamma_0 c^2}\frac{\partial}{\partial
t}\left(\frac{\partial\vec A}{\partial t}-\vec {\tilde u}\times
curl_{\vec x}\vec A+\nabla_{\vec x}\left(\vec A\cdot\vec {\tilde
u}\right)\right)\\+\frac{1}{\gamma_0 c}\left(\left(d_{\vec x}\vec
{\tilde u}+\left\{d_{\vec x}\vec {\tilde u}\right\}^T\right)\cdot
\nabla_{\vec x}\Psi_1-\left(div_{\vec x}\vec {\tilde
u}\right)\nabla_{\vec x}\Psi_1\right)+\frac{1}{\gamma_0
c}\left(\Delta_{\vec x}\Psi_1\right)\vec {\tilde
u}=\frac{4\pi}{c}\left(\vec j-\rho\vec {\tilde
u}\right)\\-\nabla_{\vec x}\left(\frac{1}{\gamma_0
c}\frac{\partial}{\partial t}\Psi_1+\frac{1}{\gamma_0 c}\vec {\tilde
u}\cdot\nabla_{\vec x}\Psi_1+\kappa_0\,div_{\vec x} \vec A\right)
-\frac{1}{\gamma_0 c^2}\frac{\partial}{\partial
t}\left(\frac{\partial\vec A}{\partial t}-\vec {\tilde u}\times
curl_{\vec x}\vec A+\nabla_{\vec x}\left(\vec A\cdot\vec {\tilde
u}\right)\right)\\+\frac{1}{\gamma_0 c}\left(\left(d_{\vec x}\vec
{\tilde u}+\left\{d_{\vec x}\vec {\tilde u}\right\}^T\right)\cdot
\nabla_{\vec x}\Psi_1-\left(div_{\vec x}\vec {\tilde
u}\right)\nabla_{\vec x}\Psi_1\right)\\-\frac{1}{\gamma_0
c^2}\left(\left(\frac{\partial}{\partial t}\left(div_{\vec x}\vec
A\right)+div_{\vec x} \left\{\left(div_{\vec x}\vec A\right)\vec
{\tilde u}\right\}\right)+div_{\vec x} \left\{\left(d_{\vec x}\vec
{\tilde u}+\left\{d_{\vec x}\vec {\tilde
u}\right\}^T\right)\cdot\vec A-\left(div_{\vec x}\vec {\tilde
u}\right)\vec A\right\}\right)\vec {\tilde u}.
\end{multline}
So we have
\begin{multline}\label{MaxVacFullPPNnnnffffffyuughjhjhjhhjhhGG}
-\kappa_0\,\Delta_{\vec x}\vec A-\frac{1}{\gamma_0 c^2}curl_{\vec x}
\left\{\vec {\tilde u}\times
%\left(-\nabla_{\vec x}\Psi-\frac{1}{c}\frac{\partial\vec A}{\partial t}+\frac{1}{c}\vec v\times curl_{\vec x}\vec A\right).
\left(\frac{\partial\vec A}{\partial t}-\vec {\tilde u}\times
curl_{\vec x}\vec A+\nabla_{\vec x}\left(\vec A\cdot\vec {\tilde
u}\right)\right)\right\}=\frac{4\pi}{c}\left(\vec j-\rho\vec {\tilde
u}\right)\\-\nabla_{\vec x}\left(\frac{1}{\gamma_0
c}\frac{\partial}{\partial t}\Psi_1+\frac{1}{\gamma_0 c}\vec {\tilde
u}\cdot\nabla_{\vec x}\Psi_1+\kappa_0\,div_{\vec x} \vec
A\right)-\frac{1}{\gamma_0 c^2}\frac{\partial}{\partial
t}\left(\frac{\partial\vec A}{\partial t}-\vec {\tilde u}\times
curl_{\vec x}\vec A+\nabla_{\vec x}\left(\vec A\cdot\vec {\tilde
u}\right)\right)\\+\frac{1}{\gamma_0 c}\left(\left(d_{\vec x}\vec
{\tilde u}+\left\{d_{\vec x}\vec {\tilde u}\right\}^T\right)\cdot
\nabla_{\vec x}\Psi_1-\left(div_{\vec x}\vec {\tilde
u}\right)\nabla_{\vec x}\Psi_1\right)\\-\frac{1}{\gamma_0
c^2}\left(\left(\frac{\partial}{\partial t}\left(div_{\vec x}\vec
A\right)+div_{\vec x} \left\{\left(div_{\vec x}\vec A\right)\vec
{\tilde u}\right\}\right)+div_{\vec x} \left\{\left(d_{\vec x}\vec
{\tilde u}+\left\{d_{\vec x}\vec {\tilde
u}\right\}^T\right)\cdot\vec A-\left(div_{\vec x}\vec {\tilde
u}\right)\vec A\right\}\right)\vec {\tilde u},
\end{multline}
that we rewrite as
\begin{multline}\label{MaxVacFullPPNnnnffffffyuughjhjhjhhjjkjhkkjhhGG}
-\kappa_0\,\Delta_{\vec x}\vec A= \frac{4\pi}{c}\left(\vec
j-\rho\vec {\tilde u}\right)+\frac{1}{\gamma_0 c}\left(\left(d_{\vec
x}\vec {\tilde u}+\left\{d_{\vec x}\vec {\tilde
u}\right\}^T\right)\cdot \nabla_{\vec x}\Psi_1-\left(div_{\vec
x}\vec {\tilde u}\right)\nabla_{\vec x}\Psi_1\right)\\-\nabla_{\vec
x}\left(\frac{1}{\gamma_0 c}\frac{\partial}{\partial
t}\Psi_1+\frac{1}{\gamma_0 c}\vec {\tilde u}\cdot\nabla_{\vec
x}\Psi_1+\kappa_0\,div_{\vec x} \vec A\right)-\frac{1}{\gamma_0
c^2}\frac{\partial}{\partial t}\left(\frac{\partial\vec A}{\partial
t}-\vec {\tilde u}\times curl_{\vec x}\vec A+\nabla_{\vec
x}\left(\vec A\cdot\vec {\tilde u}\right)\right)\\+\frac{1}{\gamma_0
c^2}curl_{\vec x} \left\{\vec {\tilde u}\times
%\left(-\nabla_{\vec x}\Psi-\frac{1}{c}\frac{\partial\vec A}{\partial t}+\frac{1}{c}\vec v\times curl_{\vec x}\vec A\right).
\left(\frac{\partial\vec A}{\partial t}-\vec {\tilde u}\times
curl_{\vec x}\vec A+\nabla_{\vec x}\left(\vec A\cdot\vec {\tilde
u}\right)\right)\right\}\\-\frac{1}{\gamma_0 c^2}\left(div_{\vec
x}\left(\frac{\partial\vec A}{\partial t}-\vec {\tilde u}\times
curl_{\vec x}\vec A+\nabla_{\vec x}\left(\vec A\cdot\vec {\tilde
u}\right)\right)\right)\vec {\tilde u},
\end{multline}
Thus by \er{MaxVacFullPPNmmmffffffhhGG} we have
\begin{multline}\label{MaxVacFullPPNmmmffffffhhtygghGG}
-\frac{1}{c}\left(\frac{\partial}{\partial t}\left(div_{\vec x}\vec
A\right)+div_{\vec x} \left\{\left(div_{\vec x}\vec A\right)\vec
{\tilde u}\right\}\right)-\Delta_{\vec
x}\Psi_1\\=4\pi\gamma_0\rho+\frac{1}{c}\,div_{\vec x}
\left\{\left(d_{\vec x}\vec {\tilde u}+\left\{d_{\vec x}\vec {\tilde
u}\right\}^T\right)\cdot\vec A-\left(div_{\vec x}\vec {\tilde
u}\right)\vec A\right\},
\end{multline}
and by \er{MaxVacFullPPNnnnffffffyuughjhjhjhhjjkjhkkjhhGG} we have
\begin{multline}\label{MaxVacFullPPNnnnffffffyuughjhjhjhhjjkjhkkjhhjhghGG}
-\Delta_{\vec x}\vec A= \frac{4\pi}{\kappa_0 c}\left(\vec j-\rho\vec
{\tilde u}\right)+\frac{1}{\kappa_0\gamma_0 c}\left(\left(d_{\vec
x}\vec {\tilde u}+\left\{d_{\vec x}\vec {\tilde
u}\right\}^T\right)\cdot \nabla_{\vec x}\Psi_1-\left(div_{\vec
x}\vec {\tilde u}\right)\nabla_{\vec x}\Psi_1\right)\\-\nabla_{\vec
x}\left(\frac{1}{\kappa_0\gamma_0 c}\left(\frac{\partial}{\partial
t}\Psi_1+\vec {\tilde u}\cdot\nabla_{\vec x}\Psi_1\right)+div_{\vec
x} \vec A\right)-\frac{1}{\kappa_0\gamma_0
c^2}\frac{\partial}{\partial t}\left(\frac{\partial\vec A}{\partial
t}-\vec {\tilde u}\times curl_{\vec x}\vec A+\nabla_{\vec
x}\left(\vec A\cdot\vec {\tilde
u}\right)\right)\\+\frac{1}{\kappa_0\gamma_0 c^2}curl_{\vec x}
\left\{\vec {\tilde u}\times
%\left(-\nabla_{\vec x}\Psi-\frac{1}{c}\frac{\partial\vec A}{\partial t}+\frac{1}{c}\vec v\times curl_{\vec x}\vec A\right).
\left(\frac{\partial\vec A}{\partial t}-\vec {\tilde u}\times
curl_{\vec x}\vec A+\nabla_{\vec x}\left(\vec A\cdot\vec {\tilde
u}\right)\right)\right\}\\-\frac{1}{\kappa_0\gamma_0
c^2}\left(div_{\vec x}\left(\frac{\partial\vec A}{\partial t}-\vec
{\tilde u}\times curl_{\vec x}\vec A+\nabla_{\vec x}\left(\vec
A\cdot\vec {\tilde u}\right)\right)\right)\vec {\tilde u}.
\end{multline}


Next if we assume the following calibration of the potentials:
\begin{equation}\label{MaxVacFullPPNjjjjffhhGGGGGG}
div_{\vec x}\vec A=0,
\end{equation}
then by \er{MaxVacFullPPNjjjjffhhGGGGGG},
\er{MaxVacFullPPNmmmffffffhhtygghGG},
\er{MaxVacFullPPNnnnffffffyuughjhjhjhhjjkjhkkjhhjhghGG} and
\er{apfrm6} we have
\begin{equation}\label{MaxVacFullPPNmmmffffffhhtygghGGGG}
-\Delta_{\vec x}\Psi_1=4\pi\gamma_0\rho+\frac{1}{c}\,div_{\vec x}
\left\{\left(d_{\vec x}\vec {\tilde u}+\left\{d_{\vec x}\vec {\tilde
u}\right\}^T\right)\cdot\vec A-\left(div_{\vec x}\vec {\tilde
u}\right)\vec A\right\},
\end{equation}
and
\begin{multline}\label{MaxVacFullPPNnnnffffffyuughjhjhjhhjjkjhkkjhhjhghGGGG}
-\Delta_{\vec x}\vec A= \frac{4\pi}{\kappa_0 c}\left(\vec j-\rho\vec
{\tilde u}\right)+\frac{1}{\kappa_0\gamma_0 c}\left(\left(d_{\vec
x}\vec {\tilde u}+\left\{d_{\vec x}\vec {\tilde
u}\right\}^T\right)\cdot \nabla_{\vec x}\Psi_1-\left(div_{\vec
x}\vec {\tilde u}\right)\nabla_{\vec
x}\Psi_1\right)\\-\frac{1}{\kappa_0\gamma_0 c}\,\nabla_{\vec
x}\left(\frac{\partial}{\partial t}\Psi_1+\vec {\tilde
u}\cdot\nabla_{\vec x}\Psi_1\right)-\frac{1}{\kappa_0\gamma_0
c^2}\frac{\partial}{\partial t}\left(\frac{\partial\vec A}{\partial
t}-\vec {\tilde u}\times curl_{\vec x}\vec A+\nabla_{\vec
x}\left(\vec A\cdot\vec {\tilde
u}\right)\right)\\+\frac{1}{\kappa_0\gamma_0 c^2}curl_{\vec x}
\left\{\vec {\tilde u}\times
%\left(-\nabla_{\vec x}\Psi-\frac{1}{c}\frac{\partial\vec A}{\partial t}+\frac{1}{c}\vec v\times curl_{\vec x}\vec A\right).
\left(\frac{\partial\vec A}{\partial t}-\vec {\tilde u}\times
curl_{\vec x}\vec A+\nabla_{\vec x}\left(\vec A\cdot\vec {\tilde
u}\right)\right)\right\}\\-\frac{1}{\kappa_0\gamma_0
c^2}\left(div_{\vec x}\left(\frac{\partial\vec A}{\partial t}-\vec
{\tilde u}\times curl_{\vec x}\vec A+\nabla_{\vec x}\left(\vec
A\cdot\vec {\tilde u}\right)\right)\right)\vec {\tilde u}=
\\
%
%
%
\frac{4\pi}{\kappa_0 c}\left(\vec j-\rho\vec {\tilde
u}\right)-\frac{1}{\kappa_0\gamma_0 c}\left(\frac{\partial}{\partial
t}\left(\nabla_{\vec x}\Psi_1\right)-curl_{\vec x}\left(\vec {\tilde
u}\times\nabla_{\vec x}\Psi_1\right)+\left(\Delta_{\vec
x}\Psi_1\right)\vec {\tilde u}\right)\\-\frac{1}{\kappa_0\gamma_0
c^2}\frac{\partial}{\partial t}\left(\frac{\partial\vec A}{\partial
t}-\vec {\tilde u}\times curl_{\vec x}\vec A+\nabla_{\vec
x}\left(\vec A\cdot\vec {\tilde
u}\right)\right)\\+\frac{1}{\kappa_0\gamma_0 c^2}curl_{\vec x}
\left\{\vec {\tilde u}\times
%\left(-\nabla_{\vec x}\Psi-\frac{1}{c}\frac{\partial\vec A}{\partial t}+\frac{1}{c}\vec v\times curl_{\vec x}\vec A\right).
\left(\frac{\partial\vec A}{\partial t}-\vec {\tilde u}\times
curl_{\vec x}\vec A+\nabla_{\vec x}\left(\vec A\cdot\vec {\tilde
u}\right)\right)\right\}\\-\frac{1}{\kappa_0\gamma_0
c^2}\left(div_{\vec x}\left(\frac{\partial\vec A}{\partial t}-\vec
{\tilde u}\times curl_{\vec x}\vec A+\nabla_{\vec x}\left(\vec
A\cdot\vec {\tilde u}\right)\right)\right)\vec {\tilde u}.
\end{multline}
On the other hand, if we assume the following alternative
calibration of the potentials:
\begin{equation}\label{MaxVacFullPPNjjjjffhhGG}
\frac{1}{\kappa_0\gamma_0 c}\left(\frac{\partial\Psi_1}{\partial
t}+\vec {\tilde u}\cdot\nabla_{\vec x}\Psi_1\right)+div_{\vec x}\vec
A=0,
\end{equation}
then by \er{MaxVacFullPPNjjjjffhhGG},
\er{MaxVacFullPPNmmmffffffhhtygghGG} and
\er{MaxVacFullPPNnnnffffffyuughjhjhjhhjjkjhkkjhhjhghGG} we have
\begin{multline}\label{MaxVacFullPPNmmmffffffiuiuhjuGG}
\frac{1}{\kappa_0\gamma_0 c^2}\left(\frac{\partial}{\partial
t}\left(\frac{\partial\Psi_1}{\partial t}+\vec {\tilde
u}\cdot\nabla_{\vec x}\Psi_1\right)+div_{\vec x}
\left\{\left(\frac{\partial\Psi_1}{\partial t}+\vec {\tilde
u}\cdot\nabla_{\vec x}\Psi_1\right)\vec {\tilde
u}\right\}\right)-\Delta_{\vec
x}\Psi_1\\=4\pi\gamma_0\rho+\frac{1}{c}\,div_{\vec x}
\left\{\left(d_{\vec x}\vec {\tilde u}+\left\{d_{\vec x}\vec {\tilde
u}\right\}^T\right)\cdot\vec A-\left(div_{\vec x}\vec {\tilde
u}\right)\vec A\right\},
\end{multline}
and
\begin{multline}\label{MaxVacFullPPNnnnffffffyuughjhjhjhhjjkjhkkjhujgGG}
-\Delta_{\vec x}\vec A= \frac{4\pi}{\kappa_0 c}\left(\vec j-\rho\vec
{\tilde u}\right)+\frac{1}{\kappa_0\gamma_0 c}\left(\left(d_{\vec
x}\vec {\tilde u}+\left\{d_{\vec x}\vec {\tilde
u}\right\}^T\right)\cdot \nabla_{\vec x}\Psi_1-\left(div_{\vec
x}\vec {\tilde u}\right)\nabla_{\vec
x}\Psi_1\right)\\-\frac{1}{\kappa_0\gamma_0
c^2}\frac{\partial}{\partial t}\left(\frac{\partial\vec A}{\partial
t}-\vec {\tilde u}\times curl_{\vec x}\vec A+\nabla_{\vec
x}\left(\vec A\cdot\vec {\tilde
u}\right)\right)\\+\frac{1}{\kappa_0\gamma_0 c^2}curl_{\vec x}
\left\{\vec {\tilde u}\times
%\left(-\nabla_{\vec x}\Psi-\frac{1}{c}\frac{\partial\vec A}{\partial t}+\frac{1}{c}\vec v\times curl_{\vec x}\vec A\right).
\left(\frac{\partial\vec A}{\partial t}-\vec {\tilde u}\times
curl_{\vec x}\vec A+\nabla_{\vec x}\left(\vec A\cdot\vec {\tilde
u}\right)\right)\right\}\\-\frac{1}{\kappa_0\gamma_0
c^2}\left(div_{\vec x}\left(\frac{\partial\vec A}{\partial t}-\vec
{\tilde u}\times curl_{\vec x}\vec A+\nabla_{\vec x}\left(\vec
A\cdot\vec {\tilde u}\right)\right)\right)\vec {\tilde u}.
\end{multline}



In particular, assume that we have the following approximation: if
the changes in space of the physical characteristics of the
electromagnetic fields become essential in the spatial landscape
$L_e$ and the changes in space of the field $\vec {\tilde u}$
becomes essential in the spatial landscape $L_{u}$, then we assume
\begin{equation}\label{MaxVacFullPPNmmmffffffhhtygghGGGGyuhggghgh}
L_e\ll L_u,\quad\text{or equivalently:}\quad \frac{|d_{\vec x}\vec
{\tilde u}|}{|\vec {\tilde u}|}\ll\frac{|d_{\vec x}\vec A|}{|\vec
A|}\quad\text{and}\quad\frac{|d_{\vec x}\vec {\tilde u}|}{|\vec
{\tilde u}|}\ll\frac{|\nabla_{\vec x}\Psi_1|}{|\Psi_1|}.
\end{equation}
i.e. the field $\vec {\tilde u}$ vary in space much weaker then
$\vec A$ and $\Psi_1$. Estimation
\er{MaxVacFullPPNmmmffffffhhtygghGGGGyuhggghgh} holds especially
good for the electromagnetic waves of high frequency for example for
the visible light. However,
\er{MaxVacFullPPNmmmffffffhhtygghGGGGyuhggghgh} is still well for
almost every electromagnetic field we meet in the common life,
except probably the magnetic field of the Earth. Then, taking into
the account \er{MaxVacFullPPNmmmffffffhhtygghGGGGyuhggghgh}, under
the calibration \er{MaxVacFullPPNjjjjffhhGGGGGG}, we rewrite
\er{MaxVacFullPPNmmmffffffhhtygghGGGG} and
\er{MaxVacFullPPNnnnffffffyuughjhjhjhhjjkjhkkjhhjhghGGGG} as
\begin{equation}\label{MaxVacFullPPNmmmffffffhhtygghGGGGgg}
-\Delta_{\vec x}\Psi_1\approx 4\pi\gamma_0\rho,
\end{equation}
and
\begin{multline}\label{MaxVacFullPPNnnnffffffyuughjhjhjhhjjkjhkkjhhjhghGGGGgg}
-\Delta_{\vec x}\vec A\approx
%
%
%
\frac{4\pi}{\kappa_0 c}\left(\vec j-\rho\vec {\tilde
u}\right)-\frac{1}{\kappa_0\gamma_0 c}\left(\frac{\partial}{\partial
t}\left(\nabla_{\vec x}\Psi_1\right)-curl_{\vec x}\left(\vec {\tilde
u}\times\nabla_{\vec x}\Psi_1\right)+\left(\Delta_{\vec
x}\Psi_1\right)\vec {\tilde u}\right)\\-\frac{1}{\kappa_0\gamma_0
c^2}\frac{\partial}{\partial t}\left(\frac{\partial\vec A}{\partial
t}-\vec {\tilde u}\times curl_{\vec x}\vec A+\nabla_{\vec
x}\left(\vec A\cdot\vec {\tilde
u}\right)\right)\\+\frac{1}{\kappa_0\gamma_0 c^2}curl_{\vec x}
\left\{\vec {\tilde u}\times
%\left(-\nabla_{\vec x}\Psi-\frac{1}{c}\frac{\partial\vec A}{\partial t}+\frac{1}{c}\vec v\times curl_{\vec x}\vec A\right).
\left(\frac{\partial\vec A}{\partial t}-\vec {\tilde u}\times
curl_{\vec x}\vec A+\nabla_{\vec x}\left(\vec A\cdot\vec {\tilde
u}\right)\right)\right\}\\-\frac{1}{\kappa_0\gamma_0
c^2}\left(div_{\vec x}\left(\frac{\partial\vec A}{\partial t}-\vec
{\tilde u}\times curl_{\vec x}\vec A+\nabla_{\vec x}\left(\vec
A\cdot\vec {\tilde u}\right)\right)\right)\vec {\tilde u}=\\
\frac{4\pi}{\kappa_0 c}\vec j-\frac{1}{\kappa_0\gamma_0
c}\left(\frac{\partial}{\partial t}\left(\nabla_{\vec
x}\Psi_1\right)-curl_{\vec x}\left(\vec {\tilde u}\times\nabla_{\vec
x}\Psi_1\right)\right)\\-\frac{1}{\kappa_0\gamma_0
c^2}\frac{\partial}{\partial t}\left(\frac{\partial\vec A}{\partial
t}-\vec {\tilde u}\times curl_{\vec x}\vec A+\nabla_{\vec
x}\left(\vec A\cdot\vec {\tilde
u}\right)\right)\\+\frac{1}{\kappa_0\gamma_0 c^2}curl_{\vec x}
\left\{\vec {\tilde u}\times
%\left(-\nabla_{\vec x}\Psi-\frac{1}{c}\frac{\partial\vec A}{\partial t}+\frac{1}{c}\vec v\times curl_{\vec x}\vec A\right).
\left(\frac{\partial\vec A}{\partial t}-\vec {\tilde u}\times
curl_{\vec x}\vec A+\nabla_{\vec x}\left(\vec A\cdot\vec {\tilde
u}\right)\right)\right\}\\-\frac{1}{\kappa_0\gamma_0
c^2}\left(div_{\vec x}\left(\frac{\partial\vec A}{\partial t}-\vec
{\tilde u}\times curl_{\vec x}\vec A+\nabla_{\vec x}\left(\vec
A\cdot\vec {\tilde u}\right)\right)\right)\vec {\tilde u}.
\end{multline}
Note that, using Proposition \ref{yghgjtgyrtrt} we deduce that the
approximate equations \er{MaxVacFullPPNmmmffffffhhtygghGGGGgg} and
\er{MaxVacFullPPNnnnffffffyuughjhjhjhhjjkjhkkjhhjhghGGGGgg} are
still invariant under the change of inertial or non-inertial
cartesian coordinate system, provided that $\vec A$ is a proper
vector field and $\Psi_1$ is a proper scalar field. So we can use
approximate equations \er{MaxVacFullPPNmmmffffffhhtygghGGGGgg} and
\er{MaxVacFullPPNnnnffffffyuughjhjhjhhjjkjhkkjhhjhghGGGGgg} in the
coordinate system $(*)$ even if
\er{MaxVacFullPPNmmmffffffhhtygghGGGGyuhggghgh} is not satisfied in
the system $(*)$, provided that
\er{MaxVacFullPPNmmmffffffhhtygghGGGGyuhggghgh} is satisfied in
another system $(**)$.


On the other hand, taking into the account
\er{MaxVacFullPPNmmmffffffhhtygghGGGGyuhggghgh}, under the
calibration \er{MaxVacFullPPNjjjjffhhGG}, we rewrite
\er{MaxVacFullPPNmmmffffffiuiuhjuGG} and
\er{MaxVacFullPPNnnnffffffyuughjhjhjhhjjkjhkkjhujgGG} as
\begin{equation}\label{MaxVacFullPPNmmmffffffiuiuhjuGGgg}
\frac{1}{\kappa_0\gamma_0 c^2}\left(\frac{\partial}{\partial
t}\left(\frac{\partial\Psi_1}{\partial t}+\vec {\tilde
u}\cdot\nabla_{\vec x}\Psi_1\right)+div_{\vec x}
\left\{\left(\frac{\partial\Psi_1}{\partial t}+\vec {\tilde
u}\cdot\nabla_{\vec x}\Psi_1\right)\vec {\tilde
u}\right\}\right)-\Delta_{\vec x}\Psi_1\approx 4\pi\gamma_0\rho.
\end{equation}
and
\begin{multline}\label{MaxVacFullPPNnnnffffffyuughjhjhjhhjjkjhkkjhujgGGgg}
-\Delta_{\vec x}\vec A\approx \frac{4\pi}{\kappa_0 c}\left(\vec
j-\rho\vec {\tilde u}\right)-\frac{1}{\kappa_0\gamma_0
c^2}\frac{\partial}{\partial t}\left(\frac{\partial\vec A}{\partial
t}-\vec {\tilde u}\times curl_{\vec x}\vec A+\nabla_{\vec
x}\left(\vec A\cdot\vec {\tilde
u}\right)\right)\\+\frac{1}{\kappa_0\gamma_0 c^2}curl_{\vec x}
\left\{\vec {\tilde u}\times
%\left(-\nabla_{\vec x}\Psi-\frac{1}{c}\frac{\partial\vec A}{\partial t}+\frac{1}{c}\vec v\times curl_{\vec x}\vec A\right).
\left(\frac{\partial\vec A}{\partial t}-\vec {\tilde u}\times
curl_{\vec x}\vec A+\nabla_{\vec x}\left(\vec A\cdot\vec {\tilde
u}\right)\right)\right\}\\-\frac{1}{\kappa_0\gamma_0
c^2}\left(div_{\vec x}\left(\frac{\partial\vec A}{\partial t}-\vec
{\tilde u}\times curl_{\vec x}\vec A+\nabla_{\vec x}\left(\vec
A\cdot\vec {\tilde u}\right)\right)\right)\vec {\tilde u}.
\end{multline}
Again note that, using Proposition \ref{yghgjtgyrtrt} we deduce that
the approximate equations \er{MaxVacFullPPNmmmffffffiuiuhjuGGgg} and
\er{MaxVacFullPPNnnnffffffyuughjhjhjhhjjkjhkkjhujgGGgg} are still
invariant under the change of inertial or non-inertial cartesian
coordinate system, provided that $\vec A$ is a proper vector field
and $\Psi_1$ is a proper scalar field. So we can use approximate
equations \er{MaxVacFullPPNmmmffffffiuiuhjuGGgg} and
\er{MaxVacFullPPNnnnffffffyuughjhjhjhhjjkjhkkjhujgGGgg} in the
coordinate system $(*)$ even if
\er{MaxVacFullPPNmmmffffffhhtygghGGGGyuhggghgh} is not satisfied in
the system $(*)$, provided that
\er{MaxVacFullPPNmmmffffffhhtygghGGGGyuhggghgh} is satisfied in
another system $(**)$.



Finally note that by \er{MaxVacFullPPNmmmffffffiuiuhjuGGgg},
\er{MaxVacFullPPNnnnffffffyuughjhjhjhhjjkjhkkjhujgGGgg} and
\er{MaxVacFullPPNmmmffffffhhtygghGGGGyuhggghgh} we can write the
further approximating equations:
\begin{equation}\label{MaxVacFullPPNmmmffffffiuiuhjuGGggFG}
\frac{1}{c^2_0}\left(\frac{\partial}{\partial
t}\left(\frac{\partial\Psi_1}{\partial t}+\vec {\tilde
u}\cdot\nabla_{\vec x}\Psi_1\right)+div_{\vec x}
\left\{\left(\frac{\partial\Psi_1}{\partial t}+\vec {\tilde
u}\cdot\nabla_{\vec x}\Psi_1\right)\vec {\tilde
u}\right\}\right)-\Delta_{\vec x}\Psi_1 \approx 4\pi\gamma_0\rho,
\end{equation}
and
\begin{equation}\label{MaxVacFullPPNnnnffffffyuughjhjhjhhjjkjhkkjhujgGGggFG}
\frac{1}{c^2_0}\left(\frac{\partial}{\partial
t}\left(\frac{\partial\vec A}{\partial t}+d_{\vec x}\vec A\cdot\vec
{\tilde u}\right)+div_{\vec x} \left\{\left(\frac{\partial\vec
A}{\partial t}+d_{\vec x}\vec A\cdot\vec {\tilde
u}\right)\otimes\vec {\tilde u}\right\}\right)-\Delta_{\vec x}\vec
A\approx \frac{4\pi}{\kappa_0 c}\left(\vec j-\rho\vec {\tilde
u}\right),
\end{equation}
where the scalar quantity $c_0$, defined by
\begin{equation}\label{gughhghfbvnbv}
c_0=c\sqrt{\kappa_0\gamma_0},
\end{equation}
is called speed of light in the medium. Note that, although the
approximate equations \er{MaxVacFullPPNmmmffffffiuiuhjuGGggFG} and
\er{MaxVacFullPPNnnnffffffyuughjhjhjhhjjkjhkkjhujgGGggFG} are
invariant under the Galilean Transformation, they are not invariant
under the more general change of non-inertial cartesian coordinate
system. However, \er{MaxVacFullPPNmmmffffffiuiuhjuGGggFG} and
\er{MaxVacFullPPNnnnffffffyuughjhjhjhhjjkjhkkjhujgGGggFG} are more
convenient then \er{MaxVacFullPPNmmmffffffiuiuhjuGGgg} and
\er{MaxVacFullPPNnnnffffffyuughjhjhjhhjjkjhkkjhujgGGgg}, since the
scalar potential $\Psi_1$ and every of the three scalar components
of the vector potential $\vec A$ in
\er{MaxVacFullPPNmmmffffffiuiuhjuGGggFG} and
\er{MaxVacFullPPNnnnffffffyuughjhjhjhhjjkjhkkjhujgGGggFG} satisfies
four decoupled equations of the same type, that differ only by the
right parts.

In the absence of charges and currents (for example for
electromagnetic waves) equations
\er{MaxVacFullPPNmmmffffffiuiuhjuGGggFG} and
\er{MaxVacFullPPNnnnffffffyuughjhjhjhhjjkjhkkjhujgGGggFG} become:
\begin{equation}\label{MaxVacFullPPNmmmffffffiuiuhjuGGggFGel}
\frac{1}{c^2_0}\left(\frac{\partial}{\partial
t}\left(\frac{\partial\Psi_1}{\partial t}+\vec {\tilde
u}\cdot\nabla_{\vec x}\Psi_1\right)+div_{\vec x}
\left\{\left(\frac{\partial\Psi_1}{\partial t}+\vec {\tilde
u}\cdot\nabla_{\vec x}\Psi_1\right)\vec {\tilde
u}\right\}\right)-\Delta_{\vec x}\Psi_1=0,
\end{equation}
and
\begin{equation}\label{MaxVacFullPPNnnnffffffyuughjhjhjhhjjkjhkkjhujgGGggFGel}
\frac{1}{c^2_0}\left(\frac{\partial}{\partial
t}\left(\frac{\partial\vec A}{\partial t}+d_{\vec x}\vec A\cdot\vec
{\tilde u}\right)+div_{\vec x} \left\{\left(\frac{\partial\vec
A}{\partial t}+d_{\vec x}\vec A\cdot\vec {\tilde
u}\right)\otimes\vec {\tilde u}\right\}\right)-\Delta_{\vec x}\vec
A=0.
\end{equation}
Therefore, by \er{vhfffngghPPNffGG}, differentiating
\er{MaxVacFullPPNmmmffffffiuiuhjuGGggFGel} and
\er{MaxVacFullPPNnnnffffffyuughjhjhjhhjjkjhkkjhujgGGggFGel} and
further usage of \er{MaxVacFullPPNmmmffffffhhtygghGGGGyuhggghgh}
gives that if the scalar field $U:=U(\vec x,t)$ is one of the three
scalar components of every of the fields $\vec E$, $\vec B$, $\vec
D$ or $\vec H$, then $U$ satisfies the following approximate scalar
equation of the wave type:
\begin{equation}\label{MaxVacFullPPNmmmffffffiuiuhjuGGggFGelGHGHGHGG}
\frac{1}{c^2_0}\left(\frac{\partial}{\partial t}\left(\frac{\partial
U}{\partial t}+\vec {\tilde u}\cdot\nabla_{\vec x}U\right)+div_{\vec
x} \left\{\left(\frac{\partial U}{\partial t}+\vec {\tilde
u}\cdot\nabla_{\vec x} U\right)\vec {\tilde
u}\right\}\right)-\Delta_{\vec x}U \approx 0,
\end{equation}
where,
\begin{equation}\label{uyuyuyy}
\vec {\tilde u}=\left(\gamma_0\vec v+(1-\gamma_0)\vec u\right)\,.
\end{equation}














\subsection{The case of quasistationary electromagnetic fields inside a slowly moving medium in a weak gravitational
field}\label{qfCM} Assume that in the given inertial or non-inertial
cartesian coordinate system $(*)$ the field $\vec {\tilde u}$ is
weak, meaning that at any instant on every point:
\begin{equation}\label{slowaetherGGaa}
\frac{1}{\kappa_0\gamma_0}\frac{|\vec {\tilde u}|^2}{c^2}\,\ll\, 1.
%\frac{|\vec v|^2}{c^2}\ll 1
\end{equation}
Here $\vec {\tilde u}=\left(\gamma_0\vec v+(1-\gamma_0)\vec
u\right)$ is the speed-like vector field, where $\vec v$ is a
vectorial gravitational potential in the system $(*)$ and $\vec u$
is the medium velocity. Furthermore, consider quasistationary
electromagnetic fields. This means the following: assume that the
changes in time of the physical characteristics of the
electromagnetic fields become essential after certain interval of
time $T_e$ and the changes in space of the physical characteristics
of the fields become essential in the spatial landscape $L_e$. Then
we assume that
\begin{equation}\label{slochangGGaa}
(\kappa_0\gamma_0)\frac{c^2T^2_e}{L^2_e}\,\gg\, 1.
\end{equation}
Next assume that we are under the calibration
\er{MaxVacFullPPNjjjjffhhGGGGGG}. Then by \er{slowaetherGGaa} and
\er{slochangGGaa} we rewrite \er{MaxVacFullPPNmmmffffffhhtygghGGGG}
and \er{MaxVacFullPPNnnnffffffyuughjhjhjhhjjkjhkkjhhjhghGGGG} as
\begin{equation}\label{MaxVacFullPPNmmmffffffhhtygghGGGGaa}
-\Delta_{\vec x}\Psi_1=4\pi\gamma_0\rho+\frac{1}{c}\,div_{\vec x}
\left\{\left(d_{\vec x}\vec {\tilde u}+\left\{d_{\vec x}\vec {\tilde
u}\right\}^T\right)\cdot\vec A-\left(div_{\vec x}\vec {\tilde
u}\right)\vec A\right\},
\end{equation}
and
\begin{equation}\label{MaxVacFullPPNnnnffffffyuughjhjhjhhjjkjhkkjhhjhghGGGGaa1}
-\Delta_{\vec x}\vec A\approx \frac{4\pi}{\kappa_0 c}\left(\vec
j-\rho\vec {\tilde u}\right)-\frac{1}{\kappa_0\gamma_0
c}\left(\frac{\partial}{\partial t}\left(\nabla_{\vec
x}\Psi_1\right)-curl_{\vec x}\left(\vec {\tilde u}\times\nabla_{\vec
x}\Psi_1\right)+\left(\Delta_{\vec x}\Psi_1\right)\vec {\tilde
u}\right).
%
%
%
\end{equation}
Moreover, by \er{slowaetherGGaa} and \er{slochangGGaa} we can
perform further approximation of
\er{MaxVacFullPPNnnnffffffyuughjhjhjhhjjkjhkkjhhjhghGGGGaa1} and we
get
\begin{multline}\label{MaxVacFullPPNnnnffffffyuughjhjhjhhjjkjhkkjhhjhghGGGGaa}
-\Delta_{\vec x}\vec A\approx\frac{4\pi}{\kappa_0 c}\left(\vec
j-\rho\vec {\tilde u}\right)-\frac{1}{\kappa_0\gamma_0
c}\left(\frac{\partial}{\partial t}\left(\nabla_{\vec
x}\Psi_1\right)-curl_{\vec x}\left(\vec {\tilde u}\times\nabla_{\vec
x}\Psi_1\right)+\left(\Delta_{\vec x}\Psi_1\right)\vec {\tilde
u}\right)
\\\approx\frac{4\pi}{\kappa_0 c}\,\vec
j-\frac{1}{\kappa_0 c}\left(\frac{\partial}{\partial
t}\left(\nabla_{\vec x}\psi_0\right)-curl_{\vec x}\left(\vec
v\times\nabla_{\vec x}\psi_0\right)\right),
\end{multline}
where $\psi_0(\vec x,t)$ is the classical Coulomb's potential which
satisfies
\begin{equation}\label{columbPPNaa}
-\Delta_{\vec x}\psi_0\equiv 4\pi\rho.
\end{equation}
So we rewrite \er{MaxVacFullPPNmmmffffffhhtygghGGGGaa} and
\er{MaxVacFullPPNnnnffffffyuughjhjhjhhjjkjhkkjhhjhghGGGGaa} as
\begin{equation}\label{MaxVacFull1bjkgjhjhgjgjgkjfhjfdghcgjhhjgkgkgugyyurhjfffhfjklhhhgkjgGGaaKK}
\begin{cases}
-\Delta_{\vec x}\vec A \approx\frac{4\pi}{\kappa_0 c}\vec {\widetilde j},\\
%\quad\text{for}\;\;(\vec x,t)\in\R^3\times[0,+\infty),\\
-\Delta_{\vec x}\Psi_1=4\pi\gamma_0\rho+\frac{1}{c}\,div_{\vec x}
\left\{\left(d_{\vec x}\vec {\tilde u}+\left\{d_{\vec x}\vec {\tilde
u}\right\}^T\right)\cdot\vec A-\left(div_{\vec x}\vec {\tilde
u}\right)\vec A\right\},
\end{cases}
\end{equation}
where we set the reduced current:
\begin{equation}\label{reducedcurrentfhfhjfhjGGaa}
\begin{cases}
\vec {\widetilde j}:=\vec j-\frac{1}{4\pi}\frac{\partial}{\partial
t} \left(\nabla_{\vec x}\psi_0\right)+\frac{1}{4\pi}curl_{\vec
x}\left(\vec {\tilde u}\times \nabla_{\vec x}\psi_0\right),\\
-\Delta_{\vec x}\psi_0= 4\pi\rho.
\end{cases}
\end{equation}
Note that by the Continuum Equation of the Conservation of Charges:
\begin{equation}\label{toksohraneniezarjadaPPNaa}
\frac{\partial\rho}{\partial t}+div_{\vec x}\vec j\equiv 0,
\end{equation}
the reduced current clearly satisfies:
\begin{equation}\label{divreducedcurrentPPNaa}
div_{\vec x}\vec {\widetilde j}\equiv 0.
\end{equation}
Moreover,  by \er{reducedcurrentfhfhjfhjGGaa} we clearly have
\begin{equation}\label{reducedcurrentPPNuighjhjaa}
\vec {\widetilde j}:=(\vec j-\rho\vec {\tilde
u})-\frac{1}{4\pi}\left(\frac{\partial}{\partial t}
\left(\nabla_{\vec x}\psi_0\right)-curl_{\vec x}\left(\vec {\tilde
u}\times \nabla_{\vec x}\psi_0\right)+\left(div_{\vec
x}\left\{\nabla_{\vec x}\psi_0\right\}\right)\vec {\tilde u}\right),
\end{equation}
and thus, by \er{reducedcurrentPPNuighjhjaa}, using Proposition
\ref{yghgjtgyrtrt}
%from the Appendix
we deduce that $\vec {\widetilde j}$ is a proper vector field.
Moreover, the approximate vectorial electromagnetic potential $\vec
A$ from
\er{MaxVacFull1bjkgjhjhgjgjgkjfhjfdghcgjhhjgkgkgugyyurhjfffhfjklhhhgkjgGGaaKK}
clearly satisfies:
\begin{equation}\label{MaxVacFull1bjkgjhjhgjgjgkjfhjfdghcgjhhjgkgkgugyyurkkkGGGGGaa}
div_{\vec x}\vec A=0.
\end{equation}
Next, since by \er{vhfffngghhjghhgPPNghghghutghffugghjhjkjjklgg} we
have:
\begin{equation}\label{vhfffngghhjghhgPPNghghghutghffugghjhjkjjklgghkhhh}
\Psi_1:=\Psi-\frac{1}{c}\vec A\cdot\vec {\tilde u},
\end{equation}
and, since by
\er{MaxVacFull1bjkgjhjhgjgjgkjfhjfdghcgjhhjgkgkgugyyurkkkGGGGGaa},
\er{apfrm6} and \er{apfrm9} we have:
\begin{multline}\label{MaxVacFullPPNnnnffffffyuughjhjhjhhjjkjhkkjhhjhghGGGGaajjkkjkjlj}
div_{\vec x} \left\{\left(d_{\vec x}\vec {\tilde u}+\left\{d_{\vec
x}\vec {\tilde u}\right\}^T\right)\cdot\vec A-\left(div_{\vec x}\vec
{\tilde u}\right)\vec A\right\}-\Delta_{\vec x}\left(\vec A\cdot\vec
{\tilde u}\right)=\\div_{\vec x} \left\{\left(d_{\vec x}\vec {\tilde
u}+\left\{d_{\vec x}\vec {\tilde u}\right\}^T\right)\cdot\vec
A-\left(div_{\vec x}\vec {\tilde u}\right)\vec A-\nabla_{\vec
x}\left(\vec A\cdot\vec {\tilde u}\right)\right\}=\\ div_{\vec x}
\left\{d_{\vec x}\vec {\tilde u}\cdot\vec A-d_{\vec x}\vec
A\cdot\vec {\tilde u}+\left(div_{\vec x}\vec A\right)\vec {\tilde
u}-\left(div_{\vec x}\vec {\tilde u}\right)\vec A-\vec {\tilde
u}\times curl_{\vec x}\vec A\right\}\\= div_{\vec x}
\left\{curl_{\vec x}\left(\vec {\tilde u}\times\vec A\right)-\vec
{\tilde u}\times curl_{\vec x}\vec A\right\}= -div_{\vec x}
\left\{\vec {\tilde u}\times curl_{\vec x}\vec A\right\},
\end{multline}
we rewrite
\er{MaxVacFull1bjkgjhjhgjgjgkjfhjfdghcgjhhjgkgkgugyyurhjfffhfjklhhhgkjgGGaaKK}
as:
\begin{equation}\label{MaxVacFull1bjkgjhjhgjgjgkjfhjfdghcgjhhjgkgkgugyyurhjfffhfjklhhhgkjgGGaa}
\begin{cases}
-\Delta_{\vec x}\vec A \approx\frac{4\pi}{\kappa_0 c}\vec {\widetilde j},\\
%\quad\text{for}\;\;(\vec x,t)\in\R^3\times[0,+\infty),\\
-\Delta_{\vec x}\Psi= 4\pi\gamma_0\rho-\frac{1}{c}\,div_{\vec
x}\left(\vec {\tilde u}\times curl_{\vec x} \vec A\right).
\end{cases}
\end{equation}
where
\begin{equation}\label{reducedcurrentfhfhjfhjGGaakklkl}
\begin{cases}
\vec {\widetilde j}:=\vec j-\frac{1}{4\pi}\frac{\partial}{\partial
t} \left(\nabla_{\vec x}\psi_0\right)+\frac{1}{4\pi}curl_{\vec
x}\left(\vec {\tilde u}\times \nabla_{\vec x}\psi_0\right),\\
-\Delta_{\vec x}\psi_0= 4\pi\rho.
\end{cases}
\end{equation}
So in order to find the scalar and the vectorial electromagnetic
potentials we just need to solve Laplace equations. Knowing the
approximate electromagnetic potentials by \er{vhfffngghPPN333yuyuGG}
we can find the approximations of of the electromagnetic fields:
\begin{equation}\label{MaxVacFull1bjkgjhjhgjgjgkjfhjfdghghligioiuittrhiguffGGaa}
\begin{cases}
\vec B= curl_{\vec x} \vec A\\
\vec E=-\nabla_{\vec x}\Psi-\frac{1}{c}\frac{\partial\vec
A}{\partial t}\\
 \vec D=-\frac{1}{\gamma_0}\nabla_{\vec
x}\Psi-\frac{1}{\gamma_0 c}\frac{\partial\vec A}{\partial t}+\frac{1}{c\gamma_0}\vec {\tilde u}\times curl_{\vec x}\vec A\\
\vec H=\kappa_0 \,curl_{\vec x} \vec A+\frac{1}{c}\,\vec {\tilde
u}\times\left(-\frac{1}{\gamma_0}\nabla_{\vec
x}\Psi-\frac{1}{\gamma_0 c}\frac{\partial\vec A}{\partial
t}+\frac{1}{\gamma_0 c}\vec {\tilde u}\times curl_{\vec x}\vec
A\right),
\end{cases}
\end{equation}
where $\Psi$ and $\vec A$ are given by
\er{MaxVacFull1bjkgjhjhgjgjgkjfhjfdghcgjhhjgkgkgugyyurhjfffhfjklhhhgkjgGGaa}.
Note also that, since $\vec {\widetilde j}$ is a proper vector
field, by Proposition \ref{yghgjtgyrtrt} we deduce that equations
\er{MaxVacFull1bjkgjhjhgjgjgkjfhjfdghcgjhhjgkgkgugyyurhjfffhfjklhhhgkjgGGaaKK}
and thus also equations
\er{MaxVacFull1bjkgjhjhgjgjgkjfhjfdghcgjhhjgkgkgugyyurhjfffhfjklhhhgkjgGGaa}
are invariant under the change of non-inertial cartesian coordinate
system, provided that $\vec A$ is a proper vector field and
$\Psi_1=\Psi-\frac{1}{c}\vec A\cdot\vec {\tilde u}$ is a proper
scalar field. So the approximate solutions in the case of
quasistationary fields in a weak gravitational field satisfy the
same transformation as the exact solutions of Maxwell Equations.
%(see \er{yuythfgfyftydtydtydtyddyyyhhddhhh}).
Therefore, if in coordinate system $(*)$ we can use the approximate
equations, given by
\er{MaxVacFull1bjkgjhjhgjgjgkjfhjfdghcgjhhjgkgkgugyyurhjfffhfjklhhhgkjgGGaa}
and \er{MaxVacFull1bjkgjhjhgjgjgkjfhjfdghghligioiuittrhiguffGGaa},
%, given by \er{MaxVacFull1bjkgjhjhgjgjgkjfhjfdghcgjhhjgkgkgugyyurhjfffhfjklhhhgkjggjgjuiyuuijk}
%and \er{MaxVacFull1bjkgjhjhgjgjgkjfhjfdghghligioiuittrhiguffgjggjgkhkk},
then we can use the similar approximation
%, given by \er{MaxVacFull1bjkgjhjhgjgjgkjfhjfdghcgjhhjgkgkgugyyurhjfffhfjklhhhgkjggjgj}
%and \er{MaxVacFull1bjkgjhjhgjgjgkjfhjfdghghligioiuittrhiguffgjghlkk}
also in coordinate system $(**)$, even in the case when in system
$(**)$ \er{slowaetherGGaa} or \er{slochangGGaa} are not satisfied.
\begin{remark}
The solutions of
\er{MaxVacFull1bjkgjhjhgjgjgkjfhjfdghcgjhhjgkgkgugyyurhjfffhfjklhhhgkjgGGaa}
and \er{MaxVacFull1bjkgjhjhgjgjgkjfhjfdghghligioiuittrhiguffGGaa}
satisfy the following equations:
\begin{equation}\label{MaxVacFull1bjkgjhjhgjaaajhfghhgGGaa}
\begin{cases}
curl_{\vec x} \left(\kappa_0\vec B+\frac{1}{c}\,\vec {\tilde
u}\times \left(- \nabla_{\vec x}\psi_0\right)\right)\equiv
\frac{4\pi}{c}\vec j+\frac{1}{c}\frac{\partial (-
\nabla_{\vec x}\psi_0)}{\partial t},\\
div_{\vec x} \vec D=4\pi\rho,\\
curl_{\vec x} \vec E+\frac{1}{c}\frac{\partial \vec B}{\partial t}=0,\\
div_{\vec x} \vec B=0\\
\vec E=\gamma_0\vec D-\frac{1}{c}\,\vec {\tilde u}\times \vec B\\
\vec H=\kappa_0\vec B+\frac{1}{c}\,\vec {\tilde u}\times \vec D,\\
\vec {\tilde u}=\left(\gamma_0\vec v+(1-\gamma_0)\vec u\right),
\end{cases}
\end{equation}
where $\psi_0$ was defined by \er{columbPPNaa}. Equations
\er{MaxVacFull1bjkgjhjhgjaaajhfghhgGGaa} differ from the original
Maxwell equations \er{MaxVacFullPPNffGG} only by neglecting the
divergence-free part of the vector field $\vec D$ on the first
equation.
\end{remark}


Next, assume that, in addition to the validity of approximation
\er{slowaetherGGaa} and \er{slochangGGaa}, the approximation
\er{MaxVacFullPPNmmmffffffhhtygghGGGGyuhggghgh} also holds. Then we
further approximate
\er{MaxVacFull1bjkgjhjhgjgjgkjfhjfdghcgjhhjgkgkgugyyurhjfffhfjklhhhgkjgGGaaKK}
as:
\begin{equation}\label{MaxVacFull1bjkgjhjhgjgjgkjfhjfdghcgjhhjgkgkgugyyurhjfffhfjklhhhgkjgGGaaKKjkjj}
\begin{cases}
-\Delta_{\vec x}\Psi_1=4\pi\gamma_0\rho,\\
-\Delta_{\vec x}\vec A \approx \frac{4\pi}{\kappa_0 c}\,\vec
j-\frac{1}{\kappa_0\gamma_0 c}\left(\frac{\partial}{\partial
t}\left(\nabla_{\vec
x}\Psi_1\right)-curl_{\vec x}\left(\vec {\tilde u}\times\nabla_{\vec x}\Psi_1\right)\right)\\
\Psi=\Psi_1+\frac{1}{c}\vec A\cdot\vec {\tilde u}.
\end{cases}
\end{equation}
Moreover, as before, we deduce that equations
\er{MaxVacFull1bjkgjhjhgjgjgkjfhjfdghcgjhhjgkgkgugyyurhjfffhfjklhhhgkjgGGaaKKjkjj}
are also invariant under the change of non-inertial cartesian
coordinate system. Therefore, as before, if in coordinate system
$(*)$ we can use the approximation equations, given by
\er{MaxVacFull1bjkgjhjhgjgjgkjfhjfdghcgjhhjgkgkgugyyurhjfffhfjklhhhgkjgGGaaKKjkjj}
%, given by \er{MaxVacFull1bjkgjhjhgjgjgkjfhjfdghcgjhhjgkgkgugyyurhjfffhfjklhhhgkjggjgjuiyuuijk}
%and \er{MaxVacFull1bjkgjhjhgjgjgkjfhjfdghghligioiuittrhiguffgjggjgkhkk},
then we can use the similar equations
%, given by \er{MaxVacFull1bjkgjhjhgjgjgkjfhjfdghcgjhhjgkgkgugyyurhjfffhfjklhhhgkjggjgj}
%and \er{MaxVacFull1bjkgjhjhgjgjgkjfhjfdghghligioiuittrhiguffgjghlkk}
also in coordinate system $(**)$, even in the case when in system
$(**)$ \er{slowaetherGGaa}, \er{slochangGGaa} or
\er{MaxVacFullPPNmmmffffffhhtygghGGGGyuhggghgh} are not satisfied.







Finally, assume that we are under the alternative calibration
\er{MaxVacFullPPNjjjjffhhGG}. Then by \er{slowaetherGGaa} and
\er{slochangGGaa} we rewrite \er{MaxVacFullPPNmmmffffffiuiuhjuGG}
and \er{MaxVacFullPPNnnnffffffyuughjhjhjhhjjkjhkkjhujgGG} as:
\begin{equation}\label{MaxVacFullPPNmmmffffffiuiuhjuGGaa}
-\Delta_{\vec x}\Psi_1\approx
4\pi\gamma_0\rho+\frac{1}{c}\,div_{\vec x} \left\{\left(d_{\vec
x}\vec {\tilde u}+\left\{d_{\vec x}\vec {\tilde
u}\right\}^T\right)\cdot\vec A-\left(div_{\vec x}\vec {\tilde
u}\right)\vec A\right\},
\end{equation}
and
\begin{equation}\label{MaxVacFullPPNnnnffffffyuughjhjhjhhjjkjhkkjhujgGGaa}
-\Delta_{\vec x}\vec A\approx \frac{4\pi}{\kappa_0 c}\left(\vec
j-\rho\vec {\tilde u}\right)+\frac{1}{\kappa_0\gamma_0
c}\left(\left(d_{\vec x}\vec {\tilde u}+\left\{d_{\vec x}\vec
{\tilde u}\right\}^T\right)\cdot \nabla_{\vec
x}\Psi_1-\left(div_{\vec x}\vec {\tilde u}\right)\nabla_{\vec
x}\Psi_1\right).
\end{equation}
Thus if we assume that in addition to the approximation
\er{slowaetherGGaa} and \er{slochangGGaa} the approximation
\er{MaxVacFullPPNmmmffffffhhtygghGGGGyuhggghgh} also holds, we
further approximate \er{MaxVacFullPPNmmmffffffiuiuhjuGGaa} and
\er{MaxVacFullPPNnnnffffffyuughjhjhjhhjjkjhkkjhujgGGaa} as:
\begin{equation}\label{MaxVacFullPPNmmmffffffiuiuhjuGGaaghgghgh}
\begin{cases}
-\Delta_{\vec x}\Psi_1\approx 4\pi\gamma_0\rho,
\\
-\Delta_{\vec x}\vec A\approx \frac{4\pi}{\kappa_0 c}\left(\vec
j-\rho\vec {\tilde u}\right)\\
\Psi=\Psi_1+\frac{1}{c}\vec A\cdot\vec {\tilde u}.
\end{cases}
\end{equation}
Moreover, as before, we deduce that equations
\er{MaxVacFullPPNmmmffffffiuiuhjuGGaaghgghgh} are also invariant
under the change of non-inertial cartesian coordinate system.
Therefore, as before, if in coordinate system $(*)$ we can use the
approximation equations, given by
\er{MaxVacFullPPNmmmffffffiuiuhjuGGaaghgghgh}
%, given by \er{MaxVacFull1bjkgjhjhgjgjgkjfhjfdghcgjhhjgkgkgugyyurhjfffhfjklhhhgkjggjgjuiyuuijk}
%and \er{MaxVacFull1bjkgjhjhgjgjgkjfhjfdghghligioiuittrhiguffgjggjgkhkk},
then we can use the similar equations
%, given by \er{MaxVacFull1bjkgjhjhgjgjgkjfhjfdghcgjhhjgkgkgugyyurhjfffhfjklhhhgkjggjgj}
%and \er{MaxVacFull1bjkgjhjhgjgjgkjfhjfdghghligioiuittrhiguffgjghlkk}
also in coordinate system $(**)$, even in the case when in system
$(**)$ \er{slowaetherGGaa}, \er{slochangGGaa} or
\er{MaxVacFullPPNmmmffffffhhtygghGGGGyuhggghgh} are not satisfied.




























































































































\subsection{Geometric optics inside a moving medium and/or in the presence of gravitational
field}\label{GO}
\subsubsection{Derivation of the Eikonal equation}\label{ekGO}
Assume that in some inertial or non-inertial cartesian coordinate
system a scalar field $U:=U(\vec x,t)$, characterizing some wave,
satisfies the following wave equation
\begin{equation}\label{MaxVacFullPPNmmmffffffiuiuhjuughbghhjj}
\frac{1}{c^2_0}\left(\frac{\partial}{\partial t}\left(\frac{\partial
U}{\partial t}+\vec {\tilde u}\cdot\nabla_{\vec x}U\right)+div_{\vec
x} \left\{\left(\frac{\partial U}{\partial t}+\vec {\tilde
u}\cdot\nabla_{\vec x} U\right)\vec {\tilde
u}\right\}\right)-\Delta_{\vec x}U=0,
\end{equation}
where $\vec {\tilde u}:=\vec {\tilde u}(\vec x,t)$ is some
moderately changing (in space and in time) speed-like vector field
and $c_0:=c_0(\vec x,t)>0$ is a moderately changing (in space and in
time) scalar quantity, that we call wave propagation speed. Note
that \er{MaxVacFullPPNmmmffffffiuiuhjuughbghhjj} coincides with
\er{MaxVacFullPPNmmmffffffiuiuhjuGGggFGelGHGHGHGG} and thus, in
particular, $U$ can represent one of the scalar components of the
electromagnetic field.

Next if we assume that the fields $\vec {\tilde u}$ and $c_0$ are
independent on the time variable, then we can write the field $U$ as
a Furier's Transform on the time variable:
\begin{equation}\label{MaxVacFullPPNmmmffffffhhtygghGGGGyuhggghghghffgggg}
U(\vec x,t)=\int \hat U(\vec x,\omega)e^{i\omega
t}d\omega\quad\text{where}\quad\hat U(\vec
x,\omega):=\frac{1}{2\pi}\int U(\vec x,t)e^{-i\omega t}dt\,.
\end{equation}
Moreover, by \er{MaxVacFullPPNmmmffffffiuiuhjuughbghhjj} we obtain
that the Furier's Transform $\hat U(\vec x,\omega)$ satisfies:
\begin{equation}\label{MaxVacFullPPNmmmffffffiuiuhjuughbghhjjugg}
\frac{1}{c^2_0}\left(i\omega\left(i\omega \hat U+\vec {\tilde
u}\cdot\nabla_{\vec x}\hat U\right)+div_{\vec x}
\left\{\left(i\omega \hat U+\vec {\tilde u}\cdot\nabla_{\vec x} \hat
U\right)\vec {\tilde u}\right\}\right)-\Delta_{\vec x}\hat U=0.
\end{equation}
Thus by \er{MaxVacFullPPNmmmffffffiuiuhjuughbghhjjugg}, for every
given $\omega$ the monochromatic wave type function
\begin{equation}\label{MaxVacFullPPNmmmffffffhhtygghGGGGyuhggghghghffgggguiu}
U_\omega(\vec x,t):=\hat U(\vec x,\omega)e^{i\omega t}
\end{equation}
is a complex solution of
\begin{equation}\label{MaxVacFullPPNmmmffffffiuiuhjuughbghhjjghghh}
\frac{1}{c^2_0}\left(\frac{\partial}{\partial t}\left(\frac{\partial
U_\omega}{\partial t}+\vec {\tilde u}\cdot\nabla_{\vec
x}U_\omega\right)+div_{\vec x} \left\{\left(\frac{\partial
U_\omega}{\partial t}+\vec {\tilde u}\cdot\nabla_{\vec x}
U_\omega\right)\vec {\tilde u}\right\}\right)-\Delta_{\vec
x}U_\omega=0.
\end{equation}
Note that equation \er{MaxVacFullPPNmmmffffffiuiuhjuughbghhjjghghh}
coincides with \er{MaxVacFullPPNmmmffffffiuiuhjuughbghhjj}.
Moreover, by \er{MaxVacFullPPNmmmffffffhhtygghGGGGyuhggghghghffgggg}
a general solution of \er{MaxVacFullPPNmmmffffffiuiuhjuughbghhjj}
can be represented as a superposition of monochromatic waves of type
$U_\omega=f(\vec x,\omega)e^{i\omega t}$ that satisfy
\er{MaxVacFullPPNmmmffffffiuiuhjuughbghhjjghghh} for every $\omega$.

Next assume that a scalar \underline{complex} field $U:=U(\vec x,t)$
satisfies \er{MaxVacFullPPNmmmffffffiuiuhjuughbghhjj}. In
particular, $U$ can be a monochromatic solution of
\er{MaxVacFullPPNmmmffffffiuiuhjuughbghhjjghghh}. Although from now
we consider that the fields $\vec {\tilde u}$ and $c_0$ can depend
on the time variable, assume however, that we have the following
approximation, analogous to
\er{MaxVacFullPPNmmmffffffhhtygghGGGGyuhggghgh}: if the changes of
the physical characteristics of the field $U$ become essential in
the spatial landscape $L_e$ and the temporal landscape $T_e$, and
the changes of the field $\vec {\tilde u}$ becomes essential in the
spatial landscape $L_{u}$ and the temporal landscape $T_{u}$ , then
we assume
\begin{equation}\label{MaxVacFullPPNmmmffffffhhtygghGGGGyuhggghghghffg}
\left(c_0T_e+L_e\right)\ll \left(c_0T_u+L_u\right),\quad\text{or
equivalently:}\quad \frac{\left(|\partial_t\vec {\tilde u}|+c_0|
d_{\vec x}\vec {\tilde u}|\right)}{|\vec {\tilde
u}|}\ll\frac{\left(|\partial_t U|+c_0| d_{\vec x}U|\right)}{|U|}.
\end{equation}
Furthermore, we represent the complex field $U$ as:
\begin{equation}\label{MaxVacFullPPNmmmffffffiuiuhjuughbghhuiiujjjjjjjj}
U(\vec x,t)=A(\vec x,t)e^{iT(\vec x,t)},
\end{equation}
where $A:=A(\vec x,t)$ and $T:=T(\vec x,t)$ are real scalar fields.
Then define
\begin{equation}\label{MaxVacFullPPNmmmffffffiuiuhjuughbghhuiiujjjjjjjjhhhjjj}
\omega:=\left<\,\left|\frac{\partial T}{\partial t}\right|\,\right>,
\end{equation}
where the sign $\left<\cdot\right>$ means the spatial and temporal
averaging. Next define $k_0$ and a scalar field $S:=S(\vec x,t)$ by
\begin{equation}\label{MaxVacFullPPNmmmffffffiuiuhjuughbghhuiiujjjjjjjjhhhjjjkk}
k_0:=\frac{\omega}{c}\quad\quad\text{and}\quad\quad S(\vec
x,t)=\frac{1}{k_0}T(\vec x,t),
\end{equation}
where $c$ is a constant in the Maxwell equations for the vacuum. So
we clearly have
\begin{equation}\label{MaxVacFullPPNmmmffffffiuiuhjuughbghhuiiujj}
U(\vec x,t)=A(\vec x,t)e^{ik_0S(\vec x,t)}\,.
\end{equation}
%Moreover, assume that the spatial and time derivatives of $\vec {\tilde u}$ of first and second order are negligible with respect to $k_0$.
Then, by \er{MaxVacFullPPNmmmffffffhhtygghGGGGyuhggghghghffg} we
approximate equation \er{MaxVacFullPPNmmmffffffiuiuhjuughbghhjj} as:
\begin{equation}\label{MaxVacFullPPNmmmffffffiuiuhjuughbghhiuijuyuggkkjj}
\frac{1}{c^2_0}\left(\frac{\partial^2 U}{\partial t^2}+2\vec {\tilde
u}\cdot\nabla_{\vec x}\left(\frac{\partial U}{\partial
t}\right)+\left(\nabla^2_{\vec x}U\cdot\vec {\tilde
u}\right)\cdot\vec {\tilde u}\right)-\Delta_{\vec x}U=0.
\end{equation}
Thus inserting \er{MaxVacFullPPNmmmffffffiuiuhjuughbghhuiiujj} into
\er{MaxVacFullPPNmmmffffffiuiuhjuughbghhiuijuyuggkkjj} we deduce:
\begin{multline}\label{MaxVacFullPPNmmmffffffiuiuhjuughbghhiuijjj}
%\frac{1}{c^2_0}\frac{\partial^2}{\partial t^2}\left(Ae^{ik_0S}\right)
-\frac{k^2_0}{c^2_0}\left(\frac{\partial S}{\partial
t}\right)^2Ae^{ik_0S}+\frac{ik_0}{c^2_0}\left(\frac{\partial^2
S}{\partial t^2}\right)Ae^{ik_0S}+\frac{2ik_0}{c^2_0}\frac{\partial
A}{\partial t}\frac{\partial S}{\partial
t}e^{ik_0S}+\frac{1}{c^2_0}\frac{\partial^2 A}{\partial
t^2}e^{ik_0S}
\\-\frac{2k^2_0}{c^2_0}\frac{\partial S}{\partial
t}\left(\vec {\tilde u}\cdot\nabla_{\vec
x}S\right)Ae^{ik_0S}+\frac{2ik_0 }{c^2_0}\left(\vec {\tilde u}\cdot
\nabla_{\vec x}\left(\frac{\partial S}{\partial
t}\right)\right)Ae^{ik_0S}+\frac{2ik_0}{c^2_0}\left(\vec {\tilde
u}\cdot\nabla_{\vec x}A\right)\frac{\partial S}{\partial
t}e^{ik_0S}\\+\frac{2ik_0}{c^2_0}\left(\vec {\tilde
u}\cdot\nabla_{\vec x}S\right)\frac{\partial A}{\partial
t}e^{ik_0S}+\frac{2}{c^2_0}\vec {\tilde u}\cdot\nabla_{\vec
x}\left(\frac{\partial A}{\partial
t}\right)e^{ik_0S}-\frac{k^2_0}{c^2}\left|\vec {\tilde
u}\cdot\nabla_{\vec
x}S\right|^2Ae^{ik_0S}\\+\frac{ik_0}{c^2_0}\left(\left(\nabla^2_{\vec
x}S\cdot\vec {\tilde u}\right)\cdot\vec {\tilde
u}\right)Ae^{ik_0S}+\frac{2ik_0}{c^2_0}\left(\vec {\tilde
u}\cdot\nabla_{\vec x}A\right)\left(\vec {\tilde u}\cdot\nabla_{\vec
x}S\right)e^{ik_0S}+\frac{1}{c^2_0}\left(\left(\nabla^2_{\vec
x}A\cdot\vec {\tilde u}\right)\cdot\vec {\tilde
u}\right)e^{ik_0S}\\+k^2_0\left|\nabla_{\vec
x}S\right|^2Ae^{ik_0S}-ik_0\left(\Delta_{\vec
x}S\right)Ae^{ik_0S}-2ik_0\left(\nabla_{\vec x}A\cdot\nabla_{\vec
x}S\right)e^{ik_0S}-\left(\Delta_{\vec x}A\right)e^{ik_0S}=0.
\end{multline}
Then:
\begin{multline}\label{MaxVacFullPPNmmmffffffiuiuhjuughbghhiuijghghghjj}
%\frac{1}{c^2_0}\frac{\partial^2}{\partial t^2}\left(Ae^{ik_0S}\right)
-\frac{k^2_0}{c^2_0}\left(\frac{\partial S}{\partial
t}\right)^2A+\frac{ik_0}{c^2_0}\left(\frac{\partial^2 S}{\partial
t^2}\right)A+\frac{2ik_0}{c^2_0}\frac{\partial A}{\partial
t}\frac{\partial S}{\partial t}+\frac{1}{c^2_0}\frac{\partial^2
A}{\partial t^2}-\frac{2k^2_0}{c^2_0}\frac{\partial S}{\partial
t}\left(\vec {\tilde u}\cdot\nabla_{\vec x}S\right)A\\+\frac{2ik_0
}{c^2_0}\left(\vec {\tilde u}\cdot \nabla_{\vec
x}\left(\frac{\partial S}{\partial
t}\right)\right)A+\frac{2ik_0}{c^2_0}\left(\vec {\tilde
u}\cdot\nabla_{\vec x}A\right)\frac{\partial S}{\partial
t}+\frac{2ik_0}{c^2_0}\left(\vec {\tilde u}\cdot\nabla_{\vec
x}S\right)\frac{\partial A}{\partial t}+\frac{2}{c^2_0}\vec {\tilde
u}\cdot\nabla_{\vec x}\left(\frac{\partial A}{\partial
t}\right)\\-\frac{k^2_0}{c^2}\left|\vec {\tilde u}\cdot\nabla_{\vec
x}S\right|^2A+\frac{ik_0}{c^2_0}\left(\left(\nabla^2_{\vec
x}S\cdot\vec {\tilde u}\right)\cdot\vec {\tilde
u}\right)A+\frac{2ik_0}{c^2_0}\left(\vec {\tilde u}\cdot\nabla_{\vec
x}A\right)\left(\vec {\tilde u}\cdot\nabla_{\vec
x}S\right)+\frac{1}{c^2_0}\left(\left(\nabla^2_{\vec x}A\cdot\vec
{\tilde u}\right)\cdot\vec {\tilde
u}\right)\\+k^2_0\left|\nabla_{\vec
x}S\right|^2A-ik_0\left(\Delta_{\vec
x}S\right)A-2ik_0\left(\nabla_{\vec x}A\cdot\nabla_{\vec
x}S\right)-\left(\Delta_{\vec x}A\right)=0.
\end{multline}
Thus, since the zero complex number has both real and imaginary part
equal to zero, by
\er{MaxVacFullPPNmmmffffffiuiuhjuughbghhiuijghghghjj} we have:
\begin{multline}\label{MaxVacFullPPNmmmffffffiuiuhjuughbghhiuijghghghhjhjhjj}
k^2_0\left(\left|\nabla_{\vec
x}S\right|^2-\frac{1}{c^2_0}\left(\frac{\partial S}{\partial t}+\vec
{\tilde u}\cdot\nabla_\vec x
S\right)^2\right)A+\frac{1}{c^2_0}\left(\frac{\partial^2 A}{\partial
t^2}+2\vec {\tilde u}\cdot\nabla_{\vec x}\left(\frac{\partial
A}{\partial t}\right)+\left(\left(\nabla^2_{\vec x}A\cdot\vec
{\tilde u}\right)\cdot\vec {\tilde u}\right)\right)-\Delta_{\vec
x}A=
%
%
%
\\-\frac{k^2_0}{c^2_0}\left(\frac{\partial S}{\partial
t}\right)^2A+\frac{1}{c^2_0}\frac{\partial^2 A}{\partial t^2}
-\frac{2k^2_0}{c^2_0}\frac{\partial S}{\partial t}\left(\vec {\tilde
u}\cdot\nabla_{\vec x}S\right)A+\frac{2}{c^2_0}\vec {\tilde
u}\cdot\nabla_{\vec x}\left(\frac{\partial A}{\partial
t}\right)-\frac{k^2_0}{c^2_0}\left|\vec {\tilde u}\cdot\nabla_{\vec
x}S\right|^2A+\frac{1}{c^2_0}\left(\left(\nabla^2_{\vec x}A\cdot\vec
{\tilde u}\right)\cdot\vec {\tilde
u}\right)\\+k^2_0\left|\nabla_{\vec x}S\right|^2A-\Delta_{\vec
x}A=0,
\end{multline}
and
\begin{multline}\label{MaxVacFullPPNmmmffffffiuiuhjuughbghhiuijghghghhhfhjj}
\frac{1}{c^2_0}\left(\frac{\partial^2 S}{\partial t^2}+2\vec {\tilde
u}\cdot \nabla_{\vec x}\left(\frac{\partial S}{\partial
t}\right)+\left(\nabla^2_{\vec x}S\cdot\vec {\tilde
u}\right)\cdot\vec {\tilde
u}\right)A\\+\frac{2}{c^2_0}\left(\frac{\partial S}{\partial t}+\vec
{\tilde u}\cdot\nabla_{\vec x}S\right)\left(\frac{\partial
A}{\partial t}+\vec {\tilde u}\cdot\nabla_{\vec
x}A\right)-\left(\Delta_{\vec x}S\right)A-2\nabla_{\vec
x}A\cdot\nabla_{\vec x}S=\\
%\frac{1}{c^2_0}\frac{\partial^2}{\partial t^2}\left(Ae^{ik_0S}\right)
\frac{1}{c^2_0}\left(\frac{\partial^2 S}{\partial
t^2}\right)A+\frac{2}{c^2_0}\frac{\partial A}{\partial
t}\frac{\partial S}{\partial t}+\frac{2 }{c^2_0}\left(\vec {\tilde
u}\cdot \nabla_{\vec x}\left(\frac{\partial S}{\partial
t}\right)\right)A+\frac{2}{c^2_0}\left(\vec {\tilde
u}\cdot\nabla_{\vec x}A\right)\frac{\partial S}{\partial
t}+\frac{2}{c^2_0}\left(\vec {\tilde u}\cdot\nabla_{\vec
x}S\right)\frac{\partial A}{\partial
t}\\+\frac{1}{c^2_0}\left(\left(\nabla^2_{\vec x}S\cdot\vec {\tilde
u}\right)\cdot\vec {\tilde u}\right)A+\frac{2}{c^2_0}\left(\vec
{\tilde u}\cdot\nabla_{\vec x}A\right)\left(\vec {\tilde
u}\cdot\nabla_{\vec x}S\right)-\left(\Delta_{\vec
x}S\right)A-2\nabla_{\vec x}A\cdot\nabla_{\vec x}S=0.
\end{multline}


Next assume the Geometric Optics approximation that is good for the
electromagnetic wave of high frequency for example for the visible
light. The Geometric Optics approximation means the following:
assume that the changes in time of $c_0$, $A$ and $S$ become
essential after certain interval of time $T_e$ and the changes in
space of $c_0$, $A$ and $S$ become essential in the spatial
landscape $L_e$. Then we assume that
\begin{equation}\label{slochangGGaaffgfgjhjggh}
k^2_0c^2_0T^2_e\,\gg\, 1\quad\text{and}\quad k^2_0L^2_e\,\gg\, 1.
\end{equation}
%together with the assumption \er{MaxVacFullPPNmmmffffffhhtygghGGGGyuhggghghghffg}.
Thus, by
\er{slochangGGaaffgfgjhjggh} we approximate
\er{MaxVacFullPPNmmmffffffiuiuhjuughbghhiuijghghghhjhjhjj} as:
\begin{equation}\label{MaxVacFullPPNmmmffffffiuiuhjuughbghhiuijghghghhjhjhhghyuyjj}
\frac{1}{c^2_0}\left(\frac{\partial S}{\partial t}+\vec {\tilde
u}\cdot\nabla_\vec x S\right)^2\approx\left|\nabla_\vec x
S\right|^2,
\end{equation}
and rewrite
\er{MaxVacFullPPNmmmffffffiuiuhjuughbghhiuijghghghhhfhjj} as:
\begin{multline}\label{MaxVacFullPPNmmmffffffiuiuhjuughbghhiuijghghghhhfhhghghguygtjj}
\frac{1}{c^2_0}\left(\frac{\partial^2 S}{\partial t^2}+2\vec {\tilde
u}\cdot \nabla_{\vec x}\left(\frac{\partial S}{\partial
t}\right)+\left(\nabla^2_{\vec x}S\cdot\vec {\tilde
u}\right)\cdot\vec {\tilde
u}\right)A\\+\frac{2}{c^2_0}\left(\frac{\partial S}{\partial t}+\vec
{\tilde u}\cdot\nabla_{\vec x}S\right)\left(\frac{\partial
A}{\partial t}+\vec {\tilde u}\cdot\nabla_{\vec
x}A\right)-\left(\Delta_{\vec
x}S\right)A-2\nabla_{\vec x}A\cdot\nabla_{\vec x}S=0\,.
\end{multline}
%
%
%
\begin{comment}
\begin{multline}\label{MaxVacFullPPNmmmffffffiuiuhjuughbghhiuijghghghhhfhhghghguygtjj}
\\
%
%
%
%\frac{1}{c^2_0}\frac{\partial^2}{\partial t^2}\left(Ae^{ik_0S}\right)
\frac{1}{c^2_0}\left(\frac{\partial^2 S}{\partial
t^2}\right)A+\frac{2}{c^2_0}\frac{\partial A}{\partial
t}\frac{\partial S}{\partial t}+\frac{2 }{c^2_0}\left(\vec {\tilde
u}\cdot \nabla_{\vec x}\left(\frac{\partial S}{\partial
t}\right)\right)A+\frac{2}{c^2_0}\left(\vec {\tilde
u}\cdot\nabla_{\vec x}A\right)\frac{\partial S}{\partial
t}+\frac{2}{c^2_0}\left(\vec {\tilde u}\cdot\nabla_{\vec
x}S\right)\frac{\partial A}{\partial
t}\\+\frac{1}{c^2_0}\left(\left(\nabla^2_{\vec x}S\cdot\vec {\tilde
u}\right)\cdot\vec {\tilde u}\right)A+\frac{2}{c^2_0}\left(\vec
{\tilde u}\cdot\nabla_{\vec x}A\right)\left(\vec {\tilde
u}\cdot\nabla_{\vec x}S\right)-\left(\Delta_{\vec
x}S\right)A-2\nabla_{\vec x}A\cdot\nabla_{\vec x}S=0\,.
\end{multline}
\end{comment}
%
%
%
Further approximation of
\er{MaxVacFullPPNmmmffffffiuiuhjuughbghhiuijghghghhhfhhghghguygtjj},
due to \er{MaxVacFullPPNmmmffffffhhtygghGGGGyuhggghghghffg} gives:
\begin{multline}\label{MaxVacFullPPNmmmffffffiuiuhjuughbghhiuijghghghhhfhhghghguygtjjjkkjjkkhh}
%\left(
\frac{1}{c^2_0}\left(\frac{\partial}{\partial t}\left(\frac{\partial
S}{\partial t}+\vec {\tilde u}\cdot\nabla_{\vec x}S\right)+div_{\vec
x} \left\{\left(\frac{\partial S}{\partial t}+\vec {\tilde
u}\cdot\nabla_{\vec x} S\right)\vec {\tilde
u}\right\}\right)A-\left(\Delta_{\vec x}S\right)A
%\right)
\\+\frac{2}{c^2_0}\left(\frac{\partial S}{\partial
t}+\vec {\tilde u}\cdot\nabla_{\vec x}S\right)\left(\frac{\partial
A}{\partial t}+\vec {\tilde u}\cdot\nabla_{\vec
x}A\right)
%-\left(\Delta_{\vec x}S\right)A
-2\nabla_{\vec x}S\cdot\nabla_{\vec x}A=0\,,
\end{multline}
and we write again the Eikonal type equation
\er{MaxVacFullPPNmmmffffffiuiuhjuughbghhiuijghghghhjhjhhghyuyjj}:
\begin{equation}\label{MaxVacFullPPNmmmffffffiuiuhjuughbghhiuijghghghhjhjhhghyuyjjjhhjhjff}
\frac{1}{c^2_0}\left(\frac{\partial S}{\partial t}+\vec {\tilde
u}\cdot\nabla_\vec x S\right)^2=\left|\nabla_\vec x S\right|^2.
\end{equation}
Then, as before, we deduce that equation
\er{MaxVacFullPPNmmmffffffiuiuhjuughbghhiuijghghghhjhjhhghyuyjjjhhjhjff}
is invariant under the change of non-inertial cartesian coordinate
system, provided that under such change we have $S'=S$. Moreover,
\er{MaxVacFullPPNmmmffffffiuiuhjuughbghhiuijghghghhhfhhghghguygtjjjkkjjkkhh}
is also invariant under the change of non-inertial cartesian
coordinate system, in the case that under such change we have
$A'=A$, provided that $S'=S$. So if the approximations
\er{MaxVacFullPPNmmmffffffhhtygghGGGGyuhggghghghffg} and
\er{slochangGGaaffgfgjhjggh} are valid in some cartesian coordinate
system $(*)$, then we can use
\er{MaxVacFullPPNmmmffffffiuiuhjuughbghhiuijghghghhjhjhhghyuyjjjhhjhjff}
and
\er{MaxVacFullPPNmmmffffffiuiuhjuughbghhiuijghghghhhfhhghghguygtjjjkkjjkkhh}
also in any other inertial or non-inertial cartesian coordinate
system $(**)$ even in the case when
\er{MaxVacFullPPNmmmffffffhhtygghGGGGyuhggghghghffg} and
\er{slochangGGaaffgfgjhjggh} are not valid in the system $(**)$,
provided that under the change of coordinate system we have $A'=A$
and $S'=S$.

\subsubsection{The case of the monochromatic wave}\label{mnGO}
Next, up to the end of this subsection, consider the case of
monochromatic wave of the constant frequency
$\nu=\frac{\omega}{2\pi}$ where the fields $\vec {\tilde u}$ and
$c_0$ are independent on the time variable i.e. the case of
\er{MaxVacFullPPNmmmffffffiuiuhjuughbghhuiiujjjjjjjj} where we have
\begin{equation}\label{MaxVacFullPPNmmmffffffiuiuhjuughbghhiuijghghghhjhjhhghyuyhhjhjihikjjjjkjljkl}
\begin{cases}
\frac{\partial T}{\partial t}=\omega\\
\frac{\partial A}{\partial t}=0\\
\frac{\partial \vec {\tilde u}}{\partial t}=0\\
\frac{\partial c_0}{\partial t}=0.
\end{cases}
\end{equation}
Then, by \er{MaxVacFullPPNmmmffffffiuiuhjuughbghhuiiujjjjjjjjhhhjjj}
and \er{MaxVacFullPPNmmmffffffiuiuhjuughbghhuiiujjjjjjjjhhhjjjkk} we
rewrite
\er{MaxVacFullPPNmmmffffffiuiuhjuughbghhiuijghghghhjhjhhghyuyhhjhjihikjjjjkjljkl}
as
\begin{equation}\label{MaxVacFullPPNmmmffffffiuiuhjuughbghhiuijghghghhjhjhhghyuyhhjhjihikjjj}
\begin{cases}
\frac{\partial S}{\partial t}=c\\
\frac{\partial A}{\partial t}=0.
\end{cases}
\end{equation}
Thus $\nabla_{\vec x}S$ is independent on $t$ and moreover, we
rewrite
\er{MaxVacFullPPNmmmffffffiuiuhjuughbghhiuijghghghhjhjhhghyuyjj} and
\er{MaxVacFullPPNmmmffffffiuiuhjuughbghhiuijghghghhhfhhghghguygtjj}
as:
\begin{equation}\label{MaxVacFullPPNmmmffffffiuiuhjuughbghhiuijghghghhjhjhhghyuyiyyujj}
\frac{c^2}{c^2_0}\left(1+\frac{1}{c}\vec {\tilde u}\cdot\nabla_\vec
x S\right)^2=\left|\nabla_\vec x S\right|^2,
\end{equation}
and
\begin{equation}\label{MaxVacFullPPNmmmffffffiuiuhjuughbghhiuijghghghhhfhhghghguygtjuuujj}
%\frac{1}{c^2_0}\frac{\partial^2}{\partial t^2}\left(Ae^{ik_0S}\right)
2\left(\nabla_{\vec x}S-\frac{c}{c_0}\left(1+\frac{1}{c}\left(\vec
{\tilde u}\cdot\nabla_{\vec x}S\right)\right)\frac{\vec {\tilde
u}}{c_0}\right)\cdot\nabla_{\vec
x}A=\left(\frac{1}{c^2_0}\left(\left(\nabla^2_{\vec x}S\cdot\vec
{\tilde u}\right)\cdot\vec {\tilde u}\right)-\left(\Delta_{\vec
x}S\right)\right)A.
\end{equation}
%\end{comment*}
%
%
%
%
%
%
%
%
\begin{comment}
\begin{multline}\label{MaxMedFullGGffgggyyojjhhjkhjyyiuhggjhhjhuyytytyuuytrrtghjtyuggyuighjuyioyyfgffhyuhhghzzrrhhkkkhhjjh}
\int\limits_a^b\sqrt{\left(1-\frac{\left|\vec {\tilde u}\left(\vec
r(s)\right)\right|^2}{c^2_0}\right)\left|\vec h\left(\vec
r(s)\right)-\frac{1}{\left|\vec {\tilde u}\left(\vec
r(s)\right)\right|^2}\left(\vec {\tilde u}\left(\vec
r(s)\right)\cdot\vec h\left(\vec r(s)\right)\right)\vec {\tilde
u}\left(\vec r(s)\right)\right|^2+\left|\frac{1}{\left|\vec {\tilde
u}\right|}\vec {\tilde u}\cdot\vec
h\right|^2}\cdot\\
\cdot\sqrt{\left(1-\frac{\left|\vec {\tilde u}\left(\vec
r(s)\right)\right|^2}{c^2_0}\right)^{-1}\left|\vec
r'(s)-\frac{1}{\left|\vec {\tilde u}\left(\vec
r(s)\right)\right|^2}\left(\vec {\tilde u}\left(\vec
r(s)\right)\cdot\vec r'(s)\right)\vec {\tilde u}\left(\vec
r(s)\right)\right|^2+\left|\frac{1}{\left|\vec {\tilde u}\left(\vec
r(s)\right)\right|}\vec {\tilde u}\left(\vec r(s)\right)\cdot\vec
r'(s)\right|^2}ds\\ =\int\limits_a^b\frac{c}{c_0\left(\vec
r(s)\right)}\cdot\\ \cdot\sqrt{\left(1-\frac{\left|\vec {\tilde
u}\left(\vec r(s)\right)\right|^2}{c^2_0}\right)^{-1}\left|\vec
r'(s)-\frac{1}{\left|\vec {\tilde u}\left(\vec
r(s)\right)\right|^2}\left(\vec {\tilde u}\left(\vec
r(s)\right)\cdot\vec r'(s)\right)\vec {\tilde u}\left(\vec
r(s)\right)\right|^2+\left|\frac{1}{\left|\vec {\tilde u}\left(\vec
r(s)\right)\right|}\vec {\tilde u}\left(\vec r(s)\right)\cdot\vec
r'(s)\right|^2}ds.
\end{multline}


\begin{multline}\label{MaxMedFullGGffgggyyojjhhjkhjyyiuhggjhhjhuyytytyuuytrrtghjtyuggyuighjuyioyyfgffhyuhhghzzrrkkhhkkkhhhjhkjhhghhgghhhihiy}
J\left(\vec r(\cdot)\right):= -\int_a^b\frac{1}{c}n^2\left(\vec
r(s)\right)\left(1-\frac{\left|\vec {\tilde u}\left(\vec
r(s)\right)\right|^2}{c^2_0}\right)^{-1}\vec {\tilde u}\left(\vec
r(s)\right)\cdot\vec r'(s)ds\\+\int\limits_a^bn\left(\vec
r(s)\right)\left(1-\frac{\left|\vec {\tilde u}\left(\vec
r(s)\right)\right|^2}{c^2_0}\right)^{-\frac{1}{2}}
\cdot\\
\cdot\sqrt{\left|\vec r'(s)-\frac{1}{\left|\vec {\tilde u}\left(\vec
r(s)\right)\right|^2}\left(\vec {\tilde u}\left(\vec
r(s)\right)\cdot\vec r'(s)\right)\vec {\tilde u}\left(\vec
r(s)\right)\right|^2+\left(1-\frac{\left|\vec {\tilde u}\left(\vec
r(s)\right)\right|^2}{c^2_0}\right)^{-1}\left|\frac{1}{\left|\vec
{\tilde u}\left(\vec r(s)\right)\right|}\vec {\tilde u}\left(\vec
r(s)\right)\cdot\vec r'(s)\right|^2}ds,
\end{multline}



\begin{equation}\label{MaxVacFullPPNmmmffffffiuiuhjuughbghhiuijghghghhhfhhghghguygtjuuujjjkyuuyhhijhhhhjihhj}
%\frac{1}{c^2_0}\frac{\partial^2}{\partial t^2}\left(Ae^{ik_0S}\right)
\vec h:=\frac{c}{c^2_0}\left(1+\frac{1}{c}\left(\vec {\tilde
u}\cdot\nabla_{\vec x}S\right)\right)\vec {\tilde u}-\nabla_{\vec
x}S,
\end{equation}

\begin{equation}\label{MaxVacFullPPNmmmffffffiuiuhjuughbghhiuijghghghhjhjhhghyuyiyyujjkhgggghggutggjkkjklkj}
\left(1-\frac{\left|\vec {\tilde
u}\right|^2}{c^2_0}\right)\left|\vec h-\frac{1}{\left|\vec {\tilde
u}\right|^2}\left(\vec {\tilde u}\cdot\vec h\right)\vec {\tilde
u}\right|^2+\left|\frac{1}{\left|\vec {\tilde u}\right|}\vec {\tilde
u}\cdot\vec h\right|^2=\left|\vec h\right|^2\left(1-\frac{\left|\vec
{\tilde u}\right|^2}{c^2_0}\right)+\frac{1}{c^2_0}\left|\vec {\tilde
u}\cdot\vec h\right|^2= \frac{c^2}{c^2_0},
\end{equation}

\begin{equation}\label{MaxVacFullPPNmmmffffffiuiuhjuughbghhiuijghghghhhfhhghghguygtjuuujjjkyuuyhhijhhhhj}
%\frac{1}{c^2_0}\frac{\partial^2}{\partial t^2}\left(Ae^{ik_0S}\right)
\frac{1}{|\vec {\tilde u}|}\vec {\tilde u}\cdot\vec
h=\frac{c}{c^2_0}|\vec {\tilde u}|-\left(\frac{1}{|\vec {\tilde
u}|}\vec {\tilde u}\cdot\nabla_{\vec x}S\right)\left(1-\frac{|\vec
{\tilde u}|^2}{c^2_0}\right),
\end{equation}
\end{comment}
%
%
%
%







































In particular, in the case of the region of the space where the
following approximation is valid:
\begin{equation}\label{ojhkk}
\frac{|\vec {\tilde u}|^2}{c^2_0}\ll 1,
\end{equation}
up to order $O\left(\frac{|\vec {\tilde u}|^2}{c^2_0}\right)$, we
rewrite
\er{MaxVacFullPPNmmmffffffiuiuhjuughbghhiuijghghghhjhjhhghyuyiyyujj}
as:
\begin{equation}\label{MaxVacFullPPNmmmffffffiuiuhjuughbghhiuijghghghhjhjhhghyuyiyyujjljk}
\left|\frac{c\vec {\tilde u}}{c^2_0}-\nabla_\vec x
S\right|^2=\frac{c^2}{c^2_0},
\end{equation}
and
\er{MaxVacFullPPNmmmffffffiuiuhjuughbghhiuijghghghhhfhhghghguygtjuuujj}
as:
\begin{equation}\label{MaxVacFullPPNmmmffffffiuiuhjuughbghhiuijghghghhhfhhghghguygtjuuujjjk}
%\frac{1}{c^2_0}\frac{\partial^2}{\partial t^2}\left(Ae^{ik_0S}\right)
\left(\frac{c\vec {\tilde u}}{c^2_0}-\nabla_{\vec
x}S\right)\cdot\nabla_{\vec x}A+\frac{1}{2}\left(-\Delta_{\vec
x}S\right)A=0.
\end{equation}
The Eikonal equation
\er{MaxVacFullPPNmmmffffffiuiuhjuughbghhiuijghghghhjhjhhghyuyiyyujjljk}
and equation of the beam propagation
\er{MaxVacFullPPNmmmffffffiuiuhjuughbghhiuijghghghhhfhhghghguygtjuuujjjk}
are two basic equations of propagation of monochromatic light in the
Geometric Optics approximation inside a moving medium or/and in the
presence of non-trivial gravitational field, provided that the field
$\vec {\tilde u}$ satisfies \er{ojhkk}.

Next if we consider an arbitrary characteristic curve $\vec
r(s):[a,b]\to\mathbb{R}^3$ of equation
\er{MaxVacFullPPNmmmffffffiuiuhjuughbghhiuijghghghhhfhhghghguygtjuuujjjk}
defined as a solution of ordinary differential equation
\begin{equation}\label{MaxVacFullPPNmmmffffffiuiuhjuughbghhiuijghghghhhfhhghghguygtjuuujjjkk}
\begin{cases}
\frac{d\vec r}{ds}(s)=\frac{c}{c^2_0\left(\vec r(s)\right)}\vec
{\tilde u}\left(\vec
r(s)\right)-\nabla_{\vec x}S\left(\vec r(s)\right)\\
\vec r(a)=\vec x_0,
\end{cases}
\end{equation}
then, as before, by
\er{MaxVacFullPPNmmmffffffiuiuhjuughbghhiuijghghghhhfhhghghguygtjuuujjjk}
and
\er{MaxVacFullPPNmmmffffffiuiuhjuughbghhiuijghghghhhfhhghghguygtjuuujjjkk}
we have
\begin{equation}\label{MaxVacFullPPNmmmffffffiuiuhjuughbghhiuijghghghhhfhhghghguygtjuuujjjkhjhjh}
%\frac{1}{c^2_0}\frac{\partial^2}{\partial t^2}\left(Ae^{ik_0S}\right)
\frac{d}{ds}\left(A\left(\vec r(s)\right)\right)=\nabla_{\vec
x}A\left(\vec r(s)\right)\cdot\frac{d\vec
r}{ds}(s)=\frac{1}{2}\left(\Delta_{\vec x}S\left(\vec
r(s)\right)\right)A\left(\vec r(s)\right),
\end{equation}
that implies
\begin{equation}\label{MaxVacFullPPNmmmffffffiuiuhjuughbghhiuijghghghhhfhhghghguygtjuuujjjkhjhjhffgg}
A\left(\vec r(s)\right)=A\left(\vec
x_0\right)e^{\frac{1}{2}\int_a^{s}\left(\Delta_{\vec x}S\left(\vec
r(\tau)\right)\right)d\tau}\quad\quad\forall s\in[a,b].
\end{equation}
In particular,
\begin{equation}\label{MaxVacFullPPNmmmffffffiuiuhjuughbghhiuijghghghhhfhhghghguygtjuuujjjkhjhjhffggiouuiiu}
A\left(\vec x_0\right)=0\;\;\text{implies}\;\; A\left(\vec
r(s)\right)=0\quad\forall s\in[a,b],\quad\text{and}\quad A\left(\vec
x_0\right)\neq 0\;\;\text{implies}\;\; A\left(\vec r(s)\right)\neq
0\quad\forall s\in[a,b].
\end{equation}
Therefore, by
\er{MaxVacFullPPNmmmffffffiuiuhjuughbghhiuijghghghhhfhhghghguygtjuuujjjkhjhjhffggiouuiiu}
we deduce that in the case of \er{ojhkk} the curve that satisfies
\er{MaxVacFullPPNmmmffffffiuiuhjuughbghhiuijghghghhhfhhghghguygtjuuujjjkk}
coincides with the beam of light that passes through the point $\vec
x_0$. So in the case of \er{ojhkk}, equality
\er{MaxVacFullPPNmmmffffffiuiuhjuughbghhiuijghghghhhfhhghghguygtjuuujjjkk}
is the equation of a beam and the vector field $\vec h$ defined for
every $\vec x$ by:
\begin{equation}\label{MaxVacFullPPNmmmffffffiuiuhjuughbghhiuijghghghhhfhhghghguygtjuuujjjkyuuy}
%\frac{1}{c^2_0}\frac{\partial^2}{\partial t^2}\left(Ae^{ik_0S}\right)
\vec h(\vec x):=\frac{c}{c^2_0(\vec x)}\vec {\tilde u}(\vec
x)-\nabla_{\vec x}S(\vec x),
\end{equation}
is the direction of the propagation of the beam that passes through
point $\vec x$. Moreover, by
\er{MaxVacFullPPNmmmffffffiuiuhjuughbghhiuijghghghhjhjhhghyuyiyyujjljk}
$\vec h$ satisfies
\begin{equation}\label{MaxVacFullPPNmmmffffffiuiuhjuughbghhiuijghghghhjhjhhghyuyiyyujjljkgghhg}
|\vec h|^2=\frac{c^2}{c^2_0}.
\end{equation}







Next, under the approximation \er{ojhkk}
%again assume that the approximate equations in \er{MaxVacFullPPNmmmffffffiuiuhjuughbghhiuijghghghhjhjhhghyuyiyyujjljk}
%and \er{MaxVacFullPPNmmmffffffiuiuhjuughbghhiuijghghghhhfhhghghguygtjuuujjjk} are valid in the given region and
consider a curve $\vec
r(s):[a,b]\to\mathbb{R}^3$ with endpoints $\vec r(a)=N$ and $\vec
r(b)=M$. Then integrating the square root of both sides of
\er{MaxVacFullPPNmmmffffffiuiuhjuughbghhiuijghghghhjhjhhghyuyiyyujjljk}
over the curve $\vec r(s)$ we deduce
\begin{equation}\label{MaxMedFullGGffgggyyojjhhjkhjyyiuhggjhhjhuyytytyuuytrrtghjtyuggyuighjuyioyyfgffhyuhhghzzrrhhkkk}
\int_a^b\left|\frac{c}{c^2_0\left(\vec r(s)\right)}\vec {\tilde
u}\left(\vec r(s)\right)-\nabla_{\vec x}S\left(\vec
r(s)\right)\right|\,\left|\vec
r'(s)\right|ds=\int_a^b\frac{c}{c_0\left(\vec r(s)\right)}\left|\vec
r'(s)\right|ds.
\end{equation}
Thus in particular,
\begin{equation}\label{MaxMedFullGGffgggyyojjhhjkhjyyiuhggjhhjhuyytytyuuytrrtghjtyuggyuighjuyioyyfgffhyuhhghzzrriuihhkkk}
\int_a^b\left(\frac{c}{c^2_0\left(\vec r(s)\right)}\vec {\tilde
u}\left(\vec r(s)\right)-\nabla_{\vec x}S\left(\vec
r(s)\right)\right)\cdot\vec
r'(s)ds\leq\int_a^b\frac{c}{c_0\left(\vec r(s)\right)}\left|\vec
r'(s)\right|ds,
\end{equation}
i.e.
\begin{equation}\label{MaxMedFullGGffgggyyojjhhjkhjyyiuhggjhhjhuyytytyuuytrrtghjtyuggyuighjuyioyyfgffhyuhhghzzrrkkijjhjhhkkk}
\left(-S(M)\right)-
\left(-S(N)\right)\leq\int_a^b\frac{c}{c_0\left(\vec
r(s)\right)}\left|\vec
r'(s)\right|ds-\int_a^b\frac{c}{c^2_0\left(\vec r(s)\right)}\vec
{\tilde u}\left(\vec r(s)\right)\cdot\vec r'(s)ds.
\end{equation}
Moreover, if
\begin{equation}\label{MaxMedFullGGffgggyyojjyugggjhhjzzrrhhkkk}
\frac{d\vec r}{ds}(s)=\sigma(s)\vec h\left(\vec
r(s)\right):=\sigma(s)\left(\frac{c}{c^2_0\left(\vec
r(s)\right)}\vec {\tilde u}\left(\vec r(s)\right)-\nabla_{\vec
x}S\left(\vec r(s)\right)\right),
\end{equation}
for some nonnegative scalar factor $\sigma=\sigma(s)$ then by
\er{MaxMedFullGGffgggyyojjyugggjhhjzzrrhhkkk} we rewrite
\er{MaxMedFullGGffgggyyojjhhjkhjyyiuhggjhhjhuyytytyuuytrrtghjtyuggyuighjuyioyyfgffhyuhhghzzrrhhkkk}
as
\begin{equation}\label{MaxMedFullGGffgggyyojjhhjkhjyyiuhggjhhjhuyytytyuuytrrtghjtyuggyuighjuyioyyfgffhyuhhghzzrrkkhhkkk}
\left(-S(M)\right)-
\left(-S(N)\right)=\int_a^b\frac{c}{c_0\left(\vec
r(s)\right)}\left|\vec
r'(s)\right|ds-\int_a^b\frac{c}{c^2_0\left(\vec r(s)\right)}\vec
{\tilde u}\left(\vec r(s)\right)\cdot\vec r'(s)ds.
\end{equation}
Thus, by comparing
\er{MaxVacFullPPNmmmffffffiuiuhjuughbghhiuijghghghhhfhhghghguygtjuuujjjkk}
with \er{MaxMedFullGGffgggyyojjyugggjhhjzzrrhhkkk} and using
\er{MaxMedFullGGffgggyyojjhhjkhjyyiuhggjhhjhuyytytyuuytrrtghjtyuggyuighjuyioyyfgffhyuhhghzzrrkkijjhjhhkkk}
and
\er{MaxMedFullGGffgggyyojjhhjkhjyyiuhggjhhjhuyytytyuuytrrtghjtyuggyuighjuyioyyfgffhyuhhghzzrrkkhhkkk},
we deduce that if we assume that the light travel from the point $N$
to the point $M$ across the curve $\vec {\tilde
r}(s):[a,b]\to\mathbb{R}^3$ such that $\vec{\tilde r(a)}=N$ and
$\vec {\tilde r}(b)=M$, then
\begin{equation}\label{MaxMedFullGGffgggyyojjhhjkhjyyiuhggjhhjhuyytytyuuytrrtghjtyuggyuighjuyioyyfgffhyuhhghzzrrkkhhkkkhhh}
\left(-S(M)\right)-
\left(-S(N)\right)=\int_a^b\frac{c}{c_0\left(\vec{\tilde
r}(s)\right)}\left|\vec {\tilde
r}'(s)\right|ds-\int_a^b\frac{c}{c^2_0\left(\vec{\tilde
r}(s)\right)}\vec {\tilde u}\left(\vec{\tilde r}(s)\right)\cdot\vec
{\tilde r}'(s)ds,
\end{equation}
and for every other curve $\vec r(s):[a,b]\to\mathbb{R}^3$ with
endpoints $\vec r(a)=N$ and $\vec r(b)=M$ we have
%\er{MaxMedFullGGffgggyyojjhhjkhjyyiuhggjhhjhuyytytyuuytrrtghjtyuggyuighjuyioyyfgffhyuhhghzzrrkkhhkkk}.
\begin{multline}\label{MaxMedFullGGffgggyyojjhhjkhjyyiuhggjhhjhuyytytyuuytrrtghjtyuggyuighjuyioyyfgffhyuhhghzzrrkkhhkkkhhhjhkjhh}
\int_a^b\frac{c}{c_0\left(\vec r(s)\right)}\left|\vec
r'(s)\right|ds-\int_a^b\frac{c}{c^2_0\left(\vec r(s)\right)}\vec
{\tilde u}\left(\vec r(s)\right)\cdot\vec r'(s)ds\geq\\
\int_a^b\frac{c}{c_0\left(\vec{\tilde r}(s)\right)}\left|\vec
{\tilde r}'(s)\right|ds-\int_a^b\frac{c}{c^2_0\left(\vec{\tilde
r}(s)\right)}\vec {\tilde u}\left(\vec{\tilde r}(s)\right)\cdot\vec
{\tilde r}'(s)ds.
\end{multline}
So we obtain the following Fermat Principle:
\begin{proposition}\label{gughghf}
Assume Geometric Optics approximation together with \er{ojhkk}. Then
the light that travels from point $N$ to point $M$ chooses the path
$\vec r(s):[a,b]\to\mathbb{R}^3$ with endpoints $\vec r(a)=N$ and
$\vec r(b)=M$ which minimizes the quantity:
\begin{equation}\label{MaxMedFullGGffgggyyojjhhjkhjyyiuhggjhhjhuyytytyuuytrrtghjtyuggyuighjuyioyyfgffhyuhhghzzrrkkhhkkkhhhjhkjhhghhggh}
J\left(\vec r(\cdot)\right):=\int_a^bn\left(\vec
r(s)\right)\left|\vec r'(s)\right|ds-\int_a^b
\frac{1}{c}n^2\left(\vec r(s)\right)\vec {\tilde u}\left(\vec
r(s)\right)\cdot\vec r'(s)ds,
\end{equation}
where we set the refraction index:
\begin{equation}\label{MaxMedFullGGffgggyyojjhhjkhjyyiuhggjhhjhuyytytyuuytrrtghjtyuggyuighjuyioyyfgffhyuhhghzzrrkkhhkkkhhhjhkjhhghhgghiuiu1}
n\left(\vec x\right):=\frac{c}{c_0\left(\vec x\right)}.
\end{equation}
Moreover, if $\vec r(s):[a,b]\to\mathbb{R}^3$ with endpoints $\vec
r(a)=N$ and $\vec r(b)=M$ is the real path of the light, then:
\begin{equation}\label{MaxMedFullGGffgggyyojjhhjkhjyyiuhggjhhjhuyytytyuuytrrtghjtyuggyuighjuyioyyfgffhyuhhghzzrrkkhhkkkoiioio}
\left(-S(M)\right)- \left(-S(N)\right)=\int_a^bn\left(\vec
r(s)\right)\left|\vec r'(s)\right|ds-\int_a^b
\frac{1}{c}n^2\left(\vec r(s)\right)\vec {\tilde u}\left(\vec
r(s)\right)\cdot\vec r'(s)ds.
\end{equation}
\end{proposition}
In particular, by Proposition \ref{gughghf} the path of travel of
the light satisfies the Euler-Lagrange equation for the functional
$J\left(\vec r(\cdot)\right)$:
\begin{multline}\label{MaxMedFullGGffgggyyojjhhjkhjyyiuhggjhhjhuyytytyuuytrrtghjtyuggyuighjuyioyyfgffhyuhhghzzrrkkhhkkkhhhjhkjhhghhgghiuiu2}
\frac{d}{ds}\left(n\left(\vec r(s)\right)\frac{1}{\left|\vec
r'(s)\right|}\vec r'(s)-\frac{1}{c}n^2\left(\vec r(s)\right)\vec
{\tilde u}\left(\vec r(s)\right)\right)=\\
\left|\vec r'(s)\right|\nabla_{\vec x}n\left(\vec
r(s)\right)-\frac{2}{c}\left(\vec {\tilde u}\left(\vec
r(s)\right)\cdot\vec r'(s)\right)n\left(\vec r(s)\right)\nabla_{\vec
x}n\left(\vec r(s)\right)-\frac{1}{c}n^2\left(\vec
r(s)\right)\left\{d_{\vec x}\vec {\tilde u}\left(\vec
r(s)\right)\right\}^T\cdot\vec r'(s),
\end{multline}
that we rewrite as:
\begin{multline}\label{MaxMedFullGGffgggyyojjhhjkhjyyiuhggjhhjhuyytytyuuytrrtghjtyuggyuighjuyioyyfgffhyuhhghzzrrkkhhkkkhhhjhkjhhghhgghiuiuhhjhj1}
\frac{1}{\left|\vec r'(s)\right|}\frac{d}{ds}\left(n\left(\vec
r(s)\right)\frac{1}{\left|\vec r'(s)\right|}\vec r'(s)\right)=\\
\nabla_{\vec x}n\left(\vec r(s)\right)+\frac{1}{c}n^2\left(\vec
r(s)\right)\left(d_{\vec x}\vec {\tilde u}\left(\vec
r(s)\right)-\left\{d_{\vec x}\vec {\tilde u}\left(\vec
r(s)\right)\right\}^T\right)\cdot\left(\frac{1}{\left|\vec
r'(s)\right|}\vec r'(s)\right)\\
+\frac{2}{c}n\left(\vec r(s)\right)\left\{\vec {\tilde u}\left(\vec
r(s)\right)\otimes\nabla_{\vec x}n\left(\vec
r(s)\right)-\nabla_{\vec x}n\left(\vec r(s)\right)\otimes\vec
{\tilde u}\left(\vec
r(s)\right)\right\}\cdot\left(\frac{1}{\left|\vec r'(s)\right|}\vec
r'(s)\right).
\end{multline}
Therefor by \er{apfrm9} and
\er{MaxMedFullGGffgggyyojjhhjkhjyyiuhggjhhjhuyytytyuuytrrtghjtyuggyuighjuyioyyfgffhyuhhghzzrrkkhhkkkhhhjhkjhhghhgghiuiuhhjhj1}
we deduce the differential equation of the path of light:
\begin{multline}\label{MaxMedFullGGffgggyyojjhhjkhjyyiuhggjhhjhuyytytyuuytrrtghjtyuggyuighjuyioyyfgffhyuhhghzzrrkkhhkkkhhhjhkjhhghhgghiuiuhhjhj}
\frac{d}{d\lambda}\left(n\left(\vec r\right)\frac{d\vec
r}{d\lambda}\right)=\frac{1}{c}n^2\left(\vec
r\right)\left(curl_{\vec x}\vec {\tilde u}\left(\vec
r\right)\right)\times\frac{d\vec r}{d\lambda}\\ +\nabla_{\vec
x}n\left(\vec r\right) +\frac{2}{c}n\left(\vec r\right)\left\{\vec
{\tilde u}\left(\vec r\right)\otimes\nabla_{\vec x}n\left(\vec
r\right)-\nabla_{\vec x}n\left(\vec r\right)\otimes\vec {\tilde
u}\left(\vec r\right)\right\}\cdot\frac{d\vec r}{d\lambda},
\end{multline}
where
\begin{equation}\label{MaxMedFullGGffgggyyojjhhjkhjyyiuhggjhhjhuyytytyuuytrrtghjtyuggyuighjuyioyyfgffhyuhhghzzrrkkhhkkkhhhjhkjhhghhgghiuiu}
\lambda:=\int_a^s\left|\vec r'(\tau)\right|d\tau,
\end{equation}
is the natural parameter of the curve.


Next, assume that the wave we consider has an electromagnetic
nature. Then by \er{gughhghfbvnbv} and \er{uyuyuyy} we have
\begin{equation}\label{gughhghfbvnbvyyuuyr}
c_0=c\sqrt{\kappa_0\gamma_0}\quad\text{and}\quad\vec {\tilde
u}=\left(\gamma_0\vec v+(1-\gamma_0)\vec u\right),
\end{equation}
where, $\vec u$ is the medium velocity and $\vec v$ is the local
vectorial gravitational potential. Moreover, assume that we consider
light traveling in some region either filled with the resting medium
of constant dielectric permeability $\gamma_0$ and magnetic
permeability $\kappa_0$ or in the vacuum. Then by
\er{gughhghfbvnbvyyuuyr} and
\er{MaxMedFullGGffgggyyojjhhjkhjyyiuhggjhhjhuyytytyuuytrrtghjtyuggyuighjuyioyyfgffhyuhhghzzrrkkhhkkkhhhjhkjhhghhgghiuiu1}
we have:
\begin{equation}\label{gughhghfbvnbvyyuuyrhiyyuiuu}
n=\frac{1}{\sqrt{\kappa_0\gamma_0}}\;\;\;\text{is a
constant,}\quad\text{and}\quad\vec {\tilde u}=\gamma_0\vec v,
\end{equation}
Then by \er{gughhghfbvnbvyyuuyrhiyyuiuu} we rewrite
\er{MaxMedFullGGffgggyyojjhhjkhjyyiuhggjhhjhuyytytyuuytrrtghjtyuggyuighjuyioyyfgffhyuhhghzzrrkkhhkkkhhhjhkjhhghhgghiuiuhhjhj}
as:
\begin{equation}\label{MaxMedFullGGffgggyyojjhhjkhjyyiuhggjhhjhuyytytyuuytrrtghjtyuggyuighjuyioyyfgffhyuhhghzzrrkkhhkkkhhhjhkjhhghhgghiuiuhhjhjiuiuyu}
\frac{d^2\vec
r}{d\lambda^2}=\frac{1}{c}\sqrt{\frac{\gamma_0}{\kappa_0}}\left(curl_{\vec
x}\vec v\left(\vec r\right)\right)\times\frac{d\vec r}{d\lambda}.
\end{equation}
In particular, if our coordinate system is inertial, or more
generally non-rotating, then $curl_{\vec x}\vec v=0$ and we deduce
that the path of the light from the point $N$ to the point $M$ is
the direct line connecting these points, provided we take in the
account estimation \er{ojhkk}.

On the other hand, if our system is rotating, then, since $\vec v$
is a speed-like vector field, we clearly deduce:
\begin{equation}\label{MaxMedFullGGffgggyyojjhhjkhjyyiuhggjhhjhuyytytyuuytrrtghjtyuggyuighjuyioyyfgffhyuhhghzzrrkkhhkkkhhhjhkjhhghhgghiuiuhhjhjiuiuyujjk}
curl_{\vec x}\vec v=-2\vec w,
\end{equation}
where $\vec w$ is the vector of the angular speed of rotation of our
coordinate system. Thus by inserting
\er{MaxMedFullGGffgggyyojjhhjkhjyyiuhggjhhjhuyytytyuuytrrtghjtyuggyuighjuyioyyfgffhyuhhghzzrrkkhhkkkhhhjhkjhhghhgghiuiuhhjhjiuiuyujjk}
into
\er{MaxMedFullGGffgggyyojjhhjkhjyyiuhggjhhjhuyytytyuuytrrtghjtyuggyuighjuyioyyfgffhyuhhghzzrrkkhhkkkhhhjhkjhhghhgghiuiuhhjhjiuiuyu}
we deduce:
\begin{equation}\label{MaxMedFullGGffgggyyojjhhjkhjyyiuhggjhhjhuyytytyuuytrrtghjtyuggyuighjuyioyyfgffhyuhhghzzrrkkhhkkkhhhjhkjhhghhgghiuiuhhjhjiuiuyuyyu}
\frac{d^2\vec
r}{d\lambda^2}=-\frac{2}{c}\sqrt{\frac{\gamma_0}{\kappa_0}}\vec
w\times\frac{d\vec r}{d\lambda}.
\end{equation}
In particular, by
\er{MaxMedFullGGffgggyyojjhhjkhjyyiuhggjhhjhuyytytyuuytrrtghjtyuggyuighjuyioyyfgffhyuhhghzzrrkkhhkkkhhhjhkjhhghhgghiuiuhhjhjiuiuyuyyu}
if we consider that $\vec w=(0,0,w)$ and $\vec r=(x,y,z)$, then
there exist three dimensionless constants $C_1$, $C_2$ and $C_3$
such that
\begin{equation}\label{MaxMedFullGGffgggyyojjhhjkhjyyiuhggjhhjhuyytytyuuytrrtghjtyuggyuighjuyioyyfgffhyuhhghzzrrkkhhkkkhhhjhkjhhghhgghiuiuhhjhjiuiuyuyyukkl}
\begin{cases}
\frac{dx}{d\lambda}=-C_1\sin{\left(\frac{2w}{c}\sqrt{\frac{\gamma_0}{\kappa_0}}\lambda\right)}+C_2\cos{\left(\frac{2w}{c}\sqrt{\frac{\gamma_0}{\kappa_0}}\lambda\right)}
\\
\frac{dy}{d\lambda}=-C_1\cos{\left(\frac{2w}{c}\sqrt{\frac{\gamma_0}{\kappa_0}}\lambda\right)}-C_2\sin{\left(\frac{2w}{c}\sqrt{\frac{\gamma_0}{\kappa_0}}\lambda\right)}
\\
\frac{dz}{d\lambda}=C_3,
\end{cases}
\end{equation}
and moreover, since $\lambda$ is a natural parameter, the constants
satisfy:
\begin{equation}\label{MaxMedFullGGffgggyyojjhhjkhjyyiuhggjhhjhuyytytyuuytrrtghjtyuggyuighjuyioyyfgffhyuhhghzzrrkkhhkkkhhhjhkjhhghhgghiuiuhhjhjiuiuyuyyuojk}
C^2_1+C^2_2+C^2_3=1.
\end{equation}
Then by
\er{MaxMedFullGGffgggyyojjhhjkhjyyiuhggjhhjhuyytytyuuytrrtghjtyuggyuighjuyioyyfgffhyuhhghzzrrkkhhkkkhhhjhkjhhghhgghiuiuhhjhjiuiuyuyyukkl}
there exist three additional constants $D_1$, $D_2$ and $D_3$ such
that
\begin{equation}\label{MaxMedFullGGffgggyyojjhhjkhjyyiuhggjhhjhuyytytyuuytrrtghjtyuggyuighjuyioyyfgffhyuhhghzzrrkkhhkkkhhhjhkjhhghhgghiuiuhhjhjiuiuyuyyukklghhg}
\begin{cases}
x(\lambda)=C_1\frac{c}{2w}\sqrt{\frac{\kappa_0}{\gamma_0}}\left(\cos{\left(\frac{2w}{c}\sqrt{\frac{\gamma_0}{\kappa_0}}\lambda\right)}-1\right)+C_2\frac{c}{2w}\sqrt{\frac{\kappa_0}{\gamma_0}}\sin{\left(\frac{2w}{c}\sqrt{\frac{\gamma_0}{\kappa_0}}\lambda\right)}+D_1
\\
y(\lambda)=-C_1\frac{c}{2w}\sqrt{\frac{\kappa_0}{\gamma_0}}\sin{\left(\frac{2w}{c}\sqrt{\frac{\gamma_0}{\kappa_0}}\lambda\right)}+C_2\frac{c}{2w}\sqrt{\frac{\kappa_0}{\gamma_0}}\left(\cos{\left(\frac{2w}{c}\sqrt{\frac{\gamma_0}{\kappa_0}}\lambda\right)}-1\right)+D_2
\\
z(\lambda)=C_3\lambda+D_3.
\end{cases}
\end{equation}
So, the curve in
\er{MaxMedFullGGffgggyyojjhhjkhjyyiuhggjhhjhuyytytyuuytrrtghjtyuggyuighjuyioyyfgffhyuhhghzzrrkkhhkkkhhhjhkjhhghhgghiuiuhhjhjiuiuyuyyukklghhg}
is the trajectory of the light in the rotating coordinate system,
provided we assume \er{ojhkk}. In particular, by
\er{MaxMedFullGGffgggyyojjhhjkhjyyiuhggjhhjhuyytytyuuytrrtghjtyuggyuighjuyioyyfgffhyuhhghzzrrkkhhkkkhhhjhkjhhghhgghiuiuhhjhjiuiuyuyyukklghhg}
and
\er{MaxMedFullGGffgggyyojjhhjkhjyyiuhggjhhjhuyytytyuuytrrtghjtyuggyuighjuyioyyfgffhyuhhghzzrrkkhhkkkhhhjhkjhhghhgghiuiuhhjhjiuiuyuyyukkl}
we have:
\begin{equation}\label{MaxMedFullGGffgggyyojjhhjkhjyyiuhggjhhjhuyytytyuuytrrtghjtyuggyuighjuyioyyfgffhyuhhghzzrrkkhhkkkhhhjhkjhhghhgghiuiuhhjhjiuiuyuyyuojkjkkkhgugg}
\begin{cases}
x(0)=D_1,\quad y(0)=D_2,\quad z(0)=D_3,\\
%\quad\text{and}\quad
\frac{dx}{d\lambda}(0)=C_2,\quad\frac{dy}{d\lambda}(0)=-C_1,\quad\frac{dz}{d\lambda}(0)=C_3.
\end{cases}
\end{equation}
%
%
%
\begin{comment} Moreover, by
\er{MaxMedFullGGffgggyyojjhhjkhjyyiuhggjhhjhuyytytyuuytrrtghjtyuggyuighjuyioyyfgffhyuhhghzzrrkkhhkkkhhhjhkjhhghhgghiuiuhhjhjiuiuyuyyukkl}
the radius of curvature of the curve satisfies:
\begin{equation}\label{MaxMedFullGGffgggyyojjhhjkhjyyiuhggjhhjhuyytytyuuytrrtghjtyuggyuighjuyioyyfgffhyuhhghzzrrkkhhkkkhhhjhkjhhghhgghiuiuhhjhjiuiuyuyyukklklihi}
R:=\frac{1}{\left|\frac{d^2\vec
r}{d\lambda^2}\right|}=\frac{c}{2w}\sqrt{\frac{\kappa_0}{\gamma_0}}\frac{1}{C^2_1+C^2_2}\,.
\end{equation}
\end{comment}
%
%
%
The constants $C_1,C_2,C_3,D_1,D_2,D_3$ can be determined either by
the initial data
\er{MaxMedFullGGffgggyyojjhhjkhjyyiuhggjhhjhuyytytyuuytrrtghjtyuggyuighjuyioyyfgffhyuhhghzzrrkkhhkkkhhhjhkjhhghhgghiuiuhhjhjiuiuyuyyuojkjkkkhgugg}
or by the beginning and the ending points $N$ and $M$ of the curve.
%
%
%
%
\begin{comment}
Consider the coordinates of the endpoints of the curve as
$N=(N_1,N_2,N_3)$ and $M=(M_1,M_2,M_3)$. Then for $\lambda=0$ we
must have $(x,y,z)=N$ i.e.
\begin{equation}\label{MaxMedFullGGffgggyyojjhhjkhjyyiuhggjhhjhuyytytyuuytrrtghjtyuggyuighjuyioyyfgffhyuhhghzzrrkkhhkkkhhhjhkjhhghhgghiuiuhhjhjiuiuyuyyuojkjkkkh}
\begin{cases}
D_1=N_1\\
D_2=N_2,\\
D_3=N_3.
\end{cases}
\end{equation}
Moreover we must have
\er{MaxMedFullGGffgggyyojjhhjkhjyyiuhggjhhjhuyytytyuuytrrtghjtyuggyuighjuyioyyfgffhyuhhghzzrrkkhhkkkhhhjhkjhhghhgghiuiuhhjhjiuiuyuyyuojk}.
Finally there must exist $\lambda_0>0$, which turns to be the length
of the curve, such that
\begin{equation}\label{MaxMedFullGGffgggyyojjhhjkhjyyiuhggjhhjhuyytytyuuytrrtghjtyuggyuighjuyioyyfgffhyuhhghzzrrkkhhkkkhhhjhkjhhghhgghiuiuhhjhjiuiuyuyyukklghhghyhjhjh}
\begin{cases}
M_1=C_1\frac{c}{2w}\sqrt{\frac{\kappa_0}{\gamma_0}}\left(\cos{\left(\frac{2w}{c}\sqrt{\frac{\gamma_0}{\kappa_0}}\lambda_0\right)}-1\right)+C_2\frac{c}{2w}\sqrt{\frac{\kappa_0}{\gamma_0}}\sin{\left(\frac{2w}{c}\sqrt{\frac{\gamma_0}{\kappa_0}}\lambda_0\right)}+D_1
\\
M_2=-C_1\frac{c}{2w}\sqrt{\frac{\kappa_0}{\gamma_0}}\sin{\left(\frac{2w}{c}\sqrt{\frac{\gamma_0}{\kappa_0}}\lambda_0\right)}+C_2\frac{c}{2w}\sqrt{\frac{\kappa_0}{\gamma_0}}\left(\cos{\left(\frac{2w}{c}\sqrt{\frac{\gamma_0}{\kappa_0}}\lambda_0\right)}-1\right)+D_2
\\
M_3=C_3\lambda_0+D_3.
\end{cases}
\end{equation}
\end{comment}
%
%
%
%
%
%
%
%
%
\begin{comment}
Moreover, by
\er{MaxMedFullGGffgggyyojjhhjkhjyyiuhggjhhjhuyytytyuuytrrtghjtyuggyuighjuyioyyfgffhyuhhghzzrrkkhhkkkhhhjhkjhhghhgghiuiuhhjhjiuiuyuyyukkl}
the radius of curvature of the curve satisfies:
\begin{equation}\label{MaxMedFullGGffgggyyojjhhjkhjyyiuhggjhhjhuyytytyuuytrrtghjtyuggyuighjuyioyyfgffhyuhhghzzrrkkhhkkkhhhjhkjhhghhgghiuiuhhjhjiuiuyuyyukklklihi}
R:=\frac{1}{\left|\frac{d^2\vec
r}{d\lambda^2}\right|}=\frac{c}{2w}\sqrt{\frac{\kappa_0}{\gamma_0}}\frac{1}{C^2_1+C^2_2}\,.
\end{equation}




Next consider the direct line, parameterized by the natural
parameter, passing through the same initial point and having the
same initial tangent vector as the curve in
\er{MaxMedFullGGffgggyyojjhhjkhjyyiuhggjhhjhuyytytyuuytrrtghjtyuggyuighjuyioyyfgffhyuhhghzzrrkkhhkkkhhhjhkjhhghhgghiuiuhhjhjiuiuyuyyukklghhg}
and
\er{MaxMedFullGGffgggyyojjhhjkhjyyiuhggjhhjhuyytytyuuytrrtghjtyuggyuighjuyioyyfgffhyuhhghzzrrkkhhkkkhhhjhkjhhghhgghiuiuhhjhjiuiuyuyyuojkjkkkhgugg}.
Then we can write this line as:
\begin{equation}\label{MaxMedFullGGffgggyyojjhhjkhjyyiuhggjhhjhuyytytyuuytrrtghjtyuggyuighjuyioyyfgffhyuhhghzzrrkkhhkkkhhhjhkjhhghhgghiuiuhhjhjiuiuyuyyukklghhgjkjk}
\begin{cases}
x_0(\lambda)=C_2\lambda+D_1
\\
y_0(\lambda)=-C_1\lambda+D_2
\\
z_0(\lambda)=C_3\lambda+D_3.
\end{cases}
\end{equation}
In particular by comparing
\er{MaxMedFullGGffgggyyojjhhjkhjyyiuhggjhhjhuyytytyuuytrrtghjtyuggyuighjuyioyyfgffhyuhhghzzrrkkhhkkkhhhjhkjhhghhgghiuiuhhjhjiuiuyuyyukklghhg}
and
\er{MaxMedFullGGffgggyyojjhhjkhjyyiuhggjhhjhuyytytyuuytrrtghjtyuggyuighjuyioyyfgffhyuhhghzzrrkkhhkkkhhhjhkjhhghhgghiuiuhhjhjiuiuyuyyukklghhgjkjk}
and using Teylor formula from the basic calculus, together with
\er{MaxMedFullGGffgggyyojjhhjkhjyyiuhggjhhjhuyytytyuuytrrtghjtyuggyuighjuyioyyfgffhyuhhghzzrrkkhhkkkhhhjhkjhhghhgghiuiuhhjhjiuiuyuyyuojk},
we deduce:
\begin{equation}\label{MaxMedFullGGffgggyyojjhhjkhjyyiuhggjhhjhuyytytyuuytrrtghjtyuggyuighjuyioyyfgffhyuhhghzzrrkkhhkkkhhhjhkjhhghhgghiuiuhhjhjiuiuyuyyukklghhgjkjkiuuui}
\begin{cases}
\left|x(\lambda)-x_0(\lambda)\right|\leq
%(C_1+C_2)
\frac{2w}{c}\sqrt{\frac{\gamma_0}{\kappa_0}}
\left(1+\frac{2w}{c}\sqrt{\frac{\gamma_0}{\kappa_0}}\left|\lambda\right|\right)\lambda^2
\\
\left|y(\lambda)-y_0(\lambda)\right|\leq
%(C_1+C_2)
\frac{2w}{c}\sqrt{\frac{\gamma_0}{\kappa_0}}
\left(1+\frac{2w}{c}\sqrt{\frac{\gamma_0}{\kappa_0}}\left|\lambda\right|\right)\lambda^2
\\
z(\lambda)=z_0(\lambda).
\end{cases}
\end{equation}
In particular we obtain the following estimate of deviation of the
path of light for the direct line with the same initial point and
initial tangent vector in a rotating coordinate system:
\begin{equation}\label{MaxMedFullGGffgggyyojjhhjkhjyyiuhggjhhjhuyytytyuuytrrtghjtyuggyuighjuyioyyfgffhyuhhghzzrrkkhhkkkhhhjhkjhhghhgghiuiuhhjhjiuiuyuyyukklghhgjkjkiuuuioiojo}
\left(\left|x(\lambda)-x_0(\lambda)\right|
+\left|y(\lambda)-y_0(\lambda)\right|+\left|z(\lambda)-z_0(\lambda)\right|\right)\,=\,|\lambda|\;O\left(\frac{2w\lambda}{c}\right),
\end{equation}
where $\lambda$ is a natural parameter of the curve.
\end{comment}
%
%
%
%
%








\subsubsection{The laws of reflection and refraction}\label{rrGO}
Next consider a
monochromatic wave of the frequency $\nu=\omega/(2\pi)$
characterized by:
\begin{equation}\label{MaxVacFullPPNmmmffffffiuiuhjuughbghhuiiujjhhjjhjhhj}
U(\vec x,t)=A(\vec x,t)e^{ik_0S(\vec x,t)},\quad\text{where}\quad
k_0=\frac{\omega}{c}\quad\text{and}\quad\frac{\partial S}{\partial
t}=c\,,
\end{equation}
and, consistently with
\er{MaxVacFullPPNmmmffffffiuiuhjuughbghhiuijghghghhhfhhghghguygtjuuujjjkyuuy}
consider a direction field:
\begin{equation}\label{MaxVacFullPPNmmmffffffiuiuhjuughbghhiuijghghghhhfhhghghguygtjuuujjjkyuuykjkjj}
%\frac{1}{c^2_0}\frac{\partial^2}{\partial t^2}\left(Ae^{ik_0S}\right)
\vec h(\vec x)=\frac{c}{c^2_0(\vec x)}\vec {\tilde u}(\vec
x)-\nabla_{\vec x}S(\vec x).
\end{equation}
Furthermore, assume that this wave undergoes reflection and/or
refraction on the surface $\mathcal{T}$ with the outcoming unit
normal $\vec n$, separating two regions characterized respectively
by $c_0=c^{(1)}_0$ and $\vec {\tilde u}=\vec {\tilde u}_1$ and by
$c^{(2)}_0$ and $\vec {\tilde u}_2$, with the formation of the
reflected wave (of the same frequency), characterized by:
\begin{equation}\label{MaxVacFullPPNmmmffffffiuiuhjuughbghhuiiujjhhjjhjhhjugh}
U_1(\vec x,t)=A_1(\vec x,t)e^{ik_0S_1(\vec
x,t)},\quad\text{where}\quad\frac{\partial S_1}{\partial t}=c\,,
\end{equation}
and by a direction field:
\begin{equation}\label{MaxVacFullPPNmmmffffffiuiuhjuughbghhiuijghghghhhfhhghghguygtjuuujjjkyuuykjkjjhjhhj}
%\frac{1}{c^2_0}\frac{\partial^2}{\partial t^2}\left(Ae^{ik_0S}\right)
\vec h_1(\vec x)=\frac{c}{c^2_0(\vec x)}\vec {\tilde u}(\vec
x)-\nabla_{\vec x}S_1(\vec x),
\end{equation}
and formation of the refracted wave (of the same frequency),
characterized by:
\begin{equation}\label{MaxVacFullPPNmmmffffffiuiuhjuughbghhuiiujjhhjjhjhhj1}
U_2(\vec x,t)=A_2(\vec x,t)e^{ik_0S_2(\vec
x,t)},\quad\text{where}\quad\frac{\partial S_2}{\partial t}=c\,.
\end{equation}
and by a direction field:
\begin{equation}\label{MaxVacFullPPNmmmffffffiuiuhjuughbghhiuijghghghhhfhhghghguygtjuuujjjkyuuykjkjj1}
%\frac{1}{c^2_0}\frac{\partial^2}{\partial t^2}\left(Ae^{ik_0S}\right)
\vec h_2(\vec x)=\frac{c}{\left(c^{(2)}_0(\vec x)\right)^2}\vec
{\tilde u_2}(\vec x)-\nabla_{\vec x}S_2(\vec x).
\end{equation}
Then the boundary conditions of $U$, $U_1$ and $U_2$ depend on the
physical meaning of these fields. However, one of the
\underline{necessary} conditions should be that
\begin{equation}\label{MaxMedFullGGffgggyyojjyugggjhhjiiuuiyuyuyyuyuzzhhkkk}
%\omega^{(1)}=\omega\quad\text{and}\quad
S_1(\vec x,t)=S_2(\vec x,t)+C_2=S(\vec x,t)\quad\quad\forall\vec
x\in\mathcal{T},
%\quad\text{and thus}\quad
\end{equation}
where $C_2$ is a real constant. In particular
\er{MaxMedFullGGffgggyyojjyugggjhhjiiuuiyuyuyyuyuzzhhkkk} implies
\begin{equation}\label{MaxMedFullGGffgggyyojjyugggjhhjiiuuiyuyuyyuyuzzhhkkkgdfg}
\nabla_{\vec x} S_1-\left(\vec n\cdot \nabla_{\vec x} S_1\right)\vec
n
%=\nabla_{\vec x} S-\left(\vec n\cdot \nabla_{\vec x} S\right)\vec n
=\nabla_{\vec x} S_2-\left(\vec n\cdot \nabla_{\vec x}
S_2\right)\vec n=\nabla_{\vec x} S-\left(\vec n\cdot \nabla_{\vec x}
S\right)\vec n\quad\quad\forall\vec x\in\mathcal{T}.
\end{equation}
In particular, for every point on the surface $\mathcal{T}$ vectors
$\nabla_{\vec x} S_1$ and $\nabla_{\vec x} S_2$ lie in the plane
formed by vectors $\vec n$ and $\nabla_{\vec x} S$. Moreover, by
\er{MaxVacFullPPNmmmffffffiuiuhjuughbghhiuijghghghhhfhhghghguygtjuuujjjkyuuykjkjj},
\er{MaxVacFullPPNmmmffffffiuiuhjuughbghhiuijghghghhhfhhghghguygtjuuujjjkyuuykjkjjhjhhj}
and \er{MaxMedFullGGffgggyyojjyugggjhhjiiuuiyuyuyyuyuzzhhkkkgdfg} we
have
\begin{equation}\label{MaxMedFullGGffgggyyojjyugggjhhjiiuuiyuyuyyuyuzzhhkkkgdfghgghgh}
\vec h_1-\left(\vec n\cdot \vec h_1\right)\vec n=\vec h-\left(\vec
n\cdot \vec h\right)\vec n\quad\quad\forall\vec x\in\mathcal{T},
\end{equation}
and in particular, for every point on the surface $\mathcal{T}$
vector $\vec h_1$ lies in the plane formed by vectors $\vec n$ and
$\vec h$.
%\begin{equation}\label{MaxMedFullGGffgggyyojjyugggjhhjiiuuiyuyuyyuyuzzhhkkkgdfg}
%\left(\nabla_{\vec x} S^{(1)}-\nabla_{\vec x} S\right)\times\vec n=\left(\nabla_{\vec x} S^{(2)}-\nabla_{\vec x} S\right)\times\vec n=0\quad\forall\vec x\in\mathcal{T}.
%\end{equation}
Next, assume that the approximate equations in
\er{MaxVacFullPPNmmmffffffiuiuhjuughbghhiuijghghghhjhjhhghyuyiyyujjljk}
and
\er{MaxVacFullPPNmmmffffffiuiuhjuughbghhiuijghghghhhfhhghghguygtjuuujjjk}
are valid in every of two regions on the both sides of
$\mathcal{T}$. Then by
\er{MaxVacFullPPNmmmffffffiuiuhjuughbghhiuijghghghhjhjhhghyuyiyyujjljkgghhg}
we have
\begin{equation}\label{MaxVacFullPPNmmmffffffiuiuhjuughbghhiuijghghghhjhjhhghyuyiyyujjljkgghhguyuy}
|\vec h_1|=|\vec h|=\frac{c}{c_0}.
\end{equation}
Then, since $\vec h_1\neq\vec h$, by
\er{MaxMedFullGGffgggyyojjyugggjhhjiiuuiyuyuyyuyuzzhhkkkgdfghgghgh}
and
\er{MaxVacFullPPNmmmffffffiuiuhjuughbghhiuijghghghhjhjhhghyuyiyyujjljkgghhguyuy}
we deduce
\begin{equation}\label{MaxMedFullGGffgggyyojjyugggjhhjiiuuiyuyuyyuyuzzhhkkkgdfghgghghijhjhhkh}
\vec n\cdot \vec h_1=-\vec n\cdot \vec h\quad\quad\forall\vec
x\in\mathcal{T}.
\end{equation}
So, by
\er{MaxVacFullPPNmmmffffffiuiuhjuughbghhiuijghghghhjhjhhghyuyiyyujjljkgghhguyuy}
and
\er{MaxMedFullGGffgggyyojjyugggjhhjiiuuiyuyuyyuyuzzhhkkkgdfghgghghijhjhhkh}
we obtain the law of reflection: vector $\vec h_1$ lies in the plane
formed by vectors $\vec n$ and $\vec h$, and we have:
\begin{equation}\label{MaxMedFulljhhjjjjj}
\theta\left(\vec h,-\vec n\right)=\theta_1\left(\vec h_1,\vec
n\right)
\end{equation}
where $\theta\left(\vec h,-\vec n\right)$ is the angle between the
incoming beam direction $\vec h$ and the incoming normal to the
surface $-\vec n$ and $\theta_1\left(\vec h_1,\vec n\right)$ is the
angle between the reflected beam direction $\vec h_1$ and the
outcoming normal $\vec n$.

Next assume that the wave we consider in
\er{MaxVacFullPPNmmmffffffiuiuhjuughbghhuiiujjhhjjhjhhj} has an
electromagnetic nature. Then by \er{gughhghfbvnbvyyuuyr} we have
\begin{equation}\label{gughhghfbvnbvyyuuy}
c_0=c\sqrt{\kappa_0\gamma_0}\quad\text{and}\quad\vec {\tilde
u}=\left(\gamma_0\vec v+(1-\gamma_0)\vec u\right),
\end{equation}
where, $\vec u$ is the medium velocity and $\vec v$ is the local
vectorial gravitational potential. Similarly, on the second side of
surface $\mathcal{T}$ we have
\begin{equation}\label{gughhghfbvnbvyyuuyGGHHG}
c^{(2)}_0=c\sqrt{\kappa^{(2)}_0\gamma^{(2)}_0}\quad\text{and}\quad\vec
{\tilde u}^{(2)}=\left(\gamma^{(2)}_0\vec v+(1-\gamma^{(2)}_0)\vec
u^{(2)}\right),
\end{equation}
where, $\vec u^{(2)}$ is the medium velocity on the second side of
surface $\mathcal{T}$. Furthermore, assume that the medium rests on
the both sides of surface $\mathcal{T}$ and the magnetic
permeability is the same on both sides of surface $\mathcal{T}$.
I.e. we have
\begin{equation}\label{gughhghfbvnbvyyuuyll}
\kappa^{(2)}_0=\kappa_0\quad\text{and}\quad\vec u^{(2)}=\vec u=0,
\end{equation}
however $\gamma^{(2)}_0$ can differ from $\gamma_0$. Then in this
particular case we rewrite \er{gughhghfbvnbvyyuuy} and
\er{gughhghfbvnbvyyuuyGGHHG} as
\begin{equation}\label{gughhghfbvnbvyyuuyggg}
c_0=c\sqrt{\kappa_0\gamma_0}\quad\text{and}\quad\vec {\tilde
u}=\gamma_0\vec v,
\end{equation}
and
\begin{equation}\label{gughhghfbvnbvyyuuyGGHHGhjhjh}
c^{(2)}_0=c\sqrt{\kappa_0\gamma^{(2)}_0}\quad\text{and}\quad\vec
{\tilde u}^{(2)}=\gamma^{(2)}_0\vec v,
\end{equation}
Then in particular, by \er{gughhghfbvnbvyyuuyggg} and
\er{gughhghfbvnbvyyuuyGGHHGhjhjh} we deduce
\begin{equation}\label{gughhghfbvnbvyyuuyGGHHGhjhjhhhjhj}
\frac{c}{\left(c^{(2)}_0\right)^2}\vec {\tilde
u}^{(2)}=\frac{c}{c^2_0}\vec {\tilde u}=\frac{1}{\kappa_0 c}\vec v.
\end{equation}
Thus, by inserting
\er{MaxVacFullPPNmmmffffffiuiuhjuughbghhiuijghghghhhfhhghghguygtjuuujjjkyuuykjkjj}
and \er{gughhghfbvnbvyyuuyGGHHGhjhjhhhjhj} into
\er{MaxMedFullGGffgggyyojjyugggjhhjiiuuiyuyuyyuyuzzhhkkkgdfg}, we
deduce:
\begin{equation}\label{MaxMedFullGGffgggyyojjyugggjhhjiiuuiyuyuyyuyuzzhhkkkgdfg1}
\vec h_2-\left(\vec n\cdot \vec h_2\right)\vec n=\vec h-\left(\vec
n\cdot \vec h\right)\vec n\quad\quad\forall\vec x\in\mathcal{T},
\end{equation}
and in particular, for every point on the surface $\mathcal{T}$
vector $\vec h_2$ lies in the plane formed by vectors $\vec n$ and
$\vec h$. On the other hand by
\er{MaxVacFullPPNmmmffffffiuiuhjuughbghhiuijghghghhjhjhhghyuyiyyujjljkgghhg}
we have:
\begin{equation}\label{MaxVacFullPPNmmmffffffiuiuhjuughbghhiuijghghghhjhjhhghyuyiyyujjljkgghhglkjlkjkjljljhhjj}
|\vec h|=\frac{c}{c_0}\quad\text{and}\quad|\vec
h_2|=\frac{c}{c^{(2)}_0}.
\end{equation}
So, by
\er{MaxMedFullGGffgggyyojjyugggjhhjiiuuiyuyuyyuyuzzhhkkkgdfg1} and
\er{MaxVacFullPPNmmmffffffiuiuhjuughbghhiuijghghghhjhjhhghyuyiyyujjljkgghhglkjlkjkjljljhhjj},
in the cases when \er{gughhghfbvnbvyyuuyll} holds, we have the
Snell's law of refraction: vector $\vec h_2$ lies in the plane
formed by vectors $\vec n$ and $\vec h$, and we have:
\begin{equation}\label{MaxMedFulljhhjjjjjhhhj}
n\sin{\left(\theta\left(\vec h,\vec
n\right)\right)}=n_2\sin{\left(\theta_2\left(\vec h_2,\vec
n\right)\right)}
\end{equation}
where $\theta\left(\vec h,\vec n\right)$ is the angle between the
incoming beam direction $\vec h$ and the normal to the surface $\vec
n$, $\theta_2\left(\vec h_2,\vec n\right)$ is the angle between the
refracted beam direction $\vec h_2$ and the normal $\vec n$ and as
in
\er{MaxMedFullGGffgggyyojjhhjkhjyyiuhggjhhjhuyytytyuuytrrtghjtyuggyuighjuyioyyfgffhyuhhghzzrrkkhhkkkhhhjhkjhhghhgghiuiu1}
we set refraction indexes:
\begin{equation}\label{MaxMedFullGGffgggyyojjhhjkhjyyiuhggjhhjhuyytytyuuytrrtghjtyuggyuighjuyioyyfgffhyuhhghzzrrkkhhkkkhhhjhkjhhghhgghiuiujjkjk}
n:=\frac{c}{c_0}\quad\text{and}\quad n_2:=\frac{c}{c^{(2)}_0}.
\end{equation}
Note, that in the case when \er{gughhghfbvnbvyyuuyll} dose not hold,
however we have $\vec {\tilde u}^{(2)}=\vec {\tilde u}=0$ instead,
the Snell's law still holds. However, in the frames of our model, in
contrast to the law of reflection, the Snell's law dose not hold
exactly in the case where the magnetic permeability $\kappa_0$ on
the one side of surface $\mathcal{T}$ differ from $\kappa^{(2)}_0$
on the another side of the surface and at the same time the field
$\vec v\neq 0$ is nontrivial.
%\begin{equation}\label{MaxVacFullPPNmmmffffffiuiuhjuughbghhiuijghghghhhfhhghghguygtjuuujjjk}
%\left(-\nabla_{\vec x}S+\frac{c\vec {\tilde u}}{c^2_0}\right)\cdot\nabla_{\vec x}A+\frac{1}{2}\left(-\Delta_{\vec x}S\right)A=0.
%\end{equation}
%
%
%
%
\begin{comment}
\begin{multline}\label{MaxVacFullPPNmmmffffffiuiuhjuughbghhiuijghghghhjhjhhghyuyiyyujjkhgggghggiuiuiu}
\frac{c^2}{c^2_0}\left(1+\frac{1}{c^2}\left|\nabla_\vec x
S\right|^2\left|\vec {\tilde u}\right|^2-\frac{1}{c^2}\left|\vec
{\tilde u}\cdot\nabla_\vec x S\right|^2\right)=\left|\nabla_{\vec
x}S-\frac{c}{c_0}\left(1+\frac{1}{c}\left(\vec {\tilde
u}\cdot\nabla_{\vec x}S\right)\right)\frac{\vec {\tilde
u}}{c_0}\right|^2=\\ \left|\nabla_{\vec
x}S\right|^2+\frac{c^2}{c^4_0}\left(1+\frac{1}{c}\left(\vec {\tilde
u}\cdot\nabla_{\vec x}S\right)\right)^2\left|\vec {\tilde
u}\right|^2-\frac{2c}{c^2_0}\left(1+\frac{1}{c}\left(\vec {\tilde
u}\cdot\nabla_{\vec x}S\right)\right)\left(\nabla_{\vec x}S\cdot\vec
{\tilde u}\right)=\\ \left|\nabla_{\vec
x}S\right|^2+\frac{c^2}{c^4_0}\left(1+\frac{2}{c}\left(\vec {\tilde
u}\cdot\nabla_{\vec x}S\right)+\frac{1}{c^2}\left(\vec {\tilde
u}\cdot\nabla_{\vec x}S\right)^2\right)\left|\vec {\tilde
u}\right|^2-\frac{2c}{c^2_0}\left(1+\frac{1}{c}\left(\vec {\tilde
u}\cdot\nabla_{\vec x}S\right)\right)\left(\nabla_{\vec x}S\cdot\vec
{\tilde u}\right).
\end{multline}
\begin{multline}\label{MaxVacFullPPNmmmffffffiuiuhjuughbghhiuijghghghhjhjhhghyuyiyyujjkhgggghggiuiuiupkkkll}
0=\left(\left|\nabla_{\vec
x}S\right|^2-\frac{c^2}{c^2_0}\right)\left(1-\frac{\left|\vec
{\tilde
u}\right|^2}{c^2_0}\right)+\frac{c^2}{c^4_0}\left(\frac{2}{c}\left(\vec
{\tilde u}\cdot\nabla_{\vec x}S\right)+\frac{1}{c^2}\left(\vec
{\tilde u}\cdot\nabla_{\vec x}S\right)^2\right)\left|\vec {\tilde
u}\right|^2-\frac{c}{c^2_0}\left(2+\frac{1}{c}\left(\vec {\tilde
u}\cdot\nabla_{\vec x}S\right)\right)\left(\nabla_{\vec x}S\cdot\vec
{\tilde u}\right).
\end{multline}
\begin{multline}\label{MaxVacFullPPNmmmffffffiuiuhjuughbghhiuijghghghhjhjhhghyuyiyyujjkhgggghggiuiuiupkkkllokljk;ljk}
0=\left(\left|\nabla_{\vec
x}S\right|^2-\frac{c^2}{c^2_0}\right)\left(1-\frac{\left|\vec
{\tilde
u}\right|^2}{c^2_0}\right)-\frac{c^2}{c^2_0}\left(\frac{2}{c}\left(\nabla_{\vec
x}S\cdot\vec {\tilde u}\right)+\frac{1}{c^2}\left|\vec {\tilde
u}\cdot\nabla_{\vec x}S\right|^2\right)\left(1-\frac{\left|\vec
{\tilde u}\right|^2}{c^2_0}\right).
\end{multline}
\end{comment}
%
%
%
%











\subsubsection{Sagnac effect}\label{seGO}
Assume again the monochromatic electromagnetic wave of the frequency
$\nu=\omega/(2\pi)$ characterized by:
\begin{equation}\label{MaxVacFullPPNmmmffffffiuiuhjuughbghhuiiujjhhjjhjhhjhjjhhjhjjh}
U(\vec x,t)=A(\vec x,t)e^{iT(\vec x,t)}=A(\vec x,t)e^{ik_0S(\vec
x,t)},\quad\text{where}\quad
k_0=\frac{\omega}{c}\quad\text{and}\quad\frac{\partial S}{\partial
t}=c\,.
\end{equation}
Then by \er{gughhghfbvnbvyyuuyr} we have
\begin{equation}\label{gughhghfbvnbvyyuuyrhjhjhj}
c_0=c\sqrt{\kappa_0\gamma_0}\quad\text{and}\quad\vec {\tilde
u}=\left(\gamma_0\vec v+(1-\gamma_0)\vec u\right),
\end{equation}
where, $\vec u$ is the medium velocity and $\vec v$ is the local
vectorial gravitational potential. Moreover, assume again that we
consider light traveling in some region either filled with the
resting medium of constant dielectric permeability $\gamma_0$ and
magnetic permeability $\kappa_0$ or in the vacuum. Then by
\er{gughhghfbvnbvyyuuyrhjhjhj} and
\er{MaxMedFullGGffgggyyojjhhjkhjyyiuhggjhhjhuyytytyuuytrrtghjtyuggyuighjuyioyyfgffhyuhhghzzrrkkhhkkkhhhjhkjhhghhgghiuiu1}
we have
\begin{equation}\label{gughhghfbvnbvyyuuyrhiyyuiuuhjh}
n=\frac{1}{\sqrt{\kappa_0\gamma_0}}\;\;\;\text{is a
constant,}\quad\text{and}\quad\vec {\tilde u}=\gamma_0\vec v.
\end{equation}
Next, assume that the light travels from point $N$ to point $M$
across the curve $\vec r(s):[a,b]\to\mathbb{R}^3$ with endpoints
$\vec r(a)=N$ and $\vec r(b)=M$ undergoing possibly certain number
of reflections from mirrors during its travel. Then by
\er{MaxMedFullGGffgggyyojjhhjkhjyyiuhggjhhjhuyytytyuuytrrtghjtyuggyuighjuyioyyfgffhyuhhghzzrrkkhhkkk},
\er{gughhghfbvnbvyyuuyrhiyyuiuuhjh} and
\er{MaxMedFullGGffgggyyojjyugggjhhjiiuuiyuyuyyuyuzzhhkkk} we have:
\begin{equation}\label{MaxMedFullGGffgggyyojjhhjkhjyyiuhggjhhjhuyytytyuuytrrtghjtyuggyuighjuyioyyfgffhyuhhghzzrrkkhhkkkhjjh}
\delta(-S):=\left(-S(M^-)\right)-
\left(-S(N^+)\right)=\frac{1}{\sqrt{\kappa_0\gamma_0}}\int_a^b
\left|\vec r'(s)\right|ds-\frac{1}{\kappa_0 c}\int_a^b\vec
v\left(\vec r(s)\right)\cdot\vec r'(s)ds.
\end{equation}
In particular, if we assume that $M=N$ i.e. our curve is closed and
moreover, our curve is the boundary of some surface $\mathcal{S}_0$,
then by Stokes Theorem we have:
\begin{multline}\label{MaxMedFullGGffgggyyojjhhjkhjyyiuhggjhhjhuyytytyuuytrrtghjtyuggyuighjuyioyyfgffhyuhhghzzrrkkhhkkkhjjhjj}
\delta(-S)=\left(-S(M^-)\right)-
\left(-S(M^+)\right)=\frac{1}{\sqrt{\kappa_0\gamma_0}}\int_a^b
\left|\vec r'(s)\right|ds-\frac{1}{\kappa_0 c}\iint \left(curl_{\vec
x}\vec v\right)\cdot\vec n
\,d\mathcal{S}_0\\=\frac{1}{\sqrt{\kappa_0\gamma_0}}\left|\partial
\mathcal{S}_0\right|-\frac{1}{\kappa_0 c}\iint \left(curl_{\vec
x}\vec v\right)\cdot\vec n \,d\mathcal{S}_0,
\end{multline}
where $\vec n$ is the unit normal to the surface.
%Note that although we assume $M=N$, $\delta(-S)$ can differ from zero since the light can undergo reflections during its travel.
In particular, if our
coordinate system is inertial, or more generally non-rotating, then
$curl_{\vec x}\vec v=0$ and by
\er{MaxMedFullGGffgggyyojjhhjkhjyyiuhggjhhjhuyytytyuuytrrtghjtyuggyuighjuyioyyfgffhyuhhghzzrrkkhhkkkhjjhjj}
we deduce
\begin{equation}\label{MaxMedFullGGffgggyyojjhhjkhjyyiuhggjhhjhuyytytyuuytrrtghjtyuggyuighjuyioyyfgffhyuhhghzzrrkkhhkkkhjjhjjhjhhjjhjjh}
\delta(-S)=\frac{1}{\sqrt{\kappa_0\gamma_0}}\left|\partial
\mathcal{S}_0\right|.
\end{equation}
On the other hand, if our system is rotating, then as in
\er{MaxMedFullGGffgggyyojjhhjkhjyyiuhggjhhjhuyytytyuuytrrtghjtyuggyuighjuyioyyfgffhyuhhghzzrrkkhhkkkhhhjhkjhhghhgghiuiuhhjhjiuiuyujjk}
we clearly deduce:
\begin{equation}\label{MaxMedFullGGffgggyyojjhhjkhjyyiuhggjhhjhuyytytyuuytrrtghjtyuggyuighjuyioyyfgffhyuhhghzzrrkkhhkkkhhhjhkjhhghhgghiuiuhhjhjiuiuyujjkjjk}
curl_{\vec x}\vec v=-2\vec w,
\end{equation}
where $\vec w$ is the vector of the angular speed of rotation of our
coordinate system. Then by
\er{MaxMedFullGGffgggyyojjhhjkhjyyiuhggjhhjhuyytytyuuytrrtghjtyuggyuighjuyioyyfgffhyuhhghzzrrkkhhkkkhhhjhkjhhghhgghiuiuhhjhjiuiuyujjkjjk}
and
\er{MaxMedFullGGffgggyyojjhhjkhjyyiuhggjhhjhuyytytyuuytrrtghjtyuggyuighjuyioyyfgffhyuhhghzzrrkkhhkkkhjjhjj}
we deduce
\begin{equation}\label{MaxMedFullGGffgggyyojjhhjkhjyyiuhggjhhjhuyytytyuuytrrtghjtyuggyuighjuyioyyfgffhyuhhghzzrrkkhhkkkhjjhjjjjk}
\delta(-S)=\frac{1}{\sqrt{\kappa_0\gamma_0}}\left|\partial
\mathcal{S}_0\right|+\frac{2}{\kappa_0 c}\iint \vec w\cdot\vec n
\,d\mathcal{S}_0.
\end{equation}
In particular, if the surface $\mathcal{S}_0$ is a part of some
plain then we rewrite
\er{MaxMedFullGGffgggyyojjhhjkhjyyiuhggjhhjhuyytytyuuytrrtghjtyuggyuighjuyioyyfgffhyuhhghzzrrkkhhkkkhjjhjjjjk}
as
\begin{equation}\label{MaxMedFullGGffgggyyojjhhjkhjyyiuhggjhhjhuyytytyuuytrrtghjtyuggyuighjuyioyyfgffhyuhhghzzrrkkhhkkkhjjhjjjjkjjkjljkl}
\delta(-S)=\frac{1}{\sqrt{\kappa_0\gamma_0}}\left|\partial
\mathcal{S}_0\right|+\frac{2}{\kappa_0 c} \left(\vec w\cdot\vec
n\right)\left|\mathcal{S}_0\right|.
\end{equation}
On the other hand, if the light travels across the same curve in the
opposite direction, then we must have:
\begin{equation}\label{MaxMedFullGGffgggyyojjhhjkhjyyiuhggjhhjhuyytytyuuytrrtghjtyuggyuighjuyioyyfgffhyuhhghzzrrkkhhkkkhjjhjjjjkjjkjljklpoo}
\delta(-S^-)=\frac{1}{\sqrt{\kappa_0\gamma_0}}\left|\partial
\mathcal{S}_0\right|-\frac{2}{\kappa_0 c} \left(\vec w\cdot\vec
n\right)\left|\mathcal{S}_0\right|.
\end{equation}
Thus, by taking the difference in two cases and using
\er{MaxVacFullPPNmmmffffffiuiuhjuughbghhuiiujjhhjjhjhhjhjjhhjhjjh},
we deduce:
\begin{equation}\label{MaxMedFullGGffgggyyojjhhjkhjyyiuhggjhhjhuyytytyuuytrrtghjtyuggyuighjuyioyyfgffhyuhhghzzrrkkhhkkkhjjhjjjjkjjkjljklkkkhjhj}
\left(\delta(-T)-\delta(-T^-)\right)=k_0\left(\delta(-S)-\delta(-S^-)\right)=\frac{4\omega}{\kappa_0
c^2}\,. \left(\vec w\cdot\vec n\right)\left|\mathcal{S}_0\right|.
\end{equation}
Here, $\gamma_0$ and $\kappa_0$ are the dielectric and the magnetic
permeability of the medium, $T$ is given in
\er{MaxVacFullPPNmmmffffffiuiuhjuughbghhuiiujjhhjjhjhhjhjjhhjhjjh},
$\left|\mathcal{S}_0\right|$ is the area of the flat surface bounded
by the closed path of the light, $\vec n$ is the unit normal to the
surface, $\omega$ is the frequency of the light and $\vec w$ is the
angular speed vector of the rotation of our coordinate system.
%
%
%
\begin{comment}
Note that by the previous subsections the exact curve of traveling
of light depends on $\vec w$. In particular, the curves of traveling
in direct and in opposite directions are not exactly the same.
However, by
\er{MaxMedFullGGffgggyyojjhhjkhjyyiuhggjhhjhuyytytyuuytrrtghjtyuggyuighjuyioyyfgffhyuhhghzzrrkkhhkkkhhhjhkjhhghhgghiuiuhhjhjiuiuyuyyukklghhgjkjkiuuuioiojo}
the correction should be of the order
\begin{equation}\label{MaxMedFullGGffgggyyojjhhjkhjyyiuhggjhhjhuyytytyuuytrrtghjtyuggyuighjuyioyyfgffhyuhhghzzrrkkhhkkkhhhjhkjhhghhgghiuiuhhjhjiuiuyuyyukklghhgjkjkiuuuioiojouiiuiuiu}
\left|\partial \mathcal{S}_0\right|O\left(\frac{2|\vec
w|\left|\partial \mathcal{S}_0\right|}{c}\right).
\end{equation}
\end{comment}
%
%
%

























\subsubsection{Fizeau experiment}\label{seGOfz}
Assume again the monochromatic electromagnetic wave of the frequency
$\nu=\omega/(2\pi)$ characterized by:
\begin{equation}\label{MaxVacFullPPNmmmffffffiuiuhjuughbghhuiiujjhhjjhjhhjhjjhhjhjjhfz}
U(\vec x,t)=A(\vec x,t)e^{iT(\vec x,t)}=A(\vec x,t)e^{ik_0S(\vec
x,t)},\quad\text{where}\quad
k_0=\frac{\omega}{c}\quad\text{and}\quad\frac{\partial S}{\partial
t}=c\,.
\end{equation}
Then by \er{gughhghfbvnbvyyuuyr} we have
\begin{equation}\label{gughhghfbvnbvyyuuyrhjhjhjfz}
c_0=c\sqrt{\kappa_0\gamma_0}\quad\text{and}\quad\vec {\tilde
u}=\left(\gamma_0\vec v+(1-\gamma_0)\vec u\right),
\end{equation}
where, $\vec u$ is the medium velocity and $\vec v$ is the local
vectorial gravitational potential. Moreover, assume that we consider
light traveling in some region filled with the moving medium of
constant dielectric permeability $\gamma_0$ and magnetic
permeability $\kappa_0$. Then by \er{gughhghfbvnbvyyuuyrhjhjhjfz}
and
\er{MaxMedFullGGffgggyyojjhhjkhjyyiuhggjhhjhuyytytyuuytrrtghjtyuggyuighjuyioyyfgffhyuhhghzzrrkkhhkkkhhhjhkjhhghhgghiuiu1}
we have
\begin{equation}\label{gughhghfbvnbvyyuuyrhiyyuiuuhjhfz}
n=\frac{c}{c_0}=\frac{1}{\sqrt{\kappa_0\gamma_0}}\;\;\;\text{is a
constant,}\quad\text{and}\quad\vec {\tilde u}=\gamma_0\vec
v+\left(1-\frac{1}{\kappa_0n^2}\right)\vec u.
\end{equation}
Next, assume that the light travels from point $N$ to point $M$
across the curve $\vec r(s):[a,b]\to\mathbb{R}^3$ with endpoints
$\vec r(a)=N$ and $\vec r(b)=M$ undergoing possibly certain number
of reflections from mirrors during its travel. Then, as before, by
\er{MaxMedFullGGffgggyyojjhhjkhjyyiuhggjhhjhuyytytyuuytrrtghjtyuggyuighjuyioyyfgffhyuhhghzzrrkkhhkkk},
\er{gughhghfbvnbvyyuuyrhiyyuiuuhjhfz} and
\er{MaxMedFullGGffgggyyojjyugggjhhjiiuuiyuyuyyuyuzzhhkkk} we have:
\begin{multline}\label{MaxMedFullGGffgggyyojjhhjkhjyyiuhggjhhjhuyytytyuuytrrtghjtyuggyuighjuyioyyfgffhyuhhghzzrrkkhhkkkhjjhfz}
\delta(-S):=\left(-S(M^-)\right)- \left(-S(N^+)\right)=\\
n\int_a^b \left|\vec r'(s)\right|ds-\frac{1}{\kappa_0 c}\int_a^b\vec
v\left(\vec r(s)\right)\cdot\vec
r'(s)ds-\frac{n^2}{c}\left(1-\frac{1}{\kappa_0n^2}\right)\int_a^b\vec
u\left(\vec r(s)\right)\cdot\vec r'(s)ds.
\end{multline}
Next assume that, either our curve is perpendicular to the direction
of the vectorial gravitational potential $\vec v$, that happens, for
example, if our path of the light is tangent to the Earth surface,
or assume that our curve is closed, i.e. $M=N$ and moreover, our
coordinate system is inertial, or more generally non-rotating. In
particular, if we assume that $M=N$ i.e. our curve is closed and
moreover, our coordinate system is inertial, or more generally
non-rotating, then, as before, by Stokes Theorem we have:
\begin{equation}\label{MaxMedFullGGffgggyyojjhhjkhjyyiuhggjhhjhuyytytyuuytrrtghjtyuggyuighjuyioyyfgffhyuhhghzzrrkkhhkkkhjjhjjfz}
\int_a^b\vec v\left(\vec r(s)\right)\cdot\vec r'(s)ds=0.
\end{equation}
On the other hand in the case that our curve is perpendicular to the
direction of the vectorial gravitational potential $\vec v$,
\er{MaxMedFullGGffgggyyojjhhjkhjyyiuhggjhhjhuyytytyuuytrrtghjtyuggyuighjuyioyyfgffhyuhhghzzrrkkhhkkkhjjhjjfz}
also trivially follows. Therefore, by inserting
\er{MaxMedFullGGffgggyyojjhhjkhjyyiuhggjhhjhuyytytyuuytrrtghjtyuggyuighjuyioyyfgffhyuhhghzzrrkkhhkkkhjjhjjfz}
into
\er{MaxMedFullGGffgggyyojjhhjkhjyyiuhggjhhjhuyytytyuuytrrtghjtyuggyuighjuyioyyfgffhyuhhghzzrrkkhhkkkhjjhfz}
in both cases we obtain:
\begin{multline}\label{MaxMedFullGGffgggyyojjhhjkhjyyiuhggjhhjhuyytytyuuytrrtghjtyuggyuighjuyioyyfgffhyuhhghzzrrkkhhkkkhjjhfzfzfz}
\delta(-S)=\left(-S(M^-)\right)- \left(-S(N^+)\right)= n\int_a^b
\left|\vec
r'(s)\right|ds-\frac{n^2}{c}\left(1-\frac{1}{\kappa_0n^2}\right)\int_a^b\vec
u\left(\vec r(s)\right)\cdot\vec r'(s)ds.
\end{multline}
Then by
\er{MaxMedFullGGffgggyyojjhhjkhjyyiuhggjhhjhuyytytyuuytrrtghjtyuggyuighjuyioyyfgffhyuhhghzzrrkkhhkkkhjjhfzfzfz}
and
\er{MaxVacFullPPNmmmffffffiuiuhjuughbghhuiiujjhhjjhjhhjhjjhhjhjjhfz}
we deduce
\begin{multline}\label{MaxMedFullGGffgggyyojjhhjkhjyyiuhggjhhjhuyytytyuuytrrtghjtyuggyuighjuyioyyfgffhyuhhghzzrrkkhhkkkhjjhfzfzfzfzffz}
\delta(-T):=\left(-T(M^-)\right)-
\left(-T(N^+)\right)=k_0\delta(-S)\\= \frac{n\omega}{c}\int_a^b
\left|\vec
r'(s)\right|ds-\frac{n^2\omega}{c^2}\left(1-\frac{1}{\kappa_0n^2}\right)\int_a^b\vec
u\left(\vec r(s)\right)\cdot\vec r'(s)ds\\
\frac{n^2\omega}{c^2}\left(c_0\int_a^b \left|\vec
r'(s)\right|ds-\left(1-\frac{1}{\kappa_0n^2}\right)\int_a^b\vec
u\left(\vec r(s)\right)\cdot\vec r'(s)ds\right).
\end{multline}
In particular, if the absolute value $\left|\vec u\left(\vec
r(s)\right)\right|$ is a constant across the curve and if the angle
between $r'(s)$ and $\vec u\left(\vec r(s)\right)$ is a constant
across the curve and equals to the value $\theta$ then denoting the
length of the path by $L$:
\begin{equation}\label{MaxMedFullGGffgggyyojjhhjkhjyyiuhggjhhjhuyytytyuuytrrtghjtyuggyuighjuyioyyfgffhyuhhghzzrrkkhhkkkhjjhjjfzuyy}
L:=\int_a^b \left|\vec r'(s)\right|ds,
\end{equation}
by
\er{MaxMedFullGGffgggyyojjhhjkhjyyiuhggjhhjhuyytytyuuytrrtghjtyuggyuighjuyioyyfgffhyuhhghzzrrkkhhkkkhjjhfzfzfzfzffz}
we deduce:
\begin{equation}\label{MaxMedFullGGffgggyyojjhhjkhjyyiuhggjhhjhuyytytyuuytrrtghjtyuggyuighjuyioyyfgffhyuhhghzzrrkkhhkkkhjjhfzfzfzfzffzghghgfz}
\delta(-T)=k_0\delta(-S)=\frac{\omega L
n^2}{c^2}\left(c_0-\left(1-\frac{1}{\kappa_0n^2}\right)|\vec
u|\cos{(\theta)}\right).
\end{equation}
Thus, if the direction of $\vec u$ coincides with the direction of
the light i.e. $\theta=0$ then
\begin{equation}\label{MaxMedFullGGffgggyyojjhhjkhjyyiuhggjhhjhuyytytyuuytrrtghjtyuggyuighjuyioyyfgffhyuhhghzzrrkkhhkkkhjjhfzfzfzfzffzghghgfzytyfz}
\delta(-T)=k_0\delta(-S)=\frac{\omega L
n^2}{c^2}\left(c_0-\left(1-\frac{1}{\kappa_0n^2}\right)|\vec
u|\right)\approx\frac{\omega
L}{\left(c_0+\left(1-\frac{1}{\kappa_0n^2}\right)|\vec u|\right)}.
\end{equation}
On the other hand, if the direction of $\vec u$ is opposite to the
direction of the light i.e. $\theta=\pi$ then
\begin{equation}\label{MaxMedFullGGffgggyyojjhhjkhjyyiuhggjhhjhuyytytyuuytrrtghjtyuggyuighjuyioyyfgffhyuhhghzzrrkkhhkkkhjjhfzfzfzfzffzghghgfzuyuyhffz}
\delta(-T)=k_0\delta(-S)=\frac{\omega L
n^2}{c^2}\left(c_0+\left(1-\frac{1}{\kappa_0n^2}\right)|\vec
u|\right)\approx\frac{\omega
L}{\left(c_0-\left(1-\frac{1}{\kappa_0n^2}\right)|\vec u|\right)}.
\end{equation}
So, in the case where the magnetic permeability is close to one,
i.e. $\kappa_0=1$, in the frames of our model we explain the results
of the Fizeau experiment.





































\subsubsection{Fermat Principle of Geometric Optics in the case when we cannot neglect effects of order $O\left(\frac{|\vec {\tilde
u}|^2}{c^2_0}\right)$}
Consider again a monochromatic wave of the
frequency $\nu=\omega/(2\pi)$ characterized by:
\begin{equation}\label{MaxVacFullPPNmmmffffffiuiuhjuughbghhuiiujjhhjjhjhhjyuyhh}
U(\vec x,t)=A(\vec x,t)e^{ik_0S(\vec x,t)},\quad\text{where}\quad
k_0=\frac{\omega}{c}\quad\text{and}\quad\frac{\partial S}{\partial
t}=c\,.
\end{equation}
As before, assume the validity of Geometric Optics approximation.
However, assume that we cannot consider anymore the case of the
approximation, given by \er{ojhkk}, i.e. we assume that we cannot
neglect anymore effects of order $O\left(\frac{|\vec {\tilde
u}|^2}{c^2_0}\right)$. This happens, for example in the case of the
Michelson-Morley experiment. Thus instead of
\er{MaxVacFullPPNmmmffffffiuiuhjuughbghhiuijghghghhjhjhhghyuyiyyujjljk}
and
\er{MaxVacFullPPNmmmffffffiuiuhjuughbghhiuijghghghhhfhhghghguygtjuuujjjk}
we need to deal with
\er{MaxVacFullPPNmmmffffffiuiuhjuughbghhiuijghghghhjhjhhghyuyiyyujj}
 and
\er{MaxVacFullPPNmmmffffffiuiuhjuughbghhiuijghghghhhfhhghghguygtjuuujj}.
On the other hand, by
\er{MaxVacFullPPNmmmffffffiuiuhjuughbghhiuijghghghhjhjhhghyuyiyyujj}
we deduce:
%\er{MaxVacFullPPNmmmffffffiuiuhjuughbghhiuijghghghhjhjhhghyuyiyyujj}
\begin{multline}\label{MaxVacFullPPNmmmffffffiuiuhjuughbghhiuijghghghhjhjhhghyuyiyyujjkhgggghgg}
\left|\nabla_{\vec x}S-\frac{c}{c_0}\left(1+\frac{1}{c}\left(\vec
{\tilde u}\cdot\nabla_{\vec x}S\right)\right)\frac{\vec {\tilde
u}}{c_0}\right|^2=\frac{c^2}{c^2_0}\left(1+\frac{1}{c^2}\left|\nabla_\vec
x S\right|^2\left|\vec {\tilde u}\right|^2-\frac{1}{c^2}\left|\vec
{\tilde u}\cdot\nabla_\vec x S\right|^2\right)=\\
\frac{c^2}{c^2_0}\left(1+\frac{1}{c^2}\left|\nabla_{\vec
x}S-\frac{c}{c_0}\left(1+\frac{1}{c}\left(\vec {\tilde
u}\cdot\nabla_{\vec x}S\right)\right)\frac{\vec {\tilde
u}}{c_0}\right|^2\left|\vec {\tilde
u}\right|^2-\frac{1}{c^2}\left|\vec {\tilde
u}\cdot\left(\nabla_{\vec
x}S-\frac{c}{c_0}\left(1+\frac{1}{c}\left(\vec {\tilde
u}\cdot\nabla_{\vec x}S\right)\right)\frac{\vec {\tilde
u}}{c_0}\right)\right|^2\right).
\end{multline}
Then we rewrite
\er{MaxVacFullPPNmmmffffffiuiuhjuughbghhiuijghghghhhfhhghghguygtjuuujj}
and
\er{MaxVacFullPPNmmmffffffiuiuhjuughbghhiuijghghghhjhjhhghyuyiyyujjkhgggghgg}
as:
\begin{equation}\label{MaxVacFullPPNmmmffffffiuiuhjuughbghhiuijghghghhhfhhghghguygtjuuujjjhy}
%\frac{1}{c^2_0}\frac{\partial^2}{\partial t^2}\left(Ae^{ik_0S}\right)
\vec h\cdot\nabla_{\vec x}A=\frac{1}{2}\left(\left(\Delta_{\vec
x}S\right)-\frac{1}{c^2_0}\left(\left(\nabla^2_{\vec x}S\cdot\vec
{\tilde u}\right)\cdot\vec {\tilde u}\right)\right)A.
\end{equation}
and
\begin{equation}\label{MaxVacFullPPNmmmffffffiuiuhjuughbghhiuijghghghhjhjhhghyuyiyyujjkhgggghggutgg}
\left(1-\frac{\left|\vec {\tilde
u}\right|^2}{c^2_0}\right)\left|\vec h-\frac{1}{\left|\vec {\tilde
u}\right|^2}\left(\vec {\tilde u}\cdot\vec h\right)\vec {\tilde
u}\right|^2+\left|\frac{1}{\left|\vec {\tilde u}\right|}\vec {\tilde
u}\cdot\vec h\right|^2=\left|\vec h\right|^2\left(1-\frac{\left|\vec
{\tilde u}\right|^2}{c^2_0}\right)+\frac{1}{c^2_0}\left|\vec {\tilde
u}\cdot\vec h\right|^2= \frac{c^2}{c^2_0},
\end{equation}
where the vector field $\vec h$ defined for every $\vec x$ by:
\begin{equation}\label{MaxVacFullPPNmmmffffffiuiuhjuughbghhiuijghghghhhfhhghghguygtjuuujjjkyuuyhh}
%\frac{1}{c^2_0}\frac{\partial^2}{\partial t^2}\left(Ae^{ik_0S}\right)
\vec h(\vec x):=\frac{c}{c^2_0(\vec x)}\left(1+\frac{1}{c}\left(\vec
{\tilde u(\vec x)}\cdot\nabla_{\vec x}S(\vec x)\right)\right)\vec
{\tilde u}(\vec x)-\nabla_{\vec x}S(\vec x),
\end{equation}
is called the direction of propagation of the beam that passes
through point $\vec x$. We clarify this name bellow.
%\begin{equation}\label{MaxVacFullPPNmmmffffffiuiuhjuughbghhiuijghghghhjhjhhghyuyiyyujjkhgggghggutggghghfgfgfg}
%\left|\vec h\right|^2=\frac{c^2}{c^2_0}\left(1+\frac{1}{c^2}\left|\vec h\right|^2\left|\vec {\tilde u}\right|^2-\frac{1}{c^2}\left|\vec{\tilde u}\cdot\vec h\right|^2\right).
%\end{equation}
The Eikonal equation
\er{MaxVacFullPPNmmmffffffiuiuhjuughbghhiuijghghghhjhjhhghyuyiyyujjkhgggghggutgg}
and equation of the beam propagation
\er{MaxVacFullPPNmmmffffffiuiuhjuughbghhiuijghghghhhfhhghghguygtjuuujjjhy}
are two basic equations of propagation of monochromatic light in the
Geometric Optics approximation inside a moving medium or/and in the
presence of non-trivial gravitational field.

Next if we consider an arbitrary characteristic curve $\vec
r(s):[a,b]\to\mathbb{R}^3$ of equation
\er{MaxVacFullPPNmmmffffffiuiuhjuughbghhiuijghghghhhfhhghghguygtjuuujjjhy}
defined as a solution of ordinary differential equation
\begin{equation}\label{MaxVacFullPPNmmmffffffiuiuhjuughbghhiuijghghghhhfhhghghguygtjuuujjjkkhh}
\begin{cases}
\frac{d\vec r}{ds}(s)=\vec h\left(\vec
r(s)\right)\\
\vec r(a)=\vec x_0,
\end{cases}
\end{equation}
then by
\er{MaxVacFullPPNmmmffffffiuiuhjuughbghhiuijghghghhhfhhghghguygtjuuujjjhy}
and
\er{MaxVacFullPPNmmmffffffiuiuhjuughbghhiuijghghghhhfhhghghguygtjuuujjjkkhh}
we have
\begin{equation}\label{MaxVacFullPPNmmmffffffiuiuhjuughbghhiuijghghghhhfhhghghguygtjuuujjjkhjhjhhh}
%\frac{1}{c^2_0}\frac{\partial^2}{\partial t^2}\left(Ae^{ik_0S}\right)
\frac{d}{ds}\left(A\left(\vec r(s)\right)\right)=\nabla_{\vec
x}A\left(\vec r(s)\right)\cdot\frac{d\vec r}{ds}(s)= \frac{1}{2}
%\left(\Delta_{\vec x}S\left(\vec r(s)\right)-\frac{1}{c^2_0\left(\vec r(s)\right)}\left(\left(\nabla^2_{\vec x}S\left(\vec r(s)\right)\cdot\vec {\tilde u}\left(\vec
%r(s)\right)\right)\cdot\vec {\tilde u}\left(\vecr(s)\right)\right)\right)
g\left(\vec r(s)\right)A\left(\vec r(s)\right),
\end{equation}
where we denote
\begin{equation}\label{MaxVacFullPPNmmmffffffiuiuhjuughbghhiuijghghghhhfhhghghguygtjuuujjjkhjhjhffgghhpiuuu}
g(\vec x):=\Delta_{\vec x}S(\vec x)-\frac{1}{c^2_0(\vec
x)}\left(\left(\nabla^2_{\vec x}S(\vec x)\cdot\vec {\tilde u}(\vec
x)\right)\cdot\vec {\tilde u}(\vec x)\right).
\end{equation}
Then
\er{MaxVacFullPPNmmmffffffiuiuhjuughbghhiuijghghghhhfhhghghguygtjuuujjjkhjhjhhh}
implies
\begin{equation}\label{MaxVacFullPPNmmmffffffiuiuhjuughbghhiuijghghghhhfhhghghguygtjuuujjjkhjhjhffgghh}
A\left(\vec r(s)\right)=A\left(\vec
x_0\right)e^{\frac{1}{2}\int_a^{s}g\left(\vec
r(\tau)\right)d\tau}\quad\quad\forall s\in[a,b].
\end{equation}
In particular,
\begin{equation}\label{MaxVacFullPPNmmmffffffiuiuhjuughbghhiuijghghghhhfhhghghguygtjuuujjjkhjhjhffggiouuiiuhh}
A\left(\vec x_0\right)=0\;\;\text{implies}\;\; A\left(\vec
r(s)\right)=0\quad\forall s\in[a,b],\quad\text{and}\quad A\left(\vec
x_0\right)\neq 0\;\;\text{implies}\;\; A\left(\vec r(s)\right)\neq
0\quad\forall s\in[a,b].
\end{equation}
Therefore, by
\er{MaxVacFullPPNmmmffffffiuiuhjuughbghhiuijghghghhhfhhghghguygtjuuujjjkhjhjhffggiouuiiuhh}
we deduce that the curve that satisfies
\er{MaxVacFullPPNmmmffffffiuiuhjuughbghhiuijghghghhhfhhghghguygtjuuujjjkkhh}
coincides with the beam of light that passes through the point $\vec
x_0$. So
\er{MaxVacFullPPNmmmffffffiuiuhjuughbghhiuijghghghhhfhhghghguygtjuuujjjkkhh}
is the equation of a beam and the vector field $\vec h$ defined for
every $\vec x$ by
\er{MaxVacFullPPNmmmffffffiuiuhjuughbghhiuijghghghhhfhhghghguygtjuuujjjkyuuyhh}
%\begin{equation}\label{MaxVacFullPPNmmmffffffiuiuhjuughbghhiuijghghghhhfhhghghguygtjuuujjjkyuuyhh}
%%\frac{1}{c^2_0}\frac{\partial^2}{\partial t^2}\left(Ae^{ik_0S}\right)
%\vec h(\vec x):=\frac{c}{c^2_0(\vec x)}\vec {\tilde u}(\vec x)-\nabla_{\vec x}S(\vec x),
%\end{equation}
is indeed the direction of propagation of the beam that passes
through point $\vec x$.
%Moreover, by \er{MaxVacFullPPNmmmffffffiuiuhjuughbghhiuijghghghhjhjhhghyuyiyyujjkhgggghggutgg} $\vec h$ satisfies
%\begin{equation}\label{MaxVacFullPPNmmmffffffiuiuhjuughbghhiuijghghghhjhjhhghyuyiyyujjljkgghhghh}
%|\vec h|^2=\frac{c^2}{c^2_0}.
%\end{equation}
%
%
%
%\begin{comment*}

Next, by
\er{MaxVacFullPPNmmmffffffiuiuhjuughbghhiuijghghghhhfhhghghguygtjuuujjjkyuuyhh}
we have:
\begin{equation}\label{MaxVacFullPPNmmmffffffiuiuhjuughbghhiuijghghghhhfhhghghguygtjuuujjjkyuuyhhijhhhhjihhyjh}
%\frac{1}{c^2_0}\frac{\partial^2}{\partial t^2}\left(Ae^{ik_0S}\right)
\left(1-\frac{|\vec {\tilde
u}|^2}{c^2_0}\right)^{-1}\left(\frac{1}{|\vec {\tilde u}|}\vec
{\tilde u}\cdot\vec h\right)=\frac{c}{c^2_0}\left(1-\frac{|\vec
{\tilde u}|^2}{c^2_0}\right)^{-1}|\vec {\tilde
u}|-\left(\frac{1}{|\vec {\tilde u}|}\vec {\tilde
u}\cdot\nabla_{\vec x}S\right),
\end{equation}
and
\begin{equation}\label{MaxVacFullPPNmmmffffffiuiuhjuughbghhiuijghghghhhfhhghghguygtjuuujjjkyuuyhhijhhhhjjhjhj}
%\frac{1}{c^2_0}\frac{\partial^2}{\partial t^2}\left(Ae^{ik_0S}\right)
\vec h-\frac{1}{\left|\vec {\tilde u}\right|^2}\left(\vec {\tilde
u}\cdot\vec h\right)\vec {\tilde u}=-\left(\nabla_{\vec
x}S-\frac{1}{\left|\vec {\tilde u}\right|^2}\left(\vec {\tilde
u}\cdot\nabla_{\vec x}S\right)\vec {\tilde u}\right),
\end{equation}
On the other hand by
\er{MaxVacFullPPNmmmffffffiuiuhjuughbghhiuijghghghhjhjhhghyuyiyyujjkhgggghggutgg}
we have
\begin{equation}\label{MaxVacFullPPNmmmffffffiuiuhjuughbghhiuijghghghhjhjhhghyuyiyyujjkhgggghggutggjkkjkll}
\left|\vec h-\frac{1}{\left|\vec {\tilde u}\right|^2}\left(\vec
{\tilde u}\cdot\vec h\right)\vec {\tilde
u}\right|^2+\left(1-\frac{\left|\vec {\tilde
u}\right|^2}{c^2_0}\right)^{-1}\left|\frac{1}{\left|\vec {\tilde
u}\right|}\vec {\tilde u}\cdot\vec h\right|^2=
\frac{c^2}{c^2_0}\left(1-\frac{\left|\vec {\tilde
u}\right|^2}{c^2_0}\right)^{-1},
\end{equation}
%again assume that the approximate equations in \er{MaxVacFullPPNmmmffffffiuiuhjuughbghhiuijghghghhjhjhhghyuyiyyujjljk}
%and \er{MaxVacFullPPNmmmffffffiuiuhjuughbghhiuijghghghhhfhhghghguygtjuuujjjk} are valid in the given region and
Therefore if we consider a curve $\vec r(s):[a,b]\to\mathbb{R}^3$
with endpoints $\vec r(a)=N$ and $\vec r(b)=M$, then integrating the
square root of both sides of
\er{MaxVacFullPPNmmmffffffiuiuhjuughbghhiuijghghghhjhjhhghyuyiyyujjkhgggghggutggjkkjkll}
over the curve $\vec r(s)$ we deduce
\begin{multline}\label{MaxMedFullGGffgggyyojjhhjkhjyyiuhggjhhjhuyytytyuuytrrtghjtyuggyuighjuyioyyfgffhyuhhghzzrrhhkkkhh}
\int\limits_a^b\sqrt{\left|\vec h\left(\vec r(s)\right)-\left(\vec
{\tilde u}\left(\vec r(s)\right)\cdot\vec h\left(\vec
r(s)\right)\right)\frac{\vec {\tilde u}\left(\vec
r(s)\right)}{\left|\vec {\tilde u}\left(\vec
r(s)\right)\right|^2}\right|^2+\left(1-\frac{\left|\vec {\tilde
u}\left(\vec r(s)\right)\right|^2}{c^2_0\left(\vec
r(s)\right)}\right)^{-1}\left|\frac{\vec {\tilde u}\left(\vec
r(s)\right)}{\left|\vec {\tilde u}\left(\vec
r(s)\right)\right|}\cdot\vec h\left(\vec r(s)\right)\right|^2}
\cdot\\
\cdot\sqrt{\left|\vec r'(s)-\left(\vec {\tilde u}\left(\vec
r(s)\right)\cdot\vec r'(s)\right)\frac{\vec {\tilde u}\left(\vec
r(s)\right)}{\left|\vec {\tilde u}\left(\vec
r(s)\right)\right|^2}\right|^2+\left(1-\frac{\left|\vec {\tilde
u}\left(\vec r(s)\right)\right|^2}{c^2_0\left(\vec
r(s)\right)}\right)^{-1}\left|\frac{\vec {\tilde u}\left(\vec
r(s)\right)}{\left|\vec {\tilde u}\left(\vec
r(s)\right)\right|}\cdot\vec r'(s)\right|^2}ds
%\left|\vec r'(s)\right|ds
\\=\int\limits_a^b\frac{c}{c_0\left(\vec
r(s)\right)}\left(1-\frac{\left|\vec {\tilde u}\left(\vec
r(s)\right)\right|^2}{c^2_0\left(\vec
r(s)\right)}\right)^{-\frac{1}{2}}
\cdot\\
\cdot\sqrt{\left|\vec r'(s)-\left(\vec {\tilde u}\left(\vec
r(s)\right)\cdot\vec r'(s)\right)\frac{\vec {\tilde u}\left(\vec
r(s)\right)}{\left|\vec {\tilde u}\left(\vec
r(s)\right)\right|^2}\right|^2+\left(1-\frac{\left|\vec {\tilde
u}\left(\vec r(s)\right)\right|^2}{c^2_0\left(\vec
r(s)\right)}\right)^{-1}\left|\frac{\vec {\tilde u}\left(\vec
r(s)\right)}{\left|\vec {\tilde u}\left(\vec
r(s)\right)\right|}\cdot\vec r'(s)\right|^2}ds.
%\left|\vec r'(s)\right|ds.
\end{multline}
Thus in particular, by inserting
\er{MaxVacFullPPNmmmffffffiuiuhjuughbghhiuijghghghhhfhhghghguygtjuuujjjkyuuyhhijhhhhjihhyjh}
and
\er{MaxVacFullPPNmmmffffffiuiuhjuughbghhiuijghghghhhfhhghghguygtjuuujjjkyuuyhhijhhhhjjhjhj}
into
\er{MaxMedFullGGffgggyyojjhhjkhjyyiuhggjhhjhuyytytyuuytrrtghjtyuggyuighjuyioyyfgffhyuhhghzzrrhhkkkhh}
and using inequality $\vec a\cdot\vec b\leq |\vec a||\vec b|$, we
deduce
\begin{multline}\label{MaxMedFullGGffgggyyojjhhjkhjyyiuhggjhhjhuyytytyuuytrrtghjtyuggyuighjuyioyyfgffhyuhhghzzrriuihhkkkhh}
\int_a^b\frac{c}{c^2_0\left(\vec
r(s)\right)}\left(1-\frac{\left|\vec {\tilde u}\left(\vec
r(s)\right)\right|^2}{c^2_0}\right)^{-1}\vec {\tilde u}\left(\vec
r(s)\right)\cdot\vec r'(s)ds-\int_a^b\nabla_{\vec x}S\left(\vec
r(s)\right)\cdot\vec r'(s)ds\\
\leq\int\limits_a^b\frac{c}{c_0\left(\vec
r(s)\right)}\left(1-\frac{\left|\vec {\tilde u}\left(\vec
r(s)\right)\right|^2}{c^2_0\left(\vec
r(s)\right)}\right)^{-\frac{1}{2}}
\cdot\\
\cdot\sqrt{\left|\vec r'(s)-\left(\vec {\tilde u}\left(\vec
r(s)\right)\cdot\vec r'(s)\right)\frac{\vec {\tilde u}\left(\vec
r(s)\right)}{\left|\vec {\tilde u}\left(\vec
r(s)\right)\right|^2}\right|^2+\left(1-\frac{\left|\vec {\tilde
u}\left(\vec r(s)\right)\right|^2}{c^2_0\left(\vec
r(s)\right)}\right)^{-1}\left|\frac{\vec {\tilde u}\left(\vec
r(s)\right)}{\left|\vec {\tilde u}\left(\vec
r(s)\right)\right|}\cdot\vec r'(s)\right|^2}ds,
\end{multline}
i.e.
\begin{multline}\label{MaxMedFullGGffgggyyojjhhjkhjyyiuhggjhhjhuyytytyuuytrrtghjtyuggyuighjuyioyyfgffhyuhhghzzrrkkijjhjhhkkkhh}
\left(-S(M)\right)-
\left(-S(N)\right)\leq-\int_a^b\frac{c}{c^2_0\left(\vec
r(s)\right)}\left(1-\frac{\left|\vec {\tilde u}\left(\vec
r(s)\right)\right|^2}{c^2_0\left(\vec r(s)\right)}\right)^{-1}\vec
{\tilde u}\left(\vec r(s)\right)\cdot\vec
r'(s)ds\\+\int\limits_a^b\frac{c}{c_0\left(\vec
r(s)\right)}\left(1-\frac{\left|\vec {\tilde u}\left(\vec
r(s)\right)\right|^2}{c^2_0\left(\vec
r(s)\right)}\right)^{-\frac{1}{2}}
\cdot\\
\cdot\sqrt{\left|\vec r'(s)-\left(\vec {\tilde u}\left(\vec
r(s)\right)\cdot\vec r'(s)\right)\frac{\vec {\tilde u}\left(\vec
r(s)\right)}{\left|\vec {\tilde u}\left(\vec
r(s)\right)\right|^2}\right|^2+\left(1-\frac{\left|\vec {\tilde
u}\left(\vec r(s)\right)\right|^2}{c^2_0\left(\vec
r(s)\right)}\right)^{-1}\left|\frac{\vec {\tilde u}\left(\vec
r(s)\right)}{\left|\vec {\tilde u}\left(\vec
r(s)\right)\right|}\cdot\vec r'(s)\right|^2}ds.
\end{multline}
Moreover, if
\begin{equation}\label{MaxMedFullGGffgggyyojjyugggjhhjzzrrhhkkkhh}
\frac{d\vec r}{ds}(s)=\sigma(s)\vec h\left(\vec r(s)\right),
\end{equation}
for some nonnegative scalar factor $\sigma=\sigma(s)$ then by
\er{MaxMedFullGGffgggyyojjyugggjhhjzzrrhhkkkhh}, exactly in the same
way as we get
\er{MaxMedFullGGffgggyyojjhhjkhjyyiuhggjhhjhuyytytyuuytrrtghjtyuggyuighjuyioyyfgffhyuhhghzzrriuihhkkkhh},
we rewrite
\er{MaxMedFullGGffgggyyojjhhjkhjyyiuhggjhhjhuyytytyuuytrrtghjtyuggyuighjuyioyyfgffhyuhhghzzrrhhkkkhh}
as
\begin{multline}\label{MaxMedFullGGffgggyyojjhhjkhjyyiuhggjhhjhuyytytyuuytrrtghjtyuggyuighjuyioyyfgffhyuhhghzzrrkkhhkkkhh}
\left(-S(M)\right)-
\left(-S(N)\right)=-\int_a^b\frac{c}{c^2_0\left(\vec
r(s)\right)}\left(1-\frac{\left|\vec {\tilde u}\left(\vec
r(s)\right)\right|^2}{c^2_0\left(\vec r(s)\right)}\right)^{-1}\vec
{\tilde u}\left(\vec r(s)\right)\cdot\vec
r'(s)ds\\+\int\limits_a^b\frac{c}{c_0\left(\vec
r(s)\right)}\left(1-\frac{\left|\vec {\tilde u}\left(\vec
r(s)\right)\right|^2}{c^2_0\left(\vec
r(s)\right)}\right)^{-\frac{1}{2}}
\cdot\\
\cdot\sqrt{\left|\vec r'(s)-\left(\vec {\tilde u}\left(\vec
r(s)\right)\cdot\vec r'(s)\right)\frac{\vec {\tilde u}\left(\vec
r(s)\right)}{\left|\vec {\tilde u}\left(\vec
r(s)\right)\right|^2}\right|^2+\left(1-\frac{\left|\vec {\tilde
u}\left(\vec r(s)\right)\right|^2}{c^2_0\left(\vec
r(s)\right)}\right)^{-1}\left|\frac{\vec {\tilde u}\left(\vec
r(s)\right)}{\left|\vec {\tilde u}\left(\vec
r(s)\right)\right|}\cdot\vec r'(s)\right|^2}ds.
\end{multline}
Thus, by comparing
\er{MaxVacFullPPNmmmffffffiuiuhjuughbghhiuijghghghhhfhhghghguygtjuuujjjkkhh}
with \er{MaxMedFullGGffgggyyojjyugggjhhjzzrrhhkkkhh} and using
\er{MaxMedFullGGffgggyyojjhhjkhjyyiuhggjhhjhuyytytyuuytrrtghjtyuggyuighjuyioyyfgffhyuhhghzzrrkkijjhjhhkkkhh}
and
\er{MaxMedFullGGffgggyyojjhhjkhjyyiuhggjhhjhuyytytyuuytrrtghjtyuggyuighjuyioyyfgffhyuhhghzzrrkkhhkkkhh},
we deduce that if we assume that the light travel from the point $N$
to the point $M$ across the curve $\vec {\tilde
r}(s):[a,b]\to\mathbb{R}^3$ such that $\vec{\tilde r(a)}=N$ and
$\vec {\tilde r}(b)=M$, then
\begin{multline}\label{MaxMedFullGGffgggyyojjhhjkhjyyiuhggjhhjhuyytytyuuytrrtghjtyuggyuighjuyioyyfgffhyuhhghzzrrkkhhkkkhhhhh}
\left(-S(M)\right)-
\left(-S(N)\right)=-\int_a^b\frac{c}{c^2_0\left(\vec{\tilde
r}(s)\right)}\left(1-\frac{\left|\vec {\tilde u}\left(\vec{\tilde
r}(s)\right)\right|^2}{c^2_0\left(\vec{\tilde
r}(s)\right)}\right)^{-1}\vec {\tilde u}\left(\vec{\tilde
r}(s)\right)\cdot\vec
r'(s)ds\\+\int\limits_a^b\frac{c}{c_0\left(\vec{\tilde
r}(s)\right)}\left(1-\frac{\left|\vec {\tilde u}\left(\vec{\tilde
r}(s)\right)\right|^2}{c^2_0\left(\vec{\tilde
r}(s)\right)}\right)^{-\frac{1}{2}}
\cdot\\
\cdot\sqrt{\left|\vec{\tilde r}'(s)-\left(\vec {\tilde
u}\left(\vec{\tilde r}(s)\right)\cdot\vec{\tilde
r}'(s)\right)\frac{\vec {\tilde u}\left(\vec{\tilde
r}(s)\right)}{\left|\vec {\tilde u}\left(\vec{\tilde
r}(s)\right)\right|^2}\right|^2+\left(1-\frac{\left|\vec {\tilde
u}\left(\vec{\tilde r}(s)\right)\right|^2}{c^2_0\left(\vec{\tilde
r}(s)\right)}\right)^{-1}\left|\frac{\vec {\tilde
u}\left(\vec{\tilde r}(s)\right)}{\left|\vec {\tilde
u}\left(\vec{\tilde r}(s)\right)\right|}\cdot\vec{\tilde
r}'(s)\right|^2}ds,
\end{multline}
and for every other curve $\vec r(s):[a,b]\to\mathbb{R}^3$ with
endpoints $\vec r(a)=N$ and $\vec r(b)=M$ we have
%\er{MaxMedFullGGffgggyyojjhhjkhjyyiuhggjhhjhuyytytyuuytrrtghjtyuggyuighjuyioyyfgffhyuhhghzzrrkkhhkkk}.
\begin{multline}\label{MaxMedFullGGffgggyyojjhhjkhjyyiuhggjhhjhuyytytyuuytrrtghjtyuggyuighjuyioyyfgffhyuhhghzzrrkkhhkkkhhhjhkjhhhh}
-\int_a^b\frac{c}{c^2_0\left(\vec
r(s)\right)}\left(1-\frac{\left|\vec {\tilde u}\left(\vec
r(s)\right)\right|^2}{c^2_0\left(\vec r(s)\right)}\right)^{-1}\vec
{\tilde u}\left(\vec r(s)\right)\cdot\vec
r'(s)ds\\+\int\limits_a^b\frac{c}{c_0\left(\vec
r(s)\right)}\left(1-\frac{\left|\vec {\tilde u}\left(\vec
r(s)\right)\right|^2}{c^2_0\left(\vec
r(s)\right)}\right)^{-\frac{1}{2}}
\cdot\\
\cdot\sqrt{\left|\vec r'(s)-\left(\vec {\tilde u}\left(\vec
r(s)\right)\cdot\vec r'(s)\right)\frac{\vec {\tilde u}\left(\vec
r(s)\right)}{\left|\vec {\tilde u}\left(\vec
r(s)\right)\right|^2}\right|^2+\left(1-\frac{\left|\vec {\tilde
u}\left(\vec r(s)\right)\right|^2}{c^2_0\left(\vec
r(s)\right)}\right)^{-1}\left|\frac{\vec {\tilde u}\left(\vec
r(s)\right)}{\left|\vec {\tilde u}\left(\vec
r(s)\right)\right|}\cdot\vec r'(s)\right|^2}ds\\ \geq
-\int_a^b\frac{c}{c^2_0\left(\vec{\tilde
r}(s)\right)}\left(1-\frac{\left|\vec {\tilde u}\left(\vec{\tilde
r}(s)\right)\right|^2}{c^2_0\left(\vec{\tilde
r}(s)\right)}\right)^{-1}\vec {\tilde u}\left(\vec{\tilde
r}(s)\right)\cdot\vec
r'(s)ds\\+\int\limits_a^b\frac{c}{c_0\left(\vec{\tilde
r}(s)\right)}\left(1-\frac{\left|\vec {\tilde u}\left(\vec{\tilde
r}(s)\right)\right|^2}{c^2_0\left(\vec{\tilde
r}(s)\right)}\right)^{-\frac{1}{2}}
\cdot\\
\cdot\sqrt{\left|\vec{\tilde r}'(s)-\left(\vec {\tilde
u}\left(\vec{\tilde r}(s)\right)\cdot\vec{\tilde
r}'(s)\right)\frac{\vec {\tilde u}\left(\vec{\tilde
r}(s)\right)}{\left|\vec {\tilde u}\left(\vec{\tilde
r}(s)\right)\right|^2}\right|^2+\left(1-\frac{\left|\vec {\tilde
u}\left(\vec{\tilde r}(s)\right)\right|^2}{c^2_0\left(\vec{\tilde
r}(s)\right)}\right)^{-1}\left|\frac{\vec {\tilde
u}\left(\vec{\tilde r}(s)\right)}{\left|\vec {\tilde
u}\left(\vec{\tilde r}(s)\right)\right|}\cdot\vec{\tilde
r}'(s)\right|^2}ds.
\end{multline}
So we obtain the following Fermat Principle:
\begin{proposition}\label{gughghfhh}
Assume Geometric Optics approximation.
%together with \er{ojhkk}.
Then
the light that travels from point $N$ to point $M$ chooses the path
$\vec r(s):[a,b]\to\mathbb{R}^3$ with endpoints $\vec r(a)=N$ and
$\vec r(b)=M$ which minimizes the quantity:
\begin{multline}\label{MaxMedFullGGffgggyyojjhhjkhjyyiuhggjhhjhuyytytyuuytrrtghjtyuggyuighjuyioyyfgffhyuhhghzzrrkkhhkkkhhhjhkjhhghhgghhh}
J\left(\vec r(\cdot)\right):= -\int_a^b\frac{1}{c}n^2\left(\vec
r(s)\right)\left(1-\frac{\left|\vec {\tilde u}\left(\vec
r(s)\right)\right|^2}{c^2_0\left(\vec r(s)\right)}\right)^{-1}\vec
{\tilde u}\left(\vec r(s)\right)\cdot\vec
r'(s)ds\\+\int\limits_a^bn\left(\vec
r(s)\right)\left(1-\frac{\left|\vec {\tilde u}\left(\vec
r(s)\right)\right|^2}{c^2_0\left(\vec r(s)\right)}\right)^{-1}
\cdot\\
\cdot\sqrt{\left(1-\frac{\left|\vec {\tilde u}\left(\vec
r(s)\right)\right|^2}{c^2_0\left(\vec r(s)\right)}\right)\left|\vec
r'(s)-\left(\vec {\tilde u}\left(\vec r(s)\right)\cdot\vec
r'(s)\right)\frac{\vec {\tilde u}\left(\vec r(s)\right)}{\left|\vec
{\tilde u}\left(\vec r(s)\right)\right|^2}\right|^2+\left|\frac{\vec
{\tilde u}\left(\vec r(s)\right)}{\left|\vec {\tilde u}\left(\vec
r(s)\right)\right|}\cdot\vec r'(s)\right|^2}ds,
\end{multline}
where we set refraction index:
\begin{equation}\label{MaxMedFullGGffgggyyojjhhjkhjyyiuhggjhhjhuyytytyuuytrrtghjtyuggyuighjuyioyyfgffhyuhhghzzrrkkhhkkkhhhjhkjhhghhgghiuiuhh}
n\left(\vec x\right):=\frac{c}{c_0\left(\vec x\right)}.
\end{equation}
\end{proposition}


























\section{Appendix}
Consider the equations:
\begin{equation}\label{MaxVacFull1bjkgjhjhgjaaaK}
\begin{cases}
curl_{\vec x} \vec H\equiv \frac{4\pi}{c}\vec
j+\frac{1}{c}\frac{\partial
\vec D}{\partial t},\\
%\quad\text{for}\;\;(\vec x,t)\in\R^3\times[0,+\infty),\\
div_{\vec x} \vec D\equiv 4\pi\rho,\\
%\quad\quad\text{for}\;\;(\vec x,t)\in\R^3\times[0,+\infty),\\
curl_{\vec x} \vec E+\frac{1}{c}\frac{\partial \vec B}{\partial t}\equiv 0,\\
%\quad\quad\text{for}\;\;(\vec x,t)\in\R^3\times[0,+\infty),\\
div_{\vec x} \vec B\equiv 0,\\
%\quad\quad\text{for}\;\;(\vec x,t)\in\R^3\times[0,+\infty),\\
\vec E=\vec D-\frac{1}{c}\,\vec v\times \vec B,\\
%\quad\quad\text{for}\;\;(\vec x,t)\in\R^3\times[0,+\infty)\\
\vec H=\vec B+\frac{1}{c}\,\vec v\times \vec D.
%\quad\quad\text{for}\;\;(\vec x,t)\in\R^3\times[0,+\infty).
\end{cases}
\end{equation}
\begin{lemma}\label{fgbfghfh}
Let $\vec D,\vec B,\vec E,\vec H,\rho,\vec j,\vec v$ be solutions of
\er{MaxVacFull1bjkgjhjhgjaaaK}. Then
\begin{multline}\label{hvkgkjgkjbjbbklnknhihiokh}
\frac{\partial}{\partial t}\left(\frac{|\vec D|^2+|\vec
B|^2}{8\pi}\right)+div_\vec x\left\{\left(\frac{|\vec D|^2+|\vec
B|^2}{8\pi}\right)\vec v\right\}=\\ \frac{1}{4\pi}div_\vec
x\left\{(\vec D\otimes \vec D+ \vec B\otimes \vec B)\cdot \vec
v-\frac{1}{2}\left(|\vec D|^2+|\vec B|^2\right)\vec v-c \vec D\times
\vec B\right\}
\\-\left\{\frac{1}{4\pi}\left(div_\vec x\left\{\vec D\otimes \vec D+\vec B\otimes
\vec B-\frac{1}{2}\left(|\vec D|^2+|\vec
B|^2\right)I\right\}\right)-\left(\rho \vec E+\frac{1}{c}\,\vec
j\times \vec B\right)\right\}\cdot \vec v-\vec j\cdot \vec E,
\end{multline}
where $I$ is the identity matrix.
\end{lemma}
\begin{proof}
By \er{MaxVacFull1bjkgjhjhgjaaaK} and  \er{apfrm3} we infer:
\begin{multline}\label{hvkgkjgkjbjbb}
\frac{1}{2c}\frac{\partial}{\partial t}\left(|\vec D|^2+|\vec
B|^2\right)=\frac{1}{c}\frac{\partial \vec D}{\partial t}\cdot \vec
D+\frac{1}{c}\frac{\partial \vec B}{\partial t}\cdot \vec
B=\left(curl_\vec x
\vec H-\frac{4\pi}{c}\vec j\right)\cdot \vec D-(curl_\vec x \vec E)\cdot \vec B=\\
\left\{curl_\vec x\left(\vec B+\frac{1}{c}\,\vec v\times \vec
D\right)\right\}\cdot \vec D-\left\{curl_\vec x
\left(\vec D-\frac{1}{c}\,\vec v\times \vec B\right)\right\}\cdot \vec B-\frac{4\pi}{c}\vec j\cdot \vec D=\\
\frac{1}{c}\vec D\cdot curl_\vec x (\vec v\times \vec
D)+\frac{1}{c}\,\vec B\cdot curl_\vec x (\vec v\times \vec B)+\vec
D\cdot curl_\vec x \vec B-\vec B\cdot curl_\vec x \vec
D-\frac{4\pi}{c}\vec
j\cdot \vec D=\\
\frac{1}{c}\vec D\cdot curl_\vec x (\vec v\times \vec
D)+\frac{1}{c}\,\vec B\cdot curl_\vec x (\vec v\times \vec
B)-div_\vec x (\vec D\times \vec B)-\frac{4\pi}{c}\vec j\cdot \vec
D.
\end{multline}
On the other hand, by \er{apfrm6} and \er{MaxVacFull1bjkgjhjhgjaaaK}
we obtain
\begin{multline}\label{hvkgkjgkjbjbbjhiuh}
\frac{1}{c}\vec D\cdot curl_\vec x (\vec v\times \vec
D)+\frac{1}{c}\,\vec B\cdot curl_\vec x
(\vec v\times \vec B)=\\
\frac{1}{c}(div_\vec x \vec D)\,\vec v\cdot \vec D
-\frac{1}{c}(div_\vec x \vec v)|\vec D|^2 +\frac{1}{c}\vec
D\cdot\left\{(d_\vec x \vec v)\cdot \vec D\right\}-\frac{1}{2c}\vec
v\cdot\nabla_\vec x |\vec D|^2\\+(div_\vec x \vec
B)\frac{1}{c}\,\vec v\cdot \vec B -(div_\vec x \vec
v)\frac{1}{c}|\vec B|^2 +\frac{1}{c}\,\vec B\cdot\left\{(d_\vec x
\vec v)\cdot \vec B\right\}-\frac{1}{2c}\vec v\cdot\nabla_\vec x
|\vec B|^2=\\
\frac{4\pi\rho}{c}\vec v\cdot \vec D-(div_\vec x \vec
v)\frac{1}{c}\left(|\vec D|^2+|\vec B|^2\right) +\frac{1}{c}\,\vec
B\cdot\left\{(d_\vec x \vec v)\cdot \vec B\right\}\\+\frac{1}{c}\vec
D\cdot\left\{(d_\vec x \vec v)\cdot \vec
D\right\}-\frac{1}{2c}\left\{\vec v\cdot\nabla_\vec x \left(|\vec
D|^2+|\vec B|^2\right)\right\}\\=\frac{4\pi\rho}{c}\vec v\cdot \vec
D -\frac{1}{c}\left(div_\vec x\left\{\vec D\otimes \vec D+\vec
B\otimes \vec B-\frac{1}{2}\left(|\vec D|^2+|\vec
B|^2\right)I\right\}\right)\cdot \vec v\\+\frac{1}{c}div_\vec
x\left\{(\vec D\otimes \vec D+ \vec B\otimes \vec B)\cdot \vec
v-\left(|\vec D|^2+|\vec B|^2\right)\vec v\right\}.
\end{multline}
Therefore, by \er{hvkgkjgkjbjbb} and \er{hvkgkjgkjbjbbjhiuh} we
obtain
\begin{multline}\label{hvkgkjgkjbjbbklnkn}
\frac{1}{2c}\frac{\partial}{\partial t}\left(|\vec D|^2+|\vec
B|^2\right)+\frac{1}{2c}div_\vec x\left\{\left(|\vec D|^2+|\vec
B|^2\right)\vec v\right\}=\\ \frac{1}{c}div_\vec x\left\{(\vec
D\otimes \vec D+ \vec B\otimes \vec B)\cdot \vec
v-\frac{1}{2}\left(|\vec D|^2+|\vec B|^2\right)\vec v-c \vec D\times
\vec B\right\}
\\-\frac{1}{c}\left(div_\vec x\left\{\vec D\otimes \vec D+\vec B\otimes
\vec B-\frac{1}{2}\left(|\vec D|^2+|\vec
B|^2\right)I\right\}\right)\cdot \vec v-\frac{4\pi}{c}(\vec j-\rho
\vec v)\cdot \vec D.
\end{multline}
Thus, since
\begin{equation}\label{hvkgkjgkjbjbbklnknihyhioguygghf}
(\vec j-\rho \vec v)\cdot \vec D=(\vec j-\rho \vec v)\cdot
\left(\vec E+\frac{1}{c}\,\vec v\times \vec B\right) =\vec j\cdot
\vec E - \vec v\cdot \left(\rho \vec E+\frac{1}{c}\,\vec j\times
\vec B\right),
\end{equation}
we rewrite \er{hvkgkjgkjbjbbklnkn} in the form
\er{hvkgkjgkjbjbbklnknhihiokh}.
\end{proof}
\begin{lemma}\label{guigiukhn}
Let $\vec D,\vec B,\vec E,\vec H,\rho,\vec j,\vec v$ be solutions of
\er{MaxVacFull1bjkgjhjhgjaaaK}. Then
\begin{multline}\label{hvkgkjgkjbjkj}
\frac{\partial}{\partial t}\left(\frac{1}{4\pi c}\,\vec D\times \vec
B\right)+div_\vec x\left\{\left(\frac{1}{4\pi c}\vec D\times \vec
B\right)\otimes \vec v\right\}=-(d_\vec x \vec
v)^T\cdot\left(\frac{1}{4\pi c}\vec D\times \vec
B\right)\\+\frac{1}{4\pi}div_\vec x\left\{\vec D\otimes \vec D+\vec
B\otimes \vec B-\frac{1}{2}\left(|\vec D|^2+|\vec
B|^2\right)I\right\}-\left(\rho \vec E+\frac{1}{c}\vec j\times \vec
B\right).
%=\\ \frac{1}{c}\left\{d_\vec x(D\times B)\right\}^T\cdot v+div_{\vec x}\left\{D\otimes D+B\otimes
%B-\frac{1}{2}\left(|D|^2+|B|^2+\frac{2}{c}v\cdot(D\times
%B)\right)I\right\}-4\pi\rho E-\frac{4\pi}{c}\vec j\times B.
\end{multline}
\end{lemma}
\begin{proof}
By \er{MaxVacFull1bjkgjhjhgjaaaK} we have:
\begin{multline}\label{hvkgkjgkj}
\frac{\partial}{\partial t}\left(\frac{1}{c}\,\vec D\times \vec
B\right)=\frac{1}{c}\frac{\partial \vec D}{\partial t}\times \vec
B+\vec D\times\frac{1}{c}\frac{\partial \vec B}{\partial
t}=\left(curl_\vec x
\vec H-\frac{4\pi}{c}\vec j\right)\times \vec B-\vec D\times curl_\vec x \vec E=\\
curl_\vec x\left(\vec B+\frac{1}{c}\,\vec v\times \vec
D\right)\times \vec B-\vec D\times curl_\vec x
\left(\vec D-\frac{1}{c}\,\vec v\times \vec B\right)-\frac{4\pi}{c}\vec j\times \vec B=\\
\frac{1}{c}\,\vec D\times curl_\vec x (\vec v\times \vec
B)+\frac{1}{c}curl_\vec x (\vec v\times \vec D)\times \vec B -\vec
D\times curl_\vec x \vec D-\vec B\times curl_\vec x \vec
B-\frac{4\pi}{c}\vec j\times \vec B.
\end{multline}
On the other hand, by \er{apfrm6} and \er{MaxVacFull1bjkgjhjhgjaaaK}
we obtain
\begin{multline}\label{hvkgkjgkhjfjfj}
\frac{1}{c}\,\vec D\times curl_\vec x (\vec v\times \vec
B)+\frac{1}{c}curl_\vec x (\vec v\times
\vec D)\times \vec B=\\
(div_\vec x \vec B)\frac{1}{c}\,\vec D\times \vec v -(div_\vec x
\vec v)\frac{1}{c}\,\vec D\times \vec B +\frac{1}{c}\,\vec
D\times\left\{(d_\vec x \vec v)\cdot \vec
B\right\}-\frac{1}{c}\,\vec D\times\left\{(d_\vec x \vec B)\cdot
\vec v\right\}\\+\frac{1}{c}(div_\vec x \vec D)\,\vec v\times \vec B
-\frac{1}{c}(div_\vec x \vec v)\,\vec D\times \vec B
+\frac{1}{c}\left\{(d_\vec x \vec v)\cdot \vec D\right\}\times
\vec B-\frac{1}{c}\left\{(d_\vec x \vec D)\cdot \vec v\right\}\times \vec B=\\
 \frac{1}{c}\,\vec D\times\left\{(d_\vec x
\vec v)\cdot \vec B\right\}+\frac{1}{c}\left\{(d_\vec x \vec v)\cdot
\vec D\right\}\times \vec B\\-2(div_\vec x \vec v)\frac{1}{c}\,\vec
D\times \vec B-\frac{1}{c}\left\{d_\vec x (\vec D\times \vec
B)\right\}\cdot \vec v+\frac{4\pi\rho}{c}\,\vec v\times \vec B
=\\
\frac{1}{c}\,\vec D\times\left\{(d_\vec x \vec v)\cdot \vec
B\right\}+\frac{1}{c}\left\{(d_\vec x \vec v)\cdot \vec
D\right\}\times \vec B\\-(div_\vec x \vec v)\frac{1}{c}\,\vec
D\times \vec B-\frac{1}{c}div_\vec x\left\{(\vec D\times \vec
B)\otimes \vec v\right\}+\frac{4\pi\rho}{c}\,\vec v\times \vec B,
\end{multline}
and by \er{apfrm8} and \er{MaxVacFull1bjkgjhjhgjaaaK} we deduce
\begin{multline}\label{hvkgkjgkjiuiokj}
-\vec D\times curl_\vec x \vec D-\vec B\times curl_\vec x \vec
B=(d_\vec x \vec D)\cdot \vec D-\frac{1}{2}\nabla_\vec x |\vec
D|^2+(d_\vec x \vec B)\cdot \vec B-\frac{1}{2}\nabla_\vec x |\vec
B|^2\\=div_\vec x\left\{\vec D\otimes \vec D+\vec B\otimes \vec
B-\frac{1}{2}\left(|\vec D|^2+|\vec B|^2\right)I\right\}-4\pi\rho
\vec D,
\end{multline}
where $I\in\R^{3\times 3}$ is the unit matrix (identity linear
operator). Thus, plugging \er{hvkgkjgkhjfjfj} and
\er{hvkgkjgkjiuiokj} into \er{hvkgkjgkj} and using \er{apfrm10}, we
obtain
\begin{multline}\label{hvkgkjgkjbjkjghiooiyiokk}
\frac{\partial}{\partial t}\left(\frac{1}{c}\,\vec D\times \vec
B\right)+div_\vec x\left\{\left(\frac{1}{c}\vec D\times \vec
B\right)\otimes \vec v\right\}=\\ \frac{1}{c}\,\vec
D\times\left\{(d_\vec x \vec v)\cdot \vec
B\right\}+\frac{1}{c}\left\{(d_\vec x \vec v)\cdot \vec
D\right\}\times \vec B-(div_\vec x \vec v)\frac{1}{c}\,\vec D\times
\vec B
\\+div_\vec x\left\{\vec D\otimes \vec D+\vec B\otimes
\vec B-\frac{1}{2}\left(|\vec D|^2+|\vec
B|^2\right)I\right\}-4\pi\rho \vec D-\frac{4\pi}{c}(\vec j-\rho \vec
v)\times \vec B\\=-\frac{1}{c}(d_\vec x \vec v)^T\cdot(\vec D\times
\vec B)+div_\vec x\left\{\vec D\otimes \vec D+\vec B\otimes \vec
B-\frac{1}{2}\left(|\vec D|^2+|\vec B|^2\right)I\right\}-4\pi\rho
\vec E-\frac{4\pi}{c}\vec j\times \vec B
\\ =\frac{1}{c}\left\{d_\vec x(\vec D\times \vec B)\right\}^T\cdot \vec v+div_\vec x\left\{\vec D\otimes \vec D+\vec B\otimes
\vec B-\frac{1}{2}\left(|\vec D|^2+|\vec B|^2+\frac{2}{c}\vec
v\cdot(\vec D\times \vec B)\right)I\right\}\\-4\pi\rho \vec
E-\frac{4\pi}{c}\vec j\times \vec B.
\end{multline}
So we finally deduce \er{hvkgkjgkjbjkj}.
\end{proof}















\begin{thebibliography}{66}


\bibitem{PC} P. Cornille, {\em Advansed Electromagnetism and Vacuum Physics}, World Scientific series in Contemporary Chemical
Physics, vol. {\bf 21}.

%\bibitem{Hertz} H. Hertz, {\em \"{U}ber die Grundgleichungen der Elektrodynamik f\"{u}r bewegte K\"{o}rper}, G\"{o}ttinger Nachr. M\"{a}rz 1890 und
%Annalen der Physik Bd. {\bf 41}, Ges. Werke, Bd. II, 2. Aufl. Leipzig 1894, S.256 - 285.



\bibitem{LL}
L.D. Landau, E.M. Lifshitz,  {\em Quantum Mechanics:
Non-Relativistic Theory. Vol. 3} (3rd ed.) (1977), Pergamon Press.


\end{thebibliography}















\end{document}
